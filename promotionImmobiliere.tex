\documentclass[11pt,a4paper]{report}
\usepackage[utf8]{inputenc}
\usepackage{amsmath}
\usepackage{amsfonts}
\usepackage{amssymb}
\usepackage{graphicx}
\usepackage{eurosym}
\usepackage[french]{babel}
\usepackage{hyperref}

\title{Promotion Immobilière}
\author{Hugues \textsc{Perinet-Marquet} \\
	Professeur à Paris II}
\begin{document}
	\maketitle
	\part{Acquisition de droits sur un sol privé}
	\chapter{L’acquisition d’un droit de propriété}
	L’achat du terrain : renvoi au droit civil général.
	7 questions toutefois :
	\section{La qualité du terrain}
	\subsection{Obligations d’information}
	Article 1112-1 du code civil
	\begin{quotation}
		Celle des parties qui connaît une information dont l'importance est déterminante pour le consentement de l'autre
	doit l'en informer dès lors que, légitimement, cette dernière ignore cette information ou fait confiance à son
	cocontractant.
	\end{quotation}
	Néanmoins, ce devoir d'information ne porte pas sur l'estimation de la valeur de la prestation.
	Ont une importance déterminante les informations qui ont un lien direct et nécessaire avec le contenu du contrat
	ou la qualité des parties.
	Il incombe à celui qui prétend qu'une information lui était due de prouver que l'autre partie la lui devait, à
	charge pour cette autre partie de prouver qu'elle l'a fournie.
	Les parties ne peuvent ni limiter, ni exclure ce devoir.
	Outre la responsabilité de celui qui en était tenu, le manquement à ce devoir d'information peut entraîner
	l'annulation du contrat dans les conditions prévues aux articles 1130 et suivants.
	Article 1112-2
	Celui qui utilise ou divulgue sans autorisation une information confidentielle obtenue à l'occasion des
	négociations engage sa responsabilité dans les conditions du droit commun.
	\subsubsection{obligation d’information de L. 514-20 C env au regard des installations classées.}
	\paragraph{obligation de tout vendeur} : Lorsqu'une installation soumise à autorisation ou à enregistrement a été
	exploitée sur un terrain, le vendeur de ce terrain est tenu d'en informer par écrit l'acheteur ; il l'informe
	également, pour autant qu'il les connaisse, des dangers ou inconvénients importants qui résultent de
	l'exploitation.
	Nécessité d’une installation classée
	Cour de cassation chambre civile 3 22 novembre 2018 \No de pourvoi: 17-26209 Publié au bulletin : « Mais
	attendu que l'article L 514-20 du code de l'environnement, qui dispose que, lorsqu'une installation classée
	soumise à autorisation ou à enregistrement a été exploitée sur un terrain, le vendeur de ce terrain est tenu d'en
	informer par écrit l'acheteur, nécessite, pour son application, qu'une installation classée ait été implantée, en
	tout ou partie, sur le terrain vendu ; qu'ayant relevé qu'aucune des installations classées implantées sur le site
	industriel de Sevran-Livry-Gargan n'avait été exploitée sur les parcelles cédées à la SCI GDLMA et retenu qu'il
	n'était pas établi qu'une installation de nature, par sa proximité ou sa connexité, à en modifier les dangers ou
	inconvénients, au sens de l'article R. 512-32 du même code, y eût été exploitée, la cour d'appel en a déduit à bon
	droit que le vendeur n'avait pas manqué à son obligation d'information ; »
	Non application de la règle à une installation en activité2
	Cass 3 e civ 9 avril 2008 Bull 2008 III \no 70
	Mais attendu que l'alinéa 1 de l'article L. 514-20 du code de l'environnement, qui dispose que lorsqu'une
	installation soumise à autorisation a été exploitée sur un terrain, le vendeur est tenu d'en informer par écrit
	l'acheteur, ne s'applique pas à la vente d'un terrain sur lequel l'exploitation d'une installation classée est en
	cours ; qu'ayant constaté que les sociétés Natexis bail, Fructicomi et Unibail avaient cédé une propriété bâtie,
	exploitée à la date de cette vente dans des conditions relevant du régime de l'autorisation, la cour d'appel en a
	déduit à bon droit que l'alinéa 1 de cet article n'était pas applicable ;
	Nécessité que l’installation ait été soumise à autorisation ou à déclaration
	Cass 3e civ 20 juin 2007, \no 06-15663
	Vu l'article L. 514-20 du code de l'environnement dans sa rédaction applicable en la cause ;
	Attendu, selon l'arrêt attaqué (Paris, 23 septembre 2004), que, le 15 juin 2001, la société Biscuiterie du Nord a
	vendu à la société Vincent Palaric un terrain situé à Saint-Ouen ; qu'assignée par la société venderesse qui lui
	reprochait d'avoir manqué à son obligation contractuelle du surélever un mur, la société Vincent Palaric a
	reconventionnellement demandé des dommages-intérêts correspondant aux frais d'enlèvement d'équipements et
	d'installations dont la société Biscuiterie du Nord lui aurait dissimulé la présence sur le terrain vendu ;
	Attendu que pour accueillir cette demande, l'arrêt retient que la présence d'installations sujettes à autorisation
	et déclaration en application de la loi du 19 juillet 1976 n'a pas été mentionnée à l'acte de vente du 15 juin 2001
	et qu'en application des dispositions de cette loi, la SCI Vincent Palarix est fondée à obtenir, sous forme de
	dommages-intérêts, la restitution d'une partie du prix de vente ;
	Qu'en statuant ainsi, sans rechercher si les installations étaient soumises à autorisation ou à déclaration, la
	cour d'appel n'a pas donné de base légale à sa décision ;
	Nécessité que l’exploitation ait été soumise à autorisation au moment où elle s’exercait
	: Cass 3 e civ, 17 nov 2004 Bulletin 2004 III \No  204 p. 184 : Attendu que lorsqu'une installation soumise à
	autorisation a été exploitée sur un terrain, le vendeur est tenu d'en informer, par écrit, l'acheteur ;
	Attendu que pour dire ce texte applicable à la vente intervenue le 30 mars 1994, l'arrêt retient que les
	dispositions de l'article L. 514-20 du Code de l'environnement sont applicables aux installations de la nature de
	celles soumises à autorisation sous l'empire de la loi du 19 juillet 1976, modifiée, alors même qu'elles auraient
	cessé d'être exploitées antérieurement à son entrée en vigueur dès lors que ces installations restent susceptibles,
	du fait de leur existence même, de présenter les dangers ou inconvénients mentionnés à l'article L. 511-1 du
	Code de l'environnement ;
	Qu'en statuant par de tels motifs qui ne permettent pas de déterminer si l'activité exercée par la société Latil
	jusqu'en 1941 était, au regard de la législation et règlementation en vigueur à cette date, soumise à autorisation,
	la cour d'appel a violé le texte susvisé ;
	Non application en matière de DPU avant 2014
	Cour de cassation chambre civile 3 15 septembre 2016 \No  de pourvoi: 15-21916 Publié au bulletin
	Attendu, selon l'arrêt attaqué (Paris, 21 mai 2015), que la Société d'aménagement et de développement des villes
	et du département du Val-de-Marne (la SADEV 94), bénéficiaire d'une délégation du droit de préemption urbain
	consentie par la commune d'Ivry-sur-Seine, a décidé d'exercer ce droit, à l'occasion d'une déclaration
	d'intention d'aliéner un terrain faite par la société Soft ADS immobilier ; que le juge de l'expropriation, saisi par
	la SADEV 94, a fixé le prix à la somme de 3 640 000 euros ; que la SADEV 94 ayant refusé de signer l'acte de
	vente, en invoquant un manquement du vendeur à l'obligation d'information environnementale, prévue par
	l'article L. 514-20 du code de l'environnement, la société Soft ADS immobilier l'a assignée en réitération de la
	vente par voie judiciaire ; ...
	Mais attendu qu'ayant exactement retenu qu'en application de l'article L. 213-2 du code de l'urbanisme, dans sa
	rédaction antérieure à la loi du 24 mars 2014, le vendeur n'avait pas l'obligation formelle d'informer le titulaire
	du droit de préemption, dans la déclaration d'intention d'aliéner, qu'une installation soumise à autorisation ou à
	enregistrement avait été antérieurement exploitée sur le terrain, objet de la vente, la cour d'appel en a déduit à
	bon droit que la SADEV 94 ne pouvait se prévaloir de l'article L. 514-20 du code de l'environnement
	-obligation du vendeur exploitant Si le vendeur est l'exploitant de l'installation, il indique également par écrit
	à l'acheteur si son activité a entraîné la manipulation ou le stockage de substances chimiques ou radioactives.
	L'acte de vente atteste de l'accomplissement de cette formalité.3
	- Dans les deux cas l’information latérale reçue par le vendeur ne dispense pas le vendeur de son
	obligation : Cass 3 e civ 12 jan 2005 Bull 2005 III \No  8 p. 7 : Attendu que lorsqu'une installation soumise à
	autorisation a été exploitée sur un terrain, le vendeur de ce terrain est tenu d'en informer par écrit l'acheteur ; il
	l'informe également, pour autant qu'il les connaisse, des dangers ou inconvénients importants qui résultent de
	l'exploitation ;
	Attendu que pour rejeter la demande de résolution formée par la commune de Dardilly, l'arrêt retient que celle-
	ci ne pouvait soutenir qu'elle ignorait qu'une installation classée était exploitée sur la parcelle acquise et
	entraînait des nuisances dès lors que des arrêtés préfectoraux de 1975, 1980, 1982 et 1988 lui avaient été
	notifiés et que des courriers avaient été échangés entre elle et la société exploitante suivis d'une réunion par elle
	organisée en 1988 ;
	Qu'en statuant ainsi, alors que la venderesse s'était abstenue d'informer par écrit l'acquéreur à l'occasion de la
	vente, la cour d'appel a violé le texte susvisé ;
	Cf Toutefois Cass 3e civ 10 sept 2008 \no 07-17086
	Attendu, selon l'arrêt attaqué (Aix-en-Provence, 12 avril 2007), qu'en 1999 la commune de Marseille a acquis,
	par préemption, un terrain appartenant aux consorts X..., sur lequel avait été exploité un dépôt de métaux, en
	vue d'y aménager des voies de circulation ; que le projet d'aménagement urbain ne s'étant pas réalisé, la société
	d'HLM Provence logis, pressentie pour acquérir le terrain, y a renoncé en raison du coût des travaux de
	dépollution nécessaires en cas de construction d'immeubles, une étude de sols effectuée à la demande de la
	commune de Marseille ayant révélé une pollution du terrain par métaux et hydrocarbures sur une profondeur de
	deux mètres ; que la commune de Marseille a assigné les consorts X... en réduction du prix de vente du bien et
	en paiement de dommages-intérêts sur le fondement des articles 1641 et suivants du code civil, L. 514-20 du
	code de l'environnement et 1116 du code civil ;
	Mais attendu qu'ayant relevé que la demande en restitution d'une partie du prix de vente était fondée sur la
	garantie des vices cachés et sur le dol, et constaté que les photographies produites, anciennes, montraient que le
	terrain servait depuis plusieurs dizaines d'années avant l'acquisition par la commune de Marseille de dépôt de
	ferrailles et matériaux industriels divers, y compris quantité de bidons métalliques vides pouvant avoir contenu
	divers liquides et huiles, et qu'il était de notoriété publique que ce terrain avait servi depuis 1945 de déchetterie
	de ferrailles diverses destinées à la récupération industrielle, la cour d'appel, qui a effectué la recherche
	prétendument omise et qui n'était pas tenue d'effectuer une recherche qui n'était pas demandée, en a
	souverainement déduit, sans violer l'article 954, dernier alinéa, du code de procédure civile, que la commune de
	Marseille, qui avait acquis le terrain en état de "friche industrielle", ne pouvait ignorer qu'il était sérieusement
	pollué et que cela entraînerait un coût de dépollution dans l'hypothèse où elle déciderait de l'utiliser ou de le
	revendre comme terrain à bâtir, a pu retenir, abstraction faite de motifs surabondants relatifs à la clause de
	non-garantie et sans être tenue de répondre à des conclusions invoquant l'article L. 514-20 du code de
	l'environnement pour écarter la clause d'exclusion de garantie des vices cachés que ses constatations rendaient
	inopérantes, que la commune était irrecevable à exercer une action estimatoire plus de quatre ans après son
	acquisition ;
	- Si l’article L 514-20 est respecté, l’information faite est suffisante ; Cass 3 e civ 18 novembre 2009 \No  08-
	19052
	Mais attendu qu'ayant constaté que le vendeur avait satisfait à son obligation d'information prévue par l'article
	L. 514 20 du code de l'environnement ainsi qu'à son obligation de remise en état du site, qu'il avait communiqué,
	préalablement à la signature de l'acte notarié, le rapport d'études des sols établi par la société Botte sondages à
	la demande de la société Chronopost, acquéreur initial, à laquelle s'était substituée la société Caprim, qui
	mentionnait la présence de " blocs " enterrés en laissant prévoir que des déplacements de pieux devraient être
	envisagés ainsi que l'existence d'odeurs très fortes, en précisant qu'il pouvait s'agir d'une pollution aux
	hydrocarbures, et qu'avait été également remis à l'acquéreur le rapport sur la réhabilitation et la dépollution du
	site réalisé par la société Céca, chargée des opérations de dépollution, lequel n'excluait pas le risque de
	découverte d'une pollution liée à l'activité qui y avait été antérieurement exercée, la cour d'appel qui, par motifs
	propres et adoptés, a, d'une part, relevé, sans être tenue de caractériser une acceptation du risque prise de façon
	délibérée et en connaissance de cause, que la société Caprim, professionnel des opérations de construction et de
	promotion immobilière, était informée de la présence d'hydrocarbures et avait conscience du risque de
	pollution, et, d'autre part, retenu que si elle n'avait pas été informée de l'importance de l'encombrement du sous
	sol, elle avait été informée de l'existence des dalles ou cuves enterrées et n'avait pas pris toutes les mesures
	requises pour s'assurer de l'ampleur de ce problème, a, sans être tenue de répondre à des conclusions que ses
	constatations rendaient inopérantes, légalement justifié sa décision ;4
	-Sanction : - A défaut, et si une pollution constatée rend le terrain impropre à la destination précisée dans le
	contrat, dans un délai de deux ans à compter de la découverte de la pollution, l'acheteur a le choix de demander
	la résolution de la vente ou de se faire restituer une partie du prix ; il peut aussi demander la réhabilitation du
	site aux frais du vendeur, lorsque le coût de cette réhabilitation ne paraît pas disproportionné par rapport au
	prix de vente.
	\subsubsection{Obligation d’information du propriétaire d’un terrain exposé à un risque (art L 125-5 C envir, loi
	22 mars 2012, O du 10 fev 2016 et R 125-25 et s D du 28 oct 2015)}
	- Obligation de fournir à l’acquéreur un état des risques
	I. - Les acquéreurs ou locataires de biens immobiliers situés dans des zones couvertes par un plan de prévention
	des risques technologiques ou par un plan de prévention des risques naturels prévisibles, prescrit ou approuvé,
	dans des zones de sismicité ou dans des zones à potentiel radon définies par voie réglementaire, sont informés
	par le vendeur ou le bailleur de l'existence de ces risques.
	II. ― En cas de mise en location de l'immeuble, l'état des risques naturels et technologiques est fourni au
	nouveau locataire dans les conditions et selon les modalités prévues à l'article 3-3 de la loi \no 89-462 du 6
	juillet 1989 tendant à améliorer les rapports locatifs et portant modification de la loi \no 86-1290 du 23
	décembre 1986.
	L'état des risques naturels et technologiques, fourni par le bailleur, est joint aux baux commerciaux mentionnés
	aux articles L. 145-1 et L. 145-2 du code de commerce.
	III. ― Le préfet arrête la liste des communes dans lesquelles les dispositions du I et du II sont applicables ainsi
	que, pour chaque commune concernée, la liste des risques et des documents à prendre en compte.
	Cass 3 e civ 19 septembre 2019 \No  de pourvoi: 18-16700 18-16935 18-17562 Publié au bulletin : Mais attendu
	qu'il résulte des dispositions combinées de l'article L. 125-5 du code de l'environnement et des articles L. 271-4
	et L. 271-5 du code de la construction et de l'habitation, dans leur rédaction alors applicable, que, si, après la
	promesse de vente, la parcelle sur laquelle est implanté l'immeuble objet de la vente est inscrite dans une zone
	couverte par un PPRNP prescrit ou approuvé, le dossier de diagnostic technique est complété, lors de la
	signature de l'acte authentique de vente, par un état des risques ou par une mise à jour de l'état des risques
	existants ; qu'ayant relevé que le terrain de camping était situé en zone rouge du plan de prévention des risques
	d'inondation approuvé par arrêté préfectoral du 25 novembre 2008 publié le 18 février 2009 au recueil des actes
	administratifs des services de l'Etat dans le département, et que le dossier de diagnostic technique annexé au
	contrat de vente n'en faisait pas état, la cour d'appel, qui a retenu à bon droit que la consultation de ce recueil
	était susceptible de renseigner utilement les cocontractants, le site internet de la préfecture n'ayant qu'une
	valeur informative, en a exactement déduit qu'à défaut d'information sur l'existence des risques visés par le
	PPRNP donnée par le vendeur dans l'acte authentique établi le 24 mars 2009, il y avait lieu de prononcer la
	résolution de la vente ;
	-
	Obligation d’informer des sinistres intervenus
	IV. ― Lorsqu'un immeuble bâti a subi un sinistre ayant donné lieu au versement d'une indemnité en application
	de l'article L. 125-2 ou de l'article L. 128-2 du code des assurances, le vendeur ou le bailleur de l'immeuble est
	tenu d'informer par écrit l'acquéreur ou le locataire de tout sinistre survenu pendant la période où il a été
	propriétaire de l'immeuble ou dont il a été lui-même informé en application des présentes dispositions. En cas de
	vente de l'immeuble, cette information est mentionnée dans l'acte authentique constatant la réalisation de la
	vente.
	-
	Sanctions
	V En cas de non-respect des dispositions du présent article, l'acquéreur ou le locataire peut poursuivre la
	résolution du contrat ou demander au juge une diminution du prix.
	Cour de cassation chambre civile 1 14 février 2018 \No  de pourvoi: 16-27263 : « Vu l'article 1382, devenu
	1240 du code civil ;
	Attendu, selon l'arrêt attaqué, que, suivant acte reçu le 29 septembre 2005 par M. C..., notaire salarié de la
	société civile professionnelle Étienne Z..., Sophie Z... , G... Z...-C...
	désormais dénommée Sophie Z... et G...
	Z... (le notaire), les consorts E... ont vendu à M. Y... un immeuble à usage d'habitation ; qu'alléguant avoir
	découvert, en juillet 2011, que l'immeuble était situé en zone inondable, M. Y... a assigné le notaire en
	responsabilité pour manquement à son devoir de conseil et d'information, et en indemnisation ;
	Attendu que, pour rejeter cette demande, l'arrêt retient que le notaire, qui a rempli son obligation de demander5
	une note de renseignements d'urbanisme sur laquelle n'apparaît aucune mention pouvant faire suspecter le
	caractère inondable de la zone ou l'existence d'un plan de prévention des risques d'inondation, n'est pas tenu de
	vérifier l'existence d'un arrêté préfectoral en ce sens, que celui-ci, régulièrement publié, peut être recherché et
	consulté par l'acquéreur, aussi bien que signalé par les vendeurs, et que le classement en zone urbaine peu
	dense ne doit pas de facto inciter le notaire à faire cette vérification sans y être expressément invité par
	l'acheteur ;
	Qu'en statuant ainsi, alors que la note de renseignements d'urbanisme ne dispensait pas le notaire de son
	obligation de s'informer sur l'existence d'un arrêté préfectoral publié, relatif à un plan de prévention des risques
	d'inondation, la cour d'appel a violé le texte susvisé »
	Cf également art R 125-23 à R 125-27
	c)
	\subsubsection{Obligation d’information pour les immeubles situés dans un secteur d’information}
	Article L 125-6
	I. ― L'Etat élabore, au regard des informations dont il dispose, des secteurs d'information sur les sols qui
	comprennent les terrains où la connaissance de la pollution des sols justifie, notamment en cas de changement
	d'usage, la réalisation d'études de sols et de mesures de gestion de la pollution pour préserver la sécurité, la
	santé ou la salubrité publiques et l'environnement.
	II. ― Le représentant de l'Etat dans le département recueille l'avis des maires des communes sur le territoire
	desquelles sont situés les projets de secteur d'information sur les sols et, le cas échéant, celui des présidents des
	établissements publics de coopération intercommunale compétents en matière d'urbanisme. Il informe les
	propriétaires des terrains concernés.
	Les secteurs d'information sur les sols sont arrêtés par le représentant de l'Etat dans le département.
	III. ― Les secteurs d'information sur les sols sont indiqués sur un ou plusieurs documents graphiques et annexés
	au plan local d'urbanisme ou au document d'urbanisme en tenant lieu ou à la carte communale.
	IV. ― L'Etat publie, au regard des informations dont il dispose, une carte des anciens sites industriels et
	activités de services. Le certificat d'urbanisme prévu à l'article L. 410-1 du code de l'urbanisme indique si le
	terrain est situé sur un site répertorié sur cette carte ou sur un ancien site industriel ou de service dont le service
	instructeur du certificat d'urbanisme a connaissance.
	V. ― Un décret en Conseil d'Etat définit les modalités d'application du présent article.
	Article L125-7
	Sans préjudice de l'article L. 514-20 et de l'article L. 125-5, lorsqu'un terrain situé en secteur d'information sur
	les sols mentionné à l'article L. 125-6 fait l'objet d'un contrat de vente ou de location, le vendeur ou le bailleur
	du terrain est tenu d'en informer par écrit l'acquéreur ou le locataire. Il communique les informations rendues
	publiques par l'Etat, en application de l'article L. 125-6. L'acte de vente ou de location atteste de
	l'accomplissement de cette formalité.
	A défaut et si une pollution constatée rend le terrain impropre à la destination précisée dans le contrat, dans un
	délai de deux ans à compter de la découverte de la pollution, l'acquéreur ou le locataire a le choix de demander
	la résolution du contrat ou, selon le cas, de se faire restituer une partie du prix de vente ou d'obtenir une
	réduction du loyer. L'acquéreur peut aussi demander la réhabilitation du terrain aux frais du vendeur lorsque le
	coût de cette réhabilitation ne paraît pas disproportionné par rapport au prix de vente.
	Un décret en Conseil d'Etat définit les modalités d'application du présent article.
	2) Nécessité de dépollution liée à une opération de construction
	a) Possibilité de substitution au dernier exploitant d’un tiers dépollueur qui peut être l’opérateur
	Article L512-21
	I. - Lors de la mise à l'arrêt définitif d'une installation classée pour la protection de l'environnement ou
	postérieurement à cette dernière, un tiers intéressé peut demander au représentant de l'Etat dans le département
	de se substituer à l'exploitant, avec son accord, pour réaliser les travaux de réhabilitation en fonction de l'usage
	que ce tiers envisage pour le terrain concerné.
	II. - Lorsque l'usage ou les usages envisagés par le tiers demandeur sont d'une autre nature que ceux définis,
	selon le cas, en application des articles L. 512-6-1, L. 512-7-6 ou L. 512-12-1, le tiers demandeur recueille
	l'accord du dernier exploitant, du maire ou du président de l'établissement public de coopération6
	intercommunale compétent en matière d'urbanisme et, s'il ne s'agit pas de l'exploitant, du propriétaire du terrain
	sur lequel est sise l'installation.
	III. - Le tiers demandeur adresse au représentant de l'Etat dans le département un mémoire de réhabilitation
	définissant les mesures permettant d'assurer la compatibilité entre l'usage futur envisagé et l'état des sols.
	IV. - Le représentant de l'Etat dans le département se prononce sur l'usage proposé dans le cas mentionné au II
	et peut prescrire au tiers demandeur les mesures de réhabilitation nécessaires pour l'usage envisagé.
	V. - Le tiers demandeur doit disposer de capacités techniques suffisantes et de garanties financières couvrant la
	réalisation des travaux de réhabilitation définis au IV pour assurer la compatibilité entre l'état des sols et
	l'usage défini.
	Toute modification substantielle des mesures prévues dans le mémoire de réhabilitation rendant nécessaires des
	travaux de réhabilitation supplémentaires pour assurer la compatibilité entre l'état des sols et le nouvel usage
	envisagé peut faire l'objet d'une réévaluation du montant des garanties financières.
	VI. - Les arrêtés préfectoraux prévus au présent article peuvent faire l'objet des mesures de police prévues au
	chapitre Ier du titre VII du livre Ier.
	VII. - En cas de défaillance du tiers demandeur et de l'impossibilité de mettre en œuvre les garanties financières
	mentionnées au V, le dernier exploitant met en œuvre les mesures de réhabilitation pour l'usage défini dans les
	conditions prévues aux articles L. 512-6-1(mise à l’arrêt définitif d’une installation soumise à autorisation), L.
	512-7-6 (mise à l’arrêt définitif d’une installation soumise à enregistrement) et L. 512-12-1 (mise à l’arrêt d’une
	installation soumise à déclaration).
	VIII. - Un décret en Conseil d'Etat définit les modalités d'application du présent article. Il prévoit, notamment,
	les modalités de substitution du tiers et le formalisme de l'accord de l'exploitant ou du propriétaire.
	Vérification nécessaire de ce que l’état du sol (dépollué) est apte à recevoir l’opération envisagée
	Article L 556-1
	Sans préjudice des articles L. 512-6-1, L. 512-7-6 et L. 512-12-1, sur les terrains ayant accueilli une installation
	classée mise à l'arrêt définitif et régulièrement réhabilitée pour permettre l'usage défini dans les conditions
	prévues par ces mêmes articles, lorsqu'un usage différent est ultérieurement envisagé, le maître d'ouvrage à
	l'initiative du changement d'usage doit définir des mesures de gestion de la pollution des sols et les mettre en
	œuvre afin d'assurer la compatibilité entre l'état des sols et la protection de la sécurité, de la santé ou de la
	salubrité publiques, l'agriculture et l'environnement au regard du nouvel usage projeté.
	Ces mesures de gestion de la pollution sont définies en tenant compte de l'efficacité des techniques de
	réhabilitation dans des conditions économiquement acceptables ainsi que du bilan des coûts, des inconvénients
	et avantages des mesures envisagées. Le maître d'ouvrage à l'initiative du changement d'usage fait attester de
	cette prise en compte par un bureau d'études certifié dans le domaine des sites et sols pollués, conformément à
	une norme définie par arrêté du ministre chargé de l'environnement, ou équivalent. Le cas échéant, cette
	attestation est jointe au dossier de demande de permis de construire ou d'aménager.
	Le cas échéant, s'il demeure une pollution résiduelle sur le terrain concerné compatible avec les nouveaux
	usages, le maître d'ouvrage à l'initiative du changement d'usage en informe le propriétaire et le représentant de
	l'Etat dans le département. Le représentant de l'Etat dans le département peut créer sur le terrain concerné un
	secteur d'information sur les sols.
	En cas de modification de la consistance du projet initial, le maître d'ouvrage à l'initiative de cette modification
	complète ou adapte, si nécessaire, les mesures de gestion définies au premier alinéa.
	Les modalités d'application du présent article sont définies par décret en Conseil d'Etat.
	Article L556-2
	Les projets de construction ou de lotissement prévus dans un secteur d'information sur les sols tel que prévu à
	l'article L. 125-6 font l'objet d'une étude des sols afin d'établir les mesures de gestion de la pollution à mettre en
	œuvre pour assurer la compatibilité entre l'usage futur et l'état des sols.
	Pour les projets soumis à permis de construire ou d'aménager, le maître d'ouvrage fournit dans le dossier de
	demande de permis une attestation garantissant la réalisation de cette étude des sols et de sa prise en compte
	dans la conception du projet de construction ou de lotissement. Cette attestation doit être établie par un bureau
	d'études certifié dans le domaine des sites et sols pollués, conformément à une norme définie par arrêté du
	ministre chargé de l'environnement, ou équivalent.
	L'attestation n'est pas requise lors du dépôt de la demande de permis d'aménager par une personne ayant
	qualité pour bénéficier de l'expropriation pour cause d'utilité publique, dès lors que l'opération de lotissement a
	donné lieu à la publication d'une déclaration d'utilité publique.
	L'attestation n'est pas requise lors du dépôt de la demande de permis de construire, lorsque la construction
	projetée est située dans le périmètre d'un lotissement autorisé ayant fait l'objet d'une demande comportant une
	attestation garantissant la réalisation d'une étude des sols et sa prise en compte dans la conception du projet7
	d'aménagement.
	Le présent article s'applique sans préjudice des dispositions spécifiques sur la pollution des sols déjà prévues
	dans les documents d'urbanisme.
	Un décret en Conseil d'Etat définit les modalités d'application du présent article.
	Article L556-3
	I. ― En cas de pollution des sols ou de risques de pollution des sols présentant des risques pour la santé, la
	sécurité, la salubrité publiques et l'environnement au regard de l'usage pris en compte, l'autorité titulaire du
	pouvoir de police peut, après mise en demeure, assurer d'office l'exécution des travaux nécessaires aux frais du
	responsable. L'exécution des travaux ordonnés d'office peut être confiée par le ministre chargé de
	l'environnement et par le ministre chargé de l'urbanisme à un établissement public foncier ou, en l'absence d'un
	tel établissement, à l'Agence de l'environnement et de la maîtrise de l'énergie. L'autorité titulaire du pouvoir de
	police peut également obliger le responsable à consigner entre les mains d'un comptable public une somme
	répondant du montant des travaux à réaliser, laquelle sera restituée au fur et à mesure de l'exécution des
	travaux. Les sommes consignées peuvent, le cas échéant, être utilisées pour régler les dépenses entraînées par
	l'exécution d'office. Lorsqu'un établissement public foncier ou l'Agence de l'environnement et de la maîtrise de
	l'énergie intervient pour exécuter des travaux ordonnés d'office, les sommes consignées lui sont réservées à sa
	demande.
	Il est procédé, le cas échéant, au recouvrement de ces sommes comme en matière de créances étrangères à
	l'impôt et au domaine. Pour ce recouvrement, l'Etat bénéficie d'un privilège de même rang que celui prévu à
	l'article 1920 du code général des impôts. Le comptable peut engager la procédure d'avis à tiers détenteur
	prévue à l'article L. 263 du livre des procédures fiscales.
	L'opposition à l'état exécutoire pris en application d'une mesure de consignation ordonnée par l'autorité
	administrative devant le juge administratif n'a pas de caractère suspensif.
	Lorsque, en raison de la disparition ou de l'insolvabilité de l'exploitant du site pollué ou du responsable de la
	pollution, la mise en œuvre des dispositions du premier alinéa du présent I n'a pas permis d'obtenir la
	réhabilitation du site pollué, l'Etat peut, avec le concours financier éventuel des collectivités territoriales,
	confier cette réhabilitation à l'Agence de l'environnement et de la maîtrise de l'énergie.
	Les travaux mentionnés au quatrième alinéa du présent I et, le cas échéant, l'acquisition des immeubles peuvent
	être déclarés d'utilité publique à la demande de l'Etat. La déclaration d'utilité publique est prononcée après
	consultation des collectivités territoriales intéressées et enquête publique menée dans les formes prévues par le
	code de l'expropriation pour cause d'utilité publique. Lorsque l'une des collectivités territoriales intéressées, le
	commissaire enquêteur ou la commission d'enquête a émis un avis défavorable, la déclaration d'utilité publique
	est prononcée par arrêté conjoint du ministre chargé de l'écologie et du ministre chargé de l'urbanisme.
	II. ― Au sens du I, on entend par responsable, par ordre de priorité :
	1\degre  Pour les sols dont la pollution a pour origine une activité mentionnée à l'article L. 165-2, une installation
	classée pour la protection de l'environnement ou une installation nucléaire de base, le dernier exploitant de
	l'installation à l'origine de la pollution des sols, ou la personne désignée aux articles L. 512-21 et L. 556-1,
	chacun pour ses obligations respectives. Pour les sols pollués par une autre origine, le producteur des déchets
	qui a contribué à l'origine de la pollution des sols ou le détenteur des déchets dont la faute y a contribué ;
	2\degre  A titre subsidiaire, en l'absence de responsable au titre du 1\degre , le propriétaire de l'assise foncière des sols
	pollués par une activité ou des déchets tels que mentionnés au 1\degre , s'il est démontré qu'il a fait preuve de
	négligence ou qu'il n'est pas étranger à cette pollution.
	III. ― Un décret en Conseil d'Etat définit les modalités d'application du présent article. Il définit notamment
	l'autorité de police chargée de l'application du présent article.
	3) Conformité du terrain vendu
	- Cour de cassation chambre civile 3 29 février 2012 \No  de pourvoi 11-10318 : Attendu que pour débouter
	la SCI de sa demande, l'arrêt retient que le vendeur avait fourni à l'acquéreur tous les éléments relatifs à l'état
	des travaux de dépollution et des mesures prises pour la réhabilitation du site d'où il ressortait qu'il avait été mis
	un terme aux sources de pollution et à l'extension de celle-ci, mais que subsistait une pollution résiduelle qui
	devait être éliminée progressivement et naturellement, que le vendeur n'avait pris aucun engagement personnel
	de dépollution du site et que l'acquéreur, qui avait connaissance avant de signer l'acte de vente de l'état des
	travaux qui avaient été exécutés, ainsi que de leurs limites, ne pouvait pas reprocher au vendeur de la
	découverte, au cours des travaux de construction, d'une nouvelle poche de contamination résiduelle aux
	hydrocarbures nécessitant une opération complémentaire de dépollution ;
	Qu'en statuant ainsi, alors qu'elle avait relevé que l'acte de vente mentionnait que l'immeuble avait fait l'objet
	d'une dépollution, ce dont il résultait que le bien vendu était présenté comme dépollué et que les vendeurs étaient
	tenus de livrer un bien conforme à cette caractéristique, la cour d'appel a violé le texte susvisé ;8
	- Cass 3 e civ 16 janvier 2013 \No  de pourvoi: 11-27101 :Mais attendu qu'ayant constaté que l'acquéreur
	déclarait être parfaitement informé du fait que le terrain avait servi de cadre à l'exploitation par la société
	SICAP d'une activité de production de résines de synthèse ayant fait l'objet d'une autorisation d'exploiter, qu'en
	annexe de l'acte de vente se trouvait un tableau de stockage recensant de façon exhaustive les produits
	chimiques stockés par l'exploitant dans ses installations bâties sur ce terrain et le rapport d'un expert faisant
	état de l'utilisation de nombreux produits toxiques pouvant avoir pollué le sous-sol et la nappe phréatique
	exigeant des travaux de recherche et des analyses importantes, et relevé que la SCI LM avait été clairement
	informée de l'existence d'un risque de pollution, qu'elle avait renoncé expressément à engager la responsabilité
	du vendeur de ce chef et que la convention des parties avait donc porté sur un terrain comportant un risque de
	pollution connu de l'acquéreur, la cour d'appel a pu, par ces seuls motifs, en déduire que la délivrance du
	terrain était conforme à la convention des parties et que la SCI devait être déboutée de ses demandes dirigées
	contre le vendeur ; A contrario, si l’absence de pollution est entrée dans le champ contractuel : Cass 3 e civ 12
	novembre 2014 \No  de pourvoi: 13-25079
	- Cass 3 e civ 28 janvier 2015 \No  de pourvoi: 13-19945 13-27050 : Mais attendu qu'ayant relevé que
	l'immeuble avait été vendu comme étant raccordé au réseau public d'assainissement et constaté que le
	raccordement n'était pas conforme aux stipulations contractuelles, la cour d'appel, qui n'était pas tenue de
	procéder à une recherche que ses constatations rendaient inopérante, en a exactement déduit que les vendeurs
	avaient manqué à leur obligation de délivrance ;
	4) les éventuelles servitudes de droit privé
	a) -Pb de la connaissance et de l’opposabilité des servitudes
	Cass 3 e civ, 16 septembre 2009 \No : 08-16499 : Mais attendu qu'une servitude est opposable à l'acquéreur de
	l'immeuble grevé si elle a été publiée, si son acte d'acquisition en fait mention, ou encore s'il en connaissait
	l'existence au moment de l'acquisition ; qu'ayant souverainement retenu que M. X... avait connaissance, au
	moment de la vente, de l'existence de la servitude dont était grevée sa parcelle, la cour d'appel en a exactement
	déduit qu'elle lui était opposable ; (adde 16 mars 2011 \no 10-13771 : Mais attendu qu'ayant retenu que la
	convention du 18 octobre 1993 était annexée à l'acte de vente et faisait l'objet d'une mention particulière dans
	cet acte aux termes de laquelle le vendeur déclarait qu'il n'avait créé ni laissé acquérir aucune servitude sur le
	bien en dehors de la servitude constituée au profit des époux Z... et que l'acquéreur déclarait avoir été informé
	du protocole d'accord annexé à l'acte de vente, établi entre Mme X..., vendeur et M. et Mme Z... concernant la
	constitution d'une servitude grevant le terrain cédé et d'une lettre de M. A..., avocat, confirmant l'intention des
	époux Z... de régulariser ladite servitude, la cour d'appel a légalement justifié sa décision en retenant que la
	servitude créée en 1993 était opposable aux acquéreurs, même à défaut de publication ; )
	- Conséquence d’une servitude occulte sur la vente ; Cass 3e civ 27 février 2013 \No  de pourvoi: 11-28783
	Vu l'article 1638 du code civil ;
	Attendu que si l'héritage vendu se trouve grevé, sans qu'il en ait été fait de déclaration, de servitudes non
	apparentes, et qu'elles soient de telle importance qu'il y ait lieu de présumer que l'acquéreur n'aurait pas acheté
	s'il en avait été instruit, il peut demander la résiliation du contrat, si mieux il n'aime se contenter d'une
	indemnité ;
	Attendu que pour condamner in solidum les époux X... et la société du canal de Provence et d'aménagement de
	la région provençale à payer aux époux Y... la somme de 30 000 euros au titre de la perte de valeur du terrain,
	l'arrêt retient que la canalisation, dont l'existence n'a été révélée aux époux Y... qu'après la vente, interdit toute
	construction sur la partie supérieure du terrain présentant plus d'attraits que la partie inférieure, nécessite la
	réalisation d'ouvrages adaptés pour pouvoir être franchie par des véhicules et diminue l'usage de ce terrain sur
	une superficie d'environ 28m 2 et que la présence de cette canalisation constitue donc un vice caché à raison
	duquel les époux X... sont tenus de la garantie prévue par l'article 1641 du code civil ;
	Qu'en statuant ainsi, alors qu'une servitude non apparente ne constitue pas un vice caché mais relève des
	dispositions de l'article 1638 du code civil, la cour d'appel a violé le texte susvisé ;
	b) -Pb de l’extinction des servitudes
	art 703 impossibilité d’user
	Les servitudes cessent lorsque les choses se trouvent en tel état qu'on ne peut plus en user.
	cf Cass 3 e civ 9 juil 2003 Bull III \no 157;
	Mais attendu qu'ayant relevé, par motifs propres et adoptés, que l'acte du 21 juillet 1982 qui avait institué la
	servitude litigieuse sur la parcelle vendue AL 210, au profit de la parcelle AL 209 demeurée la propriété du
	vendeur, précisait que cette servitude était destinée à permettre à celui-ci d'accéder au surplus de son tènement
	à l'Est, qu'il résultait des termes de l'acte que la servitude n'avait été créée qu'en vue d'assurer au propriétaire
	de la parcelle AL 209 une desserte pour un atelier par lui conservé mais devenu depuis la propriété d'un tiers, et9
	que les époux X..., qui avaient acquis en 1988, non pas l'intégralité de la parcelle AL 209, mais seulement la
	partie Nord-Est de celle-ci, n'avaient pas à se rendre dans sa partie Sud-Ouest pour atteindre la rue des Marais,
	qu'ainsi l'objet de la servitude avait disparu, la cour d'appel a pu en déduire que cette servitude, dont il résultait
	qu'elle était affectée, selon l'acte constitutif, à une destination déterminée, était éteinte dès lors que les choses se
	trouvaient en tel état qu'on ne pouvait plus en user conformément au titre ;
	23 février 2005, Bull III \no 42 p 36 ; Vu l'article 703 du Code civil, ensemble l'article 1184 de ce Code ; Attendu
	que les servitudes cessent lorsque les choses se trouvent en tel état qu'on ne peut plus en user ;
	Attendu, selon l'arrêt attaqué (Paris, 24 octobre 2002), que Mme X... a vendu aux époux Y... le lot \no  2 de sa
	propriété et a institué dans l'acte de vente une servitude de passage grevant leur fonds au profit du lot \no  1
	qu'elle a ultérieurement vendu aux époux Z... ;
	que les époux Y... ont assigné ceux-ci en nullité et, subsidiairement, en résiliation de la servitude ;
	Attendu que pour accueillir la demande de résiliation, l'arrêt retient que la répétition des manquements des
	époux Z... à leur obligation de n'utiliser la servitude qu'à titre de passage confère à ces manquements un
	caractère de gravité justifiant que soit prononcée la résiliation de la servitude sans qu'il soit dès lors besoin de
	se prononcer sur sa nullité ;
	Qu'en statuant ainsi, alors que le non-respect de ses conditions d'exercice ne peut entraîner l'extinction d'une
	servitude, la cour d'appel a violé les textes susvisés ;
	Cass 3 e civ 28 sept 2005 Bull III \no  186 p 170 : Vu l'article 703 du Code civil ; Attendu que les servitudes
	cessent lorsque les choses se trouvent en tel état qu'on ne peut plus en user ;
	Attendu que, pour rejeter les demandes de Mme X..., de M. Z... et de Mlle A..., l'arrêt retient que la sujétion très
	générale de passage mutuel sur le fonds respectif, qui résulte de l'acte de partage, n'a été imposée que pour
	permettre la "bonne exploitation des fermes" ;
	que n'est pas démontrée en l'espèce la nécessité pour les époux Y... d'utiliser un passage juste devant la maison
	d'habitation de Mme X..., de M. Z... et de Mlle A..., d'autant qu'il ressort d'une lettre de Mme X... au notaire que
	les déplacements du bétail et des engins agricoles de leur exploitation s'effectuent usuellement par l'autre accès
	au chemin ; que les conditions prévues à l'acte du 20 octobre 1931 n'étant pas remplies, il convient de constater
	que les époux Y... ne peuvent revendiquer l'exercice d'une servitude, selon les modalités qu'ils postulent alors
	même qu'ils disposent de plusieurs voies d'accès, y compris par le nord ; Qu'en statuant ainsi, sans constater que
	les époux Y... étaient dans l'impossibilité d'user de la servitude, la cour d'appel a violé le texte susvisé ;
	-réunion dans la même main art 705:
	Cour de cassation chambre civile 3 8 septembre 2016 \No  de pourvoi: 15-20371
	Vu les articles 637 et 705 du code civil ;
	Attendu, selon l'arrêt attaqué (Nîmes, 26 février 2015), que, par acte du 24 décembre 1932, Jules X... a divisé un
	terrain lui appartenant en deux lots, dénommés « article 1 » et « article 2 », et a institué une servitude non
	aedificandi au profit du lot article 2 sur le lot article 1 ; que la zone d'inconstructibilité du lot article 1 a été
	définie dans l'acte comme se trouvant au sud d'une ligne dont le tracé a été décrit ; qu'ultérieurement le lot
	article 1 a été divisé en cinq parcelles actuellement cadastrées 641 et 724, propriété des consorts Y..., 639,
	propriété de M. et Mme X..., 636, propriété de M. et Mme Z..., et 756, propriété de M. et Mme A... ; que les
	consorts Y..., ayant également acquis le lot article 2, devenu parcelle cadastrée 26, ont souhaité construire sur
	leurs parcelles ; que les consorts A..., Z... et X... s'y sont opposés ; que les consorts Y... les ont assignés, sur le
	fondement de l'article 705 du code civil, pour faire reconnaître l'extinction de la servitude grevant leurs
	parcelles 641 et 724 ;
	Attendu que, pour rejeter cette demande, l'arrêt retient que, pour l'application de l'article 705 précité, rien ne
	doit altérer l'unicité de propriétaire entre le fonds qui doit la servitude et celui auquel elle profite et que les
	consorts Y... ne remplissent pas la condition de réunion en une seule main du fonds qui doit la servitude avec
	celui auquel elle profite ;
	Qu'en statuant ainsi, alors que l'acquisition par le propriétaire du fonds dominant de parcelles issues de la
	division du fonds servant éteint la servitude grevant ces parcelles, la cour d'appel, qui a constaté que les
	consorts Y..., propriétaires du fonds dominant, avaient acquis la propriété des parcelles 641 et 724 issues de la
	division du fonds servant, a violé les textes susvisés
	-non usage pendant 30 ans art 706 La servitude est éteinte par le non-usage pendant trente ans. :
	Cass 3 e civ 27 fev 2002 Bull III \no 52 p 411 pour une servitude non aedificandi
	*application en cas de changement d’assiette d’une servitude conventionnelle Cass 3 e civ 19 janvier 2011 \No 
	10-10528
	: Attendu que, pour dire non éteinte la servitude conventionnelle Sud, l'arrêt retient que le bénéfice de cette
	servitude n'est pas dénié aux consorts Y... par les propriétaires des fonds servants, que cette servitude a toujours
	été pratiquée, certes avec de fait le déplacement de son assiette, ce qui exclut l'abandon du droit consenti au10
	fonds dominant, et que les consorts Y... ne peuvent se prévaloir d'une extinction par impossibilité d'en user au
	motif qu'elle serait devenue impraticable alors qu'il leur appartenait de l'entretenir ;
	Qu'en statuant ainsi, alors que le propriétaire d'un fonds bénéficiant d'une servitude conventionnelle de passage
	ne peut prescrire une assiette différente de celle convenue, la cour d'appel, qui n'a pas tiré les conséquences
	légales de ses constatations, a violé les textes susvisés ;
	*exceptions
	- pour des servitudes de « Cour commune » Cass 3e civ 15 déc 1999 Bull III \no  250 p 175 : « « Mais attendu
	qu'ayant constaté que la " servitude de cour commune " avait été créée pour satisfaire aux prescriptions des
	règlements, qu'elle ne profitait pas à un fonds dominant et qu'elle avait un caractère perpétuel, la cour d'appel,
	qui n'était pas tenue de répondre à des conclusions que ses constatations rendaient inopérantes, a retenu, à bon
	droit, sans violer les dispositions relatives aux servitudes et au statut de la copropriété, qu'instituée dans l'intérêt
	de la collectivité, l'obligation de maintenir cours et courettes libres de toute construction s'analysait comme une
	charge grevant à perpétuité le fonds de l'immeuble du ... et s'imposait au syndicat des copropriétaires ; » ;
	Cass 3e civ 3, 19 mai 2015 , \No  de pourvoi: 14-15384 : « Attendu, selon l'arrêt attaqué (Paris, 15 janvier
	2014), que le syndicat des copropriétaires du... à Paris 12e a assigné le syndicat des copropriétaires du... à
	Paris 12e en démolition de la construction qu'il a édifiée sur une cour en méconnaissance des prescriptions d'un
	acte du 8 juin 1912, intitulé « traité », par lequel les anciens propriétaires des terrains supportant les bâtiments
	à usage d'habitation qui appartiennent aujourd'hui à ces syndicats, s'étaient engagés à maintenir libres de
	constructions les cours contiguës faisant partie de leurs immeubles respectifs ; que le syndicat des
	copropriétaires du... a assigné en intervention forcée M. X..., propriétaire de la construction ;
	Attendu que le syndicat des copropriétaires du... à Paris 12e fait grief à l'arrêt de déclarer l'action prescrite,
	alors, selon le moyen : ...
	Mais attendu qu'ayant relevé, par motifs propres et adoptés, que M. X... avait acquis, le 6 décembre 1971, les
	lots construits en fond de cour qui existaient déjà lors de la mise en copropriété de l'immeuble le 24 mars 1965
	et que le syndicat du... n'avait pas exercé son action en démolition dans le délai trentenaire à compter de la
	construction, la cour d'appel en a exactement déduit, abstraction faite d'un motif surabondant tenant à la
	jouissance paisible et non équivoque de ces biens par M. X... pendant plus de trente ans, que l'action en
	démolition était prescrite ;
	- pour des servitudes de lotissement Cass 3 e civ 18 déc 2002 Bull III \no 272 p 235 : les servitudes imposées par
	l’autorité administrative lors de la division d’un fond, poursuivant un intérêt général ayant un caractère d’ordre
	public, ne sont pas éteintes par leur non usage pendant 30 ans.
	Cass 3e civ 21 janvier 2016 \No  de pourvoi: 15-10566 : Mais attendu qu'ayant exactement retenu que le cahier
	des charges, quelle que soit sa date, constitue un document contractuel dont les clauses engagent les colotis
	entre eux pour toutes les stipulations qui y sont contenues, la cour d'appel a décidé à bon droit qu'il n'y avait pas
	lieu à question préjudicielle devant la juridiction administrative et que ces dispositions continuaient à
	s'appliquer entre colotis ;
	Cass 3e civ 12 juin 2012 \No  de pourvoi: 10-31094 : Attendu, d'une part, qu'ayant relevé que le traité des 22
	octobre, 7 et 29 décembre 1927 et 21 mars 1928, publié au bureau des hypothèques le 13 mai 1928 entre la ville
	de Paris, la société civile immobilière Tolbiac (la SCI Tolbiac), M. Y..., aux droits duquel se trouvait la société
	civile immobilière La Colonie (la SCI), et M. Z..., aux droits duquel se trouvaient les consorts A..., stipulait que
	les propriétaires s'obligeaient tant en leur nom qu'aux noms de tous ayants cause futurs à ménager et maintenir
	libres de constructions trois cours contiguës entre elles tant que subsisteront les constructions et qu'une
	convention, expressément soumise à l'acceptation de M. Z..., avait été conclue, le 20 avril 1923 entre la SCI
	Tolbiac et M. Y..., prévoyant que le second serait autorisé à utiliser le mur jusqu'à hauteur de clôture pour y
	adosser des constructions telles que hangar ou atelier et constaté qu'il existait en 1948 et en 1967 sur le terrain
	de la SCI, un hangar, la cour d'appel sans être tenue de procéder à une recherche que ses constatations
	rendaient inopérante et répondant aux conclusions prétendument délaissées en les écartant, a, sans violer les
	articles L. 451-1 et R. 451-1 du code de l'urbanisme ni l'article 1134 du code civil, souverainement retenu que la
	servitude avait été consentie dans l'intérêt de la collectivité afin de permettre le respect des règles d'urbanisme
	et a pu en déduire que celle-ci ne s'était pas éteinte pour défaut d'utilité trente ans après la fin de la construction
	de l'immeuble de la SCI Tolbiac et que la convention du 20 avril 1923, dont il n'était pas établi qu'elle avait été
	acceptée par M. Z... et qui n'était pas reprise dans le traité de cour commune, ne pouvait avoir remis en cause ce
	dernier qui lui était postérieur ;
	Attendu, d'autre part, qu'ayant relevé qu'il n'était pas établi que la mairie de Paris avait délivré le permis de
	construire en toute connaissance de l'existence de la servitude de cour commune et retenu qu'il ne pouvait être
	déduit de la délivrance de ce permis que le préfet de la Seine avait renoncé à se prévaloir de la servitude, la11
	cour d'appel, qui n'était pas tenue de procéder à une recherche que ses constatations rendaient inopérante, a
	légalement justifié sa décision ;
	- Pb de servitudes de droit privé liées à des règles d’urbanisme
	Cass 3 e civ 26 mai 2004 Bull III \no 107. Dans cette affaire, deux voisins s’étaient entendus pour créer une
	servitude permettant le respect, par l’un d’entre eux, des dispositions du POS de Montpellier. Mais par la suite,
	ce POS fut annulé et le propriétaire du fonds servant de la servitude demanda l’annulation de cette dernière pour
	erreur. La Cour de Cassation refuse. Elle considère, en effet, qu’au moment de la signature de la convention, le
	POS existait bien et que, donc, les parties ne s’étaient ni l’une ni l’autre trompée sur les éléments de fait en
	cause. Malgré la rétroactivité de l’annulation du POS, cette rétroactivité n’a pas eu d’impact sur le mécanisme
	psychologique de formation du contrat.
	- Pté de servitude de surplomb :
	Cass 3 e civ 12 mars 2008 Bull III \no  47
	Attendu, selon l'arrêt attaqué (Bordeaux, 30 octobre 2006), que les époux X... ont assigné la société Clairsienne
	d'HLM afin de voir juger qu'ils ont acquis par prescription trentenaire une servitude de surplomb du fait d'une
	corniche construite sur leur immeuble et de voir ordonner la suspension des travaux envisagés par cette société
	y portant atteinte ;
	Attendu que la société Clairsienne d'HLM fait grief à l'arrêt d'accueillir cette demande, alors, selon le moyen :
	1\degre / qu'une servitude ne peut conférer le droit d'empiéter sur la propriété d'autrui ; qu'en décidant que les époux
	X... bénéficient d'une servitude de surplomb acquise par prescription sur le fonds voisin appartenant à la société
	Clairsienne du fait de la corniche intégrée à leur immeuble, tout en constatant que la corniche surplombe le
	fonds voisin, la cour a violé les articles 544 et 637 du code civil ;
	2\degre / qu'une servitude est une charge imposée sur un héritage pour l'usage ou l'utilité d'un héritage appartenant à
	un autre propriétaire ; que le simple élément décoratif d'un immeuble ne peut être un élément utile du fonds
	justifiant une servitude sur un autre fonds ; qu'en jugeant que la corniche intégrée à l'immeuble des époux X...
	justifiait la reconnaissance d'une servitude de surplomb sur le fonds de la société Clairsienne au motif que cette
	corniche "faisait partie de l'architecture même de l'immeuble", sans constater l'avantage de cet élément pour
	l'utilité et l'usage du fonds des époux X..., la cour d'appel a privé sa décision de base légale au regard de
	l'article 637 du code civil ;
	Mais attendu qu'ayant relevé que la corniche avait été édifiée il y a plus de trente ans avec l'immeuble, lequel, de
	type "chartreuse", ancien et de caractère, formait un tout sur le plan architectural dans lequel elle s'intégrait
	pour être surmontée d'une balustrade en pierre dans laquelle était intégré un fronton et souverainement retenu
	qu'elle présentait un avantage pour l'usage et l'utilité du fonds des époux X..., en ce qu'elle faisait partie de
	l'architecture même de leur immeuble, la cour d'appel, abstraction faite d'un motif surabondant relatif à
	l'agrément, en a exactement déduit que le fonds des époux X..., qui pouvaient se prévaloir d'une possession utile,
	bénéficiait d'une servitude de surplomb sur le fonds voisin acquise par prescription ;
	Mais une servitude ne peut conférer le droit d’empiéter sur la propriété d’autrui (Cass 3 e civ 1 er avril 2009 (\no 
	08-11079, Bull 2009, III \no  77) et elle doit avoir un véritable fonds dominant (Cass 3e civ 13 juin 2012 \No  de
	pourvoi: 10-21788 ).
	B L’achat de terrains ou de locaux HLM
	1) Champ d’application des dispositions sur la vente de logements
	Article L 443-14 du CCH : Toute décision d'aliénation d'un logement intermédiaire ou d'un élément du
	patrimoine immobilier d'un organisme d'habitations à loyer modéré autre que les logements est notifiée au
	représentant de l'Etat dans le département dans le délai d'un mois à compter de la date de l'acte la constatant.
	-A contrario ces éléments ne sont pas soumis à autorisation préalable.
	2). Importance de la convention d’utilité sociale
	Article L445-1
	Les organismes d'habitations à loyer modéré mentionnés aux deuxième à cinquième alinéas de l'article L. 411-2
	concluent avec l'Etat, sur la base du plan stratégique de patrimoine mentionné à l'article L. 411-9 , le cas
	échéant du cadre stratégique patrimonial et du cadre stratégique d'utilité sociale mentionnés à l'article L. 423-
	1-1, et en tenant compte des programmes locaux de l'habitat, une convention d'utilité sociale d'une durée de six
	ans , au terme de laquelle elle fait l'objet d'un renouvellement.12
	...
	La convention d'utilité sociale comporte :
	-l'état de l'occupation sociale de leurs immeubles ou ensembles immobiliers établi d'après les renseignements
	statistiques mentionnés à l'article L. 442-5 et décliné selon que ces immeubles ou ensembles immobiliers sont
	situés ou non sur le territoire d'un quartier prioritaire de la politique de la ville défini à l'article 5 de la loi \no 
	2014-173 du 21 février 2014 de programmation pour la ville et la cohésion urbaine ;
	-l'état du service rendu aux locataires dans les immeubles ou les ensembles immobiliers, après concertation avec
	les locataires dans les conditions fixées dans le plan de concertation locative prévu à l'article 44 bis de la loi \no 
	86-1290 du 23 décembre 1986 tendant à favoriser l'investissement locatif, l'accession à la propriété de
	logements sociaux et le développement de l'offre foncière ;
	-l'énoncé de la politique patrimoniale et d'investissement de l'organisme, comprenant notamment un plan de
	mise en vente des logements à usage locatif détenus par l'organisme et les orientations retenues pour le
	réinvestissement des fonds provenant de la vente. Cet énoncé comporte les mesures d'information à l'égard des
	locataires en cas de vente, cession ou fusion. Le plan de mise en vente comprend la liste des logements par
	commune et par établissement public de coopération intercommunale concernés que l'organisme prévoit
	d'aliéner pour la durée de la convention et soumis à autorisation en application de l'article L. 443-7 ainsi que
	les documents relatifs aux normes d'habitabilité et de performance énergétiques mentionnées au même article L.
	443-7. L'organisme est tenu de consulter la commune d'implantation ainsi que les collectivités et leurs
	groupements qui ont accordé un financement ou leurs garanties aux emprunts contractés pour la construction,
	l'acquisition ou l'amélioration des logements concernés. La commune émet son avis dans un délai de deux mois
	à compter du jour où le maire a reçu la consultation. Faute d'avis de la commune à l'issue de ce délai, celui-ci
	est réputé favorable. En cas d'opposition de la commune qui n'a pas atteint le taux de logements sociaux
	mentionné à l'article L. 302-5 ou en cas d'opposition de la commune à une cession de logements sociaux qui ne
	lui permettrait plus d'atteindre le taux précité, la vente n'est pas autorisée ;
	3) Acquéreurs potentiels
	a) Principes
	Article L443-11 Modifié par LOI \no 2018-1021 du 23 novembre 2018 - art. 97 (V)
	-
	Vente à des organismes HLM
	I. – L'organisme propriétaire peut vendre tout logement à un autre organisme d'habitations à loyer modéré ou à
	une société d'économie mixte agréée au titre de l'article L. 481-1 du présent code ou à un organisme bénéficiant
	de l'agrément relatif à la maîtrise d'ouvrage prévu à l'article L. 365-2 du présent code ou à un organisme de
	foncier solidaire défini à l'article L. 329-1 du code de l'urbanisme en vue de la conclusion d'un bail réel
	solidaire tel que défini aux articles L. 255-1 à L. 255-5 du présent code, sans qu'il soit fait application des
	conditions d'ancienneté, d'habitabilité et de performance énergétique prévues à l'article L. 443-7. La
	convention mentionnée à l'article L. 353-2 n'est pas résiliée de droit et les locataires en place continuent à
	bénéficier des conditions antérieures de location.
	Les aliénations aux bénéficiaires mentionnés au premier alinéa du présent I ne font pas l'objet de l'autorisation
	prévue à l'article L. 443-7 mais font l'objet d'une simple déclaration au représentant de l'Etat dans le
	département et au maire de la commune d'implantation des logements aliénés.
	Le prix de vente aux bénéficiaires mentionnés au premier alinéa du présent I est fixé librement par l'organisme.
	Toutefois, lorsqu'une aliénation à ces bénéficiaires conduit à diminuer de plus de 30 % le parc de logements
	locatifs détenu sur les trois dernières années par un organisme d'habitations à loyer modéré, elle doit faire
	l'objet d'une demande d'autorisation auprès du représentant de l'Etat dans le département. Cette demande
	d'autorisation doit mentionner la motivation du conseil d'administration ou du directoire et préciser si cette
	cession se fait dans le cadre d'un projet de dissolution de l'organisme. Dans ce dernier cas, l'autorisation
	d'aliéner est examinée au regard des conditions de mise en œuvre des dispositions relatives à la dissolution de
	l'organisme.
	En cas de non-respect de l'obligation prévue à l'avant-dernier alinéa du présent I, l'acte entraînant le transfert
	de propriété est entaché de nullité. L'action en nullité peut être intentée par l'autorité administrative ou par un
	tiers dans un délai de cinq ans à compter de la publication de l'acte au fichier immobilier.13
	- Vente d’un logement occupé
	II. – Un logement occupé ne peut être vendu qu'à son locataire, s'il occupe le logement depuis au moins deux
	ans. Toutefois, sur demande du locataire qui occupe le logement depuis au moins deux ans, le logement peut être
	vendu à son conjoint ou, s'ils ne disposent pas de ressources supérieures à celles qui sont fixées par l'autorité
	administrative, à ses ascendants et descendants qui peuvent acquérir ce logement de manière conjointe avec leur
	conjoint, partenaire ayant conclu un pacte civil de solidarité ou concubin.
	Tout locataire qui occupe le logement depuis au moins deux ans peut adresser à l'organisme propriétaire une
	demande d'acquisition de son logement. La réponse de l'organisme doit être motivée et adressée à l'intéressé
	dans les deux mois suivant la demande.
	Les logements occupés auxquels sont appliqués les plafonds de ressources des prêts locatifs sociaux peuvent
	aussi être vendus, s'ils ont été construits ou acquis par un organisme d'habitations à loyer modéré depuis plus de
	quinze ans, à des personnes morales de droit privé. Dans ce cas, les baux et la convention mentionnée à l'article
	L. 353-2 demeurent jusqu'au départ des locataires en place.
	- Vente de logements vacants
	III. – Les logements vacants des organismes d'habitations à loyer modéré peuvent être vendus, dans l'ordre
	décroissant de priorité :
	– à toute personne physique remplissant les conditions auxquelles doivent satisfaire les bénéficiaires des
	opérations d'accession à la propriété, mentionnées à l'article L. 443-1, parmi lesquels l'ensemble des locataires
	de logements appartenant aux bailleurs sociaux disposant de patrimoine dans le département, ainsi que les
	gardiens d'immeuble qu'ils emploient sont prioritaires ;
	– à une collectivité territoriale ou un groupement de collectivités territoriales.
	– à toute autre personne physique.
	Les logements vacants auxquels sont appliqués les plafonds de ressources des prêts locatifs sociaux peuvent être
	vendus s'ils ont été construits ou acquis par un organisme d'habitations à loyer modéré depuis plus de quinze
	ans, aux bénéficiaires mentionnés aux deuxième à avant-dernier alinéas du présent III auxquels s'ajoute, en
	dernier ordre de priorité, toute personne morale de droit privé.
	IV. – La mise en vente du ou des logements doit se faire par voie d'une publicité dont les modalités sont fixées
	par décret en Conseil d'Etat et à un prix fixé par l'organisme propriétaire en prenant pour base le prix d'un
	logement comparable, libre d'occupation lorsque le logement est vacant, ou occupé lorsque le logement est
	occupé.
	b) Exceptions
	- Pour les personnes physiques
	L 443-11 IV Lorsqu'une personne physique a acquis soit un logement auprès d'un organisme d'habitations à
	loyer modéré, soit un logement locatif appartenant à une société d'économie mixte ou à l'association mentionnée
	à l'article L. 313-34 et faisant l'objet d'une convention conclue en application de l'article L. 351-2, elle ne peut
	se porter acquéreur d'un autre logement appartenant à un organisme d'habitations à loyer modéré ou
	appartenant à une société d'économie mixte ou à l'association mentionnée à l'article L. 313-34 et faisant l'objet
	d'une convention conclue en application de l'article L. 351-2, sous peine d'entacher de nullité le contrat de vente
	de cet autre logement. Toutefois, cette interdiction ne s'applique pas en cas de mobilité professionnelle
	impliquant un trajet de plus de soixante-dix kilomètres entre le nouveau lieu de travail et le logement, ou si le
	logement est devenu inadapté à la taille du ménage ou en cas de séparation du ménage, sous réserve de la
	revente préalable du logement précédemment acquis.
	- En cas de renouvellement urbain ou de quartiers difficiles
	L 443-11 V. – Tous les logements, vacants ou occupés, peuvent également être vendus dans le cadre
	d'opérations de renouvellement urbain aux établissements publics créés en application du chapitre Ier du titre II
	du livre III du code de l'urbanisme et des articles L. 324-1 et L. 326-1 du même code, en vue de leur démolition
	préalablement autorisée par le représentant de l'Etat dans le département ; dans ce cas, les baux demeurent
	jusqu'au départ des locataires en place, le cas échéant.
	VI. – Afin d'assurer l'équilibre économique et social d'un ou plusieurs ensembles d'habitations ou d'un quartier
	connaissant des difficultés particulières, l'organisme d'habitations à loyer modéré propriétaire peut, après14
	accord du représentant de l'Etat dans le département, qui consulte la commune d'implantation, vendre des
	logements vacants à toute personne physique ou morale.
	-
	En cas de vente groupée de logements PLS
	L 443-11 VII. – Lorsqu'il est procédé à la vente d'un ensemble de plus de cinq logements d'un même immeuble
	ou ensemble immobilier, vacants ou occupés, auxquels sont appliqués les plafonds de ressources des prêts
	locatifs sociaux et qui ont été construits ou acquis depuis plus de quinze ans par un organisme d'habitations à
	loyer modéré, ces logements peuvent être cédés à toute personne morale de droit privé sans qu'il y ait lieu
	d'appliquer, pour les logements vacants, l'ordre de priorité mentionné au III du présent article. Pour les
	logements occupés, les baux et la convention mentionnée à l'article L. 353-2 demeurent jusqu'au départ des
	locataires en place.
	Les dispositions du IV du présent article et celles de l'article L. 443-12 ne sont pas applicables à la mise en
	vente d'un ensemble de logements en application du premier alinéa du présent VII. Le prix de vente est
	librement fixé par l'organisme propriétaire.
	4)
	Modalités de la vente
	Article L443-7 Modifié par LOI \no 2018-1021 du 23 novembre 2018 - art. 97 (V)
	Les organismes d'habitations à loyer modéré peuvent aliéner aux bénéficiaires prévus à l'article L. 443-11 des
	logements ou des ensembles de logements construits ou acquis depuis plus de dix ans par un organisme
	d'habitations à loyer modéré. Ils peuvent proposer à ces mêmes bénéficiaires la possibilité d'acquérir ces mêmes
	logements au moyen d'un contrat de location-accession. Ils peuvent proposer à ces mêmes bénéficiaires la
	possibilité d'acquérir ces mêmes logements au moyen d'un contrat de vente d'immeuble à rénover défini aux
	articles L. 262-1 à L. 262-11. Ces logements doivent répondre à des normes d'habitabilité minimale fixées par
	décret en Conseil d'Etat. Ces logements doivent, en outre, répondre à des normes de performance énergétique
	minimale fixées par décret. Ces normes d'habitabilité et de performance énergétique minimales doivent être
	remplies après réalisation des travaux, lorsque les logements sont cédés dans le cadre d'un contrat de vente
	d'immeuble à rénover.
	La décision d'aliéner est prise par l'organisme propriétaire. Elle ne peut porter sur des logements et immeubles
	insuffisamment entretenus. Elle ne doit pas avoir pour effet de réduire de manière excessive le parc de logements
	sociaux locatifs existant sur le territoire de la commune ou de l'agglomération concernée.
	La convention d'utilité sociale mentionnée à l'article L. 445-1 vaut autorisation de vendre pour les logements
	mentionnés dans le plan de mise en vente de la convention mentionnée au même article L. 445-1 pour la
	durée de la convention.
	La convention d'utilité sociale mentionnée audit article L. 445-1 conclue entre l'Etat et un organisme
	d'habitations à loyer modéré vaut autorisation de vendre pour les logements mentionnés dans le plan de mise en
	vente de cette convention au bénéfice de la société de vente d'habitations à loyer modéré qui les a acquis auprès
	de l'organisme ayant conclu la convention. L'autorisation de vendre au bénéfice de la société de vente est
	valable pendant la durée de la convention précitée.
	Si l'organisme propriétaire souhaite aliéner des logements qui ne sont pas mentionnés dans le plan de mise en
	vente de la convention mentionnée au même article L. 445-1, il adresse au représentant de l'Etat dans le
	département une demande d'autorisation. Le représentant de l'Etat dans le département consulte la commune
	d'implantation ainsi que les collectivités publiques qui ont accordé un financement ou leur garantie aux
	emprunts contractés pour la construction, l'acquisition ou l'amélioration des logements concernés. La commune
	émet son avis dans un délai de deux mois à compter du jour où le maire a reçu la consultation du représentant
	de l'Etat dans le département. Faute d'avis de la commune à l'issue de ce délai, l'avis est réputé favorable. En
	cas d'opposition de la commune qui n'a pas atteint le taux de logements sociaux mentionné à l'article L. 302-5
	ou en cas d'opposition de la commune à une cession de logements sociaux qui ne lui permettrait plus d'atteindre
	le taux précité, la vente n'est pas autorisée. A défaut d'opposition motivée du représentant de l'Etat dans le
	département dans un délai de quatre mois, la vente est autorisée. L'autorisation est rendue caduque par la
	signature d'une nouvelle convention mentionnée au même article L. 445-1.
	L'autorisation mentionnée au cinquième alinéa du présent article vaut autorisation de vendre au bénéfice de la
	société de vente d'habitations à loyer modéré qui a acquis les logements concernés auprès de l'organisme ayant
	reçu l'autorisation. L'autorisation de vendre au bénéfice de la société de vente est rendue caduque par la
	signature par l'organisme précité d'une nouvelle convention mentionnée au même article L. 445-1.
	Lorsque la société de vente d'habitations à loyer modéré détient des logements pour lesquels l'autorisation de
	vente est devenue caduque, elle adresse au représentant de l'Etat dans le département une demande
	d'autorisation de vendre. Le représentant de l'Etat dans le département consulte la commune d'implantation15
	ainsi que les collectivités publiques qui ont accordé un financement ou leur garantie aux emprunts contractés
	pour la construction, l'acquisition ou l'amélioration des logements concernés. La commune émet son avis dans
	un délai de deux mois à compter du jour où le maire a reçu la consultation du représentant de l'Etat dans le
	département. Faute d'avis de la commune à l'issue de ce délai, l'avis est réputé favorable. En cas d'opposition de
	la commune qui n'a pas atteint le taux de logements sociaux mentionné à l'article L. 302-5 ou en cas
	d'opposition de la commune à une cession de logements sociaux qui ne lui permettrait plus d'atteindre le taux
	précité, la vente n'est pas autorisée. A défaut d'opposition motivée du représentant de l'Etat dans le département
	dans un délai de quatre mois, la vente est autorisée. L'autorisation est caduque à l'issue d'un délai de six ans.
	Lorsque la société n'a pas obtenu d'autorisation de vendre, les logements sont cédés à un organisme mentionné
	à l'article L. 411-2 ou à une société d'économie mixte agréée en application de l'article L. 481-1 dans un délai
	de six mois à compter du refus de vendre.
	...
	En cas de non-respect de l'obligation d'autorisation de l'aliénation par le représentant de l'Etat dans le
	département ou par le président du conseil de la métropole, l'organisme vendeur est passible d'une sanction
	pécuniaire, dans la limite de 40 % du montant de la vente, hors frais d'acte, arrêtée par l'Agence nationale de
	contrôle du logement social ou le président du conseil de la métropole dans la situation prévue au cinquième
	alinéa du présent article.
	Article L443-8 Modifié par LOI \no 2018-1021 du 23 novembre 2018 - art. 97 (V)
	Lorsque des circonstances économiques ou sociales particulières le justifient, la vente de logements locatifs ne
	répondant pas aux conditions d'ancienneté définies à l'article L. 443-7 peut être autorisée par décision motivée
	du représentant de l'Etat dans le département d'implantation du logement ou du président du conseil de la
	métropole dans la situation prévue au douzième alinéa de l'article L. 443-7, après consultation de la commune
	d'implantation. La décision fixe les conditions de remboursement de tout ou partie des aides accordées par l'Etat
	pour la construction, l'acquisition ou l'amélioration de ce logement.
	...
	Article L443-12 Modifié par LOI \no 2018-1021 du 23 novembre 2018 - art. 97 (V)
	Lorsque le logement est vendu à des bénéficiaires prévus au III de l'article L. 443-11, l'organisme vend, par
	ordre de priorité défini au même article L. 443-11, à l'acheteur qui le premier formule l'offre qui correspond à
	ou qui est supérieure au prix évalué en application dudit article L. 443-11 ou, si l'offre est inférieure au prix
	évalué, qui en est la plus proche, dans des conditions définies par décret.
	5)
	Information de l’acquéreur
	Article L443-14-2 Créé par LOI \no 2018-1021 du 23 novembre 2018 - art. 97 (V)
	I.- L'organisme d'habitations à loyer modéré indique par écrit à l'acquéreur, préalablement à la vente, le
	montant des charges locatives et, le cas échéant, de copropriété des deux dernières années, et lui transmet la
	liste des travaux réalisés les cinq dernières années sur les parties communes. En tant que de besoin, il fournit
	une liste des travaux d'amélioration des parties communes et des éléments d'équipement commun qu'il serait
	souhaitable d'entreprendre, accompagnée d'une évaluation du montant global de ces travaux et de la quote-part
	imputable à l'acquéreur.
	II.- Dans les copropriétés comportant des logements vendus en application de la présente section, la liste de
	travaux mentionnée au I accompagnée de l'évaluation de leur montant global font l'objet d'une présentation
	annuelle par le syndic devant l'assemblée générale des copropriétaires.
	Lorsqu'ils sont votés par l'assemblée générale des copropriétaires, les travaux d'amélioration des parties
	communes et des éléments d'équipement commun donnent lieu à la constitution d'avances, selon des modalités
	définies par l'assemblée générale. L'organisme d'habitations à loyer modéré est dispensé de cette obligation.
	Ces avances sont déposées sur un compte bancaire ou postal séparé ouvert au nom du syndicat des
	copropriétaires avec une rubrique particulière pour chaque copropriétaire. Le compte et les rubriques ne16
	peuvent faire l'objet d'aucune convention de fusion, de compensation ou d'unité de compte.
	L'organisme d'habitations à loyer modéré est dispensé de l'obligation de versement prévue au II de l'article 14-2
	de la loi \no  65-557 du 10 juillet 1965 fixant le statut de la copropriété des immeubles bâtis et constitue dans ses
	comptes une provision correspondant à celui-ci et souscrit une caution bancaire au profit du syndicat de
	copropriétaires. Il verse sa contribution, sur appel de fond, à la réalisation du diagnostic et des travaux prévus
	aux articles L. 731-1 et L. 731-2 du présent code.
	6) Choix du syndic
	Article L443-15 Modifié par LOI \no 2018-1021 du 23 novembre 2018 - art. 97 (V)
	En cas de vente réalisée en application de la présente section, les fonctions de syndic de la copropriété sont
	assurées, sauf s'il y renonce, par l'organisme vendeur tant qu'il demeure propriétaire d'au moins un logement.
	Toutefois, l'assemblée générale des copropriétaires peut désigner un autre syndic dès lors que les
	copropriétaires autres que l'organisme vendeur détiennent au moins 60 % des voix du syndicat.
	Les fonctions de syndic de la copropriété comportant des immeubles vendus en application de la présente
	section peuvent être assurées par l'organisme vendeur conformément aux dispositions de la loi \no  65-557 du 10
	juillet 1965 fixant le statut de la copropriété des immeubles bâtis lorsque l'organisme n'est plus propriétaire
	d'aucun logement.
	Les dispositions du deuxième alinéa du I de l'article 22 de la loi \no  65-557 du 10 juillet 1965 précitée ne
	s'appliquent pas à l'organisme d'habitations à loyer modéré vendeur.
	Dans les copropriétés issues de la vente de logements locatifs réalisée en application de la présente section dans
	lesquelles un organisme d'habitations à loyer modéré détient au moins un logement, celui-ci peut, en tant que de
	besoin, mettre son personnel à disposition du syndicat des copropriétaires afin d'assurer des missions de
	gardiennage, d'agent de propreté, d'élimination des déchets, d'entretien technique courant et de veille de bon
	fonctionnement des équipements communs. Cette prestation de mise à disposition de personnel bénéficie de
	l'exonération de taxe sur la valeur ajoutée prévue à l'article 261 B du code général des impôts lorsque les
	conditions prévues par cet article sont remplies.
	En cas de cession par une société de vente d'habitations à loyer modéré d'un logement qu'elle a acquis en
	application de l'article L. 422-4 du présent code, l'organisme ou la société d'économie mixte agréée qui en était
	antérieurement propriétaire assure, en lieu et place de la société de vente, les fonctions de syndic et, le cas
	échéant, la mise à disposition de personnel en application du présent article, sauf renoncement ou convention
	contraire.
	7) Modalités particulières de revente d’un logement HLM
	Article L443-12-1 Modifié par LOI \no 2018-1021 du 23 novembre 2018 - art. 97 (V)
	L'acquéreur personne physique qui souhaite revendre son logement dans les cinq ans qui suivent l'acquisition
	est tenu d'en informer l'organisme d'habitations à loyer modéré, qui peut se porter acquéreur en priorité.
	L'acquéreur personne physique ayant acquis son logement à un prix inférieur au prix de mise en vente fixé en
	application de l'article L. 443-11 et l'ayant vendu dans les cinq ans suivant cette acquisition est tenu de verser à
	l'organisme d'habitations à loyer modéré une somme égale à la différence entre le prix de vente et le prix
	d'acquisition. Cette somme ne peut excéder l'écart constaté entre le prix de mise en vente lors de l'acquisition et
	le prix d'acquisition.
	Ces prix s'entendent hors frais d'acte et accessoires à la vente.
	Lorsque l'acquéreur personne physique a acquis son logement à un prix inférieur au prix de mise en vente fixé
	en application de l'article L. 443-11 et qu'il le loue dans les cinq ans qui suivent l'acquisition, le niveau de loyer
	ne doit pas excéder des plafonds fixés par l'autorité administrative.
	A peine de nullité, le contrat de vente entre l'acquéreur et l'organisme d'habitations à loyer modéré comporte la
	mention de ces obligations.17
	Article L443-14-1 Modifié par LOI \no 2018-1021 du 23 novembre 2018 - art. 97 (V)
	I. – Il est institué une taxe sur les plus-values réalisées à l'occasion des cessions de logements situés en France
	métropolitaine opérées au cours du dernier exercice clos par les organismes d'habitations à loyer modéré et par
	les sociétés d'économie mixte agréées en application de l'article L. 481-1.
	Cette taxe est assise sur la somme des plus-values réalisées lors des cessions de logements situés en France
	métropolitaine intervenant dans le cadre de la présente section, à l'exception des cessions intervenant dans le
	cadre du I et du troisième alinéa du III de l'article L. 443-11. Le produit de cette taxe est versé à la Caisse de
	garantie du logement locatif social. Les articles L. 452-5 et L. 452-6 sont applicables à cette taxe.
	II. – 1. La plus-value résulte de la différence entre le prix de cession et le prix d'acquisition du logement par le
	cédant, actualisé pour tenir compte de l'effet de l'érosion de la valeur de la monnaie pendant la durée de
	détention du bien....
	C L’achat avec paiement par des travaux
	-la cession de millièmes indivis
	-la cession du terrain. CE 9 avril 1999 OTTOU AJDA 1999 1149 ; Considérant que, par acte notarié du 16
	juillet 1979, Mme Ottou a cédé ses droits sur un terrain situé à Saint-Raphaël, moyennant un prix stipulé à
	l'acte, en ce qui concerne la part revenant à Mme Ottou, de 1 064 754 F converti en l'obligation pour
	l'acquéreur de lui livrer sept appartements situés dans l'immeuble à construire sur ce terrain ; que
	l'administration a retenu comme prix de cession du terrain pour la détermination de la plus-value
	imposable, non la somme mentionnée dans l'acte, mais une somme de 2 292 000 F correspondant, selon une
	méthode qui n'a pas été contestée, à l'estimation de la valeur des biens remis en dation ; que la Cour, en se
	fondant sur ces circonstances, a pu, sans commettre d'erreur de droit ni d'erreur de qualification juridique
	des faits, déduire de l'importance de l'écart existant entre le prix stipulé à l'acte et la valeur non contestée
	des droits représentatifs des locaux à construire que l'administration devait être regardée comme
	établissant la dissimulation d'une partie du prix réellement convenu et qu'elle était fondée à retenir comme
	prix de cession, pour le calcul de la plus-value effectivement réalisée par les cédants, la valeur des droits
	représentatifs des locaux à construire plutôt que le prix stipulé à l'acte ;
	- l’association d’un voisin à une opération de lotissement avec appréciation globale de constructibilité sur
	les deux unités foncières, autorisation de lotir demandée par deux personnes et remise in fine de terrains
	équipés au voisin.
	- Risques de l’opération : Cass 3 e civ 3 22 septembre 2010 \No  de pourvoi: 09-15781
	Mais attendu qu'ayant relevé que l'acte notarié prévoyait le paiement du prix de vente au profit de Mme Y...
	pour partie dans un délai de vingt-sept mois et pour partie par novation de l'obligation de payer en
	obligation de faire construire sur l'une des parcelles vendues, donner en paiement et remettre à Mme Y...
	une maison de même valeur et que la construction de la maison n'avait pas été terminée et que s'agissant
	d'une dation en paiement d'une chose à construire et donc future, le transfert de propriété au profit de la
	bénéficiaire de la dation ne pouvait s'opérer que lorsque la chose était effectivement en mesure d'être livrée
	par celui qui devait la donner, la cour d'appel en a exactement déduit que Mme Y... n'était pas fondée à
	demander le transfert de propriété de la construction ni celui de la parcelle non objet de la dation convenue
	entre les parties ; D'où il suit que le moyen n'est pas fondé ; PAR CES MOTIFS : REJETTE le pourvoi ;
	D La perte de la qualité du terrain acheté
	-problématique : erreur, vice caché lésion? ( Cf F Rouvière RDI 2010 253).
	1) vice caché:
	Fondement principal : Cass 3 e civ 1 er oct 1997 Bull III \no  181 ; 15 mars 2000 Bull III \no 61 ; 17 nov 2004
	Bull III \no  206 p 185 : Mais attendu, d'une part, qu'ayant énoncé à bon droit que les vices cachés se
	définissent comme des défauts rendant la chose impropre à sa destination, et constaté que l'action des époux
	X... était exclusivement fondée sur la présence d'anciennes carrières de gypse qui entraîneraient des
	mouvements du sol et des désordres immobiliers, la cour d'appel, qui n'a pas dénaturé les conclusions des
	parties, a exactement retenu que la garantie des vices cachés constituant l'unique fondement possible de
	l'action exercée, il n'y avait pas lieu de rechercher si le consentement des époux X... avait été donné par
	erreur ;.
	- Mais vice caché susceptible d’être mis à l’écart entre professionnels :
	cf Cass 3 e civ 26 avril 2006 \no 04-19107 : Attendu qu'ayant relevé, par motifs propres et adoptés, que le vice
	caché résidait dans l'insuffisance, au regard des constructions envisagées, du système d'évacuation des eaux18
	pluviales mis en place par le lotisseur , la cour d'appel, qui n'était pas tenue de suivre les parties dans le
	détail de leur argumentation et qui a souverainement retenu que la vente de terrains viabilisés, conclue
	entre un lotisseur et un promoteur pour son activité professionnelle consistant à acquérir des terrains en
	vue d'y construire des pavillons pour les revendre, était intervenue entre deux professionnels de
	l'immobilier, en a exactement déduit que la clause de non-garantie des vices cachés était valable ;
	cf Cour de cassation chambre civile 3 jeudi 30 juin 2016 \No  de pourvoi: 14-28839
	Attendu, selon l'arrêt attaqué (Versailles, 9 octobre 2014), que, par acte authentique du 23 mai 2006, la
	société Sevilo a vendu un immeuble à la société Sogefimur, crédit-bailleur, la société Doun en étant le
	crédit-preneur ; que la société Sevilo a fait réaliser des travaux de désamiantage entre la promesse et l'acte
	authentique de vente ; que, dans un rapport du 21 mars 2006, M. Y..., diagnostiqueur exerçant sous
	l'enseigne LM conseil a conclu à l'absence d'amiante ; que, sous la maîtrise d'ouvrage de la société Doun et
	de sa sous-locataire, la société Francare, des travaux de rénovation des locaux ont été entrepris ; que, la
	présence d'amiante ayant été détectée à la fin de l'année 2006, la société Doun a, après expertise, assigné la
	société Sevilo et M. Y..., ainsi que son assureur, la société CoveaRisks, en dommages-intérêts sur le
	fondement de la garantie des vices cachés ;
	...
	Mais attendu, d'une part, que la cour d'appel a relevé qu'il ne résultait pas des dispositions contractuelles
	que la société Sevilo avait pris l'engagement de livrer à l'acquéreur un immeuble exempt d'amiante ;
	Attendu, d'autre part, qu'ayant constaté que la vente était intervenue entre deux professionnels de même
	spécialité et que la société Doun ne rapportait pas la preuve que le vendeur avait connaissance de la
	présence d'amiante dans les locaux vendus, la cour d'appel, qui n'était pas tenue de suivre les parties dans
	le détail de leur argumentation, a pu en déduire que la clause de non-garantie des vices cachés stipulée à
	l'acte de vente devait recevoir application et que les demandes de la société Doun devaient être rejetées ;
	-Nécessité d’un vice antérieur à la vente Cass 3e civ 13 novembre 2014 \No  de pourvoi: 13-24027: Mais
	attendu qu'ayant constaté qu'au jour de la vente, le terrain était partiellement constructible et que la totalité
	de la parcelle n'avait été classée en zone inconstructible inondable que par arrêté préfectoral du 20 avril
	2006, la cour d'appel a pu en déduire que les acquéreurs ne rapportaient pas la preuve qui leur incombe
	d'un vice d'inconstructibilité antérieur à la vente ;
	Extension de la notion de professionnel :
	- Cass 3e civ 10 juillet 2013 \No  de pourvoi: 12-17149
	Attendu que pour débouter les consorts Y...-Z... et la MACIF de leurs demandes, l'arrêt retient que M. X...
	ne possédant aucune compétence particulière en matière de construction de cheminée à foyer ouvert ou
	fermé, il ne pouvait être considéré comme un professionnel présumé connaître les vices de construction
	affectant la cheminée ;
	Qu'en statuant ainsi, alors qu'elle avait relevé que M. X... avait lui-même conçu et installé la cheminée en
	foyer ouvert, puis en foyer fermé lors de nouveaux travaux, la cour d'appel a violé le texte susvisé ;
	- Cass 3e civ 27 octobre 2016 \No  de pourvoi: 15-24232
	Attendu, selon l'arrêt attaqué (Chambéry, 23 juin 2015), que la société civile immobilière Moxilotte (la SCI)
	a acquis un immeuble qu'elle a fait rénover et a vendu un appartement sur deux niveaux, dont un niveau de
	sous-sol, à Mme X... ; que, se plaignant d'une importante humidité en sous-sol, celle-ci a, après expertise,
	assigné la SCI en résolution de la vente, sur le fondement de la garantie des vices cachés, et en paiement de
	dommages-intérêts ;
	Mais attendu qu'ayant relevé, par motifs propres et adoptés, que la SCI, qui, aux termes de ses statuts, avait
	pour objet " l'acquisition par voie d'achat ou d'apport, la propriété, la mise en valeur, la transformation,
	l'aménagement, l'administration et la location de tous biens et droits immobiliers... ", avait acquis une
	vieille ferme qu'elle avait fait transformer en logements d'habitation dont elle avait vendu une partie et loué
	le reste et qu'elle avait immédiatement réinvesti les profits retirés dans une autre opération immobilière, la
	cour d'appel, qui n'était pas tenue de procéder à des recherches que ses constatations rendaient
	inopérantes, a pu en déduire, abstraction faite de motifs surabondants, que la SCI avait la qualité de
	vendeur professionnel et a légalement justifié sa décision ;
	2) erreur
	-s’apprécie traditionnellement au moment du consentement : cf Cass 3 e civ 23 mai 2007 \no 06-11889
	Vu l'article 1110 du code civil ;
	Attendu que pour accueillir la demande en annulation de la vente, l'arrêt, qui relève que le retrait fait
	disparaître rétroactivement la décision qui en fait l'objet laquelle, de ce fait, est réputée n'avoir jamais
	existé, retient qu'il est établi que le caractère constructible du terrain en cause était un motif déterminant du19
	consentement donné par les époux Z..., dans la mesure où ceux-ci avaient fait insérer dans l'acte sous seing
	privé en date du 9 novembre 1998 une condition suspensive relative à l'obtention d'un permis de construire,
	et que l'arrêté municipal du 4 février 1999 rapportant le permis de construire précédemment accordé et
	refusant tout permis consacre le caractère inconstructible du terrain en cause ;
	Qu'en statuant ainsi, alors que la rétroactivité est sans incidence sur l'erreur, qui s'apprécie au moment
	de la conclusion du contrat, la cour d'appel a violé le texte susvisé ;
	- Cf toutefois Cass 3e civ 28 janvier 2009 \No  de pourvoi: 07-20729 : Mais attendu qu'ayant relevé que pour
	consentir à la vente, les acquéreurs n'avaient pas pu prendre en compte le risque réel encouru du fait de
	l'arrêté du 30 avril 2004, révélé par la requête en annulation du permis de construire de la société Chevalier
	Maurice, souverainement retenu que ces caractéristiques du bien relatives à la constructibilité et à son
	environnement étaient déterminantes de la décision d'achat et que le consentement des acquéreurs avait en
	conséquence été vicié pour erreur sur les qualités substantielles du terrain à construire, la cour d'appel, sans
	violer le principe de la contradiction et sans être tenue de procéder à une recherche que
	ses
	constatations rendaient inopérante, a, par ces seuls motifs, légalement justifié sa décision de ce chef
	Cass 3 e civ 9 juin 2010 \No  de pourvoi: 08-13969
	Attendu, selon l'arrêt attaqué (Chambéry, 5 février 2008), que les consorts X... ont cédé par acte
	authentique du 20 mai 2003 à M. Y... leurs droits indivis sur un terrain situé à Val-d'Isère ; que l'acte
	mentionnait l'existence de divers recours et d'un pourvoi contre un permis de construire et son modificatif
	délivrés en 1995 "en raison du caractère avalancheux de la route permettant l'accès audit immeuble" ; que
	M. Y... a déclaré dans l'acte faire son affaire personnelle de ces procédures et s'obliger à en supporter toutes
	les conséquences quelles qu'elles soient, sans recours contre les vendeurs ; qu'après l'annulation le 9 juillet
	2003 par le Conseil d'Etat du permis de construire et de son modificatif comme entachés d'une erreur
	manifeste d'appréciation au regard des risques d'avalanche sur le terrain, M. Y... a assigné les consorts X...,
	principalement en résolution de la vente pour vice caché tenant au caractère inconstructible du terrain, et
	subsidiairement en nullité de la vente pour erreur sur la qualité substantielle de la chose et absence de cause
	;
	Attendu que, pour prononcer la nullité de la vente pour erreur sur les qualités substantielles de la chose,
	l'arrêt retient que n'ayant pas connu et accepté le risque que le terrain soit jugé totalement inconstructible,
	M. Y... est bien fondé à invoquer son erreur sur une qualité substantielle de la chose vendue ;
	Qu'en statuant ainsi, sans rechercher, comme il le lui était demandé, si en qualité d'architecte et de
	promoteur immobilier expérimenté dans la région de Val d'Isère, M. Y..., qui avait déclaré dans l'acte
	d'acquisition "connaître parfaitement le bien vendu", "avoir pris par lui-même tous renseignements relatifs
	aux règles d'urbanisme", faire son affaire personnelle de ces règles "la vente ayant lieu à ses risques et
	périls", ne s'était pas engagé en connaissance de cause, la cour d'appel n'a pas donné de base légale à sa
	décision ;
	- Evolution avec Cour de cassation chambre civile 3 12 juin 2014 \No  de pourvoi: 13-18446: Attendu,
	selon l'arrêt attaqué (Rouen, 23 janvier 2013), que le 27 novembre 2008, les époux X... ont vendu à M. Y... et
	Mme Z... (les consorts Y...) un terrain destiné à la construction d'une maison d'habitation ; que le permis de
	construire délivré aux acquéreurs le 13 octobre 2008 a été retiré le 5 janvier 2009 en raison de la suspicion
	de la présence d'une cavité souterraine ; que les consorts Y... ont assigné le notaire et les époux X... en
	annulation de la vente et en réparation du préjudice subi ;
	Attendu que M. et Mme X... font grief à l'arrêt d'accueillir les demandes des consorts Y..., alors, selon le
	moyen :...
	Mais attendu qu'ayant relevé que la constructibilité immédiate du terrain était un élément déterminant du
	consentement des acquéreurs et constaté que le risque lié à la présence d'une cavité souterraine existait à la
	date de la vente, la cour d'appel a pu en déduire que la décision de retrait du permis n'avait fait que prendre
	en compte la réalité de ce risque empêchant les acquéreurs de construire et que la vente était nulle ;
	- Mais critères restrictifs toujours présents : Cass 3 e civ 13 novembre 2014 \No  de pourvoi: 13-24027 :
	Mais attendu qu'ayant relevé que M. et Mme X... ne pouvaient ignorer l'enquête publique ordonnée dans le
	cadre de la révision du plan de prévention des risques naturels d'inondation et avaient accepté d'acquérir en
	toute connaissance de cause un terrain partiellement inondable, donc partiellement inconstructible et
	exactement retenu qu'ils ne pouvaient invoquer une décision administrative postérieure à la vente classant le
	terrain intégralement en zone inconstructible pour justifier leur demande d'annulation du contrat pour
	erreur sur la substance, l'extension de l'inconstructibilité à toute la surface du terrain et le refus de
	délivrance du permis de construire n'étant pas inéluctables au jour de la vente, la cour d'appel a, par ces
	seuls motifs, légalement justifié sa décision ;20
	Cour de cassation chambre civile 3 24 novembre 2016 \No  de pourvoi: 15-26226
	Attendu, selon l'arrêt attaqué (Grenoble, 1er septembre 2015), que, par acte notarié du 22 août 2006, dressé
	par MM. X...et Y..., Charles Z...et Mme Huguette A..., son épouse, ont vendu à M. B...et Mme C..., différentes
	parcelles de terrain pour lesquelles ceux-ci ont obtenu, le 26 décembre 2007, un permis de construire, qui, à
	la suite d'un recours gracieux du préfet, a été retiré par arrêté municipal du 7 juillet 2008, pour des motifs
	de sécurité, le lotissement se trouvant dans un secteur soumis à des risques naturels ; que M. B...et Mme C...,
	invoquant l'inconstructibilité du terrain, ont assigné les vendeurs, M. X..., M. Y..., la société civile
	professionnelle Mallet et Benoît et la société civile professionnelle Y... D..., en nullité du contrat de vente et
	en indemnisation de leur préjudice ; que, Charles Z...étant décédé, l'instance a été reprise par Mmes
	Maryvonne et Huguette Z...(les consorts Z...) ;
	Mais attendu qu'ayant relevé, par motifs propres et adoptés, qu'à l'acte notarié de vente, figurait un état des
	risques mentionnant que les parcelles étaient en zone inondable et étaient couvertes par un plan de
	prévention des risques et qu'au jour de la vente, le terrain litigieux était constructible, la cour d'appel, qui a
	exactement retenu que l'annulation rétroactive du permis de construire obtenu après la vente était sans
	incidence sur l'erreur devant s'apprécier au moment de la formation du contrat, a pu en déduire que le
	retrait du permis de construire ne pouvait entraîner la nullité de la vente, ni donner lieu à la garantie des
	vices cachés ;
	3) Lésion également exclue
	Cf Cass 3e civ 17 juin 2009 \no  08-15055:
	Mais attendu qu'ayant retenu que si l'annulation de la révision du plan d'occupation des sols approuvé
	le 16 mars 1999 avait pour effet de remettre en vigueur le plan d'occupation des sols immédiatement
	antérieur qui prévoyait que les terrains devaient avoir une superficie minimale de quatre mille mètres
	carrés pour être constructibles, elle ne pouvait avoir pour effet, dès lors qu'elle n'était pas notoirement
	inéluctable, de faire disparaître la dévalorisation que, de fait, à la date du 20 janvier 2000, cette révision
	faisait subir à la parcelle litigieuse dont la superficie n'était que de quatre mille quatre cent cinquante et un
	mètres carrés en prévoyant que dans le secteur où elle se trouvait, les terrains devaient avoir une superficie
	minimale de dix mille mètres carrés pour être constructibles, la cour d'appel en a déduit à bon droit, la
	rétroactivité étant sans incidence sur la lésion qui s'apprécie au moment de la conclusion du contrat, que la
	commune de Mimet devait être déboutée de sa demande en rescision de la vente ;
	4) Garantie d’éviction possible : Cass 3 e civ 7 juillet 2010 \No  de pourvoi: 09-12055
	Attendu, selon l'arrêt attaqué (Douai, 12 janvier 2009), que les époux X... ont vendu aux époux Y... un
	bien immobilier situé dans un lotissement ; que par jugement du 20 décembre 2001 confirmé par un arrêt du
	28 février 2005, ils ont été évincés d'une partie de leur bien constituée d'un espace vert qui a été reconnue
	partie commune du lotissement ; qu'ils ont assigné leurs vendeurs en indemnisation de la perte du terrain et
	paiement de dommages-intérêts ;
	Sur le moyen unique du pourvoi principal :
	Attendu que les époux X... font grief à l'arrêt d'accueillir la demande des époux Y..., alors, selon le moyen,
	que la cour d'appel a dénaturé par omission l'acte de vente du 1er février 1996 qui stipulait que l'acquéreur
	ne pourrait demander aucune indemnité, ni diminution de prix, non seulement " pour moindre mesure qui
	pourrait exister entre la contenance réelle et celle sus-indiquée " mais encore " pour quelque autre cause
	que ce soit " en violation de l'article 1134 du code civil ;
	Mais attendu qu'ayant relevé que l'acte notarié disposait que l'acquéreur s'obligeait à " prendre le bien
	vendu dans l'état où il se trouve actuellement, sans pouvoir demander aucune indemnité, ni diminution du
	prix ci-dessus fixé pour mitoyenneté, défaut d'alignement, vices de construction apparents ou cachés, vétusté
	des bâtiments, champignon, mauvais état du sol, ou du sous-sol, ou quelque autre cause que ce soit, ni pour
	moindre mesure qui pourrait exister entre la contenance réelle et celle sus-indiquée, cette différence
	excédât-elle un vingtième ", c'est par une interprétation souveraine, exclusive de dénaturation, que
	l'ambiguïté des termes de la clause rendait nécessaire, que la cour d'appel a retenu que cette clause de non-
	garantie de désignation et de contenance ne dispensait pas les vendeurs de garantir les acheteurs contre
	l'éviction de la chose vendue ;
	E- Le terrain occupé
	1) La nécessité d’un permis de démolir
	-Champ d’application21
	Article R421-27 : Doivent être précédés d'un permis de démolir les travaux ayant pour objet de démolir ou de
	rendre inutilisable tout ou partie d'une construction située dans une commune ou une partie de commune où le
	conseil municipal a décidé d'instituer le permis de démolir.
	Article R421-28 : Doivent en outre être précédés d'un permis de démolir les travaux ayant pour objet de
	démolir ou de rendre inutilisable tout ou partie d'une construction :
	a) Située dans le périmètre d'un site patrimonial remarquable classé en application de l'article L. 631-1 du code
	du patrimoine ;
	b) Située dans les abords des monuments historiques définis à l'article L. 621-30 du code du patrimoine ou
	inscrite au titre des monuments historiques ;
	c) Située dans le périmètre d'une opération de restauration immobilière définie à l'article L. 313-4 ;
	d) Située dans un site inscrit ou un site classé ou en instance de classement en application des articles L. 341-1
	et L. 341-2 du code de l'environnement ;
	e) Identifiée comme devant être protégée en étant située à l'intérieur d'un périmètre délimité par un plan local
	d'urbanisme ou un document d'urbanisme en tenant lieu en application de l'article L. 151-19 ou de l'article L.
	151-23, ou, lorsqu'elle est située sur un territoire non couvert par un plan local d'urbanisme ou un document
	d'urbanisme en tenant lieu, identifiée comme présentant un intérêt patrimonial, paysager ou écologique, en
	application de l'article L. 111-22, par une délibération du conseil municipal prise après l'accomplissement de
	l'enquête publique prévue à ce même article.
	.- Exception : Article R 421-29 Sont dispensées de permis de démolir :
	a) Les démolitions couvertes par le secret de la défense nationale ;
	b) Les démolitions effectuées en application du code de la construction et de l'habitation sur un bâtiment
	menaçant ruine ou en application du code de la santé publique sur un immeuble insalubre ;
	c) Les démolitions effectuées en application d'une décision de justice devenue définitive ;
	d) Les démolitions de bâtiments frappés de servitude de reculement en exécution de plans d'alignement approuvés
	en application du chapitre Ier du titre IV du livre Ier du code de la voirie routière ;
	e)Les démolitions de lignes électriques et de canalisations
	- Modalités d’obtention : « Art. L. 451-1. - Lorsque la démolition est nécessaire à une opération de
	construction ou d'aménagement, la demande de permis de construire ou d'aménager peut porter à la fois sur la
	démolition et sur la construction ou l'aménagement. Dans ce cas, le permis de construire ou le permis
	d'aménager autorise la démolition.
	Sur le rôle de l’ABF : Cf CE 16 mars 2015 \No  380498 : 4. Considérant qu'il suit de là qu'en jugeant que l'avis
	favorable rendu par l'architecte des Bâtiments de France le 11 janvier 2011 sur le projet litigieux ne pouvait
	être regardé comme portant sur l'ensemble de l'opération au seul motif que cet avis ne visait que la demande de
	permis de construire et non la démolition préalable à la construction, alors qu'elle avait relevé qu'une demande
	de permis de construire valant également permis de démolir avait été sollicitée, la cour administrative d'appel
	de Paris a commis une erreur de droit ;
	CE 19 juin 2015 \No  387061 (la Samaritaine): « 13. Considérant qu'aux termes de l'article L. 451-1 du code de
	l'urbanisme : " Lorsque la démolition est nécessaire à une opération de construction ou d'aménagement, la
	demande de permis de construire ou d'aménager peut porter à la fois sur la démolition et sur la construction ou
	l'aménagement. Dans ce cas, le permis de construire ou le permis d'aménager autorise la démolition " ; qu'aux
	termes de l'article R. 425-1 du même code : " Lorsque le projet est situé dans le champ de visibilité d'un édifice
	classé ou inscrit au titre des monuments historiques (...), le permis de construire (...), le permis de démolir ou la
	décision prise sur la déclaration préalable tient lieu de l'autorisation prévue à l'article L. 621-31 du code du
	patrimoine dès lors que la décision a fait l'objet de l'accord de l'architecte des Bâtiments de France " ; qu'aux
	termes de l'article R. 425-30 du même code : " Lorsque le projet est situé dans un site inscrit, la demande de
	permis ou la déclaration préalable tient lieu de la déclaration exigée par l'article L. 341-1 du code de
	l'environnement. (...) / La décision prise sur la demande de permis ou sur la déclaration préalable intervient
	après consultation de l'architecte des Bâtiments de France " ; qu'aux termes de l'article R. 425-18 du même code
	: " Lorsque le projet porte sur la démolition d'un bâtiment situé dans un site inscrit en application de l'article L.
	341-1 du code de l'environnement, le permis de démolir ne peut intervenir qu'avec l'accord exprès de l'architecte
	des Bâtiments de France " ;
	14. Considérant qu'il résulte de ces dispositions que, lorsque la démolition d'un bâtiment situé dans un site
	inscrit est nécessaire à une opération de construction et que la demande de permis de construire porte à la fois
	sur la démolition et la construction, le permis de construire, qui autorise également la démolition, ne peut
	intervenir qu'avec l'accord exprès de l'architecte des bâtiments de France ; que, lorsque la demande de permis
	de construire porte à la fois sur la démolition et sur la construction et que les documents qui y sont joints22
	présentent de manière complète les deux volets de l'opération, l'avis de l'architecte des bâtiments de France
	exigé par les articles R. 425-18 et R. 425-30 du code de l'urbanisme doit être regardé comme portant sur
	l'ensemble de l'opération projetée, sans qu'il soit nécessaire que cet avis mentionne expressément la démolition ;
	-Sur la démolition d’HLM : CE 28 jan 2015 \No  368640 : 3. Considérant qu'en déduisant de l'existence d'un
	permis de démolir, accordé par un arrêté du 24 novembre 2005 du maire de la commune de Strasbourg, que la
	démolition de l'immeuble dont l'office public de l'habitat de la communauté urbaine de Strasbourg était
	propriétaire au 3, rue Ingold à Strasbourg devait être regardée comme ayant été autorisée dans les conditions
	prévues par l'article L. 443-15-1 du code de la construction et de l'habitation et que l'office était en droit de
	bénéficier de l'exonération qu'il demandait au titre de l'année 2007, sans rechercher si cette démolition avait été
	autorisée par le représentant de l'Etat dans le département au titre de ces dispositions et après accord des autres
	personnes mentionnées par l'article L. 443-15-1, le tribunal administratif a commis une erreur de droit ; que dès
	lors et sans qu'il soit besoin d'examiner l'autre moyen du pourvoi, le ministre est fondé à demander pour ce motif
	l'annulation de l'article 1er du jugement qu'il attaque ;
	CE 21 fev 2018 \No  401043 : 5. Considérant, en dernier lieu, qu'aux termes de l'article L. 421-6 du code de
	l'urbanisme : " Le permis de construire ou d'aménager ne peut être accordé que si les travaux projetés sont
	conformes aux dispositions législatives et réglementaires relatives à l'utilisation des sols, à l'implantation, la
	destination, la nature, l'architecture, les dimensions, l'assainissement des constructions et à l'aménagement de
	leurs abords et s'ils ne sont pas incompatibles avec une déclaration d'utilité publique. / Le permis de démolir
	peut être refusé ou n'être accordé que sous réserve de l'observation de prescriptions spéciales si les travaux
	envisagés sont de nature à compromettre la protection ou la mise en valeur du patrimoine bâti ou non bâti, du
	patrimoine archéologique, des quartiers, des monuments et des sites. " ; que l'article R. 431-21 du code de
	l'urbanisme dispose que : " Lorsque les travaux projetés nécessitent la démolition de bâtiments soumis au
	régime du permis de démolir, la demande de permis de construire ou d'aménager doit : / a) Soit être
	accompagnée de la justification du dépôt de la demande de permis de démolir ; / b) Soit porter à la fois sur la
	démolition et sur la construction ou l'aménagement. " ; qu'il résulte de ces dispositions que, si le permis de
	construire et le permis de démolir peuvent être accordés par une même décision, au terme d'une instruction
	commune, ils constituent des actes distincts comportant des effets propres ; qu'en annulant l'arrêté du 21 mars
	2013 en son entier, pour des motifs tirés de la seule illégalité du permis de construire, la cour administrative
	d'appel de Nancy a commis une erreur de droit
	2) La possibilité de reconstruire à l’identique L 111-15 (ex L 111-3) C urb
	Lorsqu'un bâtiment régulièrement édifié vient à être détruit ou démoli, sa reconstruction à l'identique est
	autorisée dans un délai de dix ans nonobstant toute disposition d'urbanisme contraire, sauf si la carte
	communale, le plan local d'urbanisme ou le plan de prévention des risques naturels prévisibles en dispose
	autrement.
	Article L111-23 (ex L 111-3)
	La restauration d'un bâtiment dont il reste l'essentiel des murs porteurs peut être autorisée, sauf dispositions
	contraires des documents d'urbanisme et sous réserve des dispositions de l'article L. 111-11, lorsque son intérêt
	architectural ou patrimonial en justifie le maintien et sous réserve de respecter les principales caractéristiques
	de ce bâtiment
	CE 17 déc 2008 \no  \No  305409 Considérant qu'il ressort des pièces du dossier soumis aux juges du fond que,
	par arrêté du 24 juin 2002 le maire de Valloire a mis à jour le plan d'occupation des sols de la commune en
	modifiant le plan et le tableau des servitudes, notamment pour tenir compte de l'approbation du plan de
	prévention des risques naturels prévisibles par arrêté préfectoral du 6 mai 2002 ; que ce plan interdit les
	constructions nouvelles dans la zone d'avalanches dans laquelle est situé le terrain d'assiette du projet de M. A ;
	que, dès lors, en jugeant que l'intéressé ne pouvait se prévaloir des dispositions de l'article L. 111-3 du code de
	l'urbanisme pour bénéficier de l'autorisation de reconstruire dans cette zone un chalet précédemment détruit par
	une avalanche, la cour, qui n'avait pas à rechercher si les prescriptions dont était assorti le permis de construire
	que lui avait délivré le maire de Valloire en méconnaissance du plan de prévention des risques naturels annexé
	au plan d'occupation des sols étaient suffisantes pour éviter le danger, n'a entaché l'arrêt attaqué d'aucune
	erreur de droit
	CE 26 avril 2017 \No  400457 13. En effet, par les dispositions de l'article L. 111-3 du code de l'urbanisme, le
	législateur n'a pas entendu donner le droit de reconstruire un bâtiment dont les occupants seraient exposés à un
	risque certain et prévisible de nature à mettre gravement en danger leur sécurité. Il en va notamment ainsi
	lorsque c'est la réalisation d'un tel risque qui a été à l'origine de la destruction du bâtiment pour la23
	reconstruction duquel le permis est demandé. Dans une telle hypothèse, il y a lieu, pour l'autorité compétente, de
	refuser le permis de construire ou de l'assortir, si cela suffit à parer au risque, de prescriptions adéquates, sur le
	fondement de l'article R 111-2 du code de l'urbanisme
	CE 8 nov 2017, \No  403599 : 2. Considérant qu'aux termes de l'article L. 111-3 du code de l'urbanisme, dans sa
	rédaction en vigueur à la date des arrêtés litigieux : " La reconstruction à l'identique d'un bâtiment détruit ou
	démoli depuis moins de dix ans est autorisée nonobstant toute disposition d'urbanisme contraire, sauf si la carte
	communale, le plan local d'urbanisme ou le plan de prévention des risques naturels prévisibles en dispose
	autrement, dès lors qu'il a été régulièrement édifié. " ; qu'il résulte de ces dispositions que, dès lors qu'un
	bâtiment a été régulièrement construit, seules des dispositions expresses de la réglementation locale
	d'urbanisme prévoyant l'interdiction de la reconstruction à l'identique de bâtiments détruits par sinistre ou
	démolis peuvent faire légalement obstacle à sa reconstruction ;
	3. Considérant que, pour confirmer la légalité de l'arrêté du 11 mars 2011 portant refus de permis de construire,
	la cour administrative d'appel de Versailles, après avoir cité les dispositions du plan local d'urbanisme de la
	commune d'Evecquemont applicable à la zone selon lesquelles " sont admises les occupations suivantes : (...) la
	reconstruction à l'identique dans le cas de sinistre ", a jugé que seule la reconstruction à l'identique d'un
	bâtiment en cas de sinistre était, compte tenu de ces dispositions, légalement possible ; qu'il résulte de ce qui a
	été dit au point 2 ci-dessus qu'en statuant ainsi, la cour administrative d'appel a commis une erreur de droit ;
	que, par suite et sans qu'il soit besoin d'examiner les autres moyens du pourvoi, son arrêt doit être annulé ;
	CE 4 décembre 2017 \No  407165 7. Considérant, en troisième lieu, qu'aux termes de l'article L. 111-3 du
	code de l'urbanisme, dans sa rédaction applicable au litige : " La reconstruction à l'identique d'un bâtiment
	détruit ou démoli depuis moins de dix ans est autorisée nonobstant toute disposition d'urbanisme contraire, sauf
	si la carte communale, le plan local d'urbanisme ou le plan de prévention des risques naturels prévisibles en
	dispose autrement, dès lors qu'il a été régulièrement édifié " ; qu'en relevant que le hangar unique autorisé par
	le permis de construire contesté, qui remplace plusieurs bâtiments, n'avait pas la même implantation, la même
	surface, ni le même volume que les bâtiments détruits et en déduisant qu'il ne constituait pas une reconstruction
	à l'identique d'un bâtiment démoli pouvant être autorisée sur le fondement de l'article L. 111-3 du code de
	l'urbanisme, la cour, sans se méprendre sur la portée des écritures, a porté sur les faits de l'espèce et les pièces
	du dossier une appréciation souveraine qui est exempte de dénaturation ;
	3) L’impact de l’illégalité d’origine des constructions existantes
	-La jurisprudence THALAMY CE 9 juil 1986 \no  51172 : « que même si les documents et notamment le plan
	fourni à l'appui de la demande de permis, faisaient apparaître l'existence de cette terrasse, il appartenait au
	propriétaire de présenter une demande portant sur l'ensemble des éléments de construction qui ont eu ou qui
	auront pour effet de transformer le bâtiment tel qu'il avait été autorisé par le permis primitif ; que le maire ne
	pouvait légalement accorder un permis portant uniquement sur un élément de construction nouveau prenant
	appui sur une partie du bâtiment construite sans autorisation »
	- Confirmation CE 27 juillet 2012 Mme Sylvie B \No  316155
	6. Considérant qu'aux termes de l'article L. 421-1 du code de l'urbanisme, dans sa rédaction applicable à la
	décision contestée : " Quiconque désire entreprendre ou implanter une construction à usage d'habitation ou non,
	même ne comportant pas de fondations, doit, au préalable, obtenir un permis de construire (...) " ; que ces
	prescriptions s'appliquent également dans l'hypothèse où l'autorité administrative est saisie d'une demande
	tendant à ce que soient autorisés des travaux portant sur un immeuble qui a été édifié sans autorisation, la
	demande devant alors porter sur l'ensemble du bâtiment ;
	7. Considérant qu'il ressort des pièces du dossier soumis aux juges du fond que la construction à usage
	d'habitation située au 36, rue du 11 novembre prolongée, sur un terrain où de précédentes constructions avaient
	été entièrement démolies, a été réalisée par Mme B sans qu'ait été demandée l'autorisation de construire prévue
	à l'article L. 421-1 du code de l'urbanisme et nécessaire en l'espèce eu égard à la superficie de la construction ;
	que Mme B a déposé, le 31 mai 2006, une déclaration de travaux portant exclusivement sur la modification de
	l'aspect extérieur de cette construction alors qu'elle était tenue de déposer une demande portant sur l'ensemble
	du bâtiment ; que, dès lors, en jugeant, compte tenu des conditions dans lesquelles avait été édifié le bâtiment
	sur lequel des travaux étaient envisagés, que ceux-ci rendaient nécessaires le dépôt d'un permis de construire
	préalable portant sur l'ensemble du bâtiment, le tribunal administratif n'a ni commis d'erreur de droit, ni
	dénaturé les pièces du dossier ;
	8. Considérant ainsi qu'il vient d'être dit que, dès lors qu'une demande porte sur des travaux qui concernent un
	bâtiment édifié sans autorisation, cette demande doit porter sur l'ensemble du bâtiment ; que, par suite, en
	jugeant que le maire de la commune de Petit Couronne était tenu de s'opposer à la déclaration de travaux
	déposée par Mme B, le tribunal administratif n'a pas commis d'erreur de droit ;24
	- La limite de CE 9 jan 2009 Ville de Toulouse AJDA 2009 611 note Durand : Considérant qu'aux termes de
	l'article R. 422-2 du code de l'urbanisme dans sa rédaction alors applicable : Sont exemptés du permis de
	construire sur l'ensemble du territoire : (...) / k) les piscines non couvertes ; (...) ; qu'il ressort des pièces du
	dossier soumis au magistrat délégué par le président du tribunal administratif de Toulouse que c'est sans
	dénaturation des faits qui lui étaient soumis qu'il a pu constater que la piscine, quoique proche, n'est ni
	attenante ni structurellement liée à l'habitation principale de Mme A ; que ce magistrat n'a par suite pas commis
	d'erreur de droit en estimant que la construction de cette piscine, dissociable de l'habitation principale, ne
	nécessitait pas l'octroi d'un permis de construire, malgré l'illégalité alléguée de l'habitation principale ;
	qu'ainsi, la COMMUNE DE TOULOUSE n'est pas fondée à demander l'annulation du jugement attaqué ;
	-Extension par CE 13 déc 2013 \No  349081
	2. Considérant que, lorsqu'une construction a fait l'objet de transformations sans les autorisations d'urbanisme
	requises, il appartient au propriétaire qui envisage d'y faire de nouveaux travaux de déposer une déclaration ou
	de présenter une demande de permis portant sur l'ensemble des éléments de la construction qui ont eu ou auront
	pour effet de modifier le bâtiment tel qu'il avait été initialement approuvé ; qu'il en va ainsi même dans le cas où
	les éléments de construction résultant de ces travaux ne prennent pas directement appui sur une partie de
	l'édifice réalisée sans autorisation ; qu'il appartient à l'administration de statuer au vu de l'ensemble des pièces
	du dossier, en tenant compte, le cas échéant, de l'application des dispositions de l'article L. 111-12 du code de
	l'urbanisme issues de la loi du 13 juillet 2006 emportant régularisation des travaux réalisés depuis plus de dix
	ans ;
	-La limite de CE 3 mai 2011 Chantal Gisèle ,\no  320545
	Considérant que, dans l'hypothèse où un immeuble a été édifié sans autorisation en méconnaissance des
	prescriptions légales alors applicables, l'autorité administrative, saisie d'une demande tendant à ce que soient
	autorisés des travaux portant sur cet immeuble, est tenue d'inviter son auteur à présenter une demande portant
	sur l'ensemble du bâtiment ; que dans l'hypothèse où l'autorité administrative envisage de refuser le permis
	sollicité parce que la construction dans son entier ne peut être autorisée au regard des règles d'urbanisme en
	vigueur à la date de sa décision, elle a toutefois la faculté, dans l'hypothèse d'une construction ancienne, à
	l'égard de laquelle aucune action pénale ou civile n'est plus possible, après avoir apprécié les différents intérêts
	publics et privés en présence au vu de cette demande, d'autoriser, parmi les travaux demandés, ceux qui sont
	nécessaires à sa préservation et au respect des normes, alors même que son édification ne pourrait plus être
	régularisée au regard des règles d'urbanisme applicables ; Considérant que Mme A ne conteste pas que
	l'appentis dont elle est propriétaire, d'une surface hors oeuvre brute supérieure à 20 mètres carrés, a été réalisé
	sans autorisation d'urbanisme en méconnaissance des prescriptions légales, mais invoque l'ancienneté de sa
	construction, remontant à 1967, pour soutenir que les travaux qu'elle envisageait devaient seuls faire l'objet
	d'une autorisation de construire et relevaient dès lors du régime de la déclaration de travaux ; que, toutefois,
	contrairement à ce qu'elle soutient, les dispositions alors applicables de l'article 2262 du code civil prévoyant la
	prescription par trente ans de toutes les actions, tant réelles que personnelles, étaient sans incidence sur la
	détermination du régime d'autorisation applicable aux travaux litigieux ; que l'obligation de déposer une
	demande visant à la régularisation de l'ensemble de la construction en cause avant d'être autorisé à effectuer
	des travaux sur l'immeuble, quelle que soit leur importance, ne méconnaît ni le principe de sécurité juridique ni
	le droit de propriété, consacré notamment par l'article 17 de la Déclaration des droits de l'homme et du citoyen
	de 1789 et par l'article 1er du premier protocole additionnel à la convention européenne de sauvegarde des
	droits de l'homme et des libertés fondamentales ;
	- L’Article L 421-9 (ex L111-12)
	Lorsqu'une construction est achevée depuis plus de dix ans, le refus de permis de construire ou de déclaration de
	travaux ne peut être fondé sur l'irrégularité de la construction initiale au regard du droit de l'urbanisme.
	Les dispositions du premier alinéa ne sont pas applicables :
	a) Lorsque la construction est de nature, par sa situation, à exposer ses usagers ou des tiers à un risque de
	mort ou de blessures de nature à entraîner une mutilation ou une infirmité permanente ;
	b) Lorsqu'une action en démolition a été engagée dans les conditions prévues par l'article L. 480-13 ;
	c) Lorsque la construction est située dans un site classé en application des articles L. 341-2 et suivants du code
	de l'environnement ou un parc naturel créé en application des articles L. 331-1 et suivants du même code ;
	d) Lorsque la construction est sur le domaine public ;
	e) Lorsque la construction a été réalisée sans permis de construire ;
	f) Dans les zones visées au 1o du II de l'article L. 562-1 du code de l'environnement.
	Cf Conseil d'État 16 mars 2015 \No  369553 Publié au recueil Lebon 1ère / 6ème SSR
	2. Considérant que, lorsqu'une construction a fait l'objet de transformations sans les autorisations d'urbanisme
	requises, il appartient au propriétaire qui envisage d'y faire de nouveaux travaux de déposer une déclaration ou
	de présenter une demande de permis portant sur l'ensemble des éléments de la construction qui ont eu ou auront25
	pour effet de modifier le bâtiment tel qu'il avait été initialement approuvé ou de changer sa destination ; qu'il en
	va ainsi même dans le cas où les éléments de construction résultant de ces travaux ne prennent pas directement
	appui sur une partie de l'édifice réalisée sans autorisation ;
	3. Considérant qu'il appartient à l'autorité administrative, saisie d'une telle déclaration ou demande de permis,
	de statuer au vu de l'ensemble des pièces du dossier d'après les règles d'urbanisme en vigueur à la date de sa
	décision ; qu'elle doit tenir compte, le cas échéant, de l'application des dispositions de l'article L. 111-12 du
	code de l'urbanisme issues de la loi du 13 juillet 2006 portant engagement national pour le logement, qui
	prévoient la régularisation des travaux réalisés depuis plus de dix ans à l'occasion de la construction primitive
	ou des modifications apportées à celle-ci, sous réserve, notamment, que les travaux n'aient pas été réalisés sans
	permis de construire en méconnaissance des prescriptions légales alors applicables ; que, dans cette dernière
	hypothèse, si l'ensemble des éléments de la construction mentionnés au point 2 ne peuvent être autorisés au
	regard des règles d'urbanisme en vigueur à la date de sa décision, l'autorité administrative a toutefois la faculté,
	lorsque les éléments de construction non autorisés antérieurement sont anciens et ne peuvent plus faire l'objet
	d'aucune action pénale ou civile, après avoir apprécié les différents intérêts publics et privés en présence au vu
	de cette demande, d'autoriser, parmi les travaux demandés, ceux qui sont nécessaires à la préservation de la
	construction et au respect des normes ;
	4. Considérant, en premier lieu, qu'il ressort des constatations opérées souverainement par les juges du fond, au
	demeurant non contestées, que la demande de permis de construire des époux B...ne portait que sur les travaux
	d'extension et non sur la régularisation des travaux ayant antérieurement permis le changement de destination
	du chalet ; que la cour n'a pas commis d'erreur de droit en jugeant qu'il incombait aux époux B...de présenter
	une demande portant sur l'ensemble des travaux qui ont eu ou qui auront pour effet de transformer le bâtiment
	tel qu'il avait été autorisé par le permis de construire initial et en en déduisant que le maire de la commune de
	Saint-Gervais-les-Bains était tenu de refuser le permis ;
	CE 3 février 2017, \No  373898
	2. Considérant qu'aux termes de l'article L. 111-12 du code de l'urbanisme, dans sa rédaction alors en vigueur
	issue de la loi du 13 juillet 2006 portant engagement national pour le logement et dont les dispositions ont été
	reprises à l'actuel article L. 421-9 du même code : " Lorsqu'une construction est achevée depuis plus de dix ans,
	le refus de permis de construire ou de déclaration de travaux ne peut être fondé sur l'irrégularité de la
	construction initiale au regard du droit de l'urbanisme. / Les dispositions du premier alinéa ne sont pas
	applicables : (...) / e) Lorsque la construction a été réalisée sans permis de construire ; / (...) " ; qu'il résulte de
	ces dispositions que peuvent bénéficier de la prescription administrative ainsi définie les travaux réalisés, depuis
	plus de dix ans, lors de la construction primitive ou à l'occasion des modifications apportées à celle-ci, sous
	réserve qu'ils n'aient pas été réalisés sans permis de construire en méconnaissance des prescriptions légales
	alors applicables ; qu'à la différence des travaux réalisés depuis plus de dix ans sans permis de construire, alors
	que ce dernier était requis, peuvent bénéficier de cette prescription ceux réalisés sans déclaration préalable ;
	3. Considérant, en premier lieu, qu'il résulte de ce qui a été dit au point précédent que la cour n'a pas entaché
	son arrêt d'une erreur de droit en jugeant qu'un bâtiment édifié au dix-neuvième siècle, avant que les lois et
	règlements ne soumettent les constructions à un régime d'autorisation d'urbanisme, ne pouvait être regardé
	comme ayant été réalisé sans permis de construire pour l'application des dispositions du e) de l'article L. 111-12
	du code de l'urbanisme ;
	4. Considérant toutefois, en second lieu, que la cour a relevé que la construction litigieuse avait fait l'objet plus
	de dix ans avant l'édiction de l'arrêté litigieux de modifications qui étaient soumises à permis de construire à la
	date à laquelle elles ont été réalisées ; que, pour juger que ces travaux pouvaient néanmoins bénéficier de la
	prescription prévue à l'article L. 111-12, la cour s'est fondée sur la circonstance qu'ils avaient revêtu une
	ampleur limitée et n'avaient, dès lors, pas conduit à la réalisation d'une nouvelle construction ; que ce faisant, la
	cour a méconnu les règles rappelées au point 2 et ainsi commis une erreur de droit ; que, par suite, et sans qu'il
	soit besoin d'examiner les autres moyens du pourvoi, les requérants sont fondés, pour ce motif, à demander
	l'annulation de l'arrêt qu'ils attaquent ;
	CE 26 nov 2018 \No  411991 : « 2. Lorsqu'une construction a été édifiée sans respecter la déclaration préalable
	déposée ou le permis de construire obtenu ou a fait l'objet de transformations sans les autorisations d'urbanisme
	requises, il appartient au propriétaire qui envisage d'y faire de nouveaux travaux de déposer une déclaration ou
	de présenter une demande de permis portant sur l'ensemble des éléments de la construction qui ont eu ou auront
	pour effet de modifier le bâtiment tel qu'il avait été initialement approuvé. Il en va ainsi même dans le cas où les
	éléments de construction résultant de ces travaux ne prennent pas directement appui sur une partie de l'édifice
	réalisée sans autorisation. Il appartient à l'administration de statuer au vu de l'ensemble des pièces du dossier,
	d'après les règles d'urbanisme en vigueur à la date de sa décision, en tenant compte, le cas échéant, de26
	l'application des dispositions de l'article L. 111-12 du code de l'urbanisme issues de la loi du 13 juillet 2006,
	désormais reprises à l'article L. 421-9 de ce code, relatives à la régularisation des travaux réalisés depuis plus
	de dix ans.
	3. Toutefois, aux termes de l'article L. 462-2 du code de l'urbanisme : " L'autorité compétente mentionnée aux
	articles L. 422-1 à L. 422-3 peut, dans un délai fixé par décret en Conseil d'Etat, procéder ou faire procéder à
	un récolement des travaux et, lorsque ceux-ci ne sont pas conformes au permis délivré ou à la déclaration
	préalable, mettre en demeure le maître de l'ouvrage de déposer un dossier modificatif ou de mettre les travaux
	en conformité. Un décret en Conseil d'Etat fixe les cas où le récolement est obligatoire. / Passé ce délai,
	l'autorité compétente ne peut plus contester la conformité des travaux ". Aux termes de l'article R. 462-6 du
	même code : " A compter de la date de réception en mairie de la déclaration d'achèvement, l'autorité compétente
	dispose d'un délai de trois mois pour contester la conformité des travaux au permis ou à la déclaration. / Le
	délai de trois mois prévu à l'alinéa précédent est porté à cinq mois lorsqu'un récolement des travaux est
	obligatoire en application de l'article R. 462-7 ". Il résulte de ces dispositions que lorsque le bénéficiaire d'un
	permis ou d'une décision de non-opposition à déclaration préalable a adressé au maire une déclaration attestant
	l'achèvement et la conformité des travaux réalisés en vertu de cette autorisation, l'autorité compétente ne peut
	plus en contester la conformité au permis ou à la déclaration si elle ne l'a pas fait dans le délai, suivant les cas,
	de trois ou de cinq mois ni, dès lors, sauf le cas de fraude, exiger du propriétaire qui envisage de faire de
	nouveaux travaux sur la construction qu'il présente une demande de permis ou dépose une déclaration portant
	également sur des éléments de la construction existante, au motif que celle-ci aurait été édifiée sans respecter le
	permis de construire précédemment obtenu ou la déclaration préalable précédemment déposé »
	5) L’impact d’une construction devenue illégale
	CE 4 avril 2018 \No  407445 4. Considérant que lorsqu'une construction existante n'est pas conforme à une ou
	plusieurs dispositions d'un plan local d'urbanisme régulièrement approuvé, un permis de construire ne peut être
	légalement délivré pour la modification de cette construction, sous réserve de dispositions de ce plan
	spécialement applicables à la modification des immeubles existants, que si les travaux envisagés rendent
	l'immeuble plus conforme aux dispositions réglementaires méconnues ou s'ils sont étrangers à ces dispositions ;
	En ce qui concerne les dispositions du règlement du plan local d'urbanisme relatives aux aires de stationnement
	:
	5. Considérant qu'aux termes de l'article UJ 12 du règlement du plan local d'urbanisme de la commune de
	Grasse : " Les aires de stationnement, (y compris pour les " deux-roues "), et leurs zones de manoeuvre doivent
	être réalisées en dehors des voies ouvertes à la circulation. Il est exigé un nombre de places de stationnement
	correspondant : - aux caractéristiques de l'opération, - à son environnement. / Cependant, pour les
	constructions à usage d'habitation, il est exigé 2,5 places de stationnement par logement. (...) ;
	6. Considérant que, pour l'application de la règle rappelée au point 4, des travaux entrepris sur un immeuble
	existant qui n'impliquent pas la création de nouveaux logements mais seulement l'extension de logements
	existants doivent être regardés comme étrangers aux dispositions d'un plan local d'urbanisme imposant un
	nombre minimal de places de stationnement par logement ;
	7. Considérant qu'après avoir relevé que le permis de construire litigieux concernait deux maisons d'habitation
	situées dans une propriété comprenant en tout cinq maisons, sur laquelle huit places de stationnement avaient
	été aménagées, le jugement attaqué énonce qu'à supposer que, comme le soutiennent les défendeurs, les travaux
	ne créent aucun nouveau logement, le permis est illégal dès lors qu'il ne ressort pas des pièces du dossier qu'au
	moins cinq des places existantes soient spécialement affectées aux deux maisons en cause ; que le tribunal
	administratif a ainsi retenu que les travaux seraient effectués sur des constructions qui, faute de disposer d'au
	moins 2,5 places de stationnement par logement, n'étaient pas conformes aux dispositions précitées de l'article
	UJ 12 du règlement du plan local d'urbanisme, qu'à supposer même qu'ils n'entraînent qu'une augmentation de
	surface sans création de nouveaux logements, ils n'étaient pas étrangers à ces dispositions et qu'ils ne rendraient
	pas les constructions plus conformes à la règle méconnue ; qu'en se prononçant ainsi, alors qu'il résulte de ce
	qui a été dit au point 6 que des travaux n'entraînant pas la création de nouveaux logements devaient être
	regardés comme étrangers aux dispositions de l'article UJ 12, le tribunal a commis une erreur de droit ; que,
	par suite, les requérantes sont fondées à soutenir, sans qu'il soit besoin d'examiner les autres moyens du pourvoi
	dirigés contre le même motif du jugement, que ce motif n'est pas de nature à justifier légalement l'annulation du
	permis de construire attaqué ;27
	En ce qui concerne les dispositions du règlement du plan local d'urbanisme relatives à l'implantation des
	constructions par rapport aux voies ouvertes à la circulation :
	8. Considérant qu'aux termes de l'article UJ 6 du règlement du plan local d'urbanisme de la commune de Grasse
	(implantation des constructions par rapport aux voies et aux emprises publiques) : " Les dispositions du présent
	article s'appliquent aux voies publiques et privées ouvertes à la circulation générale, y compris aux voies
	piétonnes, ainsi qu'aux emprises publiques. / Les bâtiments doivent s'implanter à une distance au moins égale à
	75 m de l'axe de la pénétrante Cannes-Grasse (RD 6185). / Les bâtiments doivent s'implanter à une distance de
	l'alignement existant ou futur au moins égale à : - 15 mètres de la route de Cannes, / - 20 mètres de la RD 304, /
	- 10 mètres pour les autres routes départementales, / - 5 mètres pour les autres routes. (...) " ;
	9. Considérant que des travaux tendant à la surélévation d'un bâtiment implanté en méconnaissance des
	dispositions du plan local d'urbanisme relatives à l'implantation des constructions par rapport aux limites
	séparatives ou à la voie publique ne sont pas étrangers à ces dispositions et n'ont pas pour effet de rendre le
	bâtiment plus conforme à celles-ci ;
	10. Considérant qu'en relevant que les travaux projetés par le permis de construire litigieux, qui comportaient
	une surélévation d'un bâtiment implanté à l'alignement de la voie publique, n'étaient pas étrangers aux
	dispositions précitées de l'article UJ 6 du règlement du plan local d'urbanisme de la commune, qui prescrivent
	que les constructions doivent être réalisées à cinq mètres au moins de l'alignement de la voie publique, pour en
	déduire que, n'ayant pas rendu ce bâtiment plus conforme à ces dispositions, ces travaux ne pouvaient être
	légalement autorisés, le tribunal administratif, qui a suffisamment motivé son jugement sur ce point, n'a pas
	commis d'erreur de droit ;
	11. Considérant que le moyen, qui n'est pas d'ordre public et n'est pas né du jugement attaqué, tiré de ce que le
	tribunal aurait commis une erreur de droit en ne faisant pas application des dispositions de l'article L. 111-12
	du code de l'urbanisme, dont il résulte que lorsqu'une construction est achevée depuis plus de dix ans, le refus de
	permis de construire ou de déclaration de travaux ne peut être fondé sur l'irrégularité de la construction initiale
	au regard du droit de l'urbanisme est nouveau en cassation et est, par suite, inopérant ;
	12. Considérant qu'il résulte de ce qui précède que l'un des deux motifs retenus par le tribunal administratif de
	Nice, conformément aux dispositions de l'article L. 600-4-1 du code de l'urbanisme, justifie légalement
	l'annulation du permis de construire du 13 août 2013 modifié par le permis de construire du 21 novembre 2013 ;
	que, par suite, le pourvoi de Mmes C... doit être rejeté
	F - Le terrain frappé d’une clause anti-spéculative
	Cf Cass 3 e civ 23 septembre 2009 \No  de pourvoi: 08-18187
	Attendu, selon l'arrêt attaqué (Pau, 24 avril 2008), que par acte notarié du 23 mai 2003, la commune de Saint-
	Pée-sur-Nivelle a vendu à M. X... et à Mme Y... un lot d'une superficie de 999 m2 dans un lotissement communal,
	au prix de 42 685 euros ; que dans un paragraphe intitulé "conditions particulières imposées par la commune -
	Pacte de préférence", l'acte de vente comportait une clause, valable pendant vingt ans, prévoyant qu'avant toute
	revente à un tiers, le rachat du terrain devrait être proposé à la commune ; que la clause précisait que le prix de
	revente du terrain nu ne pourrait excéder le prix d'acquisition initial, réactualisé en fonction de la variation de
	l'indice INSEE du coût de la construction, et que le prix du terrain avec une construction serait égal au prix de
	vente du terrain nu majoré du prix de revient de la construction, évalué par un expert ; que M. X... et Mme Y...
	ayant, le 21 octobre 2006, signé un compromis de vente de leur terrain au prix de 120 000 euros, la commune
	les a avisés qu'elle entendait exercer son droit de priorité au prix d'acquisition réactualisé en fonction de
	l'érosion monétaire ; que M. X... et Mme Y... ont alors assigné la commune pour faire annuler la clause
	instituant, à son profit, un droit de priorité ;
	Attendu M. X... et Mme Y... font grief à l'arrêt de les débouter de leur demande, alors, selon le moyen, que le
	pacte de préférence qui impose au promettant, au cas où il déciderait d'aliéner le bien, de donner préférence au
	bénéficiaire du pacte, à un prix prédéterminé dans le contrat, constitue une atteinte au droit de propriété lorsque
	la durée de cet engagement est de vingt ans de sorte que la clause instituant un tel pacte doit être annulée ;
	qu'en décidant le contraire, tout en constatant que le contrat conclu entre M. X... et Mme Y... et la commune
	stipulait que les premiers s'engageaient, au cas où ils décideraient de vendre, à donner préférence à la
	commune, à un prix prédéterminé au contrat, pendant une durée de vingt ans, la cour d'appel a violé l'article
	544 du code civil ;
	Mais attendu qu'ayant relevé, par motifs propres et adoptés, que la stipulation avait été librement convenue,
	qu'elle avait pour but, en fixant d'ores et déjà un prix, institué pour une durée de vingt ans, d'empêcher la
	spéculation sur le bien dans un contexte marqué par la rareté de l'offre et le "décrochage" des possibilités28
	financières de la plupart des ménages par rapport à l'envolée des prix de l'immobilier, et que M. X... et Mme Y...
	avaient bénéficié en contrepartie de son acceptation de la possibilité d'accéder à un marché protégé de la
	spéculation immobilière, la cour d'appel, qui a retenu à bon droit que les modalités stipulées, notamment quant
	à la durée de validité de la clause, n'étaient pas, au regard de la nature et de l'objet de l'opération réalisée,
	constitutives d'une atteinte au droit de propriété, en a exactement déduit que la demande en nullité devait être
	rejetée ;
	Article L 3211-7 CG3P Modifié par loi Modifié par LOI \no 2017-256 du 28 février 2017
	I. – L'Etat peut procéder à l'aliénation de terrains de son domaine privé à un prix inférieur à la valeur vénale
	lorsque ces terrains, bâtis ou non, sont destinés à la réalisation de programmes comportant essentiellement des
	logements dont une partie au moins est réalisée en logement social. Pour la part du programme destinée aux
	logements sociaux, la décote ainsi consentie, qui peut atteindre 100 % de la valeur vénale du terrain, est fixée en
	fonction de la catégorie à laquelle ces logements appartiennent. Elle prend notamment en considération les
	circonstances locales tenant à la situation du marché foncier et immobilier, à la situation financière de
	l'acquéreur du terrain, à la proportion et à la typologie des logements sociaux existant sur le territoire de la
	collectivité considérée et aux conditions financières et techniques de l'opération. La décote ne saurait excéder
	50 % pour les logements financés en prêts locatifs sociaux et pour les logements en accession à la propriété
	bénéficiant des dispositifs mentionnés au VIII, à l'exception des logements en accession à la propriété en
	Guadeloupe, en Guyane, en Martinique et à La Réunion qui bénéficient d'une aide destinée aux personnes
	physiques à faibles revenus, pour financer l'acquisition de logements évolutifs sociaux.
	Pour les communes qui ne font pas l'objet d'un constat de carence, dans le cadre d'un programme de logements
	sociaux, dans les conditions fixées au présent article, une décote est possible pour la part du programme relative
	aux équipements publics destinés en tout ou partie aux occupants de ces logements. La décote ainsi consentie est
	alignée sur la décote allouée pour la part du programme consacrée aux logements sociaux. Les modalités
	d'application du présent alinéa et la liste des équipements publics concernés sont fixées par décret en Conseil
	d'Etat.
	II. – Une décote est de droit lorsque les deux conditions suivantes sont satisfaites :
	1\degre  Les terrains sont cédés au profit d'une collectivité territoriale, d'un établissement public de coopération
	intercommunale à fiscalité propre, d'un établissement public mentionné aux chapitres Ier et IV du titre II du
	livre III du code de l'urbanisme, d'un organisme agréé mentionné à l'article L. 365-2 du code de la construction
	et de l'habitation, d'un organisme mentionné à l'article L. 411-2 du même code, d'une société d'économie mixte
	mentionnée à l'article L. 481-1 dudit code ou d'un opérateur lié à une collectivité ou un établissement public de
	coopération intercommunale à fiscalité propre par une concession d'aménagement dont l'objet prévoit
	notamment la réalisation de logement social ou, en Guadeloupe, en Guyane, en Martinique et à La Réunion,
	d'un organisme agréé pour la réalisation de logements en accession à la propriété qui bénéficient d'une aide
	destinée aux personnes physiques à faibles revenus, pour financer l'acquisition de logements évolutifs sociaux ;
	2\degre  Les terrains appartiennent à une liste de parcelles établie par le représentant de l'Etat dans la région, après
	avis, dans un délai de deux mois, du comité régional de l'habitat, du maire de la commune sur le territoire de
	laquelle les terrains se trouvent et du président de l'établissement public de coopération intercommunale
	compétent. Cette liste est mise à jour annuellement. Elle peut être complétée selon les mêmes modalités, à la
	demande de l'une des personnes morales mentionnées au 1\degre , sur présentation par cette dernière d'un projet
	s'inscrivant dans une stratégie de mobilisation du foncier destinée à satisfaire des besoins locaux en matière de
	logement.
	Les présentes dispositions ne s'appliquent aux organismes agréés mentionnés à l'article L. 365-2 du code de la
	construction et de l'habitation et aux sociétés d'économie mixte mentionnées à l'article L. 481-1 du même code
	que pour les cessions de terrains destinés à des programmes de logements faisant l'objet de conventions régies
	par le chapitre III du titre V du livre III dudit code.
	II bis. – (Abrogé)
	III. – L'avantage financier résultant de la décote est exclusivement et en totalité répercuté sur le prix de revient
	des logements locatifs sociaux.
	Cette décote est également répercutée sur le prix de cession des logements en accession à la propriété
	bénéficiant des dispositifs mentionnés au VIII du présent article.
	Le primo-acquéreur d'un logement qui souhaite le revendre dans les dix ans qui suivent l'acquisition consécutive
	à la première mise en vente du bien est tenu d'en informer le représentant de l'Etat dans la région. Ce dernier en
	informe les organismes mentionnés à l'article L. 411-2 du code de la construction et de l'habitation, qui peuvent
	se porter acquéreurs du logement en priorité. Le primo-acquéreur est tenu de verser à l'Etat une somme égale à
	la différence entre le prix de vente et le prix d'acquisition de son logement. Cette somme ne peut excéder le
	montant de la décote. Pour l'application du présent alinéa, les prix s'entendent hors frais d'acte et accessoires à
	la vente.29
	Lorsque le primo-acquéreur d'un logement le loue dans les dix ans qui suivent l'acquisition consécutive à la
	première mise en vente du bien, le niveau de loyer ne doit pas excéder des plafonds fixés par le représentant de
	l'Etat dans la région. Ceux-ci sont arrêtés par référence au niveau des loyers qui y sont pratiqués pour des
	logements locatifs sociaux de catégories similaires.
	A peine de nullité, les contrats de vente comportent la mention des obligations visées aux troisième et quatrième
	alinéas du présent III et du montant de la décote consentie.
	IV. – Pour les programmes ayant bénéficié de la cession d'un terrain avec une décote dans les conditions du
	présent article :
	1\degre  Les conventions mentionnées à l'article L. 351-2 du code de la construction et de l'habitation sont d'une durée
	au moins égale à vingt ans. Cette durée ne peut être inférieure à la période restant à courir pour l'amortissement
	du prêt. Le remboursement anticipé du prêt n'a pas d'incidence sur la durée de la convention ;
	2\degre  Le délai de dix ans mentionné au premier alinéa de l'article L. 443-7 du même code est porté à vingt ans.
	Cette disposition s'applique également aux opérations des organismes agréés mentionnés à l'article L. 365-2
	dudit code.
	V. – Une convention conclue entre le représentant de l'Etat dans la région et l'acquéreur, jointe à l'acte
	d'aliénation, fixe les conditions d'utilisation du terrain cédé et détermine le contenu du programme de logements
	à réaliser.
	Les données dont l'Etat dispose sur le patrimoine naturel du terrain faisant l'objet de la cession sont annexées à
	cette convention.
	L'acte d'aliénation mentionne le montant de la décote consentie. Il prévoit, en cas de non-réalisation du
	programme de logements dans le délai de cinq ans, soit la résolution de la vente sans indemnité pour
	l'acquéreur et le versement du montant des indemnités contractuelles applicables, soit le versement du montant
	d'une indemnité préjudicielle pouvant atteindre le double de la décote consentie. Ce délai est suspendu en cas de
	recours devant la juridiction administrative contre une autorisation administrative requise pour la réalisation de
	ce programme, à compter de l'introduction du recours et jusqu'à la date à laquelle la décision de la juridiction
	devient définitive. Il est également suspendu si des opérations de fouilles d'archéologie préventive sont prescrites
	en application de l'article L. 522-2 du code du patrimoine pendant la durée de ces opérations.
	L'acte d'aliénation prévoit, en cas de réalisation partielle du programme de logements ou de réalisation dans
	des conditions différentes de celles prises en compte pour la fixation du prix de cession, le paiement d'un
	complément de prix correspondant à l'avantage financier indûment consenti.
	Lorsque la cession d'un terrain, bâti ou non, du domaine privé de l'Etat s'inscrit dans une opération
	d'aménagement, au sens de l'article L. 300-1 du code de l'urbanisme, qui porte sur un périmètre de plus de cinq
	hectares, et après accord des ministres chargés du logement et du domaine, au vu du rapport transmis par le
	représentant de l'Etat dans la région, la convention conclue entre le représentant de l'Etat dans la région et
	l'acquéreur peut prévoir une réalisation de l'opération par tranches échelonnées sur une durée totale supérieure
	à cinq ans et permettant chacune un contrôle du dispositif de décote, dans les conditions prévues aux troisième
	et quatrième alinéas du présent V.
	La convention peut prévoir, en outre, le droit de réservation d'un contingent plafonné à 10 % des logements
	sociaux du programme, au profit de l'administration qui cède son terrain avec décote, pour le logement de ses
	agents, au-delà du contingent dont dispose l'Etat.
	V bis. – L'Etat peut céder à titre onéreux à la société mentionnée au deuxième alinéa du I de l'article 141 de la
	loi \no  2006-1771 du 30 décembre 2006 de finances rectificative pour 2006 la propriété de portefeuilles de
	terrains, bâtis ou non, de son domaine privé.
	Chacune de ces cessions fait l'objet d'une convention jointe à l'acte d'aliénation, conclue entre le ministre
	chargé du domaine et l'acquéreur, après avis du ministre chargé du logement, et au vu des rapports transmis
	par les représentants de l'Etat dans les régions concernées et de l'avis de la commission nationale mentionnée
	au VII. Cette convention détermine les objectifs du programme de logements à réaliser. Elle peut prévoir une
	réalisation des opérations sur une durée totale supérieure à cinq ans. Elle prévoit les modalités permettant un
	contrôle de la réalisation des programmes et de l'application du dispositif de décote prévu au présent article.
	Le prix de cession est déterminé conformément au I. Il fait l'objet d'un versement en deux temps. Au moment de
	la cession, la société mentionnée au premier alinéa du présent V bis verse un acompte correspondant à 40 % de
	la valeur vénale cumulée des actifs du portefeuille. La valeur vénale retenue est la valeur vénale de marché du
	logement libre. Lors de l'obtention des autorisations d'urbanisme, ladite société effectue un second versement
	pour chaque actif sur le fondement du prix définitif arrêté par détermination de la décote prévue au présent
	article, en prenant en compte le programme de logement réalisé sur le bien et les circonstances locales. Si le
	prix définitif d'un actif est inférieur à 40 % de sa valeur vénale retenue dans le calcul de l'acompte, la somme à
	restituer par l'Etat s'impute sur les sommes que la société doit au titre de l'acquisition d'autres actifs du
	portefeuille.
	VI. – Le représentant de l'Etat dans la région, assisté du comité régional de l'habitat, contrôle l'effectivité de
	toute convention annexée à un acte d'aliénation et définie aux V ou V bis du présent article. A cet effet,30
	l'acquéreur des terrains mentionnés au 2\degre  du II rend compte de l'état d'avancement du programme au comité
	régional de l'habitat ainsi qu'à la commune sur le territoire de laquelle se trouve le terrain cédé. Cette
	obligation prend fin au jour de la livraison effective du programme de logements ou au jour de la résiliation de
	la convention.
	En cas de manquements constatés aux engagements pris par un acquéreur dans la convention qui accompagne
	l'acte de cession, le représentant de l'Etat dans la région, assisté du comité régional de l'habitat, mène la
	procédure contradictoire pouvant aboutir à la résolution de la vente dans les conditions prévues au V.
	Le représentant de l'Etat dans la région établit chaque année un bilan qui dresse notamment la liste des terrains
	disponibles, des terrains cédés au cours de l'année écoulée, des modalités et des prix de cession ainsi que des
	logements sociaux mis en chantier ou livrés sur les parcelles cédées. Ce bilan est transmis à la commission
	nationale mentionnée au VII chargée d'établir, pour le compte du ministre chargé du logement, le rapport
	annuel au Parlement sur la mise en œuvre du dispositif, lequel fait l'objet d'un débat devant les commissions
	permanentes.
	VII. – Il est créé, auprès des ministres chargés du logement et de l'urbanisme, une Commission nationale de
	l'aménagement, de l'urbanisme et du foncier. Elle est composée de deux membres de l'Assemblée nationale et de
	deux membres du Sénat, de représentants de l'Etat dont notamment de représentants des ministres chargés du
	logement et de l'urbanisme, de représentants du ministre chargé du Domaine, de représentants des associations
	représentatives des collectivités locales, des organismes mentionnés aux articles L. 365-1, L. 411-2 et L. 481-1
	du code de la construction et de l'habitation, des professionnels de l'immobilier, des professionnels de
	l'aménagement, des organisations de défense de l'environnement et des organisations œuvrant dans le domaine
	de l'insertion, et de personnalités qualifiées.
	La commission nationale mentionnée au présent VII est chargée de suivre le dispositif de mobilisation du foncier
	public en faveur du logement. Elle est en particulier chargée de s'assurer que la stratégie adoptée par l'Etat et
	les établissements publics concernés est de nature à favoriser la cession de biens appartenant à leur domaine
	privé au profit de programmes de logements sociaux. Le décret en Conseil d'Etat prévu au IX précise sa
	composition et fixe ses modalités de travail et de décision.
	VIII. – Pour l'application du présent article, sont assimilés aux logements locatifs mentionnés aux 3\degre  et 5\degre  de
	l'article L. 351-2 du code de la construction et de l'habitation :
	1\degre  Les structures d'hébergement temporaire ou d'urgence bénéficiant d'une aide de l'Etat ;
	2\degre  Les aires permanentes d'accueil des gens du voyage mentionnées au II de l'article 1er de la loi \no  2000-614
	du 5 juillet 2000 relative à l'accueil et à l'habitat des gens du voyage ;
	3\degre  Les logements-foyers dénommés résidences sociales, conventionnés dans les conditions définies au 5\degre  de
	l'article L. 351-2 du code de la construction et de l'habitation, ainsi que les places des centres d'hébergement et
	de réinsertion sociale mentionnées à l'article L. 345-1 du code de l'action sociale et des familles ;
	4\degre  Les résidences de logement pour étudiants, dès lors qu'elles font l'objet d'une convention définie à l'article L.
	353-1 du code de la construction et de l'habitation ;
	5\degre  Les logements en accession à la propriété en Guadeloupe, en Guyane, en Martinique et à La Réunion qui
	bénéficient d'une aide destinée aux personnes physiques à faibles revenus, pour financer l'acquisition de
	logements évolutifs sociaux.
	Outre les logements locatifs sociaux et assimilés mentionnés aux alinéas précédents, sont pris en compte pour le
	calcul de la décote prévue au présent article :
	a) Les logements occupés par des titulaires de contrats de location-accession mentionnés au 6\degre  de l'article L.
	351-2 du même code ;
	b) Les logements faisant l'objet d'une opération d'accession dans les conditions définies au huitième alinéa de
	l'article L. 411-2 dudit code.
	IX. – Un décret en Conseil d'Etat précise les conditions d'application des I à VII.
	démarche adoptée dans les opérations de lotissement de la commune de Chamonix, pour le Clos Napoléon,
	un groupe de travail avait permis de définir quatre principaux critères :
	ü la composition familiale : nécessité d'avoir un ou plusieurs enfants scolarisés dans la commune,
	ü un revenu-plafond fixé à 54 000 \euro /an/ménage (soit 4 500 \euro /mois/ménage en moyenne),
	ü l'âge : = 90 ans d'âges cumulés pour un couple, ou = 45 ans pour une famille monoparentale,
	ü l'activité professionnelle : dix ans d'activité sur la commune.
	A Chamonix, le jury d'attribution194(*) des lots du Clos Napoléon a fini par solliciter les compétences d'un
	huissier afin de réaliser un tirage au sort pour départager les vingt-deux candidats restants pour les douze lots
	disponibles. La démarche semble trouver un large écho auprès des chamoniards, et depuis 2000, une liste
	d'inscription a été ouverte : toute personne qui souhaiterait disposer de ce type de logement peut s'inscrire, et
	bien que les prochains critères d'attribution pour Les Tissières ne soient pas encore définis, plus de cinq cent
	personnes se sont manifestées.31
	Cf Nathalie MOYON : Entre convention alpine, directive territoriale d'amenagement des Alpes du nord et
	initiatives locales, quelles perspectives pour les politiques foncières volontaristes dans les Alpes ? mémoire
	grenoble 2009
	Question écrite \no  01108 JO Sénat du 12/10/2017 - page 3158 : La vente par une commune de terrains
	communaux constructibles, qui relèvent de son domaine privé, doit se faire selon les règles en vigueur. La Cour
	de justice de l'Union européenne (CJUE) a eu à se prononcer sur une problématique similaire au travers de ses
	décisions C-197/11 et C-203/11 du 8 mai 2013 sur l'application d'une disposition du droit belge. À l'occasion de
	ces deux affaires, le juge européen a considéré qu'une disposition qui subordonne l'acquisition de terrains ou
	constructions à la démonstration, par l'acquéreur, d'un lien suffisant avec la commune (domiciliation dans la
	commune, réalisation d'activités au sein de la commune ou un lien professionnel, familial, social ou économique
	en raison d'une circonstance importante ou de longue durée) était contraire au droit de l'Union européenne et
	notamment aux articles 21, 45, 56 et 63 du traité sur le fonctionnement de l'Union européenne dès lors que les
	conditions fixées sont sans rapport direct avec les aspects socio-économiques correspondant à l'objectif de
	protéger exclusivement une certaine catégorie de la population autochtone sur le marché immobilier qui, en
	l'espèce, était la population autochtone la moins fortunée.
	G Le terrain préempté
	Cass 3e civ 22 septembre 2010 \No  09-14817
	Attendu, selon l'arrêt attaqué (Paris, 2 avril 2009) que, par acte sous seing privé du 22 mai 2003, la société
	Trianon gestion a promis de vendre à M. X... qui s'est réservé la faculté d'acquérir, un immeuble pour une durée
	expirant le 24 septembre 2003, sous la condition suspensive du non exercice par leurs titulaires respectifs du
	droit de préemption ; que, par arrêté du 2 septembre 2003, la commune de Villemoisson-sur-Orge a exercé ce
	droit ; que, le 4 novembre 2003, M. X... a saisi la juridiction administrative d'une demande d'annulation pour
	excès de pouvoir de la décision de préemption ; que, par acte authentique du 27 novembre 2003, la société
	Trianon gestion a vendu l'immeuble à la commune qui, par acte authentique du même jour, l'a cédé à la
	Communauté d'agglomération du Val d'Orge ; que, par acte extrajudiciaire des 2, 3 et 4 juin 2004, M. X... a
	assigné la société Trianon gestion, la commune de Villemoisson-sur-Orge et la Communauté d'agglomération du
	Val d'Orge en annulation de ces deux ventes et en paiement de dommages-intérêts ; que, par jugement du 8 juin
	2004, devenu irrévocable, la juridiction administrative a annulé la décision de préemption de la commune ;
	Attendu que M. X... fait grief à l'arrêt de rejeter ses demandes de nullité des ventes du 27 novembre 2003 entre
	la société Trianon gestion et la commune de Villemoisson-sur-Orge et entre cette commune et la Communauté
	d'agglomération du Val d'Orge, alors, selon le moyen :
	...
	Mais attendu qu'ayant relevé que si, par l'effet de l'annulation rétroactive de la décision de préemption, la
	condition suspensive du non exercice du droit de préemption s'était réalisée, M. X... n'avait pas levé l'option, la
	cour d'appel a exactement retenu que la promesse était devenue caduque, de sorte que celui-ci ne disposait
	d'aucun droit à l'annulation de la vente ;
	Cass 3 e civ 10 octobre 2012 \No  de pourvoi: 11-15473
	Attendu, selon l'arrêt attaqué (Paris, 27 janvier 2011), que par acte sous seing privé du 16 janvier 2007, M. X...
	et Mme Y... (les consorts X...-Y...) ont vendu à M. Z... un immeuble d'habitation sous la condition suspensive du
	non-exercice du droit de préemption urbain, la date de signature de l'acte authentique étant fixée au 30 octobre
	2007 ; que par arrêté du 19 mars 2007, le maire de Champigny-sur-Marne a exercé son droit de préemption en
	offrant aux vendeurs d'acquérir l'immeuble pour un prix inférieur au prix convenu, puis, après fixation judiciaire
	du prix, a renoncé à cette acquisition par arrêté du 22 janvier 2008 ; que M. Z... a assigné en vente forcée les
	consorts X...-Y..., qui, à titre reconventionnel, ont demandé la réparation de leurs préjudices ;
	Sur le moyen unique du pourvoi principal :
	Attendu que M. Z... fait grief à l'arrêt de le débouter de sa demande, alors, selon le moyen :
	...
	Mais attendu qu'ayant constaté que le maire de Champigny-sur-Marne avait exercé son droit de préemption par
	l'arrêté du 19 mars 2007, et avait ainsi, dès cette date, évincé l'acquéreur et retenu, par un motif non critiqué,
	que la renonciation ultérieure du maire n'anéantissait pas la décision du 19 mars 2007 et n'avait d'effet que pour
	l'avenir, la cour d'appel, abstraction faite d'un motif surabondant relatif à l'application de l'article L. 213-8 du
	code de l'urbanisme, en a exactement déduit que la défaillance de la condition suspensive était acquise dès le 19
	mars 2007 entraînant la caducité de la promesse de vente ;
	H L’achat de biens du domaine privé
	Cf Y Brousolle Les particularismes de la gestion immobilière du domaine privé des collectivités publiques
	Administrer nov 200932
	1) biens communaux
	- Nécessité d’une délibération du conseil municipal Article L 2241-1 CGCL
	Le conseil municipal délibère sur la gestion des biens et les opérations immobilières effectuées par la commune,
	sous réserve, s'il s'agit de biens appartenant à une section de commune, des dispositions des articles L. 2411-1 à
	L. 2411-19.
	Le bilan des acquisitions et cessions opérées sur le territoire d'une commune de plus de 2 000 habitants par
	celle-ci, ou par une personne publique ou privée agissant dans le cadre d'une convention avec cette commune,
	donne lieu chaque année à une délibération du conseil municipal. Ce bilan est annexé au compte administratif
	de la commune.
	Toute cession d'immeubles ou de droits réels immobiliers par une commune de plus de 2 000 habitants donne
	lieu à délibération motivée du conseil municipal portant sur les conditions de la vente et ses caractéristiques
	essentielles. Le conseil municipal délibère au vue de l'avis de l'autorité compétente de l'Etat. Cet avis est réputé
	donné à l'issue d'un délai d'un mois à compter de la saisine de cette autorité.
	- Possibilité de paiement par rente viagère Article L 2241-4
	Les communes sont, sur proposition des vendeurs, autorisées à acquérir, moyennant le paiement d'une rente
	viagère, les immeubles qui leur sont nécessaires pour des opérations de restauration immobilière,
	d'aménagement ou d'équipement.
	Lorsqu'un immeuble ainsi aliéné est occupé en tout ou partie par le vendeur, le contrat de vente viagère doit
	comporter à son profit et à celui de son conjoint habitant avec lui, à la date de l'acte de vente, la réserve d'un
	droit d'habiter totalement ou partiellement ledit immeuble leur vie durant.
	- Accord du Conseil Municipal pour certains changements d’affectation Article L2241-5
	Les délibérations par lesquelles les commissions administratives chargées de la gestion des établissements
	publics communaux changent en totalité ou en partie l'affectation des locaux ou objets immobiliers ou mobiliers
	appartenant à ces établissements, dans l'intérêt d'un service public ou privé quelconque, ou mettent ces locaux et
	objets à la disposition, soit d'un autre établissement public ou privé, soit d'un particulier, ne sont exécutoires
	qu'après accord du conseil municipal.
	Les délibérations par lesquelles les conseils d'administration des établissements publics communaux
	d'hébergement des personnes âgées se prononcent sur l'affectation des immeubles sont régies par l'article L.
	315-12 du code de l'action sociale et des familles .
	Cf CE 25 sept 2009 Cne de Courtenay JCP N 2009 1327 : Une commune ne peut vendre son bien à un prix
	largement inférieur à l’estimation domaniale
	Cour de cassation chambre civile 3 16 juin 2016 \No  de pourvoi: 15-14906
	Attendu, selon les arrêts attaqués (Toulouse, 30 juin et 17 novembre 2014), que, par convention du 16 décembre
	1989, la commune d'Aulus-les-Bains (la commune) a concédé à la société Ingénierie gestion industrie commerce
	(la société IGIC) la construction des ouvrages nécessaires à la production d'énergie électrique et la gestion et
	l'exploitation de ces ouvrages pour une durée de vingt-neuf ans ; que, par acte authentique du 16 novembre
	2000, la commune a vendu à la société IGIC deux parcelles sur lesquelles étaient partiellement assises les
	installations hydroélectriques ; que, par jugement du 23 juin 2005, le tribunal administratif a autorisé M. X...,
	Mme Y... et Mme Z..., contribuables de la commune, à intenter en justice, sur le fondement de l'article L. 2132-5
	du code général des collectivités territoriales, l'action en nullité de la vente du 16 novembre 2000 que la
	commune refusait d'exercer ; que, la commune, représentée par M. X..., Mme Y... et Mme Z..., a assigné la
	société IGIC en nullité de l'acte de vente du 16 novembre 2000 ;
	Sur le premier moyen :
	Attendu que la société IGIC fait grief à l'arrêt de rejeter les exceptions d'irrecevabilité de la commune, alors,
	selon le moyen :
	Mais attendu que tout contribuable inscrit au rôle de la commune a le droit d'exercer, avec l'autorisation du
	tribunal administratif, les actions qu'il croit appartenir à la commune, et que celle-ci, préalablement appelée à
	en délibérer, a refusé ou négligé d'exercer ; que la cour d'appel, qui a constaté que, par décision du 23 juin
	2005, le tribunal administratif avait autorisé M. X..., Mme Y... et Mme Z... à intenter en justice à leurs frais et
	risques l'action en nullité de la vente du 16 novembre 2000 que la commune refusait d'exercer, a pu en déduire,
	sans dénaturation, que leur action en annulation de l'acte de vente devant le juge judiciaire était recevable ;
	D'où il suit que le moyen n'est pas fondé ;
	Sur les deuxième et troisième moyens, réunis :
	Attendu que la société IGIC fait grief à l'arrêt de prononcer la nullité de l'acte de vente du 16 novembre 2000,
	alors, selon le moyen :
	Mais attendu qu'ayant constaté que, par jugement du tribunal administratif du 8 janvier 2010, la délibération du33
	conseil municipal du 20 février 2000 décidant la vente de parcelles à la société IGIC avait été déclarée nulle et
	de nul effet, la cour d'appel, qui n'était pas tenue de suivre les parties dans le détail de leur argumentation, ni de
	rechercher si la commune n'avait pas été engagée par le maire, la théorie du mandat apparent n'étant pas
	applicable, a pu en déduire que la commune n'avait pas consenti à la vente et que l'acte authentique de vente
	devait être annulé ;
	D'où il suit que le moyen n'est pas fondé ; PAR CES MOTIFS : REJETTE le pourvoi
	%__________________________________________________________________________
	Conseil d'État 8ème - 3ème chambres réunies, 15 mars 2017 \No  393407 vente sans conditions
	1. Il ressort des pièces du dossier soumis aux juges du fond que la SARL Bowling du Hainaut a sollicité de la
	commune de Saint-Amand-les-Eaux l'acquisition des parcelles de son domaine privé cadastrées AI 331, 278 et
	363 pour un prix de 307 755 euros. Par une délibération du 21 décembre 2006, le conseil municipal de Saint-
	Amand-les-Eaux s'est prononcé favorablement sur la cession à la SARL Bowling du Hainaut, ou à toute société
	qui la substituerait, des parcelles indiquées pour le prix indiqué, et a autorisé, en premier lieu, le paiement
	échelonné du prix sur une période de cinq ans, avec un échéancier fixé de 2007 à 2011, en deuxième lieu, le
	maire à signer l'acte de transfert de propriété et, en troisième lieu, la société à déposer le permis de construire
	sur les parcelles concernées avant la signature de l'acte de transfert de propriété. Par une délibération du 30
	juin 2011, le conseil municipal a " annulé " la délibération du 21 décembre 2006 et, par une autre délibération
	du même jour, a décidé de céder les mêmes parcelles à la société Cases Investissements, ou à toute société qui la
	substituerait, au prix de 308 000 euros. La SARL Bowling du Hainaut et la SARL Bowling de Saint-Amand-les-
	Eaux, à laquelle elle s'est substituée, se pourvoient en cassation contre l'arrêt de la cour administrative d'appel
	de Douai confirmant le jugement par lequel le tribunal administratif de Lille a rejeté leur demande tendant à
	l'annulation des deux délibérations du 30 juin 2011.
	2. En premier lieu, il ressort des pièces du dossier soumis aux juges du fond que la délibération du 21 décembre
	2006 du conseil municipal de la commune de Saint-Amand-les-Eaux portait acceptation de l'offre d'acquisition
	des parcelles en litige soumise à la commune par la SARL Bowling du Hainaut au prix proposé et se bornait à
	autoriser le paiement échelonné par la société du prix du terrain cédé sur une durée de cinq ans ainsi que la
	signature par le maire de l'acte de transfert de propriété et toutes les pièces nécessaires à cet acte, sans
	subordonner expressément la réalisation de la vente à la condition tenant au paiement effectif de ce prix dans les
	délais de l'échéancier fixé ou à la signature de cet acte. En jugeant que la délibération du 21 décembre 2006
	subordonnait l'accord de la commune à des conditions, notamment financières, la cour l'a, par suite,
	inexactement interprétée.
	3. En second lieu, aux termes de l'article 1583 du code civil, dans sa rédaction applicable aux faits de l'espèce :
	la vente " est parfaite entre les parties, et la propriété acquise de droit à l'acheteur à l'égard du vendeur, dès
	qu'on est convenu de la chose et du prix, quoique la chose n'ait pas encore été livrée, ni le prix payé ". En vertu
	de l'article 1599 du même code, dans sa rédaction applicable aux faits de l'espèce : " la vente de la chose
	d'autrui est nulle (...) ".
	4. Ainsi qu'il a été dit au point 2, il ressort des termes mêmes de la délibération du 21 décembre 2006 que le
	conseil municipal de Saint-Amand-les-Eaux s'est prononcé favorablement sur l'offre de la SARL Bowling du
	Hainaut tendant à lui acheter les parcelles cadastrées AI 331, 278 et 363 pour un prix de 307 755 euros, sans
	subordonner cet accord à aucune condition. Les parties ayant ainsi marqué leur accord sur l'objet de la vente
	et sur le prix auquel elle devait s'effectuer, la délibération du 21 décembre 2006 a clairement eu pour effet, en
	application des dispositions de l'article 1583 du code civil, de parfaire la vente et de transférer à la société la
	propriété de ces parcelles. Le seul fait que la société n'ait pas honoré les engagements financiers qui lui
	incombaient en conséquence de la délibération du 21 décembre 2006 n'a pu la priver de cette propriété. Par
	suite, en jugeant que cette délibération n'avait créé aucun droit au profit des sociétés requérantes au motif que
	la SARL Bowling du Hainaut, en dépit des dossiers de permis de construire déposés, n'avait jamais consenti à
	verser aucun des acomptes prévus par l'échéancier ou demandé la passation des actes de transfert de propriété,
	de sorte que le conseil municipal pouvait légalement décider de rapporter son accord et de céder les parcelles à
	un tiers, la cour a également commis une erreur de droit.
	5. Il résulte de ce qui précède que l'arrêt attaqué doit être annulé. Il y a lieu, dans les circonstances de l'espèce,
	de régler l'affaire au fond en application des dispositions de l'article L. 821-2 du code de justice administrative.
	6. Il résulte de ce qui a été dit au point 4 que le conseil municipal de Saint-Amand-les-Eaux ne pouvait
	légalement à la date à laquelle il a statué par ses deux délibérations du 30 juin 2011, ni " annuler " la
	délibération du 21 décembre 2006 par laquelle il avait exprimé son accord sur la demande d'acquisition des
	parcelles litigieuses formée par la SARL Bowling du Hainaut au prix que celle-ci proposait ni, par voie de
	conséquence, dès lors que la commune n'en avait plus la propriété, décider de céder ces mêmes parcelles à la
	société Cases Investissements. Dès lors, et sans qu'il soit besoin d'examiner les autres moyens de la requête, la
	SARL Bowling du Hainaut et la SARL Bowling de Saint-Amand-les-Eaux sont fondées à soutenir que c'est à tort
	que le tribunal administratif de Lille a rejeté leurs conclusions tendant à l'annulation des deux délibérations du34
	30 juin 2011 du conseil municipal de Saint-Amand-les-Eaux.
	2) Département : art L 3213-2 ; Région L 4221-4,
	3) Etat Article L3211-1 CGPPP Lorsqu'ils ne sont plus utilisés par un service civil ou militaire de l'Etat ou un
	établissement public de l'Etat, les immeubles du domaine privé de l'Etat peuvent être vendus dans les conditions
	fixées par décret en Conseil d'Etat.
	Lorsque la cession de ces immeubles implique l'application des mesures prévues à l'article L. 541-2 du code de
	l'environnement ou l'élimination des pollutions pyrotechniques, l'Etat peut subordonner la cession à l'exécution,
	dans le cadre de la réglementation applicable, par l'acquéreur, de ces mesures ou de ces travaux, le coût de la
	dépollution s'imputant sur le prix de vente. Dans cette hypothèse, le coût de la dépollution peut être fixé par un
	organisme expert indépendant choisi d'un commun accord par l'Etat et l'acquéreur.
	Article L3211-6 Les immeubles bâtis et non bâtis qui font partie du domaine privé de l'Etat peuvent être cédés
	à l'amiable en vue de la réalisation d'opérations d'aménagement ou de construction et pour les cessions
	réalisées dans les conditions prévues à l'article L. 3211-7 lorsqu'elles comptent plus de 50 % de logements
	sociaux, dans les conditions fixées par décret en Conseil d'Etat. Ce décret fixe notamment les règles
	applicables à l'utilisation des biens cédés.
	Article L3211-7 voir supra
	§2 – L’achat du droit de superficie ou d’un volume
	Historique ; Développements récents.
	A Les problèmes théoriques et leurs solutions
	a)
	Exposé des deux thèses sur la superficie.
	• Thèse nominaliste : Mazeaud, Saint-Alary, Article superficie à l’Encyclopédie Dalloz)
	- droit de superficie : renonciation au droit d’accession
	- droit de nature personnelle, puis réelle, puis, en cas de démolition, personnelle
	- pas de droit réel sur une res nullius, sur un cube d’air.
	• Thèse réaliste : Gilli 1975, Savatier, la Propriété de l’espace D 65 chr 215, Congrès des
	notaires 1976, Chambelland Gringas Halloche, Les ensembles immobiliers complexes
	Defrénois 1975 art 30 993 p 1217.
	- droits réels dès l’origine
	- fondement : droit réel impossible en dehors des cas limitatifs :
	distinction bien/ chose / droit
	2) Intérêt de la discussion
	a.
	b.
	c.
	Sûreté
	Prescription du droit personnel par usucapion ou non
	Possibilité de reconstruction en cas de destruction
	3) Eléments de solution
	a. pourquoi faire une distinction entre cube d’air et cube de terre
	b. expropriation du tréfonds (droit réel sur un tréfonds inoccupé)
	c. jurisprudence Cass. 6 mars 1991 III \no  84 p 50
	d. Cass 3e civ, 26 mai 2010 \No  de pourvoi: 09-66541
	Attendu qu'ayant retenu que par l'acte du 27 mars 2004 les parties avaient constaté
	que la société civile immobilière Les Terrasses de River Park avait uniquement
	construit et achevé le bâtiment de la première tranche, que la résolution portait sur
	les volumes formant l'assiette juridique et foncière des deuxième, troisième,
	quatrième et cinquième tranches du programme ainsi que les volumes formant
	l'assiette juridique et foncière des équipements communs, l'acte précisant que tous les
	travaux et équipements communs de l'ensemble immobilier dont dépendaient les biens
	objet de la résolution, non réalisés à ce jour, étaient à la charge de la société
	d'équipements de la région mulhousienne (SERM), la cour d'appel qui, sans
	dénaturation, a déduit de ces stipulations contractuelles que les travaux de tous les35
	lots, à l'exclusion de ceux du lot AA achevé étaient à la charge de la SERM a
	légalement justifié sa décision ;
	B La concrétisation pratique moderne : la vente du volume
	1) - délimitation du volume
	- imbrication des volumes
	- Sous volumes exclus ; la division d’un volume aboutit à la disparition du volume initial et à la création de deux
	nouveaux volumes.
	2)Validation des volumes
	Loi Du 10 juillet 1965 Article 1 (différé) Modifié par Ordonnance \no 2019-1101 du 30 octobre 2019 - art. 2
	I.-La présente loi régit tout immeuble bâti ou groupe d'immeubles bâtis à usage total ou partiel d'habitation dont
	la propriété est répartie par lots entre plusieurs personnes.
	Le lot de copropriété comporte obligatoirement une partie privative et une quote-part de parties communes,
	lesquelles sont indissociables.
	Ce lot peut être un lot transitoire. Il est alors formé d'une partie privative constituée d'un droit de construire
	précisément défini quant aux constructions qu'il permet de réaliser et d'une quote-part de parties communes
	correspondante.
	La création et la consistance du lot transitoire sont stipulées dans le règlement de copropriété.
	II.-A défaut de convention y dérogeant expressément et mettant en place une organisation dotée de la
	personnalité morale et suffisamment structurée pour assurer la gestion de leurs éléments et services communs,
	la présente loi est également applicable :
	1\degre  A tout immeuble ou groupe d'immeubles bâtis à destination totale autre que d'habitation dont la propriété est
	répartie par lots entre plusieurs personnes ;
	2\degre  A tout ensemble immobilier qui, outre des terrains, des volumes, des aménagements et des services communs,
	comporte des parcelles ou des volumes, bâtis ou non, faisant l'objet de droits de propriété privatifs.
	Pour les immeubles, groupes d'immeubles et ensembles immobiliers mentionnés aux deux alinéas ci-dessus et
	déjà régis par la présente loi, la convention mentionnée au premier alinéa du présent II est adoptée par
	l'assemblée générale à l'unanimité des voix de tous les copropriétaires composant le syndicat.
	Article 28 (différé Modifié par Ordonnance \no 2019-1101 du 30 octobre 2019 - art. 32
	I.-Lorsque l'immeuble comporte plusieurs bâtiments et que la division de la propriété du sol est possible :
	a) Le propriétaire d'un ou de plusieurs lots correspondant à un ou plusieurs bâtiments peut demander que ce ou
	ces bâtiments soient retirés du syndicat initial pour constituer une propriété séparée. L'assemblée générale
	statue sur la demande formulée par ce propriétaire à la majorité des voix de tous les copropriétaires ;
	b) Les propriétaires dont les lots correspondent à un ou plusieurs bâtiments peuvent, réunis en assemblée
	spéciale et statuant à la majorité des voix de tous les copropriétaires composant cette assemblée, demander que
	ce ou ces bâtiments soient retirés du syndicat initial pour constituer un ou plusieurs syndicats séparés.
	L'assemblée générale du syndicat initial statue à la majorité des voix de tous les copropriétaires sur la demande
	formulée par l'assemblée spéciale.
	II.-Dans les deux cas, l'assemblée générale du syndicat initial statue à la même majorité sur les conditions
	matérielles, juridiques et financières nécessitées par la division.
	L'assemblée générale du ou des nouveaux syndicats, sauf en ce qui concerne la destination de l'immeuble,
	procède, à la majorité de l'article 24, aux adaptations du règlement initial de copropriété et de l'état de
	répartition des charges rendues nécessaires par la division.
	La répartition des créances et des dettes est effectuée selon les principes suivants :
	1\degre  Les créances du syndicat initial sur les copropriétaires anciens et actuels et les hypothèques du syndicat
	initial sur les lots des copropriétaires sont transférées de plein droit aux syndicats issus de la division auquel le
	lot est rattaché, en application de l'article 1346 du code civil ;
	2\degre  Les dettes du syndicat initial sont réparties entre les syndicats issus de la division à hauteur du montant des
	créances du syndicat initial sur les copropriétaires transférées aux syndicats issus de la division.
	III.-Si l'assemblée générale du syndicat initial décide de constituer une union de syndicats pour la création, la
	gestion et l'entretien des éléments d'équipements communs qui ne peuvent être divisés, cette décision est prise à
	la majorité de l'article 24.
	Le règlement de copropriété du syndicat initial reste applicable jusqu'à l'établissement d'un nouveau règlement
	de copropriété du syndicat ou de chacun des syndicats selon le cas.
	La division ne prend effet que lorsque sont prises les décisions mentionnées aux alinéas précédents. Elle emporte
	la dissolution du syndicat initial.36
	IV.- La procédure prévue au présent article peut également être employée pour la division en volumes d'un
	ensemble immobilier complexe comportant soit plusieurs bâtiments distincts sur dalle, soit plusieurs entités
	homogènes affectées à des usages différents, pour autant que chacune de ces entités permette une gestion
	autonome.
	La procédure ne peut en aucun cas être employée pour la division en volumes d'un bâtiment unique.
	En cas de division en volumes, la décision de constituer une union de syndicats pour la création, la gestion et
	l'entretien des éléments d'équipements à usage collectif est prise à la majorité mentionnée à l'article 25.
	Par dérogation au troisième alinéa de l'article 29, les statuts de l'union peuvent interdire à ses membres de se
	retirer de celle-ci.
	Cass 3 e civ 18 janvier 2012 \No  de pourvoi: 10-27396
	Mais attendu qu'ayant retenu à bon droit, d'une part, que l'état descriptif proposé par M. B..., en ce qu'il divisait
	l'immeuble en considération de deux régimes de propriété qui s'y appliquent, se bornait à constater une situation
	juridique existante pour la transposer, avec exactitude, sur un support juridique publiable à la conservation des
	hypothèques en application de l'article 71 du décret \no  55-1350 du 14 octobre 1955 et, d'autre part, que la
	destination de la fraction indivise de l'immeuble, qualifiée de passage commun, impliquait nécessairement, au
	profit de chacun des propriétaires indivis, un droit de passage, la cour d'appel, répondant aux conclusions, et
	abstraction faite de motifs surabondants en a exactement déduit que l'état descriptif de division en volumes ne
	restreignait ni ne modifiait la consistance des droits réels des consorts Z...- A..., et n'avait pas pour effet de
	modifier le régime juridique de la fraction indivise de l'immeuble ;
	Cass 3 e civ 19 septembre 2012 \No  de pourvoi: 11-13679 11-13789
	Vu l'article 1er, alinéa 2, de la loi du 10 juillet 1965 ;
	Attendu qu'à défaut de convention contraire créant une organisation différente, la loi est applicable aux
	ensembles immobiliers qui, outre des terrains, des aménagements et des services communs, comportent des
	parcelles, bâties ou non, faisant l'objet de droits de propriété privatifs ;
	Attendu, selon l'arrêt attaqué (Paris, 10 novembre 2010), que la Ville de Paris a, par acte du 17 août 1988,
	consenti à la société civile immobilière Habitat Ramponeau (la SCI) un bail emphytéotique pour une durée de 55
	ans, à charge pour cette dernière d'y édifier un bâtiment et de le remettre en fin de location à la Ville de Paris ;
	qu'un état descriptif de division a été établi par acte notarié du 31 mai 1990 qui divise l'immeuble en 24 lots de
	volumes, dont certains ont été placés sous le régime de la copropriété selon un règlement de copropriété du 3
	juillet 1990 ; que par acte authentique des 9 et 14 mars 1995, la SCI a fait apport à l'association Or Thora
	éducation juive du 20e arrondissement (l'association) des droits qu'elle détenait du bail emphytéotique sur les
	locaux constituant le volume 4 ; que la SCI et le syndicat des copropriétaires 38-40 rue Ramponeau (le syndicat)
	ont assigné l'association en payement de certaines sommes représentant la quote part des charges générales de
	l'ensemble immobilier incombant au lot \no  4 ;
	Attendu que, pour débouter la SCI et le syndicat de leur demande, l'arrêt, qui constate que le lot \no  4 ne fait pas
	partie des lots de volumes soumis au règlement de copropriété du 3 juillet 1990, relève que l'état descriptif de
	division stipule que l'ensemble immobilier ne sera pas régi par la loi du 10 juillet 1965 et qu'à cette fin, l'acte
	identifie des volumes immobiliers de pleine propriété dans le cadre du régime du droit de superficie, et énonce
	l'ensemble des servitudes issues de l'imbrication de ces volumes qui permettent leur coexistence ainsi que
	l'attribution 3026/10.000èmes des charges générales au lot \no  4, retient que l'état descriptif de division
	constitue, relativement à ce lot, la convention contraire visée à l'article 1er, alinéa 2, de la loi du 10 juillet 1965;
	Qu'en statuant ainsi, sans constater la création d'une organisation différente, au sens de la loi, pour la gestion
	des éléments communs de l'ensemble immobilier, la cour d'appel a violé le texte susvisé ;
	Article 28 IV de la loi du 10 juillet 1965
	IV.- Après avis du maire de la commune de situation de l'immeuble et autorisation du représentant de l'Etat dans
	le département, la procédure prévue au présent article peut également être employée pour la division en volumes
	d'un ensemble immobilier complexe comportant soit plusieurs bâtiments distincts sur dalle, soit plusieurs entités
	homogènes affectées à des usages différents, pour autant que chacune de ces entités permette une gestion
	autonome. Si le représentant de l'Etat dans le département ne se prononce pas dans les deux mois, son avis est
	réputé favorable.
	La procédure ne peut en aucun cas être employée pour la division en volumes d'un bâtiment unique.
	En cas de division en volumes, la décision de constituer une union de syndicats pour la création, la gestion et
	l'entretien des éléments d'équipements à usage collectif est prise à la majorité mentionnée à l'article 25.
	Par dérogation au troisième alinéa de l'article 29, les statuts de l'union peuvent interdire à ses membres de se
	retirer de celle-ci.
	3) volume et urbanisme
	- SPC37
	- DIA : cf formulaire annexé à l’article A 213-1 : création d’une case vente en lots de volumes
	- Lotissement
	Cf CAA Paris 7 juillet 2005 (\no  01PA00808, Ville de Paris ). Dans cette affaire, la Ville de Paris demandait à la
	Cour d’annuler un jugement du Tribunal Administratif de Paris qui avait annulé lui-même l’arrêté accordant un
	permis de construire à une société Immobilière au motif de l’absence d’autorisation de lotir lors d’une division en
	volumes. Elle obtient gain de cause:
	« Considérant qu'aux termes de l'article R. 315-2 du code de l'urbanisme : Ne constituent pas des lotissements et
	ne sont pas soumises aux dispositions du présent chapitre : / ... d) Les divisions par ventes ou locations
	effectuées par un propriétaire au profit de personnes qu'il a habilitées à réaliser une opération immobilière sur
	une partie de sa propriété et qui ont elles-mêmes déjà obtenu une autorisation de lotir ou un permis de
	construire portant sur la création d'un groupe de bâtiments ou d'un bâtiment comportant plusieurs logements ...;
	Considérant que par acte du 12 décembre 1996, le syndicat des transports parisiens a vendu à la société
	d'aménagement Denfert-Montsouris, filiale de la régie autonome des transports parisiens (RATP), un lot de
	volumes non utiles à l'exploitation ferroviaire de la parcelle de 21 078 m2 dont il était alors seul propriétaire et la
	RATP affectataire, 71 à 83 boulevard Saint-Jacques, 3 place Denfert-Rochereau, 1 à 9 avenue René Coty et 2 rue
	Hallé à Paris (14ème) ; que la société d'aménagement Denfert-Montsouris a, par divers actes notariés, divisé à
	son tour en jouissance le volume ainsi acquis pour céder à trois sociétés différentes, dont la société
	IMMOBILIÈRE 3F, des droits de construire à certains emplacements, assortis du droit de superficie et du droit
	de propriété exclusif des immeubles bâtis, y compris celui de les détruire et reconstruire sans limitation de durée
	; qu'il ressort des pièces du dossier que l'acte de vente avec la société IMMOBILIÈRE 3F a été signé les 21 et 24
	décembre 1998 ; qu'auparavant, les sociétés bénéficiaires desdits lots avaient obtenu chacune un permis de
	construire, délivrés respectivement le 10 septembre 1996 pour les deux premières et le 8 septembre 1997 pour la
	société IMMOBILIÈRE 3F ; qu'ainsi, la division des lots étant intervenue après la délivrance des permis de
	construire susmentionnés, celle-ci ne présentait pas le caractère d'un lotissement ; que, par suite, le permis de
	construire litigieux délivré à la société IMMOBILIÈRE 3F n'avait pas à être précédé de l'octroi d'une autorisation
	de lotir ; que c'est dès lors à tort que le Tribunal administratif de Paris s'est fondé sur le motif qu'une autorisation
	de lotir aurait dû être délivrée préalablement au permis de construire accordé à la société IMMOBILIÈRE 3F
	pour annuler ce permis de construire »
	CE 30 nov 2007 \No  271897 Ville de Strasbourg
	Considérant qu'il ressort des pièces du dossier soumis aux juges du fond que la VILLE DE STRASBOURG a
	délivré à la société civile immobilière C+K, les 18 novembre 1997 et 19 janvier 1998, deux permis de construire
	autorisant la construction de deux immeubles sur une parcelle sur laquelle deux droits de superficie perpétuels
	avaient été cédés à cette société civile immobilière par la société d'économie mixte de la région de Strasbourg ;
	que le tribunal administratif de Strasbourg a rejeté, par deux jugements du 25 janvier 2000, les requêtes par
	lesquelles M. et Mme A demandaient l'annulation de ces permis de construire ; que la cour administrative d'appel
	de Nancy, a, par un arrêt du 24 juin 2004, annulé les jugements et les permis attaqués ; que la VILLE DE
	STRASBOURG se pourvoit en cassation contre cet arrêt ; Sans qu'il soit besoin d'examiner l'autre moyen du
	pourvoi ; Considérant qu'aux termes de l'article R. 315-1 du code de l'urbanisme alors en vigueur : Constitue un
	lotissement au sens du présent chapitre toute division d'une propriété foncière en vue de l'implantation de
	bâtiments qui a pour objet ou qui, sur une période de moins de dix ans, a eu pour effet de porter à plus de deux le
	nombre de terrains issus de ladite propriété (...) ; qu'il résulte de ces dispositions que l'édification sur une parcelle
	de plusieurs constructions ne peut être regardée comme constitutive d'un lotissement que si la parcelle servant
	d'assiette aux constructions a été divisée en jouissance ou en propriété ; Considérant que, par suite, en jugeant,
	pour annuler les jugements du tribunal administratif de Strasbourg du 25 janvier 2000 et les permis litigieux, qu'il
	ressortait des pièces du dossier qui lui était soumis, et notamment des clauses de l'acte de vente aux termes
	desquelles chaque lot comporte la pleine propriété des volumes et chaque propriétaire de lot sera propriétaire des
	constructions, que l'acte de vente avait pour objet de procéder à une division en lots, alors que la division de ces
	lots entre les futurs copropriétaires des immeubles collectifs autorisés par les permis de construire litigieux
	n'emporte pour eux ni propriété ni jouissance exclusive et particulière du sol d'assiette de la parcelle, la cour
	administrative d'appel a entaché son arrêt d'une erreur de qualification juridique ; que son arrêt doit être annulé
	pour ce motif ; Considérant qu'il y a lieu, dans les circonstances de l'espèce, de faire application des dispositions
	de l'article L. 821-2 du code de justice administrative et de juger l'affaire au fond ; Considérant qu'ainsi qu'il a
	été dit, en l'absence de toute division, en jouissance ou en propriété, de la parcelle servant d'assiette aux
	constructions autorisées par les deux permis de construire litigieux, l'édification de ces constructions sur cette
	parcelle ne saurait être regardée comme constitutive d'un lotissement ;
	4) volume et servitudes
	- nécessité de prévoir quelques servitudes d’appui38
	- Les servitudes créatrices de volumes
	Article L2113-du code des transports Créé par la LOI \no  2015-992 du 17 août 2015 - art. 52 (V)
	Le maître d'ouvrage d'une infrastructure souterraine de transport public ferroviaire ou guidé déclarée d'utilité
	publique, ou la personne agissant pour son compte, peut demander à tout moment à l'autorité administrative
	compétente d'établir une servitude d'utilité publique en tréfonds.
	La servitude en tréfonds confère à son bénéficiaire le droit d'occuper le volume en sous-sol nécessaire à
	l'établissement, à l'aménagement, à l'exploitation et à l'entretien de l'infrastructure souterraine de transport.
	Elle oblige les propriétaires et les titulaires de droits réels concernés à s'abstenir de tout fait de nature à nuire
	au bon fonctionnement, à l'entretien et à la conservation de l'ouvrage.
	La servitude en tréfonds ne peut être établie qu'à partir de quinze mètres au-dessous du point le plus bas du
	terrain naturel, sous réserve du caractère supportable de la gêne occasionnée.
	La servitude est établie, par décision de l'autorité administrative compétente, dans les conditions fixées aux
	articles L. 2113-2 à L. 2113-5.
	Décret d’application \no  2015-1572 du 2 décembre 2015 relatif à l'établissement d'une servitude d'utilité publique
	en tréfonds
	-
	Modification du code des transports par l’ordonnance \no  2015-1495 du 18 novembre 2015
	Art. L. 1251-3. - La déclaration de projet ou la déclaration d'utilité publique d'une infrastructure de transport
	par câbles en milieu urbain relevant de l'article L. 2000-1 confère aux autorités mentionnées à l'article L. 1231-
	1 et à l'article L. 1241-1 le droit à l'établissement par l'autorité administrative compétente de l'Etat de servitudes
	d'utilité publique de libre survol, de passage et d'implantation de dispositifs de faible ampleur indispensables à
	la sécurité du système de transport par câbles, sur des propriétés privées ou faisant partie du domaine privé
	d'une collectivité publique, bâties ou non bâties, fermées ou non fermées de murs ou clôtures équivalentes.
	« Le point le plus bas du survol ne peut être situé à moins de dix mètres des propriétés survolées.
	« Art. L. 1251-4. - La servitude de libre survol confère à son bénéficiaire le droit d'occuper le volume aérien
	nécessaire à l'exploitation, l'entretien et la sécurité de l'ouvrage.
	« La servitude de passage confère à son bénéficiaire le droit :
	« - d'accéder, à titre exceptionnel, aux propriétés privées survolées lorsque aucun autre moyen pour réaliser
	l'installation, l'entretien et l'exploitation ne peut être envisagé ;
	« - d'établir les cheminements nécessaires aux opérations d'évacuation et d'entretien des infrastructures.
	« Les servitudes obligent les propriétaires et les titulaires de droits réels concernés à s'abstenir de tout fait de
	nature à nuire au bon fonctionnement, à l'entretien et à la conservation de l'ouvrage.
	Décret \no  2015-1581 du 3 décembre 2015 relatif à l'instauration de servitudes d'utilité publique pour le transport
	par câbles en milieu urbain
	5) Volume et domaine public -> renvoi.
	6) Utilisation marginale du droit de superficie
	Le droit de superficie n’est plus utilisé que pour régulariser les constructions sur le terrain d’autrui cf J Lafond,
	Construction sur terrain d’autrui et publicité foncière JCP N 2005 I 1449
	Cf également A Fournier Division en volume et pté Foncière JCP ed N 2006 I 1330
	Section 2 – L’acquisition d’un droit réel conféré par un bail.
	- bail emphytéotique art L 451-1 et s. du code rural
	Article L451-1 : Le bail emphytéotique de biens immeubles confère au preneur un droit réel susceptible d'hypothèque ;
	ce droit peut être cédé et saisi dans les formes prescrites pour la saisie immobilière.
	Ce bail doit être consenti pour plus de dix-huit années et ne peut dépasser quatre-vingt-dix-neuf ans ; il ne peut se
	prolonger par tacite reconduction.39
	Article L451-2 : Le bail emphytéotique ne peut être valablement consenti que par ceux qui ont le droit d'aliéner, et sous
	les mêmes conditions, comme dans les mêmes formes.
	Les immeubles appartenant à des mineurs ou à des majeurs sous tutelle peuvent être donnés à bail emphytéotique en
	vertu d'une délibération du conseil de famille.
	Lorsque les époux restent soumis au régime dotal, le mari peut donner à bail emphytéotique les immeubles dotaux
	avec le consentement de la femme et l'autorisation de justice.
	Article L451-3 : La preuve du contrat d'emphytéose s'établit conformément aux règles du code civil en matière de baux.
	A défaut de conventions contraires, il est régi par les dispositions suivantes.
	Article L451-4 : Le preneur ne peut demander la réduction de la redevance pour cause de perte partielle du fonds, ni
	pour cause de stérilité ou de privation de toute récolte à la suite de cas fortuits.
	Article L451-5 : A défaut de paiement de deux années consécutives, le bailleur est autorisé, après une sommation restée
	sans effet, à faire prononcer en justice la résolution de l'emphytéose.
	La résolution peut également être demandée par le bailleur en cas d'inexécution des conditions du contrat ou si le
	preneur a commis sur le fonds des détériorations graves.
	Néanmoins, les tribunaux peuvent accorder un délai suivant les circonstances.
	Article L451-6 : Le preneur ne peut se libérer de la redevance, ni se soustraire à l'exécution des conditions du bail
	emphytéotique en délaissant le fonds.
	Article L451-7 : Le preneur ne peut opérer dans le fonds aucun changement qui en diminue la valeur.
	Si le preneur fait des améliorations ou des constructions qui augmentent la valeur du fonds, il ne peut les détruire, ni
	réclamer à cet égard aucune indemnité.
	Article L451-8 : Le preneur est tenu de toutes les contributions et charges de l'héritage.
	En ce qui concerne les constructions existant au moment du bail et celles qui auront été élevées en exécution de la
	convention, il est tenu des réparations de toute nature, mais il n'est pas obligé de reconstruire les bâtiments, s'il prouve
	qu'ils ont été détruits par cas fortuit, par force majeure ou qu'ils ont péri par le vice de la construction antérieure au bail.
	Il répond de l'incendie, conformément à l'article 1733 du code civil.
	Article L451-9 : L'emphytéote peut acquérir au profit du fonds des servitudes actives, et les grever, par titres, de
	servitudes passives, pour un temps qui n'excédera pas la durée du bail à charge d'avertir le propriétaire.
	Article L451-10 : L'emphytéote profite du droit d'accession pendant la durée de l'emphytéose.
	Article L451-11 : Le preneur a seul le droit de chasse et de pêche et exerce à l'égard des mines, carrières et tourbières
	tous les droits de l'usufruitier.
	Article L451-12 : Les articles L. 451-1 et L. 451-9 sont applicables aux emphytéoses établies avant le 25 juin 1902 si le
	contrat ne contient pas de stipulations contraires.
	Article L451-13 : Ainsi qu'il est dit à l'article 689 du code général des impôts, l'acte constitutif de l'emphytéose est assujetti
	à la taxe de publicité foncière et aux droits d'enregistrement aux taux prévus pour les baux à ferme ou à loyer d'une
	durée limitée.
	Les mutations de toute nature ayant pour objet soit le droit du bailleur, soit le droit du preneur, sont soumises aux
	dispositions du code général des impôts concernant les transmissions de propriété d'immeubles. Le droit est liquidé sur
	la valeur vénale déterminée par une déclaration estimative des parties.
	Article L451-14 : Les articles L. 451-1 à L. 451-12 sont applicables à Mayotte
	- bail à construction art L 251-1 et CCH
	- bail à réhabilitation art L 252-1 et s. du CCH. 121 baux à réhabilitation conclus en 1996.
	- concession immobilière, loi du 30 déc. 1967 exclue car acquisition du droit de construire secondaire. En fait
	dérogation au bail commercial pas utilisée.
	- Bail réel immobilier Article 4 de l' Ordonnance \no  2014-159 du 20 février 2014 relative au logement
	intermédiaire , art L 254-1 et S CCH
	- Bail réel solidaire : Ordonnance \no 2016-985 du 20 juillet 2016 : figure aux art L 255-1 et s du CCH
	§1 – Eléments communs40
	A Maintien de la propriété du sol au bailleur
	Cass 3 e civ, 16 mai 2001AJDI 2001 719 qui parle de droit de construire du preneur
	Attendu qu'après avoir exactement relevé que le bail à construction conférait au preneur, pour la durée du bail,
	un droit réel immobilier consistant en la disposition du sol mais que ce droit était limité au projet de
	construction contractuellement fixé par les parties, puis, ce droit de construire étant consommé, en la propriété
	des constructions ainsi édifiées, également pendant la seule durée du bail, avec possibilité de cession ou de
	location, le terrain restant à tout moment la propriété du bailleur, la cour d'appel, qui retient, sans dénaturer les
	écritures de la société Intercoop, que cette société ne formulait aucune demande chiffrée pour perte du droit de
	construire et qui n'était pas tenue de procéder à des recherches qui ne lui étaient pas demandées, a, par ces
	seuls motifs, légalement justifié sa décision ;
	B Droit réel
	- L 451-1 Code Rural, Le bail emphytéotique de biens immeubles confère au preneur un droit réel susceptible
	d'hypothèque ; ce droit peut être cédé et saisi dans les formes prescrites pour la saisie immobilière.
	, L 251-3 : Le bail à construction confère au preneur un droit réel immobilier.
	Ce droit peut être hypothéqué, de même que les constructions édifiées sur le terrain loué ; il peut être saisi dans
	les formes prescrites pour la saisie immobilière.
	L 252-2 : Le preneur est titulaire d'un droit réel immobilier. Ce droit peut être hypothéqué ; il peut être saisi
	dans les formes prescrites pour la saisie immobilière.
	L 254-1 : Constitue un contrat dénommé " bail réel immobilier ” le bail par lequel un propriétaire personne
	physique ou personne morale de droit privé, dans le périmètre mentionné à l'article L. 302-16, consent, pour une
	longue durée, à un preneur, avec obligation de construire ou de réhabiliter des constructions existantes, des
	droits réels en vue de la location ou de l'accession temporaire à la propriété de logements
	L 255-1 : Constitue un contrat dénommé “bail réel solidaire” le bail par lequel un organisme de foncier
	solidaire consent à un preneur, dans les conditions prévues à l'article L. 329-1 du code de l'urbanisme et pour
	une durée comprise entre dix-huit et quatre-vingt-dix-neuf ans, des droits réels en vue de la location ou de
	l'accession à la propriété de logements, avec s'il y a lieu obligation pour ce dernier de construire ou réhabiliter
	des constructions existantes.
	1) Possibilité de saisie et d’hypothèque du droit L 251-3, L 252-2, L 451-1, L 254-3, L 255-9
	2) - Possibilité d’attribution préférentielle du droit Cour de cassation chambre civile 1 12 juin 2013 \No 12-11724:
	Vu les articles 831-2, 1\degre , et 1476 du code civil ;
	Attendu que, selon ces textes, en cas de dissolution de la communauté par divorce, un époux peut demander
	l'attribution préférentielle de la propriété ou du droit au bail du local qui lui sert effectivement d'habitation s'il y
	avait sa résidence ;
	Attendu, selon l'arrêt attaqué, que M. X... et Mme Y... se sont mariés sans contrat préalable le 23 janvier 1971 ;
	que par acte du 18 mai 1982 leur a été consenti un bail emphytéotique sur une villa qui a été le logement de la
	famille et qui sera attribuée à l'épouse pendant la procédure de divorce introduite par assignation du 12
	novembre 2007 ; que celle-ci a sollicité l'attribution préférentielle du droit à ce bail par application de l'article
	1751 du code civil ;
	Attendu que, pour déclarer irrecevable cette demande, l'arrêt retient que ce texte n'est pas applicable,
	l'emphytéose étant régie par les articles L. 451-1 et suivants du code rural et de la pêche et la jurisprudence
	étant venue à plusieurs reprises rappeler la spécificité de ce type de bail auquel les règles qui régissent le louage
	ordinaire n'ont jamais été applicables ;
	Attendu qu'en statuant ainsi la cour d'appel a violé par refus d'application les textes susvisés ;
	3)- Droit réel ayant une valeur propre :Cass. 3 ème Civ 17 juillet 1997 Bull III \no  169 (3).
	Attendu, selon l'arrêt attaqué (Rouen, 20 décembre 1994), que la société civile immobilière Lemoine (la SCI) a,
	en 1984, consenti à la société Inter-Coop un bail à construction sur un terrain lui appartenant situé sur le
	territoire de la commune de Bois-Guillaume ; que la société Inter-Coop a édifié sur ce terrain plusieurs
	bâtiments à usage commercial ; qu'en 1989, une partie de ces bâtiments a été détruite par un incendie ; que le
	juge de l'expropriation a, en 1991, transféré au profit de la commune de Bois-Guillaume la propriété du terrain
	et, en 1993, a prononcé l'expropriation du bail à construction ; que la commune ayant demandé au juge de
	l'expropriation de fixer l'indemnité due à la société Inter-Coop du chef de l'expropriation du bail à construction
	sans avoir saisi ce juge d'une demande en fixation de l'indemnité relative à l'expropriation du terrain donné à
	bail, cette société a assigné la SCI Lemoine en intervention forcée afin de lui rendre opposable l'évaluation du
	prix de ce terrain ;
	Sur le premier moyen du pourvoi principal : (sans intérêt) ;
	Mais sur le second moyen du pourvoi principal :41
	Vu l'article L. 13-13 du Code de l'expropriation, ensemble l'article L. 251-3 du Code de la construction et de
	l'habitation ;
	Attendu que, pour fixer à une certaine somme le montant de l'indemnité due par la commune de Bois-Guillaume
	à la société Inter-Coop, à la suite de l'expropriation du bail à construction, l'arrêt retient que la valeur du droit
	réel immobilier conféré au preneur par ce bail correspond à la valeur de l'utilisation du terrain faisant l'objet du
	bail, c'est-à-dire à la valeur des constructions édifiées sur ce terrain ;
	Qu'en statuant ainsi, sans rechercher si ce droit n'avait pas une valeur propre distincte de la seule valeur des
	constructions édifiées sur le terrain en vertu de ce bail, la cour d'appel n'a pas donné de base légale à sa
	décision de ce chef ;
	- Toutefois, en cas de destruction par incendie et non reconstruction, le preneur ne peut, en cas d’expropriation,
	être indemnisé que pour les bâtiments édifiés par lui et qui existent à la date de référence (Cass 3 e civ, 16 mai
	2001 précité)
	- l’emphytéote est titulaire d’un droit réel immobilier qui peut fonder un crédit bail au sens de l’article 313-7 du
	code monétaire et financier (CA Paris 31 oct 2002 CU 2003 71)
	4) Mais droit réel qui n’est pas un droit de propriété et n’est donc pas soumis au droit de préemption
	(Cass 3 e civ, 11 mai 2000 Bull III \no 108 p 73) ()
	Mais attendu, d'une part, qu'ayant relevé que les deux sociétés souhaitaient réaliser concomitamment la vente de
	leurs droits portant l'une sur un immeuble, l'autre sur le bail à construction consenti par la première sur une
	partie de l'immeuble, la cour d'appel, qui n'a pas statué par des motifs dubitatifs, a retenu, à bon droit, que le
	droit de préemption de la commune ne pouvait concerner le bail à construction dont la société CIEN était
	titulaire et que la transmission simultanée du terrain et du droit au bail ne pouvait y faire échec et en a
	exactement déduit qu'il ne pouvait être fait grief au notaire rédacteur de ne pas avoir fait apparaître la
	solidarité voulue par les deux venderesses dès lors que cette condition était inopposable à la commune pour
	l'application de son droit de préemption ;
	Attendu, d'autre part, que seules les conditions notifiées au titulaire du droit de préemption lui étant opposables,
	le moyen qui soutient que le but recherché par les sociétés venderesses pouvait être atteint en stipulant dans la
	promesse de vente des immeubles à l'acquéreur une condition suspensive de non-exercice du droit de préemption
	par la commune est sans portée ;
	Toutefois fraude en cas de bail emphytéotique de 99 ans avec pacte de préférence alors que les intentions de la
	commune étaient connues (Cass 3e civ 27 nov 2002 CU 2003 78 : Mais attendu qu'ayant relevé que la SNCF ET
	M. X... avaient conclu le bail après avoir pris conscience de ce que l'échange des parcelles nécessaires au projet
	de la ligne "TGV" avec les terrains litigieux se heurtait à l'obligation d'une déclaration d'aliéner, laquelle faisait
	jouer le droit de préemption prioritaire de la commune dont les intentions étaient connues de la SNCF et
	affichées pour avoir été soumises à l'approbation du conseil municipal, en 1987 ; que l'intérêt bien compris des
	parties passait par le sacrifice des droits de cette commune et permettait à la SNCF de différer la notification de
	cette déclaration, dans le but de mener à bien son projet et que la signature d'un bail emphytéotique d'une durée
	maximale, assorti d'un "pacte de préférence", assimilable, dans ce contexte, à une vente déguisée, démontrait la
	fraude, la cour d'appel qui n'a pas fondé sa décision sur les dispositions de l'article 1167 du Code civil, a, par
	ces seuls motifs, légalement justifié sa décision ;
	5) Droit réel entrainant une division du sol en jouissance : Cass 3 e civ 17 nov 2004 Bull 2004 III \No  203 p.
	183, RD Imm 2005 23: Mais attendu, d'une part, qu'ayant exactement relevé qu'en vertu de l'article R. 315-1 du
	Code de l'urbanisme, constitue un lotissement une division de propriété foncière en vue de l'implantation de
	bâtiments ayant pour objet de porter à plus de deux le nombre de terrains issus de cette division, que cette
	définition s'applique notamment aux divisions de propriété ou de jouissance résultant de mutations à titre gratuit
	ou onéreux, de partage ou de location et qu'en ce qui concerne les divisions s'opérant par voie de location, sont
	visés aussi bien les locations de terrains que les baux à construction, les baux emphytéotiques dès lors que la
	location est consentie en vue de l'implantation de bâtiments, l'arrêt retient à bon droit que les preneurs à bail,
	bénéficiaires de la jouissance d'un terrain consenti en vue de l'implantation de bâtiments, ont la qualité de
	colotis, tenus dès lors au respect du cahier des charges à l'égard de la commune bailleresse, et sont fondés à se
	prévaloir entre eux des stipulations contractuelles du cahier des charges auxquelles ils ont adhéré ;
	6)Mais droit réel qui n’est pas un droit permettant de contester l’expropriation du terrain
	Cass civ 3 30 janvier 2008 , Bulletin 2008, III, \No  19
	Vu les articles L. 12-1 et L. 12-2 du code de l'expropriation pour cause d'utilité publique ;
	Attendu que par ordonnance du 23 juin 2006, le juge de l'expropriation du département du Tarn a transféré à la
	commune de Mazamet la propriété de parcelles appartenant à " l'indivision Y... " ; que M.X..., titulaire d'un bail
	emphytéotique sur ces parcelles, a formé un pourvoi en cassation contre cette ordonnance ;42
	Attendu que seuls les propriétaires, ou les titulaires d'un droit réel lorsque l'expropriation porte uniquement sur
	ce droit, ayant qualité pour former un pourvoi en cassation contre une ordonnance portant transfert de
	propriété, le pourvoi n'est pas recevable ;
	PAR CES MOTIFS : DECLARE IRRECEVABLE le pourvoi ;
	C Longue durée du bail
	- L 451-1 : Ce bail doit être consenti pour plus de dix-huit années et ne peut dépasser quatre-vingt-dix-neuf
	ans ; il ne peut se prolonger par tacite reconduction.
	- L 252-1 : Le bail à réhabilitation est consenti par ceux qui ont le droit d'aliéner et dans les mêmes conditions
	et formes que l'aliénation. Il est conclu pour une durée minimale de douze ans. Il ne peut se prolonger par tacite
	reconduction.
	- L 251-1 Il est conclu pour une durée comprise entre dix-huit et quatre-vingt-dix-neuf ans. Il ne peut se
	prolonger par tacite reconduction .
	- L 254-1 « Le bail réel immobilier est consenti pour une durée de dix-huit à quatre-vingt-dix-neuf ans par les
	personnes qui ont le droit d'aliéner
	- L 255-1 : « pour une durée comprise entre dix-huit et quatre-vingt-dix-neuf ans »
	- donc nécessité de la capacité de disposer
	1) Interdiction de principe d’une clause de résiliation de plein droit
	- Impossibilité d’une clause de résiliation triennale : TA Dijon, 9 mars 1999 AJDI 1999 917) () Com 23 fev
	1999 AJDI 1999 917
	- La précarité résultant d’une clause de résiliation de plein droit est incompatible avec le bail
	emphytéotique (Cass 3 civ 14 nov 2002 Bull III \no 223) : Attendu que le bail emphytéotique de biens immeubles
	confère au preneur un droit réel susceptible d'hypothèque ; qu'à défaut de paiement de deux années
	consécutives, le bailleur est autorisé, après une sommation restée sans effet, à faire prononcer en justice la
	résolution de l'emphytéose ;
	Attendu, selon l'arrêt attaqué (Toulouse, 31 mai 2001), que, par actes des 27 décembre 1949 et 10 mars 1952,
	M. X..., aux droits duquel viennent les consorts Y..., a consenti à la société Laitière du Midi, aux droits de
	laquelle vient la société Groupe Lactalis, deux baux relatifs à deux ensembles de bâtiments contigus, l'un pour
	une durée de 98 ans et 5 jours, l'autre pour 95 ans et 291 jours ; qu'en décembre 1997, la locataire a donné
	congé pour le 1er juillet 1998, en visant les dispositions du statut des baux commerciaux ; que, par acte du 21
	juillet 1998, les consorts Y... l'ont assignée afin qu'il soit jugé que le bail avait un caractère emphytéotique et
	pour qu'elle soit condamnée à leur payer une certaine somme au titre des loyers restant à courir, et une autre à
	titre de travaux ;
	Attendu que, pour qualifier les contrats liant les parties de baux emphytéotiques, l'arrêt retient que le caractère
	emphytéotique ne peut être écarté par l'effet de la clause de résiliation, en faveur du bailleur en cas de non-
	paiement du loyer, la précarité imposée à l'emphytéote ayant son origine dans son propre fait ;
	Qu'en statuant ainsi, alors qu'une clause de résolution de plein droit confère à la jouissance du preneur une
	précarité incompatible avec la constitution d'un droit réel, la cour d'appel a violé les textes susvisés ;
	Solution différente en matière de bail à construction : Cass 3e civ 28 janvier 2004, Bull III \no  13 p 11 ; 5 novembre
	2008 \no  07-18174 : « Mais attendu qu'ayant retenu, par motifs propres et adoptés, que le " bail " du 16 décembre
	1971 prévoyait que le terrain loué ne pourrait être utilisé que pour l'édification d'un snack-bar de montagne, et,
	en son article 9, sanctionnait l'inexécution de cette obligation par la résiliation de plein droit du contrat si la
	construction n'était pas achevée dans les trois ans à compter du permis de construire, et constaté que le preneur
	bénéficiait d'une durée de trente ans et que les parties avaient fait publier cet acte à la conservation des
	hypothèques, la cour d'appel, qui a pu en déduire, abstraction faite d'un motif erroné mais surabondant, qu'il y
	avait lieu de qualifier le bail en cause de bail à construction, a légalement justifié sa décision de ce chef ; » ; 1 er
	juin 2011 \no  09-70502, Bull III \no  89).
	2) Modalités de reconduction du bail
	- Pas de tacite reconduction cf art L 251-1, L 252-1, L 254-1, L 255-6 du CCH et L 451-1 du C rural;
	D Transfert de toutes les charges, taxes et contributions relatifs aux terrains et aux constructions
	-L 451-8 : Le preneur est tenu de toutes les contributions et charges de l'héritage.
	-L 251-4 : Le preneur est tenu de toutes les charges, taxes et impôts relatifs tant aux constructions qu'au terrain.43
	1400 II C.G.I. : » II.-Lorsqu'un immeuble est grevé d'usufruit ou loué soit par bail emphytéotique, soit par bail à
	construction, soit par bail à réhabilitation ou fait l'objet d'une autorisation d'occupation temporaire du domaine
	public constitutive d'un droit réel, la taxe foncière est établie au nom de l'usufruitier, de l'emphytéote, du
	preneur à bail à construction ou à réhabilitation ou du titulaire de l'autorisation.
	-L 254-2 : Le preneur est tenu de toutes les charges, taxes et impôts relatifs tant à l'immeuble donné à bail
	qu'aux constructions réalisées
	Rien de prévu pour le bail réel solidaire
	Cf en matière de taxes d’urbanisme : CE 23 déc 2011 \no  313306 « Considérant qu'aux termes des premier et
	troisième alinéas de l'article L. 251-1 du code de la construction et de l'habitation : Constitue un bail à
	construction le bail par lequel le preneur s'engage, à titre principal, à édifier des constructions sur le terrain du
	bailleur et à les conserver en bon état d'entretien pendant toute la durée du bail. / (...) / Il est conclu pour une
	durée comprise entre dix-huit et quatre-vingt-dix-neuf ans (...) ; qu'aux termes de l'article L. 251-2 du même
	code : Les parties conviennent de leurs droits respectifs de propriété sur les constructions existantes et sur les
	constructions édifiées. A défaut d'une telle convention, le bailleur en devient propriétaire en fin de bail et profite
	des améliorations ; qu'aux termes du premier alinéa de l'article L. 251-3 du même code : Le bail à construction
	confère au preneur un droit réel immobilier ; qu'il résulte de ces dispositions que, pendant la durée du bail, le
	preneur est, sauf stipulation contraire, propriétaire des constructions qu'il édifie et bénéficie d'un droit réel
	immobilier sur le terrain du bailleur ; qu'il suit de là que, lorsque le permis de construire a été délivré au
	bailleur, le preneur du bail à construction doit être regardé comme un ayant cause du titulaire de l'autorisation
	de construire au sens des dispositions précitées des articles 1723 decies et 1929 du code général des impôts ;
	Considérant qu'il résulte de ce qui précède qu'en jugeant que le preneur d'un bail à construction ne figurait pas
	parmi les personnes tenues solidairement avec le titulaire du permis de construire, en vertu des dispositions
	précitées, au paiement du versement pour dépassement du plafond légal de densité et de la taxe locale
	d'équipement, le tribunal administratif de Paris a commis une erreur de droit ; »
	Considérant, en troisième lieu, que si la société soutient que les dispositions de l'article L. 251-4 du code de la
	construction et de l'habitation, en vertu desquelles le preneur d'un bail à construction est tenu de toutes les
	charges, taxes et impôts relatifs tant aux constructions qu'au terrain, régissent les seules relations entre les
	parties, ce moyen ne peut qu'être écarté dès lors que la responsabilité solidaire de l'ayant cause du bénéficiaire
	de l'autorisation de construire à l'égard de ce dernier pour le paiement du versement pour dépassement du
	plafond légal de densité et de la taxe locale d'équipement découle directement des dispositions précitées des
	articles 1723 decies et 1929 du code général des impôts ;
	E Le droit du preneur est cessible
	1) - Emphytéose :L 451-1 et Req. 8 mars 1861 S 61.1.713 et Cass. 3 ème Civ 10 avril 1991 Bull III \no  114 p 65 ; 7
	avril 2004 Bull III \no  67 ; 29 avril 2009 \no  08-10944 pour dire que le bail conclu le 23 juin 1958 est un bail
	emphytéotique, l'arrêt retient que s'il est de principe que le preneur jouisse d'un libre droit de cession de ses
	droits, la disposition du contrat qui semble limiter ce droit par la nécessité d'un accord du bailleur, aussitôt
	corrigée par celle selon laquelle cette autorisation n'est pas requise dès lors que le cessionnaire est un
	"successeur dans l'exploitation commerciale", ne permet aucune limitation effective de ce droit ;
	Qu'en statuant ainsi, alors qu'elle avait constaté que le bail comportait une clause limitant la cession, la cour
	d'appel, qui n'a pas tiré les conséquences légales de ses propres constatations, a violé le texte susvisé ;
	Cass 3 e civ 16 juin 2011 \No  de pourvoi: 10-17169 : Mais attendu qu'ayant relevé que la durée et la clause de
	cession du bail étaient contraires aux dispositions légales relatives au bail emphytéotique
	2) - Bail à construction : L 251-3 : Le preneur peut céder tout ou partie de ses droits ou les apporter en société. Les
	cessionnaires ou la société sont tenus des mêmes obligations que le cédant qui en reste garant jusqu'à
	l'achèvement de l'ensemble des constructions que le preneur s'est engagé à édifier en application de l'article L.
	251-1.(ce texte est d’ordre public en vertu de L 251-8)
	a)- Elément essentiel : Cour de cassation chambre civile 3 24 septembre 2014 \No  de pourvoi: 13-22357
	Attendu que la société Immobilière Carrefour fait grief à l'arrêt de la débouter de sa demande en résiliation
	judiciaire du bail, alors, selon le moyen, que la clause d'agrément, stipulée dans un bail à construction,
	subordonnant la cession du contrat à l'accord du bailleur, est valable, car elle se borne à restreindre le droit du
	preneur à céder son bail, sans lui interdire de le faire ; qu'en annulant la clause d'agrément figurant dans le bail
	à construction liant la société Immobilière Carrefour à la SCI du Centre commercial de Stains, sous prétexte
	qu'elle restreignait la liberté de cession du preneur titulaire d'un droit réel immobilier, la cour d'appel a violé
	les articles L. 251-3 et L. 251-8 du code de la construction et de l'habitation ;
	Mais attendu que la cour d'appel a retenu, à bon droit, que, le bail à construction conférant au preneur un droit44
	réel immobilier, la clause soumettant la cession à l'agrément du bailleur, qui constitue une restriction au droit
	de céder du preneur contraire à la liberté de cession, est nulle et de nul effet ;
	b)- En cas de cession partielle, le cessionnaire est tenu aux mêmes obligations que le cédant :
	Cass 3 e civ 1 er mars 2006 \no  04-18327 : Attendu, selon l'arrêt attaqué (Basse-Terre, 7 juin 2004), rendu en
	matière de référé, qu'en 1977, les époux B..., aux droits desquels viennent les consorts B..., ont consenti à la
	société Omnium Tourisme Antilles un bail à construction sur un terrain leur appartenant, avec possibilité de
	rétrocéder à diverses personnes des quote-parts de ce bail, ce que cette société a fait, au profit, notamment, de
	M. X... ; ...
	que faisant état de diverses irrégularités les consorts B... ont assigné la société Omnium Tourisme Antilles et les
	cessionnaires, aux fins, notamment, d'obtenir justification de la souscription d'une assurance et de la notification
	des cessions, et de remise en état du toit d'un bâtiment, le tout sous astreinte ;
	Mais attendu que par application de l'article L. 251-3 du Code de la construction et de l'habitation le preneur
	d'un bail à construction peut céder tout ou partie de ses droits et les cessionnaires sont tenus des mêmes
	obligations que le cédant ; qu'ayant relevé que M. X..., tiers au contrat de bail à construction initial, était devenu
	cessionnaire dans le contrat de cession de quote-parts de ce bail liant les consorts B... à la société Omnium
	Tourisme Antilles, la cour d'appel, qui ne s'est pas déterminée par référence à des pièces qui n'auraient pas été
	communiquées, et qui a souverainement constaté l'inutilité d'appeler à l'instance le syndicat des copropriétaires
	de l'immeuble, compte tenu de l'objet du litige, en a exactement déduit, répondant aux conclusions, que M. X...
	était tenu des mêmes obligations que le preneur, notamment en matière d'assurance, et de notification des
	cessions intervenues ;
	c)- Non application à la cession de l’article L 251-6 : Cass 3 e civ, 28 juin 2000 Bull III \no 129 p 88 : Attendu,
	selon l'arrêt attaqué (Basse-Terre, 20 avril 1998), que les consorts Magras ayant consenti, jusqu'au 31
	décembre 2015, un bail à construction à la société Omnium Tourisme Antilles (OTA), celle-ci a, suivant un acte
	du 31 décembre 1992, cédé partiellement son droit à la société Meyronne ; que, le 15 janvier 1993, la société
	Meyronne a emprunté une somme à la Banque des Antilles françaises (BDAF) garantie par une hypothèque sur
	les biens acquis selon la convention de cession partielle du bail à construction ; que le solde du prix de cession
	n'ayant pas été réglé, la société OTA a délivré un commandement de payer à la société Meyronne ; que la
	société Meyronne a formé opposition à ce commandement ; que, par un précédent arrêt du 2 décembre 1996, la
	cour d'appel de Basse-Terre a débouté la société Meyronne et constaté la résolution de la cession du bail à
	construction ; que la BDAF ayant entre-temps cédé sa créance hypothécaire à la société Saint-Barth
	investissements management (SBIM), ces sociétés ont formé tierce opposition à l'encontre de l'arrêt du 2
	décembre 1996 ;
	Attendu que les sociétés SBIM, BDAF et Meyronne font grief à l'arrêt de rejeter la tierce opposition alors, selon
	le moyen, 1\degre  que si le bail à construction prend fin par résiliation judiciaire ou amiable, les privilèges et
	hypothèques nés du chef du preneur et inscrits avant la publication de la demande en justice tendant à obtenir
	cette résiliation ou avant la publication de l'acte ou de la convention la constatant ne s'éteignent qu'à la date
	primitivement convenue pour l'expiration du bail ; que ces dispositions étaient applicables à l'hypothèque
	inscrite le 16 mars 1993 par la BDAF en garantie du prêt par elle consenti à la société Meyronne, preneur du
	bail à construction, et transmise avec la créance au titre du prêt à la SBIM ; que la cour d'appel, en statuant
	comme elle l'a fait, a violé l'article L. 251-6 du Code de la construction et de l'habitation ; 2\degre  que les juges ne
	peuvent interpréter les dispositions légales que si celles-ci sont obscures ou ambiguës ; que les dispositions de
	l'article L. 251-6 du Code de la construction et de l'habitation, claires et précises, s'imposaient à la cour d'appel
	sans que celle-ci puisse avancer une interprétation restrictive venant en limiter la portée ; qu'ainsi, en statuant
	comme elle l'a fait, la cour d'appel a encore violé le texte susvisé ; 3\degre  qu'en matière civile seuls les textes de
	procédure ou relatifs aux mesures disciplinaires sont d'interprétation stricte ; qu'à supposer même que les
	dispositions de l'article L. 251-6 du Code de la construction et de l'habitation puissent nécessiter une
	interprétation, la cour d'appel ne pouvait en proposer une interprétation stricte s'agissant d'une loi de fond ;
	qu'elle a donc derechef violé ce texte ;
	Mais attendu que la cour d'appel a retenu, à bon droit, que les dispositions de l'article L. 251-6 du Code de la
	construction et de l'habitation ne devaient pas recevoir application au-delà de leurs propres prévisions,
	lesquelles ne visent que la résiliation du bail et ne concernaient pas une opération de cession totale ou partielle
	de bail à construction, qui se réalise entre cédant et cessionnaire indépendamment du bailleur ;
	3)Bail à réhabilitation
	-L 252-2 : Ce droit est cessible nonobstant toute convention contraire. La cession ne peut être consentie qu'à
	l'un des organismes mentionnés au premier alinéa de l'article L. 252-1, avec l'accord du bailleur. Le droit ne45
	peut être cédé que s'il porte sur la totalité de l'immeuble loué. Le cédant demeure garant de l'exécution du bail
	par le cessionnaire.
	4) Bail réel immobilier
	L 254-2 : Le preneur peut céder son propre droit au bail, céder son droit de propriété temporaire sur les
	constructions édifiées et librement consentir des baux sur l'immeuble, dans les conditions prévues aux articles L.
	254-4 à L. 254-6.
	5)Bail réel solidaire
	Règles particulières avec agrément obligatoire
	F Renonciation à l’accession pendant la durée du bail
	-L 451-10 : L'emphytéote profite du droit d'accession pendant la durée de l'emphytéose.
	-L 251-2 qui n’est pas d’ordre public : Les parties conviennent de leurs droits respectifs de propriété sur les
	constructions existantes et sur les constructions édifiées. A défaut d'une telle convention, le bailleur en devient
	propriétaire en fin de bail et profite des améliorations.
	-L 252-1 implicite... En fin de bail, les améliorations réalisées bénéficient au bailleur sans indemnisation.
	-L 254-2 : Le preneur peut jouir librement des constructions de l'immeuble et des installations ou constructions
	qu'il a édifiées ou rénovées, dès lors qu'il n'est pas porté atteinte à la destination de l'immeuble telle qu'elle est
	mentionnée à l'article L. 254-1, ou à l'état dans lequel il a été convenu que ces constructions seraient remises en
	fin de bail.
	- art L 255-7 : Les constructions et améliorations réalisées par le preneur demeurent sa propriété en cours de
	bail et deviennent la propriété de l'organisme de foncier solidaire à l'expiration du bail.
	Le preneur peut jouir librement des droits réels immobiliers et des installations ou constructions qu'il a édifiées,
	rénovées ou réhabilitées, dès lors qu'il n'est pas porté atteinte à la destination de l'immeuble, ou à l'état dans
	lequel il a été convenu que ces droits réels immobiliers seraient remis en fin de bail.
	-Conséquences : gestion des constructions et améliorations pour le preneur.
	- La résiliation du bail à construction avant terme ne le transforme pas, d’un point de vue fiscal, en bail
	ordinaire Cass. 3 ème Civ 24 juin 1997 Bull 20 : Attendu que la SCI Agnel-Teissonnière reproche au jugement
	d'avoir rejeté sa demande alors, selon le pourvoi, d'une part, que l'article 738.1\degre  du Code général des impôts
	dispose que les résiliations de " baux à durée limitée de biens de toute nature " sont enregistrées au droit fixe de
	430 francs, et que ce texte énumère limitativement les exceptions au principe qu'il pose, exceptions qui ne
	comprennent pas les baux à construction qui sont donc au nombre des " baux à durée limitée de biens de toute
	nature ", d'où il suit qu'en statuant comme il a fait le Tribunal a violé ladite disposition ; alors, d'autre part, que
	si l'article 1378 ter du Code général des impôts soumet aux dispositions fiscales applicables aux mutations
	d'immeubles les mutations de toute nature ayant pour objet, en matière de bail à construction, les droits du
	preneur comme ceux du bailleur, cet article ne peut viser que les cessions desdits droits à des tiers et n'est pas
	susceptible de s'appliquer aux hypothèses de résiliation anticipée des baux à construction par lesquelles le
	bailleur accède à la propriété des constructions, de sorte qu'en statuant comme il a fait le Tribunal a violé ladite
	disposition par fausse interprétation ; et alors, enfin, que la résiliation du bail à construction, quelle qu'en soit
	la cause, entraîne acquisition au profit du bailleur de la propriété des biens construits par accession, mode
	d'acquisition originaire de la propriété, dont le bail à construction ne fait que retarder la réalisation ; qu'en
	décidant que la résiliation du bail intervenue le 20 juin 1990 avait entraîné un transfert de propriété des
	immeubles construits, mode d'acquisition dérivé, pour la contraindre à verser des droits de mutation prévus
	pour la mutation d'immeubles, le Tribunal a violé les articles 554, 555, 711 et 712 du Code civil ensemble
	l'article 683 du Code général des impôts ;
	Mais attendu qu'ayant relevé que dans le bail à construction le preneur bénéficie sur le terrain d'un droit réel
	immobilier et sur les constructions d'un droit de propriété temporaire et le bailleur, propriétaire du sol, devient,
	sauf convention contraire, propriétaire des constructions en fin de bail, le jugement en déduit, à bon droit, que
	par l'effet de la résiliation le preneur perd ce droit de propriété temporaire et permet au bailleur d'accéder à la
	propriété des immeubles construits avant l'expiration du bail, de sorte qu'entraînant transfert de propriété des
	immeubles la résiliation constitue une mutation soumise aux droits d'enregistrement des mutations d'immeubles ;
	que le moyen n'est fondé en aucune de ses trois branches ;
	.
	- Les constructions ne rentrent fiscalement dans le patrimoine du bailleur qu’à l’expiration du bail : TA
	Dijon, 3 fev 1998 AJDI 1999 25
	G Possibilité de rémunérer le bailleur par la remise des constructions ou améliorations46
	1) Bail à construction cf L 251-5 « Le prix du bail peut consister, en tout ou partie, dans la remise au bailleur,
	à des dates et dans des conditions convenues, d'immeubles ou de fractions d'immeubles ou de titres donnant
	vocation à la propriété ou à la jouissance de tels immeubles.».
	- élément fondamental cf Cass. 1 ère Civ 19 déc 1995 JCP 1997 II N. p 95 obs B. Stemmer :
	Attendu, selon l'arrêt attaqué, que le Syndicat intercommunal à vocation unique (SIVU) de Nistos, a été créé en
	1988, afin de mettre en place les installations nécessaires à la création d'un stade de ski de fond, dans le cadre
	d'une Unité de séjour touristique (UST) destinée à développer le tourisme dans la région ; que, par acte
	authentique du 8 mars 1989, le SIVU a conclu, avec le Syndicat des montagnes et forêts de Nistos (le syndicat)
	un bail à construction d'une durée de 25 ans portant sur une parcelle de 2 000 m2 comprise dans le domaine à
	vocation forestière et pastorale géré par celui-ci et destinée à l'édification d'un gîte d'étape de 410 m2,
	moyennant un loyer annuel d'un franc ; que la valeur résiduelle de la construction était fixée à 500 000 francs
	en fin de bail ; que le syndicat s'engageait, à l'expiration du bail, à ne pas changer l'affectation des locaux et à
	en garantir l'utilisation au gestionnaire du stade de ski au moyen d'une nouvelle convention à passer avec celui-
	ci ; que le syndicat a assigné le SIVU en nullité de ce bail devant le tribunal de grande instance de Tarbes ;
	...
	Mais sur le second moyen :
	Vu l'article 1131 du Code civil ;
	Attendu que, pour annuler le bail à construction, conclu entre le SIVU et le syndicat, l'arrêt attaqué énonce que
	la contrepartie de ce bail est, pour le bailleur, un droit de propriété indéfiniment dévalorisé, dans la mesure où,
	à l'expiration du bail, il deviendra propriétaire d'une construction sans détenir les pouvoirs de jouissance et de
	disposition qui caractérisent le droit de propriété, puisqu'il devra, de nouveau, louer les lieux au gestionnaire en
	place ;
	Qu'en statuant ainsi, alors qu'il résultait des constatations des juges du fond que le syndicat, qui avait mis à la
	disposition du SIVU un terrain sans grande valeur vénale obtenait en fin de bail la propriété des bâtiments
	réalisés, ce qui constituait une contrepartie sérieuse, dans la mesure où rien n'interdisait au bailleur d'exiger du
	preneur un loyer substantiel, lors de la conclusion d'un bail ultérieur, la cour d'appel n'a pas légalement justifié
	sa décision au regard du texte susvisé ;
	-remise des constructions libres de droits des tiers, notamment de propriété commerciale
	Cass civ 3 14 novembre 2007 Bulletin 2007, III, \No  204
	Attendu, selon l'arrêt attaqué (Pau, 26 juin 2006), que la société Hôtel Sunset, aux droits de laquelle est venue la
	société civile immobilière Alpha (la SCI), titulaire d'un bail à construction, a consenti, sur un immeuble compris
	dans ce bail, à la société Belsa le renouvellement, par actes des 30 décembre 1992 et 6 mars 1993, d'un bail
	commercial jusqu'au 25 avril 2002, terme du bail à construction ; qu'en fin de bail, la SCI et la société civile
	foncière Motels, bailleresse à construction, ont poursuivi, par la voie du référé, l'expulsion de la société Belsa ;
	que cette dernière les a assignées pour voir dire qu'elle bénéficiait de la propriété commerciale et obtenir une
	indemnité d'éviction ;
	Attendu que la société Belsa fait grief à l'arrêt de la débouter de sa demande de dommages-intérêts, alors, selon
	le moyen ...
	Mais attendu qu'ayant exactement retenu que le bail commercial renouvelé par l'acte des 30 décembre 1992 et 6
	mars 1993 portant sur un immeuble compris dans un bail à construction, se trouvait révoqué par l'effet de la loi
	à la date d'expiration de ce bail à construction, et relevé que la société Belsa, en acceptant les stipulations de
	l'acte selon lesquelles l'expiration du bail commercial coïnciderait avec la fin du bail à construction, avait
	reconnu que son titre d'occupation ne lui donnait pas droit au renouvellement ni, partant, au paiement d'une
	indemnité d'éviction, la cour d'appel a légalement justifié sa décision ;
	PAR CES MOTIFS :REJETTE le pourvoi ;
	2)Emphytéose : L 451-7 : Le preneur ne peut opérer dans le fonds aucun changement qui en diminue la valeur.
	Si le preneur fait des améliorations ou des constructions qui augmentent la valeur du fonds, il ne peut les
	détruire, ni réclamer à cet égard aucune indemnité
	- mais Cass. 3 ème Civ 12 oct. 1994 Bull III \no  175 : Mais attendu qu'ayant répondu aux conclusions en relevant
	que, malgré les dispositions de l'article 861 du Code rural applicables à la date de la convention, les parties
	pouvaient choisir de conclure un bail emphytéotique et que si ce terme n'était pas indiqué dans le contrat, il
	appartenait au juge, à défaut d'indication contraire, de rechercher la commune intention des parties, la cour
	d'appel a légalement justifié sa décision en retenant que la convention, consentie pour une durée de 30 ans,
	stipulait la libre cession du droit au bail, que les améliorations culturales entreprises, sans nécessité
	d'autorisation du bailleur, resteraient de plein droit la propriété de celui-ci, sans droit à indemnité pour la
	société et en constatant le caractère modeste du loyer ; 5 dec 2001 Bull III \no  142 p 111 : Mais attendu qu'ayant47
	retenu que la redevance était modique par rapport à la valeur des plantations et améliorations qui, en fin de
	bail, devaient être attribuées au bailleur et que les parties avaient entendu conférer au preneur un droit réel sur
	l'immeuble loué, la cour d'appel a pu en déduire, sans dénaturation, que le contrat du 12 janvier 1967 devait
	être qualifié de bail emphytéotique ;
	3) Bail à réhabilitation : L 252-1 : En fin de bail, les améliorations réalisées bénéficient au bailleur sans
	indemnisation.
	4) bail réel immobilier : L 254-1 : Les constructions et améliorations réalisées par le preneur demeurent sa
	propriété en cours de bail et deviennent la propriété du bailleur à son expiration. A l'issue du bail, le preneur ne
	peut réclamer, sauf stipulations contraires, d'indemnité au regard des améliorations qu'il a effectuées
	H Obligation de conserver les constructions en bon état d’entretien.
	-L 251-4 : Il est tenu du maintien des constructions en bon état d'entretien et des réparations de toute nature. Il
	n'est pas obligé de reconstruire les bâtiments s'ils ont péri par cas fortuit ou force majeure ou, s'agissant des
	bâtiments existant au moment de la passation du bail, par un vice de construction antérieur audit bail. Il répond
	de l'incendie des bâtiments existants et de ceux qu'il a édifiés.
	-L 451-8 : En ce qui concerne les constructions existant au moment du bail et celles qui auront été élevées en
	exécution de la convention, il est tenu des réparations de toute nature, mais il n'est pas obligé de reconstruire les
	bâtiments, s'il prouve qu'ils ont été détruits par cas fortuit, par force majeure ou qu'ils ont péri par le vice de la
	construction antérieure au bail. Il répond de l'incendie, conformément à l'article 1733 du code civil.
	- L 252-1 : et à le conserver en bon état d'entretien et de réparations de toute nature en vue de louer cet
	immeuble à usage d'habitation pendant la durée du bail.
	-L 254-2 : Le preneur ne peut effectuer aucun changement qui diminue la valeur de l'immeuble et ne peut, sauf
	stipulation contraire, démolir en vue de les reconstruire les ouvrages existants ou qu'il a édifiés. Il doit
	maintenir en bon état d'entretien et de réparations les constructions existant lors de la conclusion du bail et
	celles qu'il réalise pendant la durée de celui-ci. Il est tenu des réparations de toute nature en ce qui concerne les
	constructions existant au moment du bail et celles qui auront été élevées, mais il n'est pas obligé de les
	reconstruire s'il prouve qu'elles ont été détruites par cas fortuit, force majeure, ou qu'elles ont péri par le vice de
	la construction antérieure au bail .
	- L 255-7 : « Le preneur doit maintenir en bon état d'entretien et de réparations les constructions existantes
	lors de la conclusion du bail et celles qu'il réalise pendant la durée de celui-ci. Il est tenu des réparations de
	toute nature en ce qui concerne les constructions existantes au moment du bail et celles qui auront été édifiées,
	mais il n'est pas obligé de les reconstruire s'il prouve qu'elles ont été détruites par cas fortuit, force majeure, ou
	qu'elles ont péri par le vice de la construction antérieure au bail. En cas de sinistre entraînant résiliation du
	bail, le preneur est indemnisé de la valeur de ses droits réels, dans les conditions prévues par le bail. »
	1)-Modalités d’application de la théorie des risques en la matière
	Cf Cass 3e civ 17 déc 2003 RD Imm 2005 p 22 : Attendu qu'ayant, par motifs propres et adoptés, relevé que le
	bail à construction mettait expressément à la charge du preneur, pendant tout le cours du bail, l'obligation de
	"conserver en bon état d'entretien les constructions édifiées et tous aménagements qu'il aura apportés et
	effectuer à ses frais les réparations de toute nature, y compris les grosses réparations, ainsi que le remplacement
	de tous éléments de la construction et de son aménagement, au fur et à mesure que le tout se révélera
	nécessaire" et constaté que les ouvrages de charpente corrodés par l'humidité normale de l'air ambiant
	n'avaient pas fait l'objet d'une réfection régulière faisant partie de l'entretien courant et que les éléments de
	couverture devenus impropres auraient dû être remplacés en partie ou en totalité depuis le moment où les
	premiers désordres importants avaient été constatés et que les derniers travaux réalisés par la société civile
	immobilière La Vigne aux loups (SCI) pour mettre le bâtiment hors d'eau, approximatifs et non conformes aux
	normes, documents techniques unifiés et règles de l'art, devaient être considérés comme étant des mesures
	conservatoires provisoires, la cour d'appel a pu retenir, sans trancher une contestation sérieuse, que la SCI,
	dont les obligations contractuelles étaient clairement définies dans la convention, devait être condamnée à payer
	une provision correspondant au coût des travaux de réparation et d'entretien nécessaires déterminés par
	l'expertise ;
	D'où il suit que le moyen n'est pas fondé ;
	- En conséquence résiliation du bail aux torts du preneur si celui-ci démolit les constructions édifiées avant la fin
	du bail (CA Rouen 24 mai 2005 Constr Urb 2005 240 obs Cornille , le preneur ayant démoli pour respecter la
	législation CDEC après avoir obtenu le droit de transférer son activité).
	2)-mais possibilité d’une clause exigeant la remise du terrain nu au bailleur48
	Cass civ 3 , 30 janvier 2008 \No  de pourvoi: 06-21292
	Attendu, selon l'arrêt attaqué (Saint-Denis,4 septembre 2006), que, par acte sous seing privé du 30 mai 2001, les
	époux X... ont donné à bail à la Société touristique d'hôtellerie et de casino de la Réunion (la société STHCR)
	une parcelle de terrain nu pour une durée de dix-huit ans afin que le preneur y construise et y installe un poste
	de transformation d'électricité ; que les bailleurs ayant assigné la société en paiement des loyers, celle-ci a
	reconventionnellement soutenu que le contrat s'analysait en un bail à construction nul, notamment en raison du
	refus opposé par la direction départementale de l'équipement à cette installation ;
	Sur le premier moyen :
	Vu l'article L. 251-1 du code de la construction et de l'habitation ;
	Attendu que constitue un bail à construction, le bail par lequel le preneur s'engage, à titre principal, à édifier
	des constructions sur le terrain du bailleur et à les conserver en bon état d'entretien pendant toute la durée du
	bail ;
	Attendu que, pour exclure la qualification de bail à construction, l'arrêt retient qu'il apparaît aux termes de
	l'article 12 du contrat signé entre les parties que le preneur s'engage à l'expiration du bail à remettre les lieux
	dans l'état dans lequel ils se trouvaient lors de sa prise de possession et qu'en présence de l'obligation de
	débarrasser le terrain du poste transformateur, ce contrat n'a pas la nature juridique d'un bail à construction ;
	Qu'en statuant ainsi, la cour d'appel a violé le texte susvisé ;
	PAR CES MOTIFS et sans qu'il y ait lieu de statuer sur les deuxième et troisième moyens :
	CASSE ET ANNULE
	I Possibilité de consentir des servitudes passives
	- L 451-9 : L'emphytéote peut acquérir au profit du fonds des servitudes actives, et les grever, par titres, de
	servitudes passives, pour un temps qui n'excédera pas la durée du bail à charge d'avertir le propriétaire.
	- L 251-3 : Le preneur peut consentir les servitudes passives indispensables à la réalisation des constructions
	prévues au bail.
	- L 254-2 : Le preneur peut acquérir des servitudes actives et consentir les servitudes passives indispensables à
	la réalisation des constructions et ouvrages édifiés en application du contrat de bail. Sauf accord du bailleur, il
	ne peut constituer des servitudes passives au-delà de la durée du bail.
	- Art L 255-7 : « Le preneur peut acquérir des servitudes actives et consentir les servitudes passives
	indispensables à la réalisation des constructions et ouvrages édifiés ou à réhabiliter en application du contrat de
	bail. Sauf accord de l'organisme de foncier solidaire, il ne peut constituer des servitudes passives au-delà de la
	durée du bail. »
	- Le preneur peut se voir interdire de consentir à des servitudes qui ne sont pas nécessaires à la réalisation
	de la construction (cf Cass. 3e civ, 16 juil 1998 CU 1998 299 Bull III \no 164 p 109 : Mais attendu qu'ayant
	exactement relevé qu'en application de l'article L. 251-8 du Code de la construction et de l'habitation, seules les
	dispositions législatives relatives aux servitudes passives indispensables à la réalisation des constructions
	prévues au bail à construction, ce qui n'était pas le cas de la servitude de passage consentie par la SCI Marly à
	la SCI IRMTS, sont d'ordre public, la cour d'appel en a justement déduit que les parties pouvaient déroger par
	convention aux dispositions législatives régissant les servitudes passives autres que celles indispensables à la
	réalisation de la construction sur le terrain faisant l'objet du bail ;
	- sort des servitudes L 251-6 : « Les servitudes passives, autres que celles mentionnées au quatrième alinéa de
	l'article L. 251-3, privilèges, hypothèques ou autres charges nées du chef du preneur et, notamment, les baux et
	titres d'occupation de toute nature portant sur les constructions, s'éteignent à l'expiration du bail.
	...Par ailleurs, si le bail prend fin par résiliation judiciaire ou amiable, les privilèges et hypothèques mentionnés
	au premier alinéa et inscrits, suivant le cas, avant la publication de la demande en justice tendant à obtenir cette
	résiliation ou avant la publication de l'acte ou de la convention la constatant, ne s'éteignent qu'à la date
	primitivement convenue pour l'expiration du bail. »
	§2 – Particularités de chaque contrat
	A Le bail à construction
	1) une obligation de construire
	cf L 251-1 : Constitue un bail à construction le bail par lequel le preneur s'engage, à titre principal, à édifier
	des constructions sur le terrain du bailleur et à les conserver en bon état d'entretien pendant toute la durée du
	bail.
	pb sanction cf infra49
	a)- obligation essentielle.cf Cass 3 e civ 13 jan 1987 Bull III \no 93 ; 20 mai 1992 GP 1992 2 pan 243
	Requalification possible du contrat en son absence : CA Paris, 26 sept 2000 CU 2001 143
	Cass 3 e civ 12 mai 2010 \No  de pourvoi: 09-14387
	Sur le premier moyen :
	Vu l'article L. 251-1 du code de la construction et de l'habitation ;
	Attendu, selon l'arrêt attaqué (Paris, 12 mars 2009), que la société Immobilière Carrefour (société Carrefour),
	venant aux droits de la société Euromarché, est devenue propriétaire d'un ensemble immobilier situé à Stains,
	dont le lot \no  2, qui a été aménagé à usage de cafétéria, avait, selon acte sous seing privé du 14 février 1992, été
	donné à bail à construction à la société Eris restauration, aux droits de laquelle se trouve la société civile
	immobilière du Centre commercial de Stains (la SCI du Centre commercial de Stains) ; que par acte authentique
	reçu par M. X..., notaire, le 12 février 2004, cette société a cédé le bail à construction à la société civile
	immobilière Synergie HM (la SCI Synergie) ; qu'après avoir infructueusement fait délivrer le 26 avril 2004 à la
	société du Centre commercial de Stains un commandement de payer une somme au titre du loyer du 2ème
	trimestre 2004, visant la clause résolutoire, la société Carrefour a assigné cette société, sollicitant notamment, à
	titre principal, que la convention du 14 février 1992 soit qualifiée de bail commercial régi par les dispositions
	des articles L. 145-1 et suivants du code de commerce, à titre subsidiaire, que la clause d'agrément insérée dans
	le bail à construction soit validée, que la clause résolutoire soit déclarée acquise, à titre très subsidiaire, que la
	résiliation judiciaire du bail à construction soit prononcée pour violation de la clause d'agrément relative à la
	cession de bail, défaut d'exploitation des lieux loués et absence de paiement des loyers pendant deux ans et demi
	; que la SCI du Centre commercial de Stains a appelé en garantie M. X... ; que la SCI Synergie est intervenue
	volontairement à l'instance ;
	Attendu que pour débouter la société Carrefour de sa demande de requalification de la convention du 14 février
	1992 en bail commercial, l'arrêt retient que le bail à construction suppose l'édification de travaux ayant un
	caractère à la fois immobilier et substantiel, que le bail initial porte, en l'espèce, sur un terrain de 3470 m2 et de
	deux volumes sans précision d'ouvrages existants, que cet élément de fait ajouté à la qualification de bail à
	construction (article L. 251-1 du code de la construction) que les parties ont très explicitement énoncé dans leur
	contrat ne permet pas d'effectuer la requalification sollicitée sans dénaturer leurs intentions au 14 février 1992 ;
	Qu'en statuant ainsi, sans rechercher si le preneur s'était engagé à édifier des constructions sur le terrain du
	bailleur, la cour d'appel n'a pas donné de base légale à sa décision de ce chef ;
	- doit être requalifié en bail à construction un bail emphytéotique qui fait peser une obligation de construire sur
	les preneurs : CA Paris, 24 fev 2005 Constr urb 2005 136 obs Pagès de Varenne
	- peut être passé par une société d’attribution dont la construction d’immeubles est l’objet (Cass 3 e civ 31 mars
	2005 Bull III \no  81 p 75)
	b)-Impact d’un retrait du permis de construire
	Cass 3e civ 1 er juin 2011 \no 09-70.502 pour un bail conclu sous condition suspensive d’obtention du permis
	Mais attendu, d’une part, qu’ayant relevé que le contrat avait été conclu sous condition suspensive de la
	délivrance d’un permis de construire, que le permis avait été délivré le 7 mars 2007 sans remise en cause des
	caractéristiques des ouvrages et équipements à mettre en place, qu’un premier arrêté municipal du 7 décembre
	2007 avait ordonné l’interruption des travaux et que l’annulation ou retrait du permis de construire était
	intervenue le 7 novembre 2008 au motif de précautions sanitaires, la cour d’appel, qui n’a pas constaté que les
	décisions administratives résultaient d’agissements fautifs de la preneuse, a pu, tirant les conséquences légales
	de ses propres constatations, retenir un revirement de l’autorité administrative imprévisible lors de la formation
	du contrat et dans le cours de son exécution tel le fait du prince;
	Attendu, d’autre part, qu’ayant pu retenir que l’interruption des travaux puis le retrait du permis de construire
	constituaient des événements insurmontables s’agissant de décisions administratives s’imposant immédiatement
	quels que soient les recours possibles et contraignant la société locataire à interrompre sur le champ puis à
	cesser les travaux, la cour d’appel, qui n’était pas tenue d’apprécier le mérite d’un éventuel recours devant les
	juridictions administratives, a pu en déduire que l’impossibilité d’exécuter le contrat caractérisait la force
	majeure et, sans violer l’article 1184 du code civil, décider que devait être prononcée pour ce motif la résiliation
	du contrat à compter de la date de l’arrêté d’annulation ;
	c)- impact sur le paiement du constructeur et l’éventuelle faute commise par le bailleur cf Cass 3 e civ, 12
	jan 2000, AJDI 2000 345
	Attendu, selon l'arrêt attaqué (Metz, 19 juin 1997), qu'en 1977, la société civile immobilière (SCI) de Gungling,
	propriétaire de terrains, a consenti à la société Hantz un bail à construction, stipulant que le preneur
	demeurerait propriétaire des immeubles à édifier pendant toute la durée du bail, dont le terme a été fixé en 1980
	à l'année 2011 ; qu'en 1992, la société Hantz a fait réaliser des travaux par la société Les Fils de Ferdinand
	Beck (société Beck), entrepreneur, ne les a pas entièrement réglés, et a été placée en redressement judiciaire ;
	que la société Beck a assigné la SCI, en paiement du solde du prix de ces travaux ;50
	Attendu que la société Beck fait grief à l'arrêt de rejeter cette demande, alors, selon le moyen, "que ni l'effet
	relatif des obligations ni les règles entourant la mise en oeuvre de l'action de in rem verso, ne pouvaient
	permettre aux juges du fond d'éluder les règles régissant la fraude invoquée par la société Les Fils de Ferdinand
	Beck, et que l'existence de la procédure collective n'interdisait nullement à la juridiction civile d'apprécier le
	caractère frauduleux du montage reproché à la SCI, personne juridique distincte de la SA Hantz et titulaire d'un
	droit de propriété certain en son principe sur les constructions réalisées sur son sol, dès leur édification ;
	qu'ainsi, la cour d'appel, qui s'est fondée sur des motifs purement inopérants, a violé, par refus d'application, le
	principe Fraus omnia corrumpit" ;
	Mais attendu qu'ayant relevé, par motifs propres et adoptés que nonobstant la qualité des associés, la
	composition du capital de la société Hantz, et la répartition des parts de la SCI, ces sociétés étaient des
	personnes morales distinctes notamment par leur objet, que la société Hantz était propriétaire des constructions
	jusqu'en 2011, et qu'il ne pouvait être imputé à faute à la SCI d'avoir contracté avec la société Hantz dans des
	conditions qui correspondaient à un cadre juridique déterminé et connu, la cour d'appel a pu retenir que la
	responsabilité de la SCI n'était pas engagée vis-à-vis de la société Beck sur le fondement délictuel ;
	d)- Impact sur la possibilité, pour le bailleur, d’exercer l’action en responsabilité décennale avec le
	preneur : CA Paris, 27 mai 2002 Constr urb 2002 \no 247 : Les bailleurs à construction ont intérêts à agir aux
	côtés du preneur au cours de son action en réparation des désordres dirigée contre les constructeurs, lié au fait
	qu'ils doivent recouvrir la propriété des constructions édifiées pendant le cours du bail, et au fait d'autre part que
	la situation financière fragile du preneur risque de conduire à la résiliation des baux en cours.
	Cour de cassation chambre civile 3 7 octobre 2014 \No  de pourvoi: 13-19448
	Attendu, selon l'arrêt attaqué (Bordeaux, 9 avril 2013), que la société civile immobilière Saint-Louis immobilier
	1 (la SCI) a consenti à la société Bordeaux distribution un bail à construction d'une durée de trente ans à
	compter du 1er janvier 1983 ; qu'en exécution de ce bail, la société Bordeaux distribution a fait édifier un
	bâtiment sous la forme d'un « hypermarché » exploité par la société Sofibor ; que des désordres étant apparus
	sur le carrelage refait, en 2001, par la société JM Branger, assurée par la société Swisslife assurance de biens
	(la société Swisslife), la SCI et les sociétés Bordeaux distribution et Sofibor ont assigné la société JM Branger et
	la société Swisslife, sur le fondement décennal, en réparation de leurs préjudices ;
	...
	Mais attendu qu'ayant retenu que la SCI ne serait propriétaire de l'ouvrage qu'à l'expiration du bail à
	construction, d'une durée de trente ans, dont elle n'a pas mentionné la date de prise d'effet et que les travaux
	avaient été commandés par la société Bordeaux distribution, propriétaire de l'ouvrage, la cour d'appel, sans
	contradiction et abstraction faite de l'utilisation sans conséquence d'un terme impropre, a pu en déduire,
	répondant aux conclusions, que seule la société Bordeaux distribution avait la qualité de maître d'ouvrage et
	que la SCI et la société Sofibor étaient sans qualité pour agir sur le fondement des dispositions de l'article 1792
	du code civil ;
	2) un loyer en argent révisable
	L 251-5 : S'il est stipulé un loyer périodique payable en espèces, ce loyer est affecté d'un coefficient révisable
	par périodes triennales comptées à partir de l'achèvement des travaux. Toutefois, la première révision a lieu au
	plus tard dès l'expiration des six premières années du bail.
	La variation du coefficient est proportionnelle à celle du revenu brut des immeubles. Le revenu pris pour base de
	la variation du coefficient est celui de la première année civile qui suit celle de l'achèvement des travaux.
	Les contestations relatives à l'application des dispositions des deux précédents alinéas sont portées devant le
	président du tribunal de grande instance.
	En cas de perte des bâtiments, le loyer est maintenu au taux qu'il avait atteint à la date de cette perte jusqu'à
	reconstruction éventuelle des bâtiments détruits et R 251-1 et suivants.
	- Réduction du loyer en cas d’expropriation partielle du terrain : Cass 3 e civ, 1 er mars 2000 Constr Urb 2000
	\no 125 obs Sizaire Mais attendu qu'ayant relevé qu'il résultait clairement des actes conclus, comportant avec
	précision la contenance des terrains objet du bail, que le loyer avait été calculé en fonction de la surface louée,
	même si celle-ci ne constituait pas la seule base de calcul modulée selon la valeur de chaque parcelle, son
	emplacement et sa nature constructible ou non, que la perte partielle de terrain était distincte du préjudice
	commercial indemnisé par les autorités expropriantes, et constaté que l'expert, conformément à la mission qu'il
	avait reçue, avait recherché la valeur locative des parcelles expropriées aprés avoir analysé les données
	spécifiques pour un bail à construction, la cour d'appel, abstraction faite d'un motif surabondant relatif à
	l'application de l'article 1722 du Code civil, appréciant la commune intention des parties, a pu en déduire que la
	demande de réduction de loyer, dont elle a souverainement fixé le montant, devait être accueillie ;
	- En cas d’option pour la TVA, nécessité que un loyer en argent soit prévu sinon l’administration
	détermine un loyer forfaitaire (CAA Paris, 19 déc 1989 JCP 1990 ed N p 228)51
	- le contrat de bail à construction conclu pour un prix dérisoire ou vil n'est pas inexistant mais nul pour
	défaut de cause ... l'action en nullité de ce contrat, qui relève d'intérêt privé, est, s'agissant d'une nullité relative,
	soumise à la prescription quinquennale de l'article 1304 du code civil (Cass 3 e civ 21 septembre 2011
	\No  de pourvoi: 10-21900)
	- Lésion : Cour de cassation chambre civile 3 4 avril 2013 \No  de pourvoi: 12-14134
	Mais attendu qu'ayant constaté que l'indivisibilité entre le bail à construction et la promesse de vente n'avait pas
	été stipulée dans l'acte, alors que les parties avaient expressément prévu l'indivisibilité du bail à construction et
	du pacte de préférence organisés par la même convention, ce dont elle a souverainement déduit qu'elles avaient
	entendu traiter différemment les rapports entre les diverses conventions, et relevé que les parties avaient conclu
	un bail à construction assorti d'une promesse unilatérale de vente conférant une option au preneur, lequel s'était
	donc réservé le droit d'acquérir ou non, que la SIEMP ne rapportait pas la preuve que le loyer stipulé était
	supérieur à celui qu'elle aurait payé si elle avait seulement contracté un bail sur le terrain et inclurait une partie
	du prix de vente et que les deux conventions existaient et pouvaient être exécutées indépendamment l'une de
	l'autre, la cour d'appel a pu en déduire que le bail à construction et la promesse de vente n'étaient pas
	indivisibles et que l'action en rescision pour lésion de Mme X... était recevable
	3) un régime faiblement d’ordre public. Cf L 251-8 : Les dispositions des troisième et quatrième alinéas de
	l'article L. 251-3, ainsi que celles de l'avant dernier alinéa de l'article L. 251-5 sont d'ordre public.
	a)-Possibilité d’une location vente des terrains
	cf en matière de lésion Cass 3 e civ 7 juil 2010 \no  09-14579 : « Attendu que pour déclarer parfaite la vente
	intervenue entre Mme X... et la SIEMP au prix de 114 598,80 euros, ordonné la réitération de cette vente par
	acte authentique et débouter Mme X... de ses demandes, l'arrêt retient que l'acte du 23 novembre 1981 comporte
	plusieurs conventions qui forment un ensemble indivisible, à savoir un bail permettant à la SIEMP, preneuse, de
	prendre la jouissance du terrain avec l'obligation d'en faire édifier un bâtiment, après démolition des anciennes
	constructions existant sur le terrain et, en fin de bail, une promesse de vente du terrain d'assiette, qu'il a été
	convenu que la vente, si elle se réalise, aura lieu moyennant un prix équivalent à dix années de loyers du bail à
	construction en sorte que le transfert de la propriété du terrain d'assiette quoique reportée en fin de bail serait
	payé sur la base des loyers réglés, cette clause rendant les deux opérations indissociables, et que le prix résiduel
	était calculé sur des loyers révisés, ce qui confère à la vente un caractère aléatoire interdisant l'application des
	articles 1674 et 1675 du code civil ;
	Qu'en statuant ainsi, par des motifs qui ne suffisent à caractériser l'indivisibilité entre le bail à construction et la
	vente et alors que l'aléa doit s'apprécier au jour de la réalisation de la vente, soit en l'espèce au jour de la levée
	de l'option, la cour d'appel a violé les textes susvisés ; »
	b)- Possibilité, pour le bailleur d’imposer des restrictions à l’activité du preneur Cass 3 e civ 7 avril 2004
	Bull III \no 70 : Mais attendu qu'ayant énoncé à bon droit que s'agissant des droits et obligations des parties, les
	articles L. 251-1 à L. 251-9 du Code de la construction et de l'habitation, relatifs au bail à construction, opèrent
	une distinction entre les dispositions supplétives de la volonté des parties et celles qui, déclarées d'ordre public,
	s'imposent aux parties nonobstant toute stipulation contraire, la cour d'appel, qui a exactement retenu qu'à
	l'exception des dispositions visées par l'article L. 251-8, les parties conservaient entière leur liberté
	contractuelle et pouvaient, dans le silence de la loi, insérer une clause imposant des restrictions à l'activité du
	preneur, a pu en déduire que la commune de Marseille, qui n'avait fait qu'appliquer la convention en opposant
	un refus au changement d'activité du preneur, n'avait commis aucune faute contractuelle ; cf JL Tixier La
	destination dans le bail à construction Constr Urb 2005 chr \no 4
	c)- possibilité, pour le bailleur de soumettre toute construction nouvelle à autorisation
	Cass civi 3 5 décembre 2007 Bulletin 2007, III, \No  215
	Sur le premier moyen :
	Vu l'article L. 251-8 du code de la construction et de l'habitation ;
	Attendu qu'il résulte de ce texte que sont seules d'ordre public les dispositions des troisième et quatrième alinéas
	de l'article L. 251-3 ainsi que celles de l'avant dernier alinéa de l'article L. 251-5 relatifs au bail à construction,
	énonçant que le preneur peut céder tout ou partie de ses droits ou les apporter en société et consentir les
	servitudes passives indispensables à la réalisation des constructions prévues au bail et que les contestations
	relatives à la révision du loyer périodique payable en espèces sont portées devant le président du tribunal de
	grande instance ;
	....
	Attendu que pour débouter Mme X... de ses demandes, l'arrêt retient que la clause du bail obligeant le preneur à
	se munir de l'autorisation écrite du bailleur pour apporter des modifications aux constructions édifiées sur le
	terrain loué doit être tenue pour nulle dès lors que l'essence du droit réel immobilier, dont le preneur est52
	titulaire aux termes de l'article L. 251-3 du code de la construction et de l'habitation, est la liberté de son usage,
	le preneur pouvant en exercer toutes les prérogatives, et notamment celui de construire selon les besoins de son
	activité, sans que le bail puisse le lui interdire ou soumettre ce droit à restriction ;
	Qu'en statuant ainsi, alors que les dispositions des articles L. 251-1 à L. 251-9 du code de la construction et de
	l'habitation régissant les droits et obligations des parties au bail à construction, opérant une distinction entre les
	dispositions supplétives de la volonté des parties et celles qui, déclarées d'ordre public, s'imposent à elles
	nonobstant toutes stipulations contraires, ne prohibent pas l'insertion dans ce bail d'une clause particulière
	subordonnant à l'autorisation du bailleur l'édification par le preneur de constructions nouvelles en cours de
	bail, la cour d'appel a violé le texte susvisé ;
	PAR CES MOTIFS, CASSE ET ANNULE,
	d)- libre rédaction des clauses de réalisation : Cass 3e civ 20 octobre 2010 \No  de pourvoi: 09-69645
	Sur le moyen unique :
	Attendu, selon l'arrêt attaqué (Grenoble, 23 juin 2009) que, par acte sous seing privé du 17 décembre 2003, la
	société civile immobilière Valcouriol (la SCI) et la société Mc Donald's France ont signé une promesse de bail à
	construction dont la réitération était soumise à huit conditions suspensives, notamment l'obtention d'un permis
	de construire devenu définitif, la modification des documents d'urbanisme de la ville devant être réalisée avant
	le 31 décembre 2005 puis après report du terme, avant le 30 juin 2006 ; que le 28 juin 2006, alors qu'elle n'avait
	pas obtenu le permis de construire, la société Mc Donald's France a fait savoir à la SCI qu'elle renonçait aux
	conditions suspensives et sollicitait la réitération du bail à construction ; que la SCI ayant refusé de signer ce
	bail, la société Mc Donald's France l'a assignée en réalisation forcée et en payement de dommages et intérêts et
	que la SCI a formé une demande reconventionnelle en paiement de dommages et intérêts ;
	Attendu que la société Mc Donald's France fait grief à l'arrêt de la débouter de ses demandes et de la
	condamner à payer à la SCI une certaine somme à titre de dommages et intérêts, alors, selon le moyen :
	1\degre / que l'article H de la promesse de bail à construction du 17 décembre 2003 stipulait que "Le preneur se
	réserve le droit de demander que le bail soit réalisé malgré la non-réalisation d'une ou plusieurs des conditions
	suspensives en envoyant au bailleur une simple lettre recommandée avec accusé de réception à cet effet, avant
	l'expiration du délai ci-dessus, éventuellement prolongé" ; que la société Mc Donald's France, preneur, se voyait
	ainsi reconnaître la faculté de renoncer au bénéfice des conditions suspensives et d'exiger la conclusion du
	contrat de bail à construction, malgré le défaut de réalisation d'une ou plusieurs conditions suspensives ; qu'en
	affirmant néanmoins que cette clause ne permettait pas à la société Mc Donald's France d'exiger la conclusion
	du contrat de bail à construction, en renonçant au bénéfice des conditions suspensives, et que la SCI conservait
	la possibilité de s'y opposer dès lors que toutes les conditions suspensives n'étaient pas réalisées, la cour d'appel
	a dénaturé les termes clairs et précis de l'article susvisé, en violation de l'article 1134 du code civil ;
	2\degre / que constitue un bail à construction, le bail par lequel le preneur s'engage, à titre principal, à édifier des
	constructions sur le terrain du bailleur et à les conserver en bon état d'entretien pendant toute la durée du bail ;
	que l'édification de la construction doit donc être possible lors de l'exécution du contrat de bail, et non lors de la
	formation de celui-ci, de sorte que l'obtention d'un permis de construire ne constitue pas une condition de
	formation du contrat de bail ; qu'en décidant néanmoins le contraire, pour en déduire qu'à défaut pour la société
	Mc Donald's France d'avoir obtenu un permis de construire dans le délai prévu pour la conclusion du contrat de
	bail à construction, la promesse de bail était devenue caduque, la cour d'appel a violé l'article L. 251-1 du code
	de la construction et de l'habitation" ;
	Mais attendu qu'ayant, par motifs propres et adoptés, relevé que l'article de l'acte énumérant les conditions
	suspensives stipulait que les parties n'auraient pas contracté sans elles, chacune étant déterminante, et qu'il
	énonçait expressément que le preneur bénéficiait du droit de demander la réalisation du bail, la cour d'appel a,
	par une interprétation souveraine de la volonté des parties exclusive de dénaturation, retenu que la clause ne
	permettait à la société Mc Donald's France que de demander la réalisation du bail et non l'exiger, la
	renonciation aux conditions suspensives ne pouvant résulter que du consentement des deux parties et a,
	abstraction faite d'un motif erroné mais surabondant tenant à l'exigence de l'obtention d'un permis de construire
	pour la formation du contrat, légalement justifié sa décision ;
	c)
	absence de clauses exorbitantes
	Sont possibles des clauses subordonnant à l'accord préalable du bailleur la modification des caractéristiques
	essentielles de l'espace restauration ainsi que toute modification substantielle de l'aspect extérieur des biens et
	constructions, de même que celle imposant au preneur l'obligation de consulter le bailleur sur le choix du
	gestionnaire et les modifications intervenant dans la gestion des locaux, dès lors que les sujétions imposées au
	preneur, compatibles avec la libre jouissance de la chose louée, étaient conformes au but recherché par les parties
	et correspondaient au contrôle normal, au regard de l'objet du bail, du respect de la chose louée, l'article L. 251-2
	du code de la construction et de l'habitation disposant que le bailleur devient, en fin de bail, propriétaire des
	constructions édifiées et profite des améliorations. De telles clauses n’étant pas exorbitantes du droit commun le53
	contrat qui les contient ne revêt pas un caractère administratif en raison de ses clauses, même s’il est passé par
	une personne publique (Cass 1re civ, 4 octobre 2017 \No  de pourvoi: 16-21693).
	4) un régime fiscal favorable
	a)- Principe
	- Amortissement des constructions sur la durée du bail et non sur la durée d’utilisation de l’immeuble art 39 D.
	C.G.I. logique compte tenu de l’accession des constructions en fin de bail
	- Loyer= revenu foncier
	- Atténuation de l’impôt payé par le bailleur lors de la remise des constructions
	* pas d’imposition si la durée du bail est > à 30 ans (33 ter du CGI et ann III art 2 sexiès)
	* décote de 8% du revenu par année de 18 à 30 ans
	* étalement du revenu supplémentaire sur 15 ans transmissible aux héritiers
	- pour le calcul du revenu supplémentaire prise en compte du prix de revient des constructions et non de leur
	valeur vénale.
	CE 26 mars 2012 \No  340883 ; CE 5 nov 2014 \No  366231 :
	4. Considérant, en premier lieu, qu'il résulte des dispositions de l'article 33 ter du code général des impôts que,
	lorsque le prix d'un bail à construction consiste, en tout ou partie, dans la remise au bailleur des constructions
	érigées sur son terrain par le preneur, le revenu foncier ou le bénéfice qui en résulte pour le bailleur doit être
	calculé d'après le prix de revient des biens qui lui ont été remis à l'expiration du bail ; que ce revenu ou ce
	bénéfice, que le bailleur peut répartir sur l'année au cours de laquelle les constructions ont été remises ainsi que
	sur les quatorze années suivantes, est par ailleurs réduit par application d'une décote de 8 % du prix de revient
	par année de bail au-delà de la dix-huitième année, lorsque la durée du bail à construction est comprise entre
	dix-huit ans et trente ans ; qu'enfin, la remise au bailleur des constructions à l'expiration du bail n'entraîne
	aucune imposition pour le bailleur lorsque la durée du bail à construction a été au moins égale à trente ans ;
	que, dans cette hypothèse, le bailleur peut prétendre au bénéfice de l'exonération d'imposition, dans la limite du
	prix de revient des constructions qui lui ont été remises, sans qu'y fasse obstacle la circonstance qu'il aurait
	comptabilisé ces constructions à leur valeur vénale ; que, dans le cas où le bailleur a inscrit les constructions
	reçues à leur valeur vénale, l'administration est fondée, conformément aux dispositions du 2 de l'article 38 du
	code général des impôts, à réintégrer dans le bénéfice imposable du contribuable l'écart entre la valeur vénale
	et le prix de revient des constructions qui lui ont été remises ; qu'il suit de là que c'est sans commettre d'erreur
	de droit que la cour a jugé, après avoir relevé que la société Casino Guichard-Perrachon avait comptabilisé à
	leur valeur vénale les constructions qui lui ont été remises gratuitement à l'issue du bail à construction conclu
	pour une durée de trente ans avec la SCI Economiques Troyens Barberey, que la société requérante n'avait droit
	au bénéfice de l'exonération d'imposition prévue par les dispositions du II de l'article 33 ter que dans la limite
	du prix de revient des constructions édifiées sur son terrain et en en déduisant que l'administration était fondée à
	réintégrer dans son résultat l'écart entre la valeur vénale et le prix de revient des constructions ; que la cour, qui
	a relevé, par une appréciation qui n'est pas arguée de dénaturation, que la société requérante, alors même
	qu'elle y avait été invitée par le ministre, n'avait fourni aucun élément justifiant du prix de revient des
	constructions remises et qu'elle n'avait pas conclu, fût-ce à titre subsidiaire, à la décharge de la fraction des
	impositions en litige assises sur le prix de revient des constructions, n'a pas davantage commis d'erreur de droit
	en jugeant que l'administration était dans ces conditions fondée à réintégrer dans son résultat imposable un
	montant égal à la valeur vénale des constructions ;
	5. Considérant, en second lieu, que lorsqu'un contrat de bail à construction prévoit, en faveur du bailleur, la
	remise gratuite en fin de bail des constructions réalisées par le preneur, le revenu foncier ou le bénéfice
	correspondant à cet avantage est imposable au titre de l'année au cours de laquelle le bail arrive à expiration ou
	fait l'objet, avant son terme normal, d'une résiliation ; que, dès lors, la cour n'a commis aucune erreur de droit
	en jugeant que la valeur des biens remis à l'issue d'un bail à construction, alors même qu'elle est assimilée à un
	complément de prix du bail, devait être imposée l'année de la remise des constructions ;
	- Mais pas de régime favorable pour les plus-values.
	Au contraire les constructions n’étant censées entrer dans le patrimoine qu’à la fin du bail, attendre 15 ans après
	la fin de ce bail. De plus valeur d’acquisition nulle.
	Cf toutefois CE 3 juil 2002 Cts Geffroy Constr Urb 2002 257 :
	Considérant, en premier lieu, qu'il ressort des pièces du dossier soumis aux juges du fond qu'aux termes du bail
	conclu le 20 octobre 1964 entre les époux Y... et la société Férisol, "toutes les constructions que la société a pu
	ou pourra édifier sur le terrain, après accord de M. Y..., deviendront en fin de bail la propriété du bailleur, sans
	que la société puisse prétendre à aucune indemnité, quelle que soit la date des constructions" ; que ces54
	stipulations contractuelles relatives au régime de propriété des constructions édifiées sur les terrains loués par
	les époux Y... avant le 30 juillet 1974, et qui prévoient que la propriété de ces constructions sera acquise aux
	époux Y... par l'effet d'une obligation née du contrat, font obstacle à ce que l'appropriation desdites
	constructions par les époux Y... puisse être regardée comme ayant été effectuée par la voie de l'accession régie
	par l'article 551 du code civil ;
	Considérant, en second lieu, qu'il ressort également des pièces du dossier soumis aux juges du fond que les
	stipulations précitées du bail du 20 octobre 1964 excluent que les époux Y... versent au locataire une indemnité
	en remboursement de la valeur des constructions édifiées sur le terrain loué et que le loyer stipulé par le bail ne
	s'élevait qu'à 5 000 F par an en 1964 et à 8 000 F par an en 1973 ; que la modicité de ce loyer ne peut qu'être
	regardée comme la contrepartie de l'absence d'obligation, pour les époux Y..., de verser à la société preneuse,
	en fin de bail, une indemnité en remboursement de la valeur des constructions édifiées par celle-ci sur le terrain
	pendant la durée du bail ; que, par suite, la valeur de l'avantage ainsi consenti par les époux Y... à la société
	Férisol doit être prise en compte dans la détermination du prix d'acquisition des constructions en cause, qui ne
	peut être égal à zéro et qui est nécessaire au calcul de la plus-value imposable en application des dispositions
	précitées de l'article 150H du code général des impôts ;
	-Régime en cas de cession : - par le bailleur : droit d’enregistrement et plus values
	- par le preneur : même chose.(TVA si les constructions datent de moins de 5 ans) cf
	Mortier, fiscalité du bail à construction, JCP 1997 ed N 629 et CE 30 mars 2007\No  263428
	b)- Difficultés en cas de cessation anticipées du bail
	- Cessation sans vente par le bailleur du terrain : droits d’enregistrement
	L 251-6 et cf Cass. Com 24 juin 1997 Bull \no  202; JCP 1998 10059 note Cornille
	Mais attendu qu'ayant relevé que dans le bail à construction le preneur bénéficie sur le terrain d'un droit réel
	immobilier et sur les constructions d'un droit de propriété temporaire et le bailleur, propriétaire du sol, devient,
	sauf convention contraire, propriétaire des constructions en fin de bail, le jugement en déduit, à bon droit, que
	par l'effet de la résiliation le preneur perd ce droit de propriété temporaire et permet au bailleur d'accéder à la
	propriété des immeubles construits avant l'expiration du bail, de sorte qu'entraînant transfert de propriété des
	immeubles la résiliation constitue une mutation soumise aux droits d'enregistrement des mutations d'immeubles ;
	que le moyen n'est fondé en aucune de ses trois branches ;
	Mais attendu que le bail à construction obligeant le preneur à construire un immeuble dont il devient
	propriétaire et à le conserver en bon état d'entretien pendant toute sa durée diffère, par la nature des droits qu'il
	confère au preneur, des autres baux qui ont pour principal objet de lui conférer la jouissance temporaire d'un
	bien contre le paiement d'un loyer ; que la résiliation d'un bail à construction avant son terme n'a pas pour effet
	de le transformer en un bail ordinaire mais de créer une situation qu'il convient d'apprécier eu égard aux
	conditions qu'elle prévoit et à l'ensemble des effets qui s'ensuivent ; qu'en retenant que la résiliation du bail à
	construction transfère au bailleur la propriété des constructions édifiées par le preneur et que l'indemnité due à
	celui-ci constitue l'assiette des droits d'enregistrement dus pour cette cession d'immeuble, le jugement se borne à
	énoncer les conséquences nécessaires de l'acte du 20 juin 1990, tel qu'il a été présenté à l'enregistrement ; que
	la bonne foi du redevable, qui n'avait procédé à aucune dissimulation ou déguisement d'acte, n'étant pas mise en
	cause, le Tribunal a pu statuer comme il a fait ; que le moyen n'est fondé en aucune de ses deux branches ;
	, RM du 4 juin 2001 et Cass com 19 juin 2001 JCP N 2002 1136
	-RM publiée dans le JO Sénat du 10/06/2010 - page 1462
	Aux termes de l'article L. 251-1 du code de la construction et de l'habitation (CCH), le bail à construction est la
	convention par laquelle le preneur s'engage, à titre principal, à édifier des constructions sur le terrain et à les
	conserver en bon état d'entretien pendant toute la durée du bail. L'article L. 251-2 du CCH prévoit que, dans le
	cadre d'un bail à construction, les parties conviennent de leurs droits respectifs de propriété sur les
	constructions existantes et sur les constructions édifiées. À défaut d'une telle convention, le bailleur en devient
	propriétaire à l'échéance du bail et profite des améliorations. En vertu des dispositions combinées des articles
	33 bis et 33 ter du code général des impôts (CGI), lorsque le prix d'un bail à construction consiste, en tout ou
	partie, dans la remise gratuite d'immeubles ou fractions d'immeubles, ou de titres donnant vocation à la
	propriété ou à la jouissance de tels immeubles, le bailleur doit déclarer un revenu foncier à raison de cet
	avantage en nature, l'avantage étant calculé d'après le prix de revient des immeubles ou des titres qui lui sont
	remis. Lorsqu'en fin de bail le bailleur devient propriétaire des constructions, il convient de faire une distinction
	selon la durée du bail : si la durée du bail est au moins égale à trente ans, la remise des constructions ne donne
	lieu à aucune imposition à l'impôt sur le revenu au titre des revenus fonciers ; si la durée du bail est inférieure à
	trente ans, l'imposition est due sur une valeur réduite en fonction de la durée du bail : le revenu brut foncier
	imposable à l'impôt sur le revenu correspondant à la valeur des constructions remises sans indemnité en fin de
	bail est égal au prix de revient de ces constructions, sous déduction d'une décote égale à 8 par année de bail au-
	delà de la dix-huitième année (art. 2 sexies de l'annexe III au CGI). Dans les cas évoqués par l'auteur de la
	question, et notamment en cas d'apport ou de cession au preneur du terrain objet du bail avant la fin du contrat,55
	la jurisprudence constante du Conseil d'État considère que l'apport ou la vente produit, sur le plan fiscal, les
	mêmes effets qu'une résiliation amiable tacite du bail, et doit être regardé comme impliquant la remise des
	immeubles au bailleur préalablement à la vente, faisant naître au bénéfice du bailleur le complément de loyer
	correspondant à la valeur des constructions édifiées par le preneur qui lui reviennent au terme du bail. À l'appui
	de cette jurisprudence, il est rappelé que le bail à construction est un bail à titre onéreux dans lequel la remise
	gratuite de l'immeuble au bailleur en fin de bail correspond à tout ou partie du loyer. Si la cession du terrain
	avant le terme du bail n'était pas analysée fiscalement comme une résiliation du contrat entraînant la taxation
	du bailleur pour la valeur des immeubles, les loyers courus constitués sous forme de remise de l'immeuble en fin
	de bail ne seraient pas taxables. Le Conseil d'État a considéré qu'il n'était pas logique, au regard de l'égalité
	devant l'impôt, que le loyer du bail à construction, constitué par la remise des constructions en fin de bail,
	puisse échapper à l'impôt en cas d'apport ou de cession du terrain au preneur, alors qu'en cas de résiliation du
	bail ce loyer est imposable. À défaut, le bailleur échapperait à toute imposition. Toute autre interprétation aurait
	pour effet de faire échapper la valeur des constructions à l'imposition, comme s'il n'y avait eu aucun loyer versé.
	La proposition de créer un report d'imposition en cas de cession anticipée du bail à construction ne paraît pas
	pertinente car elle conduirait en fait à une exonération définitive. Dès lors que le bail à construction a pris fin
	du fait de la réunion des qualités de bailleur et de preneur sur une seule tête, il n'existerait en effet pas
	d'événement ultérieur permettant de mettre fin au report d'imposition du bailleur dans la catégorie des revenus
	fonciers.
	- Cessation par vente du terrain au preneur
	Principe de réintégration de la valeur des constructions dans les revenus du bailleur
	CE 11 Avril 2008 \No  287961 et 21 nov 2011 \No  340777
	Considérant que la cour a souverainement interprété le contrat de cession du 19 décembre 2003 comme devant
	avoir, au regard de la loi fiscale, et quelle qu'ait été l'intention des parties, les mêmes effets qu'une résiliation
	amiable tacite du bail impliquant la remise à la SCI " 9 rue Félibien " des immeubles construits par la SAS
	Clinique Jules Verne préalablement à la vente ; qu'elle n'a pas commis d'erreur de droit en en déduisant que
	l'anticipation ainsi convenue du terme du bail devait entraîner l'application des dispositions des articles 33 bis
	et 33 ter du code général des impôts et en jugeant que l'administration était fondée à réintégrer dans les revenus
	de la SCI la valeur des constructions édifiées par la SAS Clinique Jules Verne et à imposer M. et Mme A à
	raison de ce revenu au prorata de leurs droits dans la société dans la catégorie des revenus fonciers au titre de
	l'année 2003 ;
	position différente de la Cass dans les baux commerciaux : Cass com 4 décembre 2012 \No  de pourvoi: 11-
	25958
	Attendu que, pour rejeter la demande de la société Fuxedis, l'arrêt retient que la vente du bien immobilier, par le
	bailleur au locataire, avait éteint le bail par confusion des droits locatifs et de propriété sur la tête de la même
	personne, avant son terme normal ; qu'il retient encore que cette extinction avait produit les mêmes effets qu'une
	résiliation amiable tacite anticipée du bail qui devait être regardée comme impliquant la remise du bien
	immobilier, dans toutes ses composantes, au bailleur préalablement à la vente et qu'elle constituait une mutation
	soumise aux droits d'enregistrement pour le tout en sorte que les travaux d'amélioration réalisés par la société
	Fuxedis ne pouvaient échapper à la taxation correspondante ;
	Attendu qu'en statuant ainsi, alors que l'acquisition par le preneur n'avait pas entraîné la résiliation anticipée
	du bail commercial mais son extinction par confusion des droits au sens de l'article 1300 du code civil et
	qu'aucun transfert de la propriété des constructions réalisées par le preneur ne s'était produit entre son
	patrimoine et celui du bailleur avant cette acquisition, la cour d'appel a violé les textes susvisés ;
	- Maintien de la position du juge administratif : Conseil d'État 27 février 2013\No  350663
	6. Considérant qu'aux termes du 2 de l'article 38 du code général des impôts : " Le bénéfice net est défini comme
	la différence entre les valeurs de l'actif net à la clôture et à l'ouverture de la période dont les résultats doivent
	servir de base à l'impôt diminuée des suppléments d'apport et augmentée des prélèvements effectués au cours de
	cette période par l'exploitant ou par les associés / L'actif net s'entend de l'excédent des valeurs d'actif sur le total
	formé au passif par les créances de tiers, les amortissements et les provisions justifiées (...) " ;
	7. Considérant que, dans le cas de la vente, avant le terme du bail à construction, par le bailleur au profit du
	preneur du terrain faisant l'objet de ce bail, le contrat de cession produit, au regard de la loi fiscale, pour
	l'ensemble des parties, les mêmes effets qu'une résiliation amiable tacite du bail et doit être regardé comme
	impliquant la remise des immeubles au bailleur préalablement à la vente ; que le transfert des constructions
	dans le patrimoine du bailleur implique nécessairement leur sortie du patrimoine du preneur, lequel perd ainsi
	les droits qu'il détenait sur les immeubles qu'il a édifiés ; que l'acquisition par le preneur du terrain d'assiette
	des constructions emporte ensuite immédiatement transfert dans son patrimoine de l'ensemble immobilier,
	composé du terrain et des constructions ; que si la vente par le bailleur au preneur a entraîné sur le plan civil
	une confusion des qualités de bailleur et de preneur en vertu de l'article 1300 du code civil et si cette confusion56
	n'a pas entraîné la résiliation anticipée du bail à construction mais son extinction, ces dispositions du code civil
	ne font pas échec, pour l'application de la loi fiscale, à la mise en oeuvre des règles exposées ci-dessus ;
	formule intégralement reprise par CAA Marseille 27 janvier 2015 \No  11MA03817 ; CAA de LYON 26 janvier
	2017 trois arrêts \No  15LY01376, \No  15LY01377, \No  15LY01378
	c)-Situation en cas de prorogation du contrat :CE 25 janvier 2006 \No  271523
	Considérant qu'il ressort des pièces du dossier soumis aux juges du fond qu'aux termes d'un bail à construction
	conclu en 1971 pour vingt ans, la SA Immobilière du Parc devait devenir propriétaire en 1991 de la clinique à
	construire par le preneur ; qu'en dépit de ce qu'en 1985, les parties à ce contrat ont convenu de le proroger pour
	dix ans, l'administration fiscale a estimé que cette société devait être imposée au titre de 1991 sur le revenu
	foncier en nature correspondant au transfert de propriété de la clinique et a redressé en conséquence les
	résultats déclarés par la société ; que le MINISTRE DE L'ECONOMIE, DES FINANCES ET DE L'INDUSTRIE
	se pourvoit en cassation contre l'arrêt en date du 1er juillet 2004 par lequel la cour administrative d'appel de
	Lyon a rejeté l'appel qu'il avait formé contre le jugement du tribunal administratif de Dijon en date du 3 février
	1998, déchargeant la société du supplément d'impôt sur les sociétés impliqué par ce redressement ; Considérant
	qu'en vertu des dispositions combinées des articles 33 ter et 151 quater du code général des impôts, l'immeuble
	qui, en fin de bail, revient sans indemnité au bailleur à construction constitue pour lui un revenu foncier
	imposable au titre de l'année de la fin du bail ; que si l'article L. 251-1 du code de la construction et de
	l'habitation interdit qu'un bail à construction puisse se prolonger par tacite reconduction, cette disposition
	n'interdit pas aux parties de convenir de proroger l'échéance initialement prévue pour un tel bail ; que la cour
	administrative d'appel n'a pas commis d'erreur de droit en jugeant qu'une telle prorogation, qui n'implique pas,
	par elle-même, la naissance d'un nouveau contrat, avait pour effet de reporter au nouveau terme convenu la date
	à laquelle le bailleur deviendrait propriétaire de la clinique construite par le preneur, et par suite l'imposition
	du revenu foncier en nature correspondant ;
	5) Des règles spécifiques en cas de montage « pass foncier»
	art L 251-1 dernier alinéa : Toutefois, lorsque le bail prévoit une possibilité d'achat du terrain par le preneur
	dans le cadre d'une opération d'accession sociale à la propriété dans les conditions prévues par la section 1 du
	chapitre III du titre IV du livre IV du présent code et que le preneur lève l'option, le bail prend fin à la date de la
	vente, nonobstant les dispositions du troisième alinéa.
	Art L 256-al 2 : Cependant, lorsque le bail prévoit une possibilité d'achat du terrain par le preneur dans le
	cadre d'une opération d'accession sociale à la propriété et que le preneur lève l'option conformément au
	quatrième alinéa de l'article L. 251-1, les privilèges et hypothèques du chef du preneur inscrits avant la levée de
	l'option ne s'éteignent pas à l'expiration du bail mais conservent leurs effets, jusqu'à leur date d'extinction, sur
	l'immeuble devenu la propriété du constituant. Ils s'étendent de plein droit au terrain et peuvent garantir les
	prêts consentis pour l'acquisition dudit terrain.
	Cf F Aubry, Le pass Foncier, un dispositif à manier avec précautions Défrénois 2009 38938
	6) Sort du bail d’habitation
	Article L 251-6 : Les servitudes passives, autres que celles mentionnées au quatrième alinéa de l'article L. 251-
	3, privilèges, hypothèques ou autres charges nées du chef du preneur et, notamment, les baux et titres
	d'occupation de toute nature portant sur les constructions, s'éteignent à l'expiration du bail sauf pour les
	contrats de bail de locaux d'habitation.
	Cour de cassation chambre civile 3 4 avril 2019 \No  de pourvoi: 18-14049 : Mais attendu qu'ayant relevé
	que, conformément à l'article L. 251-6 du code de la construction et de l'habitation, dans sa version antérieure à
	celle issue de la loi du 24 mars 2014, prévoyant que les contrats de location consentis par le preneur d'un bail à
	construction s'éteignent à l'expiration du bail, le contrat de bail à construction mentionnait que le preneur
	pourrait louer les constructions pour une durée ne pouvant excéder celle du bail, que la société d'HLM, qui,
	seule, pouvait fixer le terme des baux qu'elle avait consentis sur les appartements, ne disposait de droits sur les
	immeubles que jusqu'au 19 juillet 2010, date d'expiration du délai contractuel de vingt-cinq ans figurant au
	contrat de bail à construction du 19 juillet 1985, et que les trois appartements des consorts K... ne leur avaient
	été restitués respectivement qu'en novembre 2010 et novembre 2011, la cour d'appel en a exactement déduit que
	la société d'HLM avait manqué à son obligation de restituer les lieux libres de tous occupants ;
	7) Juge compétent
	Cour de cassation chambre civile 1 4 octobre 2017 \No  de pourvoi: 16-21693
	Attendu, selon l'arrêt attaqué (Douai, 12 mai 2016), que, suivant acte du 9 avril 2010, la commune de
	Dunkerque (la commune) a consenti à la société GHM une promesse synallagmatique de bail à construction57
	portant sur une parcelle dépendant de son domaine privé ; que, reprochant à son cocontractant d'avoir refusé de
	réitérer la promesse par acte authentique, la société Dunotel, venant aux droits de la société GHM, a saisi la
	juridiction judiciaire aux fins d'exécution forcée et, subsidiairement, en indemnisation de son préjudice ; que la
	commune a soulevé une exception d'incompétence au profit de la juridiction administrative ;
	Attendu que la commune fait grief à l'arrêt de rejeter cette exception, alors, selon le moyen : ...
	Mais attendu, d'une part, qu'il résulte de l'article 1er de la directive 2004/18/CE du Parlement européen et du
	Conseil du 31 mars 2004, relative à la coordination des procédures de passation des marchés publics de
	travaux, de fournitures et de services, alors en vigueur, que constitue un marché public de travaux tout contrat à
	titre onéreux conclu entre un ou plusieurs opérateurs économiques et un ou plusieurs pouvoirs adjudicateurs, et
	ayant pour objet la réalisation, par quelque moyen que ce soit, d'un ouvrage répondant aux besoins précisés par
	le pouvoir adjudicateur ; que la circonstance que la personne publique n'assure pas la maîtrise d'ouvrage des
	travaux envisagés ne fait pas obstacle à une telle qualification ; que, cependant, la qualification de marché
	public de travaux, au sens des dispositions précitées, ne suffit pas, à elle seule, à conférer au contrat un
	caractère administratif ; que celui-ci ne revêt un tel caractère que s'il porte sur l'exécution de travaux
	immobiliers exécutés pour le compte de la personne publique et dans un but d'intérêt général ou s'il a pour objet
	l'exécution même d'un service public ; que la cour d'appel a constaté, d'abord, que les travaux prévus par la
	promesse de bail à construction ne seraient pas réalisés sous la maîtrise d'ouvrage de la commune, ce dont il
	résultait que le contrat litigieux ne pouvait être qualifié de marché public de travaux au sens du code des
	marchés publics, alors en vigueur, et ne constituait pas, par suite, un contrat administratif par détermination de
	la loi, en application de l'article 2 de la loi \no  2001-1168 du 11 décembre 2001 portant mesures urgentes de
	réformes à caractère économique et financier ; qu'elle a relevé, ensuite, que la promesse avait pour objet des
	travaux de construction d'un immeuble hôtelier et de restauration destiné à être géré par une société
	commerciale pour une durée de quatre-vingt-dix-neuf ans ; qu'ayant ainsi fait ressortir que ce contrat ne portait
	ni sur l'exécution de travaux publics ni sur l'exécution même d'un service public, elle en a exactement déduit
	qu'il n'avait pas un caractère administratif en raison de son objet ;
	Et attendu, d'autre part, que la clause exorbitante du droit commun est celle qui, notamment par les prérogatives
	reconnues à la personne publique contractante dans l'exécution du contrat, implique, dans l'intérêt général, qu'il
	relève du régime exorbitant des contrats administratifs ; que la cour d'appel a retenu, à bon droit, que ne
	pouvaient recevoir une telle qualification les clauses subordonnant à l'accord préalable du bailleur la
	modification des caractéristiques essentielles de l'espace restauration ainsi que toute modification substantielle
	de l'aspect extérieur des biens et constructions, de même que celle imposant au preneur l'obligation de consulter
	le bailleur sur le choix du gestionnaire et les modifications intervenant dans la gestion des locaux, dès lors que
	les sujétions imposées au preneur, compatibles avec la libre jouissance de la chose louée, étaient conformes au
	but recherché par les parties et correspondaient au contrôle normal, au regard de l'objet du bail, du respect de
	la chose louée, l'article L. 251-2 du code de la construction et de l'habitation disposant que le bailleur devient,
	en fin de bail, propriétaire des constructions édifiées et profite des améliorations ; qu'elle en a exactement
	déduit que la promesse litigieuse ne revêtait pas un caractère administratif en raison de ses clauses ;
	B Le bail emphytéotique
	Article L 451-4 : Le preneur ne peut demander la réduction de la redevance pour cause de perte partielle du fonds,
	ni pour cause de stérilité ou de privation de toute récolte à la suite de cas fortuits.
	Article L 451-5 : A défaut de paiement de deux années consécutives, le bailleur est autorisé, après une sommation
	restée sans effet, à faire prononcer en justice la résolution de l'emphytéose.
	La résolution peut également être demandée par le bailleur en cas d'inexécution des conditions du contrat ou
	si le preneur a commis sur le fonds des détériorations graves.
	Néanmoins, les tribunaux peuvent accorder un délai suivant les circonstances.
	Article L 451-6 : Le preneur ne peut se libérer de la redevance, ni se soustraire à l'exécution des conditions du bail
	emphytéotique en délaissant le fonds.
	- Pas d’obligation d’exploitation personnelle par le titulaire du bail (Cass 3e civ 27 oct 2004 , Bull III \no  180
	p 163)
	-Pas d’obligation de construire : Cass 3e civ 8 septembre 2016 \No  de pourvoi: 15-21.381 Mais attendu
	qu’ayant relevé que les termes du bail, qui prévoyait seulement la faculté de faire édifier tous immeubles et
	notamment un casino, ne mettaient à la charge de la société Cannes Balnéaire aucune obligation de construire
	et retenu, sans la dénaturer, que la clause stipulant que, “dans le cas où la ville de Cannes ne donnerait pas à la
	société Cannes Balnéaire les autorisations nécessaires à l’exploitation d’un casino, il est entendu que le présent
	bail n’aura aucun effet”, n’était pas une clause résolutoire mais une condition concernant l’exploitation du
	casino et n’édictait aucune obligation de construire, la cour d’appel en a déduit à bon droit que le contrat devait
	être qualifié de bail emphytéotique ; »58
	- Liberté absolue d’utilisation du droit. Peut être utilisé pour la construction de logements pour les personnes
	défavorisées dans le cadre d’une mission de service public. Les marchés passés peuvent alors être considérés
	comme des documents administratifs communicables (CAA Nantes 26 avril 2001 Gindreau RD Imm 2001 498)
	Cour de cassation chambre civile 3 15 décembre 2016 \No  de pourvoi: 15-22416 Mais attendu qu'ayant
	constaté que le bail conférant un droit réel au preneur prévoyait que celui-ci pourrait édifier des constructions
	nouvelles et souverainement retenu que la bailleresse ne rapportait la preuve ni d'un manquement au contrat
	justifiant sa résolution ni de l'existence de détériorations graves du fonds engendrées par les travaux du preneur,
	la cour d'appel a légalement justifié sa décision
	- Sur la possibilité pour le preneur d’un bail emphytéotique de passer des baux commerciaux Cass 3 e civ 9
	fév 2005 Bull 2005 III \No  34 p. 29 : Vu l'article L. 145-3 du Code de commerce, ensemble l'article L. 145-32 de
	ce Code ;
	Attendu que les dispositions du chapitre V du titre IV du livre premier du Code de commerce ne sont pas
	applicables aux baux emphytéotiques, sauf en ce qui concerne la révision du loyer ; que toutefois, elles
	s'appliquent, dans les cas prévus aux articles L. 145-1 et L. 145-2, aux baux passés par les emphytéotes, sous
	réserve que la durée du renouvellement consenti à leurs sous-locataires n'ait pas pour effet de prolonger
	l'occupation des lieux au-delà de la date d'expiration du bail emphytéotique ;
	...
	Qu'en statuant ainsi, alors que la durée du bail consenti par un emphytéote ne pouvant excéder celle du bail
	emphytéotique, le sous-locataire ne peut prétendre à l'expiration de celui-ci à aucun droit au renouvellement et,
	partant, au paiement d'une indemnité d'éviction, la cour d'appel a violé les textes susvisés ;
	- Sur l’opposabilité du bail d’habitation passé par l’emphytéote : Cass 3 e civ 2 juin 2010 \no  08-17731
	« Attendu que pour accueillir la demande d'expulsion formée par l'APHP, l'arrêt retient que la société ne
	pouvait consentir à des tiers plus de droits qu'elle n'en avait elle-même, que la sous-locataire, occupant du chef
	de l'emphytéote, locataire principal, ne peut opposer au bailleur plus de droits qu'il n'en résulte du bail
	d'habitation et que n'en détient la société à l'égard de l'APHP en vertu du bail emphytéotique, que ce principe ne
	saurait être tenu en échec par l'absence de disposition légale expresse limitant à la durée du bail emphytéotique
	la durée des baux d'habitation consentis par l'emphytéote conformément aux textes en vigueur en matière de
	baux d'habitation, que le bail liant Mme X... à la société ayant pris fin le 1er avril 2002 par l'effet du terme du
	bail emphytéotique, Mme X... est devenue occupante sans droit ni titre, peu important l'absence de signification
	de congé sur le fondement de l'article 15 de la loi du 6 juillet 1989 ;
	Qu'en statuant ainsi, alors que le bail d'habitation régulièrement consenti à Mme X... par l'emphytéote était
	opposable à l'APHP et qu'aucun texte n'affranchissait celle-ci de l'obligation de respecter les dispositions
	d'ordre public de la loi du 6 juillet 1989 qui lui étaient applicables, la cour d'appel, qui n'a pas constaté qu'il
	avait été mis fin au bail conformément aux dispositions de cette loi, a violé les textes susvisés ; »
	Cf art 10 al 6 loi du 6 juillet 1989 reproduit à l’article L 451-2 C rural : « Concernant les locaux à usage
	d'habitation, régis par les dispositions d'ordre public de la présente loi, le contrat de bail conclu par
	l'emphytéote avec le locataire se poursuit automatiquement avec le propriétaire de l'immeuble jusqu'au terme du
	bail prévu par le contrat de location, lorsque le bail à construction ou le bail emphytéotique prend fin avant la
	fin du contrat de location. Toute clause contraire est réputée non écrite. »
	- sur l’attribution préférentielle du droit au bail: Cour de cassation chambre civile 1 12 juin 2013 \No  de
	pourvoi: 12-11724
	Sur le moyen relevé d'office après que les parties ont été invitées à présenter leurs observations :
	Vu les articles 831-2, 1\degre , et 1476 du code civil ;
	Attendu que, selon ces textes, en cas de dissolution de la communauté par divorce, un époux peut demander
	l'attribution préférentielle de la propriété ou du droit au bail du local qui lui sert effectivement d'habitation s'il y
	avait sa résidence ;
	Attendu, selon l'arrêt attaqué, que M. X... et Mme Y... se sont mariés sans contrat préalable le 23 janvier 1971 ;
	que par acte du 18 mai 1982 leur a été consenti un bail emphytéotique sur une villa qui a été le logement de la
	famille et qui sera attribuée à l'épouse pendant la procédure de divorce introduite par assignation du 12
	novembre 2007 ; que celle-ci a sollicité l'attribution préférentielle du droit à ce bail par application de l'article
	1751 du code civil ;
	Attendu que, pour déclarer irrecevable cette demande, l'arrêt retient que ce texte n'est pas applicable,
	l'emphytéose étant régie par les articles L. 451-1 et suivants du code rural et de la pêche et la jurisprudence
	étant venue à plusieurs reprises rappeler la spécificité de ce type de bail auquel les règles qui régissent le louage
	ordinaire n'ont jamais été applicables ;
	Attendu qu'en statuant ainsi la cour d'appel a violé par refus d'application les textes susvisés ;
	- Sur la non application du droit des baux commerciaux : Cour de cassation chambre civile 3 19 février
	2014 \No  de pourvoi: 12-1927059
	Vu les articles L. 451-1 et L. 451-3 du code rural et de la pêche maritime ensemble L. 145-3 et L. 145-33 du
	code de commerce ;
	Attendu, selon l'arrêt attaqué (Aix-en-Provence, 20 janvier 2012), que Mme X...est propriétaire d'un terrain
	qui, en 1981, a été donné à M. Y...à bail emphytéotique de vingt-six ans, à usage de création et exploitation
	d'un camping-caravaning, village de chalets de vacances, terrains de sports et de jeux, piscine et attractions
	diverses ; que par acte du 11 août 1982, une partie du terrain a été exclue de ce bail cédé pour le surplus et
	pour une durée plus importante, à M. Z... aux droits duquel se trouve Mme A..., à charge pour le locataire
	autorisé à créer et exploiter tous commerces de son choix, de faire à ses frais les routes, aménagements et
	constructions nécessaires ; qu'en 2004, Mme X...se prévalant d'une augmentation de la valeur locative de plus
	de dix pour cent, a sollicité la révision et la fixation judiciaire du loyer ;
	Attendu que pour fixer à une certaine somme le montant du loyer à compter du 15 juin 2004 avec intérêts au
	taux légal à compter du 28 mars 2006, l'arrêt retient qu'il existe une modification matérielle des facteurs
	locaux de commercialité ayant une incidence sur l'activité exercée et que l'examen des loyers de terrain de
	camping dans la région constituant la référence la plus adéquate car elle concerne une activité
	d'hébergement de loisirs proche de celle exploitée, aboutit après actualisation au 1er trimestre 2004, à un
	montant de loyer au mètre carré à pondérer en raison de la spécificité du terrain donné en location soit à un
	loyer annuel de 96 576 euros pour les 68 011 mètres carrés loués et exploitables pour l'année 2004 alors que
	par application du seul jeu de l'indice du coût de la construction le loyer dû par le preneur était de 4 999, 54
	euros ce qui démontre une évolution de plus de 10 % de la valeur locative ;
	Qu'en statuant ainsi alors que les dispositions des articles L. 145-3 et L. 145-33 du code de commerce ne
	s'appliquent pas au loyer du bail emphytéotique prévu à l'article L. 451-3 du code rural et de la pêche
	maritime au terme duquel le preneur, titulaire d'un droit réel pendant sa durée, ne bénéficie d'aucun droit au
	renouvellement ni à indemnité d'éviction ; la cour d'appel a violé les textes susvisés ;
	Cass 3e civ 8 septembre 2016 \No  de pourvoi: 15-21.381 : « Attendu, d’autre part, qu’ayant retenu que la
	valeur locative était étrangère à l’économie du contrat de bail emphytéotique, la contrepartie de la jouissance
	du preneur étant pour le bailleur, non le payement du loyer, mais l’absence de renouvellement et l’accession
	sans indemnité en fin de bail de tous travaux et améliorations faits par le preneur, la cour d’appel en a
	exactement déduit que les bailleurs ne pouvaient saisir le juge des loyers commerciaux d’une demande de
	révision du loyer pour le faire correspondre à la valeur locative, fût-ce en invoquant une évolution favorable des
	facteurs locaux de commercialité »
	- Sur le bénéfice d’un décision d’expulsion : CE 0 octobre 2012 \No  352770
	2. Considérant que la responsabilité de l'Etat née du refus de prêter le concours de la force publique pour
	assurer l'exécution d'une décision de justice peut être engagée à l'égard de la personne au profit de laquelle a
	été rendue cette décision ou de la personne investie ultérieurement de ses droits ; qu'un bailemphytéotique
	consenti par un propriétaire au profit duquel a été rendue une décision de l'autorité judiciaire ordonnant
	l'expulsion des occupants d'un immeuble, est de nature à investir le preneur des droits que le propriétaire tient
	de cette décision, sous réserve des stipulations contractuelles contraires ;
	3. Considérant que, pour rejeter la demande de Paris Habitat OPH, le tribunal administratif de Melun s'est
	fondé sur la seule circonstance que le demandeur n'était pas le propriétaire du logement, au profit duquel avait
	été rendu le jugement du 15 février 2001 du tribunal d'instance de Boissy-Saint-Léger, sans rechercher si le
	bailemphytéotique conclu le 21 octobre 2004 avec ce propriétaire, sur lequel Paris Habitat OPH fondait sa
	demande d'indemnité, avait investi celui-ci des droits du propriétaire ; qu'il n'a pas, ainsi, justifié légalement sa
	décision ; que, par suite, sans qu'il soit besoin d'examiner les autres moyens du pourvoi, Paris Habitat OPH est
	fondé à demander l'annulation du jugement qu'il attaque ;
	-Sur l’hypothèque inscrite sur un bail emphytéotique : Cass 3 e civ 7 oct 2009 \no 08-14962
	Vu les articles L. 451-1 du code rural et 2488 du code civil, ensemble les articles 2461 et 2480 du code civil ;
	Attendu, selon l'arrêt attaqué (Pau, 6 février 2008) que le 31 janvier 1963, la commune de Campan a donné à
	bail emphytéotique de trente ans, à M. et Mme X..., une parcelle destinée à l'édification d'un local commercial,
	que le 26 avril 1988 les époux X... ont cédé le droit au bail à la société civile immobilière des Neiges laquelle a
	loué le local à Mme Y... suivant bail commercial du 30 septembre 1993 avant d'être mise en liquidation
	judiciaire le 18 février 1994, que la créance des époux X... a été admise à titre hypothécaire, que le 30 décembre
	1995 la commune de Campan a consenti un nouveau bail emphytéotique à M. Z... agissant en qualité de
	mandataire liquidateur de la société civile immobilière des Neiges pour une durée de trente années, à compter
	du 1er janvier 1991, que par ordonnance du 19 janvier 1999 le juge commissaire a autorisé la cession du bail
	emphytéotique à la société civile immobilière JIV en cours de constitution entre les époux Y..., que cette société a
	mis en oeuvre la procédure de purge à la suite de laquelle les époux Y... ont fait délivrer une réquisition de
	surenchère ;
	Attendu que pour dire valable la surenchère formée par les époux X... et ordonner la vente sur surenchère,60
	l'arrêt retient qu'il importe peu que le bail renouvelé constitue un nouveau bail dès lors qu'il poursuit la relation
	contractuelle du bail venu à expiration, le preneur tenant son droit à ce que sa situation soit maintenue par
	application du bail précédent, la commune intention des parties ayant été en ce sens lors du renouvellement, que
	de la lecture des deux baux successifs il résulte "l'absence de solution de continuité" dans la propriété du
	preneur à la faveur du renouvellement ce dont il résulte que les sûretés inscrites sur les biens qui restent la
	propriété du vendeur demeurent ;
	Qu'en statuant ainsi alors que l'hypothèque inscrite sur un bail emphytéotique disparaît à l'expiration de ce bail
	et qu'elle avait constaté que le bail emphytéotique du 30 décembre 1995 constituait un nouveau bail distinct de
	celui du 31 janvier 1963, la cour d'appel a violé les textes susvisés ;
	- Sur une action en révocation du bail par des héritiers : Conseil d'État 29 mai 2017 \No  393448
	1. Il ressort des pièces du dossier soumis aux juges du fond que la fondation Pierre D..., qui est propriétaire à
	Plouescat de bâtiments, les a mis gratuitement à disposition de l'association de gestion du lycée professionnel
	rural privé " Pierre D...", lequel y a dispensé un enseignement jusqu'à la fermeture de cet établissement en juin
	2009. Le conseil d'administration de cette fondation a alors approuvé le principe d'une mise à disposition
	gracieuse de ces locaux à la commune de Plouescat à l'effet d'y accueillir des associations à caractère éducatif
	ou de loisir, à charge pour la commune de s'acquitter de l'entretien, de la mise aux normes, des assurances et
	des diverses taxes. Un projet de bail emphytéotique a été établi pour une durée de trente ans et soumis à
	l'autorisation du préfet du Finistère. Cette autorisation a été délivrée par un arrêté du 27 novembre 2009. Par
	un jugement du 8 mars 2013, le tribunal administratif de Rennes a rejeté comme irrecevable la demande de M.
	C...tendant à l'annulation de cet arrêté. M. C...se pourvoit en cassation contre l'arrêt de la cour administrative
	d'appel de Nantes du 30 juin 2015 qui a confirmé ce jugement.
	2. Aux termes de l'article 954 du code civil : " Dans le cas de la révocation pour cause d'inexécution des
	conditions, les biens rentreront dans les mains du donateur, libres de toutes charges et hypothèques du chef du
	donataire ; et le donateur aura, contre les tiers détenteurs des immeubles donnés, tous les droits qu'il aurait
	contre le donataire lui-même ". Aux termes de l'article 1046 du même code : " Les mêmes causes qui, suivant
	l'article 954 et les deux premières dispositions de l'article 955, autoriseront la demande en révocation de la
	donation entre vifs, seront admises pour la demande en révocation des dispositions testamentaires ". En
	application de ces dispositions, les héritiers, directs ou indirects, de l'auteur d'un legs ou d'une donation ont
	intérêt à en demander la révocation au juge judiciaire pour inexécution des charges dont il ou elle est grevé. De
	même, ces héritiers ont intérêt à demander l'annulation d'un acte administratif approuvant un contrat de bail
	emphytéotique permettant la constitution de droits réels immobiliers sur des biens donnés ou légués par le
	testateur à une fondation reconnue d'utilité publique, lorsque le legs ou la donation en cause est grevé de
	charges et encore susceptible, au moment de l'introduction de la demande, de faire l'objet d'une action en
	révocation pour inexécution, notamment au regard des règles de prescription applicables à une telle action.
	C Le bail à réhabilitation
	- limitation des preneurs potentiels art. L 252-1 : Est qualifié de bail à réhabilitation et soumis aux
	dispositions du présent chapitre le contrat par lequel soit un organisme d'habitations à loyer modéré, soit une
	société d'économie mixte dont l'objet est de construire ou de donner à bail des logements, soit une collectivité
	territoriale, soit un organisme bénéficiant de l'agrément relatif à la maîtrise d'ouvrage prévu à l'article L. 365-2
	s'engage à réaliser dans un délai déterminé des travaux d'amélioration sur l'immeuble du bailleur et à le
	conserver en bon état d'entretien et de réparations de toute nature en vue de louer cet immeuble à usage
	d'habitation pendant la durée du bail.
	- cession limitée à des organismes sociaux L 252-2
	- obligation de la conclusion d’une convention de l’article L 351-2, précisant les conditions du logement
	locatif conventionné éligible à l’APL L 252-3.
	- Dispositions spécifiques pour éviter le maintien de locataires dans les locaux objets du bail (articles L
	252-4 et L 252-5)
	- Absence de pouvoir du preneur sur les contrats en cours
	Cf Cass 3 e civ 8 mars 2006 Bull III \no 58 : Mais attendu que le bail à réhabilitation n'emporte pas, par lui-
	même, novation des baux d'habitation en cours par substitution de bailleur ; qu'ayant relevé qu'en vertu du
	contrat qui lui avait été consenti, la SEM avait pris à bail à réhabilitation les différents immeubles visés au
	contrat, dans les termes de l'article L. 252-1 à L. 252-4 du Code de la construction et de l'habitation, pour y
	effectuer des travaux d'amélioration en vue de les louer à usage exclusif d'habitation, que le contrat précisait à
	cet égard que le preneur pourrait procéder à la location des locaux réhabilités à des personnes présentant toutes
	garanties d'honorabilité et de solvabilité, qu'il ne prévoyait rien en ce qui concernait les baux en cours, la cour
	d'appel en a exactement déduit que la SEM n'avait pas qualité pour délivrer commandement de payer aux époux
	X... ni solliciter la résiliation de leur bail ;
	- Règles particulières en cas d’immeuble en copropriété : Article L252-1-161
	Par dérogation à l'article 23 de la loi \no  65-557 du 10 juillet 1965 précitée, si le bail à réhabilitation porte sur
	un ou plusieurs lots dépendant d'un immeuble soumis au statut de la copropriété, le preneur est de droit le
	mandataire commun prévu au second alinéa du même article. Par dérogation au troisième alinéa du I de
	l'article 22 de cette même loi, ce preneur peut recevoir plus de trois délégations de vote des bailleurs.
	Le preneur du bail à réhabilitation supporte seul, pendant la durée du bail, toutes les provisions prévues aux
	articles 14-1 et 14-2 de ladite loi.
	Le preneur mandataire commun doit disposer d'un mandat exprès du bailleur avant de voter sur les décisions
	relatives à des travaux de toute nature qui ne sont pas mis à la charge du preneur par le contrat de bail à
	réhabilitation et dont la prise en charge n'est pas prévue dans le bail à réhabilitation ou dont le paiement
	n'incombera pas à titre définitif au preneur.
	Le bail à réhabilitation précise la répartition des charges en fin de bail et le sort des avances et provisions
	appelées pendant la durée du bail à réhabilitation ainsi que des régularisations de charges intervenant après la
	fin du bail. Ces clauses sont inopposables au syndicat des copropriétaires
	- Régime fiscal moyen en fin de bail
	a. Améliorations -> ajout comme revenu 
neutralité
-> dépenses déduites
 art. 29 C.G.I.
b. Construction, reconstruction, agrandissement
Exonération en fin de bail, art 33 quinquiès CGI découlant de la loi du 29
juillet 1998
- Fiscalité classique des plus –values ; Dépenses d’amélioration comptées pour 0 : Cf R.M. du 29 déc. 1998
D Bail réel immobilier
1) Particularités de l’objet : « Art. L. 254-1
Constitue un contrat dénommé " bail réel immobilier " le bail par lequel un propriétaire personne physique ou
personne morale de droit privé consent, pour une longue durée, à un preneur, avec obligation de construire ou
de réhabiliter des constructions existantes, des droits réels en vue de la location ou de l'accession temporaire à
la propriété de logements :
1\degre  Destinés, pendant toute la durée du contrat, à être occupés, à titre de résidence principale, par des personnes
physiques dont les ressources n'excèdent pas des plafonds, fixés par décret en fonction de la typologie du
ménage, de la localisation et du mode d'occupation du logement, lesquels ne sauraient être inférieurs, pour les
logements donnés en location, aux plafonds prévus au chapitre unique du titre III du livre III ;
2\degre  Dont, pendant toute la durée du contrat, le prix d'acquisition ou, pour les logements donnés en location, le
loyer n'excède pas des plafonds fixés par décret en fonction de la localisation du logement, de son type et, le cas
échéant, de son mode de financement, lesquels ne sauraient être inférieurs, pour les logements donnés en
location, aux plafonds prévus au chapitre unique du titre III du livre III.
Ce contrat peut également être conclu par les collectivités territoriales, leurs groupements, leurs établissements
publics ainsi que par les établissements publics fonciers de l'Etat.
Il est régi par les dispositions du présent chapitre.
Le bail réel immobilier est consenti pour une durée de dix-huit à quatre-vingt-dix-neuf ans par les personnes qui
ont le droit d'aliéner. Il ne peut prévoir aucune faculté de résiliation unilatérale ni faire l'objet d'une tacite
reconduction.
Le bail réel immobilier oblige le preneur à effectuer les travaux mentionnés aux trois premiers alinéas sur
l'immeuble objet du bail, dans le respect des règles applicables à de telles opérations. Le preneur ne peut, sauf
stipulations contraires, exécuter d'autres ouvrages ou travaux que ceux prévus initialement.
Les constructions et améliorations réalisées par le preneur demeurent sa propriété en cours de bail et
deviennent la propriété du bailleur à son expiration. A l'issue du bail, le preneur ne peut réclamer, sauf
stipulations contraires, d'indemnité au regard des améliorations qu'il a effectuées.
Le contrat de bail détermine, le cas échéant, les activités accessoires qui pourront être exercées dans l'immeuble
objet du bail et subordonne à l'accord du bailleur tout changement d'activité.
Art. L. 2222-5-1 CG3P.-Un bien immobilier appartenant au domaine privé des collectivités territoriales, à leurs
groupements ou à leurs établissements publics ainsi qu'aux établissements publics fonciers de l'Etat peut faire
l'objet d'un bail réel immobilier prévu à l'article L. 254-1 du code de la construction et de l'habitation en vue de
la location ou de l'accession temporaire à la propriété de logements relevant du régime du logement
intermédiaire défini à l'article L. 302-16 du même code
Art. L. 254-8.-Les contrats conclus en méconnaissance de l'article L. 254-1 sont frappés de nullité. Les titulaires
et conditions d'exercice de l'action en nullité sont précisés par décret en Conseil d'Etat62
2)Particularités de la rémunération du bailleur : « Art. L. 254-3.-Le preneur s'acquitte du paiement d'une
redevance dont le montant tient compte des conditions d'occupation des logements, objet du bail réel immobilier.
Il ne peut se libérer de la redevance, ni se soustraire à l'exécution des conditions du bail réel immobilier en
délaissant l'immeuble.
« Le bail réel immobilier peut prévoir l'obligation pour le preneur de se libérer, par avance, du paiement de la
redevance, pour tout ou partie de la durée du bail.
« A défaut pour le preneur d'exécuter ses obligations contractuelles, notamment en cas de défaut de paiement de
la redevance non régularisé six mois après une mise en demeure signifiée par acte extrajudiciaire, le bailleur
peut demander la résiliation par le juge du bail réel immobilier. En cas de résiliation amiable ou par le juge, les
baux d'habitation conclus par le preneur sont transférés de plein droit au bailleur.
3)Particularités de location et de cession des droits réels : « Art. L. 254-4.-Lorsque le titulaire des droits réels
relatifs au logement, objet du bail réel immobilier, décide de le mettre en location, le contrat de location
reproduit en caractères apparents, sous peine de nullité, les dispositions de l'article L. 254-1 et du troisième
alinéa de l'article L. 254-2.
« Dans les baux qu'il consent en application de la loi \no  89-462 du 6 juillet 1989 tendant à améliorer les
rapports locatifs et portant modification de la loi \no  86-1290 du 23 décembre 1986 et des chapitres II et III du
titre III du livre VI du présent code, le preneur mentionne, en caractères apparents, la date d'extinction du bail
réel immobilier et son effet sur le contrat de bail en cours.
« A défaut, et par dérogation aux dispositions du troisième alinéa de l'article L. 254-2, les bénéficiaires du droit
au bail d'habitation consenti en application de la loi \no  89-462 du 16 juillet 1989 tendant à améliorer les
rapports locatifs et portant modification de la loi \no  86-1290 du 23 décembre 1986 ont le droit de se maintenir
dans les lieux pendant une durée de trente-six mois à compter de la date d'expiration du bail réel immobilier
moyennant une indemnité d'occupation égale au dernier loyer d'habitation expiré et payable dans les mêmes
conditions. Cette durée est réduite à douze mois pour les bénéficiaires de baux consentis en application des
chapitres II et III du titre III du livre VI du présent code.
« Art. L. 254-5.-Pour tout projet de cession des droits réels afférents aux logements, objet du bail réel
immobilier, l'acquéreur reçoit de la part du cédant une offre préalable d'acquisition mentionnant expressément,
en caractères apparents, le caractère temporaire du droit réel, sa date d'extinction, et reproduisant en termes
apparents les dispositions du présent chapitre.
« Le cédant est tenu de maintenir son offre préalable pour une durée de trente jours minimum à compter de sa
réception par l'acquéreur potentiel. Cette offre préalable ne peut être acceptée par l'acquéreur potentiel, par la
signature d'une promesse de vente ou d'une vente, avant un délai de dix jours à compter de sa réception.
« Les règles fixées aux alinéas précédents sont prescrites à peine de nullité de la vente.
« La preuve du contenu et de la notification de l'offre pèse sur le cédant.
« Art. L. 254-6.-Les dispositions des articles L. 271-1 à L. 271-3 relatives à la protection de l'acquéreur sont
applicables aux actes conclus en vue de l'acquisition des droits réels afférents aux logements, objet du bail réel
immobilier.
« Art. L. 254-7.-Sans préjudice des articles 515-6,763 et 764 du code civil et par dérogation à l'article L. 254-4,
les conditions de ressources définies à l'article L. 302-16 ne sont applicables ni aux transmissions successorales
des logements au conjoint survivant, quel que soit le régime matrimonial, ni au partenaire d'un pacte civil de
solidarité, prévu aux articles 515-1 à 515-7-1 du code civil, quel que soit le régime de ce pacte, ni aux personnes
mentionnées à l'article 14 de la loi \no  89-462 du 6 juillet 1989 tendant à améliorer les rapports locatifs et
portant modification de la loi \no  86-1290 du 23 décembre 1986, ni au conjoint survivant cotitulaire du bail dans
les conditions définies à l'article 1751 du code civil, lesquels pourront continuer à occuper les lieux.
E Bail réel solidaire
1-Caractère social de l’opération
Article L255-1
Constitue un contrat dénommé “bail réel solidaire” le bail par lequel un organisme de foncier solidaire consent
à un preneur, dans les conditions prévues à l'article L. 329-1 du code de l'urbanisme et pour une durée comprise
entre dix-huit et quatre-vingt-dix-neuf ans, des droits réels en vue de la location ou de l'accession à la propriété
de logements, avec s'il y a lieu obligation pour ce dernier de construire ou réhabiliter des constructions
existantes.
Ces logements sont destinés, pendant toute la durée du contrat, à être occupés à titre de résidence principale.
Article L255-263
Le bail réel solidaire peut être consenti à un preneur qui occupe le logement. Les plafonds de prix de cession des
droits réels et de ressources du preneur sont fixés par décret en Conseil d'Etat.
L'organisme de foncier solidaire peut, en fonction de ses objectifs et des caractéristiques de chaque opération,
appliquer des seuils inférieurs.
Le contrat de bail peut, en fonction de ses objectifs et des caractéristiques de chaque opération, prévoir que le
preneur doit occuper le logement objet des droits réels sans pouvoir le louer.
Article L255-3
Le bail réel solidaire peut être consenti à un opérateur qui, le cas échéant, construit ou réhabilite des logements
et qui s'engage à vendre les droits réels immobiliers attachés à ces logements à des bénéficiaires répondant aux
conditions de ressources fixées en application de l'article L. 255-2 et à un prix fixé en application du même
article, ou à proposer la souscription de parts ou actions permettant la jouissance du bien par ces bénéficiaires,
dans le respect des conditions prévues à l'article L. 255-2.
Dans le cas d'une vente, chacun des acquéreurs de droits réels immobiliers doit être agréé par l'organisme de
foncier solidaire dans les conditions et délais fixés aux articles L. 255-10 à L. 255-15.
Un bail réel solidaire portant sur les droits réels immobiliers acquis par chaque preneur est signé avec
l'organisme de foncier solidaire. Ces droits sont automatiquement retirés du bail réel solidaire initial conclu
entre l'opérateur et l'organisme de foncier solidaire. Lorsque la totalité des droits sont retirés du bail réel
solidaire initial, ce dernier s'éteint.
Article L255-4
Le bail réel solidaire peut être consenti à un opérateur qui, le cas échéant, construit ou réhabilite des logements
et qui s'engage à les mettre en location.
Les plafonds du loyer applicable et des ressources du locataire sont fixés par décret en Conseil d'Etat.
L'organisme de foncier solidaire peut, en fonction de ses objectifs et des caractéristiques de chaque opération,
imposer des seuils inférieurs.
Article L255-6
Le bail réel solidaire ne peut prévoir aucune faculté de résiliation unilatérale de la part du bailleur en dehors
des cas prévus au présent chapitre, ni faire l'objet d'une tacite reconduction
art R 255-1).Les plafonds du loyer sont ceux fixés pour les conventions portant sur les logements mentionnés à
l'article L. 351-2. L'ensemble des ressources du locataire, à la date d'entrée dans les lieux, est au plus égal à un
montant déterminé par l'arrêté conjoint des ministres chargés de la construction et de l'habitation et des finances
prévu à l'article R. 331-12, pour les logements mentionnés au I de l'article R. 331-1
2) Particularité de la rémunération
Article L 255-8
Le preneur s'acquitte du paiement d'une redevance dont le montant tient compte des conditions d'acquisition du
patrimoine par l'organisme de foncier solidaire et, le cas échéant, des conditions financières et techniques de
l'opération de construction ou de réhabilitation des logements et des conditions d'occupation des logements,
objet du bail réel solidaire. Il ne peut ni se libérer de la redevance, ni, s'il est l'opérateur mentionné aux articles
L. 255-3 et L. 255-4, se soustraire à l'exécution des conditions du bail réel solidaire en délaissant l'immeuble.
A défaut pour le preneur d'exécuter ses obligations contractuelles, notamment en cas de défaut de paiement de
la redevance, le bail est résilié, après indemnisation de la valeur des droits réels immobiliers tenant compte du
manquement ayant entraîné la résiliation du bail, selon les modalités prévues au contrat.
3-Particularité de la cession
- Prix de cession
Article L255-5
En cas de mutation, le prix de vente ou la valeur maximale des droits réels immobiliers, parts et actions
permettant la jouissance du bien sont limités à leur valeur initiale, actualisée selon des modalités définies par
décret en Conseil d'Etat.
L'acquéreur, le donataire ou l'ayant droit de la succession des droits réels immobiliers ou de parts ou actions64
donnant droit à l'attribution en jouissance du bien doit répondre aux conditions définies aux articles L. 255-2, L.
255-3 ou L. 255-4.
Article R 255-3
En cas de mutation, le prix de cession des droits réels, parts et actions permettant la jouissance du bien
n'excède pas le prix d'acquisition des droits réels, actualisé par application de la variation d'un indice choisi par
l'organisme de foncier solidaire, et majoré de la valorisation des travaux effectués entre l'acquisition et la
cession. Les modalités de valorisation et la nature des travaux sont déterminées par le bail réel solidaire liant le
preneur et l'organisme de foncier solidaire.
Le contrat de bail peut fixer une méthode d'évaluation du prix de cession des droits réels, parts et actions
permettant la jouissance du bien.
Le prix ainsi convenu ne peut excéder celui défini à l'article R. 255-1, lequel s'entend pour son montant, taxe sur
la valeur ajoutée en vigueur au moment de la mutation comprise
- Agrément
Article 255-10 : Pour tout projet de cession ou donation des droits réels afférents au bien objet du bail réel
solidaire, l'acquéreur ou donataire reçoit, de la part du cédant ou donateur, une offre préalable de cession ou de
donation mentionnant expressément le caractère temporaire du droit réel, sa date d'extinction, la nouvelle durée
du bail réel solidaire résultant de l'application de l'article L. 255-12 si l'organisme foncier solidaire agrée la
transmission des droits réels dans les conditions prévues à l'article L. 255-11, les conditions de délivrance de cet
agrément par l'organisme de foncier solidaire, les modalités de calcul du prix de vente ou de la valeur donnée,
telles que prévues au bail. L'offre reproduit les dispositions du présent chapitre.
Le cédant ou donateur est tenu de maintenir son offre préalable pour une durée de trente jours minimum à
compter de sa réception par l'acquéreur ou donataire potentiel. Cette offre ne peut être acceptée par l'acquéreur
ou donataire potentiel avant un délai de dix jours à compter de sa réception.
Le cédant ou donateur informe l'organisme de foncier solidaire de son intention de céder les droits réels
immobiliers qu'il tient de son bail réel solidaire, dans les trente jours qui suivent la réception par lui de
l'acceptation de l'offre préalable de cession ou donation des droits réels et sollicite l'agrément de l'acquéreur ou
donataire par l'organisme de foncier solidaire. Il joint à sa demande l'offre préalable de cession ou de donation,
les pièces permettant d'établir l'éligibilité de l'acquéreur ou du donataire choisi par lui, ainsi que le dossier de
diagnostic immobilier prévu à l'article L. 271-4 du code de la construction et de l'habitation.
Article L255-11
La vente ou la donation de droits réels afférents au bien objet du bail réel solidaire est subordonnée à
l'agrément de l'acquéreur ou du donataire par l'organisme de foncier solidaire.
L'organisme de foncier solidaire dispose d'un délai de deux mois à compter de la date de transmission de l'offre
préalable de cession ou de donation pour délivrer son agrément. Celui-ci est fondé sur la vérification du respect
des conditions d'éligibilité de l'acquéreur ou du donataire à la conclusion d'un bail réel solidaire définies à
l'article L. 255-2, L. 255-3 ou L. 255-4, de la conformité de l'offre préalable de cession ou de donation avec le
bail en vigueur, notamment du respect des stipulations concernant les modalités de calcul du prix de vente ou de
l'évaluation des droits réels appartenant au vendeur ou au donateur, et, le cas échéant, de la validité du plan de
financement de l'acquisition.
Les règles fixées aux alinéas précédents sont prescrites à peine de nullité de la vente ou de la donation. La
preuve du contenu et de la notification de l'offre préalable de vente ou de donation pèse sur le cédant ou le
donateur.
Article L255-12
Si cet agrément est délivré, la durée du bail est de plein droit prorogée afin de permettre à tout nouveau preneur
de bénéficier d'un droit réel d'une durée égale à celle prévue dans le contrat initial.
Article L255-13
En cas de refus d'agrément lors d'une cession, le cédant peut demander à l'organisme de foncier solidaire de lui
proposer un acquéreur répondant aux conditions d'éligibilité prévues à l'article L. 255-2, L. 255-3 ou L. 255-4.
Les conditions d'acquisition respectent les modalités de calcul du prix de vente stipulées dans le bail. Dans le
cas où l'organisme de foncier solidaire n'est pas en mesure de proposer un acquéreur dans les six mois suivant
la demande du cédant, il se porte acquéreur des droits réels afférents au bien objet du bail réel solidaire.
En cas de refus d'agrément lors d'une donation des droits réels immobiliers, le bail réel solidaire peut être
résilié conventionnellement et le preneur est indemnisé de la valeur de ses droits réels immobiliers, dans les65
conditions prévues par le bail.
Article L255-14
En cas du décès du preneur, les droits réels afférents au bien objet du bail réel solidaire sont transmis à son
ayant droit. Le bail fait l'objet de plein droit d'une prorogation de telle manière que l'ayant droit bénéficie d'un
bail d'une durée identique à celle prévue dans le bail initial, s'il répond aux conditions d'éligibilité mentionnées
à l'article L. 255-2, L. 255-3 ou L. 255-4.
Ces conditions d'éligibilité ne sont pas opposables au conjoint survivant, quel que soit le régime matrimonial, ou
au partenaire de pacte civil de solidarité.
Si l'ayant droit ne satisfait pas aux conditions d'éligibilité, il dispose d'un délai de douze mois à compter du
décès pour céder les droits réels afférents au bien objet du bail réel solidaire à un acquéreur répondant aux
conditions d'éligibilité susmentionnées et agréé par l'organisme de foncier solidaire. Ce délai peut être prorogé
par l'organisme de foncier solidaire pour une durée correspondant aux délais de la régularisation par acte
notarié de la cession des droits réels immobiliers. A défaut de cession dans ces délais, le bail réel solidaire est
résilié et l'ayant droit est indemnisé par l'organisme de foncier solidaire de la valeur de ses droits réels
immobiliers, dans les conditions prévues par le bail.
Article L255-15
L'organisme de foncier solidaire dispose d'un droit de préemption à son profit, mentionné dans le bail réel
solidaire, à l'occasion de toute cession ou donation. Dans ce cas, il peut racheter les droits réels afférents au
bien objet du bail réel solidaire ou les faire acquérir par un bénéficiaire répondant aux conditions d'éligibilité
définies à l'article L. 255-2. L'organisme de foncier solidaire fait connaître sa décision d'exercer son droit de
préemption dans un délai de deux mois à compter de la transmission de l'offre préalable de cession ou de
donation. Ce délai est prorogé d'un mois en cas de refus de l'agrément.
Le preneur est indemnisé de la valeur de ses droits réels immobiliers dans le respect des modalités de calcul du
prix de vente stipulées dans le bail et de la valeur maximale mentionnée à l'article L. 255-5.
Article L255-16
A l'expiration du bail, les droits réels immobiliers du preneur deviennent la propriété de l'organisme de foncier
solidaire après indemnisation de la valeur de ses droits réels immobiliers, dans les conditions prévues par le
bail et dans la limite de la valeur maximale mentionnée à l'article L. 255-5.
Dans les baux qu'il consent, le preneur du bail réel solidaire mentionne, en caractères apparents, la date
d'extinction du bail réel solidaire et son effet sur le contrat de bail en cours.
A défaut de cette mention, les bénéficiaires du droit au bail d'habitation ont le droit de se maintenir dans les
lieux pendant une durée de trente-six mois à compter de la date d'expiration du bail réel solidaire moyennant
une indemnité d'occupation égale au dernier loyer d'habitation expiré et payable dans les mêmes conditions.
Cette durée est réduite à douze mois pour les bénéficiaires de baux consentis en application des chapitres II et
III du titre III du livre VI du présent code.
Article R255-6
Le contrat de bail prévoit la liste des pièces devant être remises par le cédant ou donateur des droits réels
afférents au bien objet d'un bail réel solidaire pour solliciter l'agrément d'un acquéreur ou d'un donataire en
application de l'article L. 255-10.
En complément, dans le cas d'un acquéreur ou donataire souhaitant occuper le logement, l'organisme de foncier
solidaire sollicite auprès de l'acquéreur ou du donataire potentiel une copie des avis d'imposition ou de situation
déclarative établis au titre de l'avant-dernière année précédant celle de la transmission. Ces avis doivent
permettre d'apprécier la situation fiscale de l'ensemble des membres du ménage appelé à jouir des droits réels
du bien objet d'un bail réel solidaire.
L'organisme de foncier solidaire saisi d'une demande d'agrément vérifie la complétude des documents transmis.
Dans le cas où ces documents seraient incomplets, il peut suspendre le délai défini au deuxième alinéa de
l'article L. 255-11 par lettre recommandée avec demande d'avis de réception, ou par voie électronique, adressée
au donateur ou au cédant, ainsi qu'au donataire ou à l'acquéreur. La lettre précise les compléments à apporter.
Cette suspension est levée à la réception de ces documents par l'organisme de foncier solidaire.
Lors de la cession ou de la donation des droits réels, le contrat de bail réel solidaire est adapté s'il y a lieu pour
tenir compte du nouveau preneur et de l'usage du bien.
Article R255-7
En application de l'article L. 255-14, l'ayant droit précise à l'organisme son intention d'occuper ou de donner le
bien en location. Dans le premier cas, il transmet à l'organisme de foncier solidaire une copie de ses avis
d'imposition ou de situation déclarative établis au titre de l'avant-dernière année précédant celle de la66
transmission. Ces avis doivent permettre d'apprécier la situation fiscale de l'ensemble des membres du ménage
appelé à jouir des droits réels du bien objet d'un bail réel solidaire.
L'organisme de foncier solidaire saisi dispose d'un délai de deux mois pour agréer la transmission. Il vérifie la
complétude des documents transmis. Dans le cas où ces documents seraient incomplets, il peut suspendre le
délai par lettre recommandée avec demande d'avis de réception, ou par voie électronique, adressée à l'ayant
droit. La lettre précise les compléments à apporter à la transmission initiale. Cette suspension est levée à la
réception de ces documents par l'organisme de foncier solidaire.
En cas d'éligibilité, le contrat de bail réel solidaire est adapté pour tenir compte du nouveau preneur et de
l'usage du bien.
Section 3 L’acquisition d’un droit réel découlant d’un usufruit
§1 Bail avec convention d’usufruit
- L’objectif visé
- Les difficultés du montage
- Les solutions apportées par les articles L 253-1 et s CCH
Article L253-1
L'usufruit d'un logement ou d'un ensemble de logements peut être établi par convention au profit d'une personne
morale, pour une durée minimale de quinze années, en vue de la location de ce ou ces logements.
Article L253-1-1
I. ― La convention d'usufruit précise la répartition des dépenses de l'immeuble entre nu-propriétaire et
usufruitier. L'usufruitier supporte seul, pendant la durée de la convention, le paiement des provisions prévues
aux articles 14-1 et 14-2 de la loi \no  65-557 du 10 juillet 1965 fixant le statut de la copropriété des immeubles
bâtis qui lui incombent au titre de la convention.
II. ― Par dérogation à l'article 23 de la loi \no  65-557 du 10 juillet 1965 précitée, si la convention d'usufruit
porte sur un ou plusieurs lots dépendant d'un immeuble soumis au statut de la copropriété, l'usufruitier est de
droit le mandataire commun prévu au second alinéa du même article. Il bénéficie d'une délégation de vote pour
prendre les décisions mentionnées à l'article 24, et à l'article 25, à l'exclusion du n, et au c de l'article 26 de
cette même loi et dont, au titre de la convention d'usufruit, il assume seul la charge financière définitive. Il doit
obtenir un mandat exprès pour les autres décisions. Par dérogation au troisième alinéa du I de l'article 22 de
ladite loi, il peut recevoir plus de trois délégations de vote des nus-propriétaires. Lorsque la convention
d'usufruit porte sur l'intégralité des lots, l'usufruitier ne bénéficie pas de délégation de vote pour prendre la
décision mentionnée au c de l'article 25.
III. ― La convention d'usufruit précise la répartition des charges à son expiration, le sort des avances et
provisions appelées pendant la durée de la convention ainsi que les régularisations de charges intervenant après
l'extinction de l'usufruit. Ces clauses sont inopposables au syndicat des copropriétaires.
Article L253-2
Les logements dont l'usufruit est détenu par les bailleurs visés à l'article L. 253-1 peuvent être financés par des
prêts aidés dans des conditions définies par décret.
Ils peuvent faire l'objet d'une convention mentionnée à l'article L. 351-2, conclue pour une durée identique à
celle de l'usufruit.
Dans l'hypothèse où la convention mentionnée au deuxième alinéa du présent article est conclue, les articles L.
353-7 et L. 353-16 sont applicables aux locataires et occupants présents au moment de la conclusion de ladite
convention.
.
Article L253-3
Le bail doit expressément indiquer, de manière apparente, le statut juridique du logement, préciser le terme
ultime du contrat tel que prévu à l'article L. 253-4 et reproduire les termes des articles L. 253-5 à L. 253-7.
Article L253-4
Le bail conclu dans le cadre d'un usufruit prend fin de plein droit au plus tard à la date de l'extinction des droits
d'usufruit sur le bien loué.
Les deuxième à quatrième alinéas de l'article 595 du code civil ne s'appliquent pas aux baux soumis au présent
article.
Article L253-5
Six mois avant l'extinction de l'usufruit, le nu-propriétaire, par lettre recommandée avec demande d'avis de
réception, peut :
-soit, s'il est occupant, informer l'usufruitier de son intention de renouveler la convention d'usufruit ;67
-soit proposer au locataire un nouveau bail prenant effet au terme de l'usufruit, conforme aux dispositions de la
loi \no  89-462 du 6 juillet 1989 tendant à améliorer les rapports locatifs et portant modification de la loi \no  86-
1290 du 23 décembre 1986 ;
-soit donner congé pour vendre ou pour occuper le bien selon les modalités prévues à l'article 15 de la loi \no  89-
462 du 6 juillet 1989 précitée, avec effet au terme de l'usufruit. Le congé est dans ce cas valablement donné par
le seul nu-propriétaire au locataire.
La notification reproduit les termes du II de l'article L. 253-6 et de l'article L. 253-7.
Article L253-6
I. - Un an avant l'extinction de l'usufruit, le bailleur rappelle au nu-propriétaire et au locataire les dispositions
de l'article L. 253-5.
II. - Trois mois avant l'extinction de l'usufruit, le bailleur propose au locataire qui n'a pas conclu un nouveau
bail avec le nu-propriétaire et qui remplit les conditions de ressources fixées par décret la location d'un
logement correspondant à ses besoins et à ses possibilités.
Le non-respect par l'usufruitier-bailleur de cette obligation est inopposable au nu-propriétaire.
Article L253-7
Le locataire qui n'a pas conclu le contrat de location proposé par le nu-propriétaire ni accepté l'offre de
relogement faite par l'usufruitier-bailleur est déchu de tout titre d'occupation sur le logement à l'expiration de
l'usufruit.
Article L253-8
Les dispositions du présent chapitre sont d'ordre public.
- Les difficultés subsistantes
§2 Usufruit se substituant à un bail
Cf B Tournafond, L’usufruit social Et foncières nov 2008 p 4968
\part{Acquisition de droits sur le domaine public}
\section{Les conceptions d’origine}
Les droits des concessionnaires de voirie (contrat souvent accessoire d’une concession de service
public) et permissionnaires (titulaires d’une autorisation unilatérale) :
- droits précaires. Retrait possible. Droits réels discutés (cf Gaudemet \no 407)
CE 17 juin 2013 \No  352772 Ministre du budget : 3. Considérant qu'en adoptant ces dispositions,
notamment celles de l'article 3 de la loi du 25 juillet 1994, le législateur a entendu que le titulaire d'une
autorisation d'occupation du domaine public ne soit susceptible de détenir des droits réels que sur les
seuls ouvrages, constructions et installations de caractère immobilier qu'il a réalisés en vertu d'un titre
délivré, modifié ou renouvelé postérieurement à l'entrée en vigueur de la loi du 25 juillet 1994 ou qui,
autorisés par un titre antérieur, ont été réhabilités, rénovés ou étendus postérieurement à cette entrée
en vigueur, sous réserve de la délivrance d'un nouveau titre ; qu'il suit de là que la cour, en jugeant que
la société concessionnaire, qui était titulaire d'une autorisation d'occupation temporaire du domaine
public, était détentrice de droits réels sur les immeubles concédés, sans rechercher si elle avait,
conformément aux dispositions combinées de l'article L. 34-1 du code du domaine de l'Etat et de
l'article 3 de la loi du 25 juillet 1994, construit les ouvrages concernés, ou obtenu un nouveau titre
après avoir réhabilité, rénové ou étendu, postérieurement à l'entrée en vigueur de cette loi, a commis
une erreur de droit ; que les articles 1er, 2 et 3 de son arrêt doivent, dès lors, être annulés ;
- Biens de reprise ( propriété du concessionnaire dont le sort final doit être déterminé par le contrat) et
biens de retour (appartenant dès l’origine au concédant qui les met à disposition du concessionnaire)
CE 21 oct 2013 \No  358873 1. Considérant qu'il ressort des pièces du dossier soumis aux juges du fond
que le département des Bouches-du-Rhône a, par une convention du 23 décembre 1996, concédé à la
société d'économie mixte SEMIDEP la gestion du port maritime de commerce et de pêche de La Ciotat ;
qu'à l'issue de la vérification de comptabilité dont cette société a fait l'objet en 2008, l'administration
fiscale l'a assujettie à la taxe foncière sur les propriétés bâties à raison d'aménagements qu'elle avait
réalisés en sa qualité de concessionnaire sur les installations et constructions du domaine public
portuaire de cette commune ; que la société a demandé la décharge des cotisations de taxe foncière sur
les propriétés bâties auxquelles elle a été assujettie au titre de l'année 2009 au motif qu'elle n'en était
pas le redevable légal ; que le ministre du budget, des comptes publics et de la réforme de l'Etat, porte-
parole du Gouvernement, se pourvoit en cassation contre le jugement du 21 février 2012 par lequel le
tribunal administratif de Marseille a fait droit à la demande de la société ;
2. Considérant qu'aux termes de l'article 1400 du code général des impôts : " I. Sous réserve des
dispositions des articles 1403 et 1404, toute propriété, bâtie ou non bâtie, doit être imposée au nom du
propriétaire actuel. / II. Lorsqu'un immeuble est grevé d'usufruit ou loué soit par bail emphytéotique,
soit par bail à construction, soit par bail à réhabilitation ou fait l'objet d'une autorisation d'occupation
temporaire du domaine public constitutive d'un droit réel, la taxe foncière est établie au nom de
l'usufruitier, de l'emphytéote, du preneur à bail à construction ou à réhabilitation ou du titulaire de
l'autorisation. /... " ;
3. Considérant que, dans le cadre d'une délégation de service public ou d'une concession de travaux
mettant à la charge du cocontractant les investissements correspondant à la création ou à l'acquisition
des biens nécessaires au fonctionnement du service public, l'ensemble de ces biens, meubles ou
immeubles, appartient, dans le silence de la convention, dès leur réalisation ou leur acquisition, à la
personne publique ;
4. Considérant que, lorsque des ouvrages nécessaires au fonctionnement du service public, et ainsi
constitutifs d'aménagements indispensables à l'exécution des missions de ce service, sont établis sur la
propriété d'une personne publique, ils relèvent, de ce fait, du régime de la domanialité publique ; que la
faculté offerte aux parties au contrat d'en disposer autrement ne peut s'exercer, en ce qui concerne les
droits réels dont peut bénéficier le cocontractant sur le domaine public de l'Etat et de ses établissements
publics, que selon les modalités et dans les limites définies aux articles L. 34-1 à L. 34-8 du code du
domaine de l'Etat, issus de la loi du 25 juillet 1994 relative à la constitution de droits réels sur le
domaine public, puis aux articles L. 2122-6 à L. 2122-14 du code général de la propriété des personnes
publiques, à compter de l'entrée en vigueur le 1er juillet 2006 de ce code, et à condition que la nature et
l'usage des droits consentis ne soient pas susceptibles d'affecter la continuité du service public ;
5. Considérant qu'à l'expiration de la convention, les biens qui sont entrés, en application des principes
énoncés ci-dessus, dans la propriété de la personne publique et ont été amortis au cours de l'exécution
du contrat font nécessairement retour à celle-ci gratuitement, sous réserve des clauses contractuelles
permettant à la personne publique, dans les conditions qu'elles déterminent, de faire reprendre par son
cocontractant les biens qui ne seraient plus nécessaires au fonctionnement du service public ;69
6. Considérant que la circonstance que le contrat de concession prévoie le versement au
concessionnaire, à l'expiration de la concession, d'une indemnité destinée à compenser la valeur non
amortie des biens nécessaires au fonctionnement du service public ne fait nullement obstacle,
contrairement à ce que soutient le ministre, à ce que ces biens appartiennent, dès leur réalisation ou
leur acquisition, à la personne publique ;
7. Considérant que le tribunal administratif a relevé, par des motifs qui ne sont pas contestés, qu'il
n'était pas établi que la société SEMIDEP serait titulaire de droits réels sur le domaine public en vertu
du contrat de concession, ni que les ouvrages qu'elle avait réalisés auraient été affectés, en tout ou en
partie, à ses seuls besoins, ni qu'ils n'auraient pas été réalisés en vue de répondre aux besoins du
service public et ne constitueraient pas un accessoire indispensable du domaine public portuaire ; qu'en
jugeant, en conséquence, par un jugement suffisamment motivé, que ces ouvrages appartenaient dès
leur réalisation à l'autorité concédante, sans qu'y fasse obstacle la clause contractuelle prévoyant le
versement à la société SEMIDEP d'une indemnité égale à la valeur nette comptable des biens faisant
retour au département des Bouches-du-Rhône à l'expiration de la concession, le tribunal n'a pas
commis d'erreur de droit ;
- CE 29 juin 2018 \No  402251 Publié au recueil Lebon 3. Considérant, en premier lieu, que,
dans le cadre d'une concession de service public mettant à la charge du cocontractant les
investissements correspondant à la création ou à l'acquisition des biens nécessaires au
fonctionnement du service public, l'ensemble de ces biens, meubles ou immeubles, appartient, dans
le silence de la convention, dès leur réalisation ou leur acquisition à la personne publique ; que le
contrat peut attribuer au concessionnaire, pour la durée de la convention, la propriété des
ouvrages qui, bien que nécessaires au fonctionnement du service public, ne sont pas établis sur la
propriété d'une personne publique, ou des droits réels sur ces biens, sous réserve de comporter les
garanties propres à assurer la continuité du service public, notamment la faculté pour la personne
publique de s'opposer à la cession, en cours de concession, de ces ouvrages ou des droits détenus
par la personne privée ;
4. Considérant, en deuxième lieu, qu'à l'expiration de la convention, les biens qui sont entrés, en
application de ces principes, dans la propriété de la personne publique et ont été amortis au cours
de l'exécution du contrat font nécessairement retour à celle-ci gratuitement, sous réserve des
clauses contractuelles permettant à la personne publique, dans les conditions qu'elles déterminent,
de faire reprendre par son cocontractant les biens qui ne seraient plus nécessaires au
fonctionnement du service public ; que le contrat qui accorde au concessionnaire, pour la durée de
la convention, la propriété des biens nécessaires au service public autres que les ouvrages établis
sur la propriété d'une personne publique, ou certains droits réels sur ces biens, ne peut, sous les
mêmes réserves, faire obstacle au retour gratuit de ces biens à la personne publique en fin de
concession ;
5. Considérant, en troisième lieu, que lorsque la convention arrive à son terme normal ou que la
personne publique la résilie avant ce terme, le concessionnaire est fondé à demander
l'indemnisation du préjudice qu'il subit à raison du retour des biens à titre gratuit dans le
patrimoine de la collectivité publique, en application des principes énoncés ci-dessus, lorsqu'ils
n'ont pu être totalement amortis, soit en raison d'une durée du contrat inférieure à la durée de
l'amortissement de ces biens, soit en raison d'une résiliation à une date antérieure à leur complet
amortissement ; que lorsque l'amortissement de ces biens a été calculé sur la base d'une durée
d'utilisation inférieure à la durée du contrat, cette indemnité est égale à leur valeur nette
comptable inscrite au bilan ; que, dans le cas où leur durée d'utilisation était supérieure à la durée
du contrat, l'indemnité est égale à la valeur nette comptable qui résulterait de l'amortissement de
ces biens sur la durée du contrat ; que si, en présence d'une convention conclue entre une personne
publique et une personne privée, il est loisible aux parties de déroger à ces principes, l'indemnité
mise à la charge de la personne publique au titre de ces biens ne saurait en toute hypothèse
excéder le montant calculé selon les modalités précisées ci-dessus ;
6. Considérant que les règles énoncées ci-dessus, auxquelles la loi du 9 janvier 1985 n'a pas70
entendu déroger, trouvent également à s'appliquer lorsque le cocontractant de l'administration
était, antérieurement à la passation de la concession de service public, propriétaire de biens qu'il
a, en acceptant de conclure la convention, affectés au fonctionnement du service public et qui sont
nécessaires à celui-ci ; qu'une telle mise à disposition emporte le transfert des biens dans le
patrimoine de la personne publique, dans les conditions énoncées au point 3 ; qu'elle a également
pour effet, quels que soient les termes du contrat sur ce point, le retour gratuit de ces biens à la
personne publique à l'expiration de la convention, dans les conditions énoncées au point 4 ; que les
parties peuvent prendre en compte cet apport dans la définition de l'équilibre économique du
contrat, à condition que, eu égard notamment au coût que représenterait l'acquisition ou la
réalisation de biens de même nature, à la durée pendant laquelle les biens apportés peuvent être
encore utilisés pour les besoins du service public et au montant des amortissements déjà réalisés, il
n'en résulte aucune libéralité de la part de la personne publique ;
7. Considérant que, dans l'hypothèse où la commune intention des parties a été de prendre en
compte l'apport à la concession des biens qui appartenaient au concessionnaire avant la signature
du contrat par une indemnité, le versement d'une telle indemnité n'est possible que si l'équilibre
économique du contrat ne peut être regardé comme permettant une telle prise en compte par les
résultats de l'exploitation ; qu'en outre, le montant de l'indemnité doit, en tout état de cause, être
fixé dans le respect des conditions énoncées ci-dessus afin qu'il n'en résulte aucune libéralité de la
part de la personne publique
§ 1 – Les grandes lignes du nouveau code
- Distinction de la propriété qui relève des principes de droit privé et de la domanialité liée à l’affectation et qui
peut supporter un régime particulier lié à cette affectation
Régime fonctionnel de la domanialité
A - Restriction du champ du domaine public :
-
-
-
-
Par la définition même du domaine public : Article L. 2111-1
Sous réserve de dispositions législatives spéciales, le domaine public d'une personne publique
mentionnée à l'article L. 1 est constitué des biens lui appartenant qui sont soit affectés à l'usage
direct du public, soit affectés à un service public pourvu qu'en ce cas ils fassent l'objet d'un
aménagement indispensable à l'exécution des missions de ce service public.
Par la définition de l’accessoire : Article L. 2111-2 : Font également partie du domaine public les
biens des personnes publiques mentionnées à l'article L. 1 qui, concourant à l'utilisation d'un bien
appartenant au domaine public, en constituent un accessoire indissociable.
Cf CE 28 avril 2014 \No  349420 : 10. Considérant que l'exploitation des pistes de ski constitue un
service public industriel et commercial ; qu'aux termes de l'article L. 445-2 du code de l'urbanisme
alors en vigueur, désormais repris à l'article L. 473-1 du même code : " L'aménagement de pistes
de ski alpin est soumis à l'autorisation délivrée par l'autorité compétente en matière de permis de
construire " ; qu'une piste de ski alpin qui n'a pu être ouverte qu'en vertu d'une telle autorisation a
fait l'objet d'un aménagement indispensable à son affectation au service public de l'exploitation des
pistes de ski ; que, par suite, font partie du domaine public de la commune qui est responsable de
ce service public les terrains d'assiette d'une telle piste qui sont sa propriété ; qu'en vertu de
l'article L. 2111-2 du code général de la propriété des personnes publiques, le sous-sol de ces
terrains fait également partie du domaine public de la commune s'il comporte lui-même des
aménagements ou des ouvrages qui, concourant à l'utilisation de la piste, en font un accessoire
indissociable de celle-ci
CE 2 novembre 2015, \no  373896 2. Considérant qu'en vertu de l'article L. 2111-14 du code
général de la propriété des personnes publiques, le domaine public routier communal comprend
l'ensemble des biens appartenant à la commune et affectés aux besoins de la circulation terrestre, à
l'exception des voies ferrées ; que, selon l'article L. 2111-2 du même code, font également partie du
domaine public communal les biens de la commune qui, concourant à l'utilisation d'un bien
appartenant au domaine public, en constituent un accessoire indissociable ;71
3. Considérant qu'il ressort des énonciations de l'arrêt attaqué que, pour qualifier la parcelle
litigieuse de dépendance du domaine public communal, la cour, d'une part, après avoir relevé que
cette parcelle, propriété de la commune, était située à l'intersection de deux voies communales,
dans le prolongement des trottoirs bordant ces voies, sans obstacle majeur à la circulation des
piétons, en a déduit que cette parcelle était affectée aux besoins de la circulation terrestre ; que, s'il
lui appartenait de se prononcer sur l'existence, l'étendue et les limites du domaine public routier
communal, la cour, en statuant ainsi, sans rechercher si la commune avait affecté la parcelle en
cause aux besoins de la circulation terrestre, a commis une erreur de droit ; que la cour a, d'autre
part, jugé que la parcelle litigieuse constituait l'accessoire d'une dépendance du domaine public
routier ; que, toutefois, en ne recherchant pas si cette parcelle était indissociable du bien relevant du
domaine public dont elle était supposée être l'accessoire, la cour a méconnu les dispositions de
l'article L. 2111-2 du code général de la propriété des personnes publiques ; que, par suite et sans
qu'il soit besoin d'examiner les autres moyens du pourvoi, son arrêt doit être annulé ;
-
CE 26 janvier 2018 : 2. Avant l'entrée en vigueur, le 1er juillet 2006, du code général de la
propriété des personnes publiques, l'appartenance d'un bien au domaine public était subordonnée
à la condition que le bien ait été affecté au service public et spécialement aménagé en vue du
service public auquel il était destiné ou affecté à l'usage direct du public après, si nécessaire, son
aménagement.
3. Il résulte de l'instruction que la parcelle appartenant alors à la RATP et occupée par la société
Var Auto est située sur une dalle en béton recouvrant la voûte du tunnel permettant notamment le
passage de la ligne A du Réseau express régional sous l'avenue de Joinville à Nogent-sur-Marne.
Cette dalle n'est pas elle-même affectée à l'usage direct du public ou à une activité de service
public.
4. Par ailleurs, si le tunnel, y compris sa voûte, constitue un ouvrage d'art affecté au service public
du transport ferroviaire des voyageurs et spécialement aménagé à cet effet, il ne résulte pas de
l'instruction que la dalle de béton, qui est située physiquement au-dessus de la voûte du tunnel,
présente une utilité directe pour cet ouvrage, notamment sa solidité ou son étanchéité, et qu'elle en
constituerait par suite l'accessoire.
5. Il résulte de tout ce qui précède qu'à la date du 9 avril 1987, la parcelle en litige appartenait au
domaine privé de la RATP. Dès lors, sans qu'il soit besoin de se prononcer sur les autres moyens
de la requête, la société Var Auto est fondée à soutenir que c'est à tort que, par le jugement
attaqué, le tribunal administratif de Melun a jugé que la parcelle cadastrée section T nos 63 et 66,
située 13 avenue de Joinville à Nogent-sur-Marne, appartenait au domaine public de la RATP
-
Par l’article L 2211-1 Font partie du domaine privé les biens des personnes publiques mentionnées
à l'article L. 1, qui ne relèvent pas du domaine public par application des dispositions du titre Ier
du livre Ier.
Il en va notamment ainsi des réserves foncières et des biens immobiliers à usage de bureaux, à
l'exclusion de ceux formant un ensemble indivisible avec des biens immobiliers appartenant au
domaine public
Cf CE 11 déc 2008 Mme perreau pollier \no 309260 : les appartements du crédit municipal de Paris ne sont pas
des accessoires du domaine public
B - Limites à l’inaliénabilité du domaine public
Article L. 3211-2 : Les biens immobiliers à usage de bureaux mentionnés à l'article L. 2211-1, qui sont la
propriété de l'Etat, peuvent être aliénés alors qu'ils continuent à être utilisés par les services de l'Etat. Dans ce
cas, l'acte d'aliénation comporte des clauses permettant de préserver la continuité du service public.
C - Règles d’entrée dans la domaine public
Maintien de la domanialité publique virtuelle CE 1 er oct 2013 \No  349099
5. Considérant, en second lieu, d'une part, qu'il résulte de l'instruction qu'en vertu de la convention par laquelle
la commune d'Ozoir-la-Ferrière a confié la gestion de la résidence pour personnes âgées à l'AACHA, la72
commune disposait d'un pouvoir de contrôle sur la gestion financière de la résidence tant que la situation
financière de l'AACHA ne lui permettrait pas de faire face seule aux dépenses de gestion ; que la gestion de la
résidence devait être effectuée conformément à la convention tripartite conclue entre l'Etat, la commune et la
société anonyme d'habilitation à loyer modéré du personnel de la préfecture de police, conclue le 5 septembre
1986 ; que les budgets de la résidence devaient être approuvés par le conseil municipal ; qu'ainsi, eu égard à
l'intérêt général de sa mission, aux conditions de son organisation et de son fonctionnement, aux obligations qui
lui étaient imposées ainsi qu'aux mesures prises pour vérifier que les objectifs qui lui étaient assignés étaient
atteints, l'AACHA était chargée de l'exécution d'une mission de service public ;
6. Considérant, d'autre part, qu'à la date de la délibération autorisant le maire d'Ozoir-la-Ferrière à conclure
les conventions litigieuses, la commune avait prévu de manière certaine l'affectation du terrain à ce service
public, moyennant la réalisation des aménagements nécessaires à son exécution ;
7. Considérant, dès lors, que le terrain sur lequel la société anonyme d'habitation à loyer modéré du personnel
de la préfecture de police devait édifier la résidence pour personnes âgées litigieuse, qui devait être affecté à un
service public en vue duquel il serait spécialement aménagé, doit être regardé comme une dépendance du
domaine public communal ; que, par suite, le juge administratif est compétent pour connaître du présent litige
10. Considérant, d'une part, qu'eu égard à la date à laquelle l'ensemble contractuel litigieux a été conclu, la
commune ne pouvait légalement concéder à la société anonyme d'habitation à loyer modéré du personnel de la
préfecture de police un droit réel sur une dépendance de son domaine public ; qu'il suit de là que l'objet de
l'ensemble contractuel litigieux est illicite ;
CE 8 avril 2013 Association ALTARL \No  363738
2. Considérant que l'association soutient, à l'appui de son pourvoi, que le juge administratif des référés n'était
manifestement pas compétent pour statuer sur la demande d'expulsion dont il était saisi, dès lors que celle-ci
portait sur des parcelles qui, en l'absence d'aménagement spécial, n'auraient jamais fait partie du domaine
public ; que, toutefois, ainsi qu'il ressort des écritures des parties, avant la date à laquelle l'association a été
autorisée par l'Etat, par la convention du 15 juin 2005, à occuper les parcelles cadastrées section B \no  1081,
1480, 1072 et 1069 situées sur le territoire de la commune de Villeneuve-les-Béziers, ces parcelles avaient été
acquises par l'Etat en vue de la réalisation des travaux, déclarés d'utilité publique par décret du 30 mars 2000,
de raccordement de l'autoroute A75 à l'autoroute A9 aux abords de l'échangeur de Béziers Est ; qu'ainsi, la
personne publique avait prévu de manière certaine de réaliser les aménagements nécessaires ; que, par suite,
ces parcelles étaient soumises aux principes de la domanialité publique ; que la circonstance qu'elles n'aient
finalement pas été utilisées pour la réalisation des infrastructures de transport ainsi envisagées, ainsi qu'il
résulte d'une déclaration d'utilité publique modificative du 16 novembre 2007, est sans incidence, en l'absence
de décision de déclassement, sur leur appartenance au domaine public ; que, par suite, les emplacements
occupés par l'association, alors même qu'ils n'ont fait l'objet ni des aménagements projetés en 2000 ni d'autres
travaux d'aménagement ferroviaire, ne sont pas manifestement insusceptibles d'être qualifiés de dépendance du
domaine public dont le contentieux relève de la juridiction administrative ; que le moyen tiré de ce que le juge
des référés n'aurait manifestement pas été compétent pour statuer sur la demande du préfet de l'Hérault doit
donc être écarté;
CE 13 avril 2016 \No  391431 Cne de Baillargues : 3. Considérant que, quand une personne publique a pris la
décision d'affecter un bien qui lui appartient à un service public et que l'aménagement indispensable à
l'exécution des missions de ce service public peut être regardé comme entrepris de façon certaine, eu égard à
l'ensemble des circonstances de droit et de fait, telles que, notamment, les actes administratifs intervenus, les
contrats conclus, les travaux engagés, ce bien doit être regardé comme une dépendance du domaine public ;
4. Considérant que le tribunal administratif, devant lequel il n'était pas contesté que la commune avait pris la
décision d'affecter les terrains en cause au service public, a, par un motif qui n'est argué d'aucune dénaturation,
relevé que les travaux de réalisation du projet avaient été engagés ; qu'il résulte de ce qui a été dit au point 3 ci-
dessus qu'en jugeant que les terrains n'étaient pas incorporés au domaine public de la commune, sans
rechercher s'il résultait de l'ensemble des circonstances de droit et de fait, notamment des travaux dont il
constatait l'engagement, que l'aménagement indispensable à l'exécution des missions du service public auquel la
commune avait décidé d'affecter ces terrains pouvait être regardé comme entrepris de façon certaine, le tribunal
a commis une erreur de droit ; que, dès lors, la commune est fondée à demander l'annulation du jugement
qu'elle attaque ;
D – Règles de sortie du domaine public
1) règles normales de sortie73
Article L2141-1 : Un bien d'une personne publique mentionnée à l'article L. 1, qui n'est plus affecté à un service
public ou à l'usage direct du public, ne fait plus partie du domaine public à compter de l'intervention de l'acte
administratif constatant son déclassement.
Article L2141-3 : Par dérogation à l'article L. 2141-1, le déclassement d'un bien affecté à un service public
peut, afin d'améliorer les conditions d'exercice de ce service public, être prononcé en vue de permettre un
échange avec un bien d'une personne privée ou relevant du domaine privé d'une personne publique. Cet échange
s'opère dans les conditions fixées à l'article L. 3112-3.
-CE 23 mars 2011 \no  340089 Cté de communes du Queyras
Considérant qu’en faisant ainsi application des stipulations d’un contrat qui, telles qu’elles les a souverainement
interprétées, prévoyaient le transfert à une personne privée, sans désaffectation ni déclassement préalables, de la
propriété de dépendances du domaine public, sans relever d’office, eu égard au principe d’inaliénabilité de ce
domaine, le caractère illicite de leur contenu et en écarter l’application, la cour administrative d’appel de
Marseille a commis une erreur de droit ; qu’il y a lieu, par suite, pour ce motif, sans qu’il soit besoin d’examiner
les moyens du pourvoi incident, d’annuler son arrêt en tant qu’il statue sur ces trois demandes ;
2) déclassement anticipé : Ordonnance \no  2017-562 du 19 avril 2017 relative à la propriété des personnes
publiques
Article L. 2141-2 : « Par dérogation à l'article L. 2141-1, le déclassement d'un immeuble appartenant au
domaine public artificiel des personnes publiques et affecté à un service public ou à l'usage direct du public peut
être prononcé dès que sa désaffectation a été décidée alors même que les nécessités du service public ou de
l'usage direct du public justifient que cette désaffectation ne prenne effet que dans un délai fixé par l'acte de
déclassement. Ce délai ne peut excéder trois ans. Toutefois, lorsque la désaffectation dépend de la réalisation
d'une opération de construction, restauration ou réaménagement, cette durée est fixée ou peut être prolongée
par l'autorité administrative compétente en fonction des caractéristiques de l'opération, dans une limite de six
ans à compter de l'acte de déclassement. En cas de vente de cet immeuble, l'acte de vente stipule que celle-ci
sera résolue de plein droit si la désaffectation n'est pas intervenue dans ce délai. L'acte de vente comporte
également des clauses relatives aux conditions de libération de l'immeuble par le service public ou de
reconstitution des espaces affectés à l'usage direct du public, afin de garantir la continuité des services publics
ou l'exercice des libertés dont le domaine est le siège. »
Art. L. 3112-4. « -Un bien relevant du domaine public peut faire l'objet d'une promesse de vente ou
d'attribution d'un droit réel civil dès lors que la désaffectation du bien concerné est décidée par l'autorité
administrative compétente et que les nécessités du service public ou de l'usage direct du public justifient que
cette désaffectation permettant le déclassement ne prenne effet que dans un délai fixé par la promesse.
« A peine de nullité, la promesse doit comporter des clauses précisant que l'engagement de la personne publique
propriétaire reste subordonné à l'absence, postérieurement à la formation de la promesse, d'un motif tiré de la
continuité des services publics ou de la protection des libertés auxquels le domaine en cause est affecté qui
imposerait le maintien du bien dans le domaine public.
« La réalisation de cette condition pour un tel motif ne donne lieu à indemnisation du bénéficiaire de la
promesse que dans la limite des dépenses engagées par lui et profitant à la personne publique propriétaire. »
Article 11 de l’ordonnance de 2017 : « Les biens des personnes publiques qui, avant l'entrée en vigueur de la
présente ordonnance, ont fait l'objet d'un acte de disposition et qui, à la date de cet acte, n'étaient plus affectés à
un service public ou à l'usage direct du public peuvent être déclassés rétroactivement par l'autorité compétente
de la personne publique qui a conclu l'acte de disposition en cause, en cas de suppression ou de transformation
de cette personne, de la personne venant aux droits de celle-ci ou, en cas de modification dans la répartition des
compétences, de la personne nouvellement compétente.
Les dispositions des articles L. 3112-1 et L. 3112-2 du code général de la propriété des personnes publiques sont
applicables aux cessions et échanges entre personnes publiques réalisés antérieurement à l'entrée en vigueur de
l'ordonnance du 21 avril 2006 susvisée »
E - Distinction d’une occupation conforme et d’une occupation compatible
- Utilisation conforme à l'affectation (Chapitre Ier) : Article L. 2121-1
Les biens du domaine public sont utilisés conformément à leur affectation à l'utilité publique.
Aucun droit d'aucune nature ne peut être consenti s'il fait obstacle au respect de cette affectation.74
- Utilisation compatible avec l'affectation (Chapitre II)
Article L. 2122-1 : Nul ne peut, sans disposer d'un titre l'y habilitant, occuper une dépendance du domaine
public d'une personne publique mentionnée à l'article L. 1 ou l'utiliser dans des limites dépassant le droit
d'usage qui appartient à tous
Le titre mentionné à l'alinéa précédent peut être accordé pour occuper ou utiliser une dépendance du domaine
privé d'une personne publique par anticipation à l'incorporation de cette dépendance dans le domaine public,
lorsque l'occupation ou l'utilisation projetée le justifie.
Dans ce cas, le titre fixe le délai dans lequel l'incorporation doit se produire, lequel ne peut être supérieur à six
mois, et précise le sort de l'autorisation ainsi accordée si l'incorporation ne s'est pas produite au terme de ce
délai.
Article L. 2122-2 : L'occupation ou l'utilisation du domaine public ne peut être que temporaire.
Article L. 2122-3 : L'autorisation mentionnée à l'article L. 2122-1 présente un caractère précaire et révocable.
Article L. 2122-4 : Des servitudes établies par conventions passées entre les propriétaires, conformément à
l'article 639 du code civil, peuvent grever des biens des personnes publiques mentionnées à l'article L. 1, qui
relèvent du domaine public, dans la mesure où leur existence est compatible avec l'affectation de ceux de ces
biens sur lesquels ces servitudes s'exercent.
Impossible avant cf Cass. 1 ère Civ 2 mars 1994 Bull I \No  85 p. 66 RD Im 1994 644
Attendu que les biens du domaine public sont inaliénables ;
Attendu que pour accueillir la demande des époux Lavis en ce qu'elle était dirigée contre la société Escota, l'arrêt
énonce que les fonds dépendant du domaine public peuvent être grevés d'une servitude de passage à la condition
que l'exercice du passage ne soit pas incompatible avec la destination de ces biens et ne compromette pas le
service normal auquel ils sont affectés ; qu'il ajoute que le tracé préconisé par l'expert est le plus court et le moins
dommageable, étant implanté essentiellement sur des terrains longeant l'autoroute " sans aucun usage particulier
", et qu'il ne porte pas atteinte à la destination de l'ouvrage public, " l'intégrité de l'autoroute étant préservée " ;
Attendu qu'en se déterminant de la sorte, alors qu'il résulte du principe de l'inaliénabilité des biens du domaine
public qu'ils ne peuvent être grevés de servitudes légales de droit privé, et notamment d'un droit de passage en
cas d'enclave, la cour d'appel a violé le texte susvisé ;
Conventions d’exercice impossibles : CE8 nov 2019 \no  421491
1. Il ressort des pièces du dossier soumis aux juges du fond que l'association Club seynois multisport, qui a pour
objet le développement de la pratique du sport, notamment le tennis, par les habitants de la Seyne-sur-Mer, a
acquis diverses parcelles sur le territoire de cette commune et fait construire des bâtiments et installations en vue
de la pratique du tennis. Par un " acte administratif de cession amiable " du 25 mars 1975, les parcelles cadastrées
AK \no  338, 371 et 708 ont été cédées par l'association à la commune, cet acte prévoyant que l'ensemble de ces
parcelles, ainsi que l'extension future du complexe sportif seraient exclusivement réservées " aux activités de la
section tennis du club sportif municipal seynois ". Par un acte notarié du 31 mars 2010, les parcelles cadastrées
AK \no  339 et 381 ont également été cédées par l'association à la commune. L'ensemble de ces parcelles formant,
avec les parcelles AK \no  382, 383, 709 et 722, également propriété de la commune, le complexe tennistique
Barban. Par lettre du 27 août 2014, la commune a notifié à l'association son intention de ne pas renouveler, à son
échéance quinquennale, la convention du 31 mars 2010 par laquelle les parties avaient convenu des modalités de
la mise à disposition à l'association des équipements du complexe tennistique. L'association ayant refusé de signer
le projet de nouvelle convention qui lui était proposée par la commune, cette dernière a saisi le tribunal
administratif de Toulon de conclusions tendant à ce que soit ordonnée l'expulsion de l'association des dépendances
du domaine public communal qu'elle occupait sans droits ni titre. Par un jugement du 12 octobre 2017, ce tribunal
a enjoint l'association de libérer le complexe tennistique sous astreinte de 200 euros par jour de retard à
l'expiration d'un délai de six mois à compter de la notification de sa décision. L'association Club seynois multisport
se pourvoit en cassation contre l'arrêt du 13 avril 2018 par lequel la cour administrative d'appel de Marseille a
rejeté l'appel qu'elle avait formé contre ce jugement ainsi que sa demande tendant à ce qu'il soit sursis à son
exécution.
2. Lorsque le juge administratif est saisi d'une demande tendant à l'expulsion d'un occupant d'une dépendance
appartenant à une personne publique, il lui incombe, pour déterminer si la juridiction administrative est
compétente pour se prononcer sur ces conclusions, de vérifier que cette dépendance relève du domaine public à
la date à laquelle il statue. A cette fin, il lui appartient de rechercher si cette dépendance a été incorporée au
domaine public, en vertu des règles applicables à la date de l'incorporation, et, si tel est le cas, de vérifier en outre
qu'à la date à laquelle il se prononce, aucune disposition législative ou, au vu des éléments qui lui sont soumis,
aucune décision prise par l'autorité compétente n'a procédé à son déclassement. Avant l'entrée en vigueur de la75
partie législative du code général de la propriété des personnes publiques, intervenue le 1er juillet 2006,
l'appartenance d'un bien au domaine public était, sauf si ce bien était directement affecté à l'usage du public,
subordonnée à la double condition qu'il ait été affecté à un service public et spécialement aménagé en vue du
service public auquel il était destiné. Il résulte par ailleurs des dispositions de l'article L. 2111-1 de ce code que
le domaine public d'une personne publique est constitué des biens lui appartenant qui sont soit affectés à l'usage
direct du public, soit affectés à un service public pourvu qu'en ce cas ils fassent l'objet d'un aménagement
indispensable à l'exécution des missions de ce service public.
3. En premier lieu, la cour administrative d'appel a relevé, par des énonciations non contestées, que les parcelles
cadastrées AK \no  338, 371 et 708, acquises par la commune le 25 mars 1975 avaient été affectées au service public
communal d'activités sportives et de loisir alors qu'elles supportaient déjà, à cette date, des équipements
tennistiques constitutifs d'aménagements spéciaux. Elle a également relevé, par des énonciations non contestées,
que les parcelles cadastrées AK \no  381 et 339, acquises par la commune le 31 mars 2010, avaient été affectées à
ce même service public communal, la première supportant un bâtiment abritant un court de tennis constituant un
aménagement indispensable et la seconde un parc de stationnement à l'usage des utilisateurs des équipements
tennistiques constituant un accessoire indispensable pour l'exécution de ce service public. Par suite, la cour
administrative d'appel a pu en déduire, sans entacher son arrêt d'erreur de droit ni d'erreur de qualification
juridique des faits que ces parcelles, qui n'avaient fait l'objet d'aucune mesure de déclassement, constituaient des
dépendances du domaine public communal.
4. En deuxième lieu, la cour administrative d'appel n'a ni commis d'erreur de droit, ni méconnu la compétence de
la juridiction administrative en jugeant que la clause du contrat du 25 mars 1975 prévoyant que le complexe, ainsi
que son extension future, seraient exclusivement réservés aux activités de la section tennis de l'association, à
supposer qu'elle doive être interprétée comme emportant pour celle-ci un droit d'utilisation perpétuelle de ces
installations, était incompatible avec le régime de la domanialité publique. La cour a pu en déduire sans erreur
de droit que l'association ne pouvait tirer de cette clause, qui n'a en tout état de cause pas la nature d'une servitude
conventionnelle en l'absence de tout fonds servant ou dominant, un droit d'occupation des dépendances
domaniales en litige. Elle n'a pas davantage entaché son arrêt d'erreur de droit en s'abstenant de déduire de
l'incompatibilité de cette clause avec le régime de la domanialité publique qu'elle aurait fait obstacle à l'entrée
des parcelles en litige dans le domaine public communal.
5. En troisième lieu, après avoir relevé, par une appréciation souveraine non arguée de dénaturation, que le refus
de la commune de renouveler la convention conclue le 31 mars 2010 était devenu définitif et que l'association était
depuis le 31 mars 2015, date du terme de cette convention, dépourvue de tout titre d'occupation des parcelles en
cause, la cour n'a pas commis d'erreur de droit en jugeant que le tribunal administratif était tenu de faire droit à
la demande de la commune tendant à ce que soit ordonnée son expulsion des dépendances du domaine public
qu'elle occupait .
- Utilisation à titre onéreux : Cf Conseil Constitutionnel 1994 346 rec. P 96 ,Revue Française de Droit
Constitutionnel 1994 p 814
Cf toutefois CE 19 juillet 2011 : « Les dispositions de l’article 17 de la déclaration des droits de l’homme ne
concernent pas seulement la propriété privée des particuliers mais également la propriété de l’Etat et des autres
personnes publiques et font obstacle à ce que le domaine public puisse être durablement grevé de droits réels sans
contrepartie appropriée eu égard à la valeur réelle de ce patrimoine comme aux missions de service public
auxquelles il est affecté.
Considérant qu'il résulte des dispositions précitées de la loi du 9 décembre 1905 que les collectivités publiques
peuvent seulement financer les dépenses d'entretien et de conservation des édifices servant à l'exercice public d'un
culte dont elles sont demeurées ou devenues propriétaires lors de la séparation des Eglises et de l'Etat ou accorder
des concours aux associations cultuelles pour des travaux de réparation d'édifices cultuels et qu'il leur est interdit
d'apporter une aide à l'exercice d'un culte ; que les collectivités publiques ne peuvent donc, aux termes de ces
dispositions, apporter aucune contribution directe ou indirecte à la construction de nouveaux édifices cultuels ;
Considérant, toutefois, que, ainsi que l'a jugé la cour sans commettre d'erreur de droit, l'article L. 1311-2 du code
général des collectivités territoriales, dont la portée exacte sur ce point a été explicitée par l'ordonnance précitée
du 21 avril 2006, a ouvert aux collectivités territoriales la faculté, dans le respect du principe de neutralité à
l'égard des cultes et du principe d'égalité, d'autoriser un organisme qui entend construire un édifice du culte ouvert
au public à occuper pour une longue durée une dépendance de leur domaine privé ou de leur domaine public,
dans le cadre d'un bail emphytéotique, dénommé bail emphytéotique administratif et soumis aux conditions
particulières posées par l'article L. 1311-3 du code général des collectivités territoriales ; que le législateur a
ainsi permis aux collectivités territoriales de conclure un tel contrat en vue de la construction d'un nouvel édifice
cultuel, avec pour contreparties, d'une part, le versement, par l'emphytéote, d'une redevance qui, eu égard à la76
nature du contrat et au fait que son titulaire n'exerce aucune activité à but lucratif, ne dépasse pas, en principe,
un montant modique, d'autre part, l'incorporation dans leur patrimoine, à l'expiration du bail, de l'édifice
construit, dont elles n'auront pas supporté les charges de conception, de construction, d'entretien ou de
conservation ; qu'il a, ce faisant, dérogé aux dispositions précitées de la loi du 9 décembre 1905 ;
F Application dans le temps du nouveau code
cf (CE 3 oct. 2012, Commune de Port-Vendres, req. \no  353915 : Considérant, en premier lieu, qu'avant l'entrée
en vigueur, le 1er juillet 2006, du code général de la propriété des personnes publiques, l'appartenance au
domaine public d'un bien était, sauf si ce bien était directement affecté à l'usage du public, subordonnée à la
double condition que le bien ait été affecté au service public et spécialement aménagé en vue du service public
auquel il était destiné ; qu'en l'absence de toute disposition en ce sens, l'entrée en vigueur de ce code n'a pu, par
elle-même, avoir pour effet d'entrainer le déclassement de dépendances qui appartenaient antérieurement au
domaine public et qui, depuis le 1er juillet 2006, ne rempliraient plus les conditions désormais fixées par son
article L. 2111-1
CE 8 avril 2013 Association ALTARL \No  363738
1. Considérant qu'avant l'entrée en vigueur, le 1er juillet 2006, du code général de la propriété des personnes
publiques, l'appartenance d'un bien au domaine public était, sauf si ce bien était directement affecté à l'usage du
public, subordonnée à la double condition que le bien ait été affecté au service public et spécialement aménagé
en vue du service public auquel il était destiné ; que le fait de prévoir de façon certaine un tel aménagement du
bien concerné impliquait que celui-ci était soumis, dès ce moment, aux principes de la domanialité publique ;
qu'en l'absence de toute disposition en ce sens, l'entrée en vigueur de ce code n'a pu, par elle-même, avoir pour
effet d'entraîner le déclassement de dépendances qui, n'ayant encore fait l'objet d'aucun aménagement,
appartenaient antérieurement au domaine public en application de la règle énoncée ci-dessus, alors même qu'en
l'absence de réalisation de l'aménagement prévu, elles ne rempliraient pas l'une des conditions fixées depuis le
1er juillet 2006 par l'article L. 2111-1 du code général de la propriété des personnes publiques qui exige, pour
qu'un bien affecté au service public constitue une dépendance du domaine public, que ce bien fasse déjà l'objet
d'un aménagement indispensable à l'exécution des missions de ce service public ;
Solution identique : CE 28 juillet 2017 \No  395911
G Contentieux :
CE 29 avril 2013 \No  364058 :5. Considérant que les contestations portant sur le contrat de vente d'un bien
appartenant au domaine privé d'une personne publique doivent, sauf dispositions législatives contraires et dès
lors que ce contrat ne comporte pas de clause exorbitante du droit commun, être portées devant le juge judiciaire
; que, d'une part, s'il est soutenu que ce contrat est entaché de nullité au motif que ce bien appartenait au domaine
public de cette personne publique, cette allégation, dans le cas où elle présenterait un caractère sérieux, justifierait
le renvoi par le juge judiciaire de cette question au juge administratif, seul compétent pour y répondre, mais ne
saurait avoir pour effet de donner compétence à la juridiction administrative pour statuer sur la validité de ce
contrat ; que, d'autre part, les servitudes créées par un acte du même jour que l'acte de cession du 21 décembre
2011 relatives, d'une part, à la circulation des piétons sur l'esplanade, d'autre part, à l'évacuation des usagers des
transports publics par la sortie de secours du RER, ne sauraient manifestement être regardées comme constituant
des clauses exorbitantes du droit commun de nature à fonder la compétence du juge administratif des référés ;
que, dès lors, il n'appartient pas à la juridiction administrative de connaître de la demande du syndicat
d'agglomération nouvelle du Val d'Europe et de la commune de Chessy tendant à ce que soit suspendue l'exécution
du contrat de vente ; qu'ainsi, en statuant sur la demande des requérants, le juge des référés a méconnu l'étendue
de la compétence du juge administratif ; que, par suite et sans qu'il soit besoin d'examiner les moyens du pourvoi
sur ces conclusions, son ordonnance doit, dans cette mesure, être annulée ;
§2 La problématique des montages en volume sur le domaine public.
CE 11 fév. 1994 JCP 1994 II 22338 Cie d’assurance la préservatrice foncière
Considérant que les règles essentielles du régime de la copropriété telles qu'elles sont fixées par la loi du 10 juillet
1965, et notamment la propriété indivise des parties communes, - au nombre desquelles figurent, en particulier,
outre le gros oeuvre de l'immeuble, les voies d'accès, passages et corridors -, la mitoyenneté présumée des cloisons
et des murs séparant les parties privatives, l'interdiction faite aux copropriétaires de s'opposer à l'exécution, même
à l'intérieur de leurs parties privatives, de certains travaux décidés par l'assemblée générale des copropriétaires
se prononçant à la majorité, la garantie des créances du syndicat des copropriétaires à l'encontre d'un77
copropriétaire par une hypothèque légale sur son lot, sont incompatibles tant avec le régime de la domanialité
publique qu'avec les caractères des ouvrages publics ; que, par suite, des locaux acquis par l'Etat, fût-ce pour les
besoins d'un service public, dans un immeuble soumis au régime de la copropriété n'appartiennent pas au domaine
public et ne peuvent être regardés comme constituant un ouvrage public ; que, par conséquent, les dommages qui
trouveraient leur source dans l'aménagement ou l'entretien de ces locaux ne sont pas des dommages de travaux
publics ;
Cass 3 e civ 25 fev 2009 \no  07-15772: Qu'en statuant ainsi après avoir relevé qu'il résultait de l'article 5 du cahier
des charges de l'adjudication que les portiques resteraient toujours du domaine public, ce dont il résultait qu'il
s'agissait d'un bien appartenant au domaine public avant la division de l'immeuble par lots, alors que les biens du
domaine public sont imprescriptibles et inaliénables et qu'un règlement de copropriété ne peut soustraire au
domaine public d'une commune un ouvrage public préexistant à la copropriété, la cour d'appel a violé les articles
susvisés ;
§ 3 – La possibilité de se faire consentir des droits réels temporaires sur le domaine public :
Avant la réforme : Conseil d'Etat Chambres réunies 11 Mai 2016 \No  390118
5. Considérant que, contrairement à ce que relève l'arrêt attaqué, le droit réel dont bénéficie, en vertu de
l'article L. 34-1 du code du domaine de l'Etat, repris à l'article L. 2122-6 du code général de la propriété des
personnes publiques, le titulaire d'une autorisation d'occupation temporaire du domaine de l'Etat, ne porte pas
uniquement sur les ouvrages, constructions et installations que réalise le preneur mais inclut le terrain d'assiette
de ces constructions ; que, par suite, en jugeant que la convention, qualifiée par les parties de bail à
construction, conclue le 21 mars 2005 sur le domaine public du port autonome de Marseille était incompatible
avec les règles de gestion du domaine public au seul motif qu'un bail à construction confère au preneur, en vertu
des dispositions citées ci-dessus, des droits réels sur le sol au-delà des seuls " constructions, ouvrages et
installations " mentionnés à l'article L. 34-1 du code du domaine de l'Etat, la cour administrative d'appel de
Marseille a commis une erreur de droit ;
6. Considérant qu'il résulte de ce qui précède, sans qu'il soit besoin d'examiner les autres moyens du pourvoi,
que la communauté urbaine Marseille-Provence-Métropole est fondée à demander l'annulation de l'arrêt du 12
mars 2015 en tant que, par celui-ci, la cour a annulé la délibération \no  AGER 001 et rejeté le surplus de ses
conclusions d'appel ;
11. Considérant qu'avant l'entrée en vigueur, le 1er juillet 2006, du code général de la propriété des personnes
publiques, l'appartenance d'un bien au domaine public était, sauf si ce bien était directement affecté à l'usage du
public, subordonnée à la double condition que le bien ait été affecté au service public et spécialement aménagé
en vue du service public auquel il était destiné ; que le fait de prévoir de façon certaine de réaliser un tel
aménagement impliquait que le bien concerné était soumis, dès ce moment, aux principes de la domanialité
publique ; qu'en l'absence de toute disposition en ce sens, l'entrée en vigueur de ce code n'a pu, par elle-même,
avoir pour effet d'entraîner le déclassement de dépendances qui, n'ayant encore fait l'objet d'aucun
aménagement, appartenaient antérieurement au domaine public en application de la règle énoncée ci-dessus,
alors même qu'en l'absence de réalisation de l'aménagement prévu, elles ne rempliraient pas l'une des
conditions fixées depuis le 1er juillet 2006 par l'article L. 2111-1 du code général de la propriété des personnes
publiques qui exige, pour qu'un bien affecté au service public constitue une dépendance du domaine public, que
ce bien fasse déjà l'objet d'un aménagement indispensable à l'exécution des missions de ce service public ;
12. Considérant par ailleurs que la condition d'affectation au service public est regardée comme remplie alors
même que le service public en cause est géré par une collectivité publique différente de la collectivité publique
qui est propriétaire ;
13. Considérant qu'il ressort des pièces du dossier que le terrain sur lequel a été implantée l'unité de traitement
des déchets ménagers et assimilés, qui appartenait au domaine privé du port autonome de Marseille, a été
affecté au service public du traitement des déchets ménagers et assimilés par la communauté urbaine Marseille-
Provence-Métropole, ainsi que le révèlent les stipulations de la convention qualifiée de bail à construction
approuvée par une délibération du 9 juillet 2004 et conclue le 21 mars 2005, qui prévoient expressément que le
preneur exercera exclusivement un ensemble d'activités industrielles liées aux traitements thermiques et
biologiques des déchets ménagers et assimilés avec valorisation énergétique ; que, par suite, le terrain sur
lequel a été édifiée l'unité de traitement des déchets était entré dans le domaine public du port autonome de78
Marseille dès la conclusion de la convention, soit le 21 mars 2005, nonobstant la circonstance que le service
public auquel il a été affecté est géré par la communauté urbaine Marseille-Provence-Métropole ;
14. Considérant que si la constitution de droits réels sur le domaine public de l'Etat suppose en principe la
délivrance d'une autorisation temporaire d'occupation du domaine public, aucune disposition ni aucun principe
n'interdit que l'Etat et ses établissements publics puissent autoriser l'occupation d'une dépendance du domaine
public en vertu d'une convention par laquelle l'une des parties s'engage, à titre principal, à édifier des
constructions sur le terrain de l'autre partie et à les conserver en bon état d'entretien pendant toute la durée de
la convention et qui, comme les autorisations d'occupation constitutives de droits réels, confère un droit réel
immobilier, à condition toutefois que les clauses de la convention ainsi conclue respectent, ainsi que le prévoit
l'article L. 34-5 du code du domaine de l'Etat, repris à l'article L. 2122-11 du code général de la propriété des
personnes publiques cité au point 4, les dispositions applicables aux autorisations d'occupation temporaires du
domaine public de l'Etat constitutives de droits réels, qui s'imposent aux conventions de toute nature ayant pour
effet d'autoriser l'occupation du domaine public ;
15. Considérant qu'il ressort des pièces du dossier que la convention qualifiée de bail à construction conclue le
21 mars 2005, annexée à la délibération attaquée, a une durée de soixante-dix ans, qui n'excède pas la durée
fixée par l'article L. 34-1 du code du domaine de l'Etat, repris à l'article L. 2122-6 du code général de la
propriété des personnes publiques ; que, cependant, elle permet au preneur de consentir des servitudes passives
indispensables à la réalisation des ouvrages, constructions et installations prévues par la convention, toute autre
servitude ne pouvant être conférée qu'avec le consentement du port autonome, alors qu'antérieurement à l'entrée
en vigueur du code général de la propriété des personnes publiques, aucune servitude ne pouvait être
valablement instituée sur le domaine public ; que les stipulations de cette convention ne soumettent pas à
l'agrément préalable de l'autorité compétente la cession des droits réels conférés par le contrat, ainsi que l'exige
l'article L. 34-2 du code du domaine de l'Etat, repris à l'article L. 2122-7 du code général de la propriété des
personnes publiques, mais se bornent à prévoir la notification d'une telle cession à cette autorité ; qu'elles
autorisent le titulaire à grever son droit à la convention, ainsi que les constructions, ouvrages et installations
qu'il aura édifiés, de privilèges et d'hypothèques, sans prévoir, conformément à l'article L. 34-2 du code du
domaine de l'Etat, repris à l'article L. 2122-8 du code général de la propriété des personnes publiques, que ces
hypothèques pourront seulement servir à garantir les emprunts contractés en vue de financer la réalisation, la
modification ou l'extension des ouvrages, constructions et installations de caractère immobilier situés sur la
dépendance domaniale occupée ; qu'elles prévoient également que, pour le financement des ouvrages, le
délégataire de service public, ou cessionnaire, pourra notamment faire appel à un crédit-bail, alors que l'article
L. 34-7 du code du domaine de l'Etat, applicable à la date de la conclusion de la convention, interdit la
conclusion de tels contrats pour la réalisation des ouvrages, constructions et installations affectés à un service
public et aménagés à cet effet ou affectés directement à l'usage du public ainsi que des travaux exécutés pour
une personne publique dans un but d'intérêt général ; qu'ainsi, cette convention, d'une part, ne comporte pas
toutes les clauses requises par les dispositions du code du domaine de l'Etat et reprises par le code général de la
propriété des personnes publiques, applicables aux autorisations d'occupation temporaire du domaine public de
l'Etat constitutives de droits réels et aux conventions de toute nature ayant pour effet d'autoriser l'occupation du
domaine public de l'Etat, de nature à garantir l'utilisation du domaine public conformément à son affectation à
l'utilité publique et, d'autre part, contient des clauses incompatibles avec le droit du domaine public avant sa
modification par le code général de la propriété des personnes publiques ; qu'il suit de là que l'acte de cession
de la convention conclue le 21 mars 2005 est illégal
Article R 2122-1 L'autorisation d'occupation ou d'utilisation du domaine public peut être consentie, à titre
précaire et révocable, par la voie d'une décision unilatérale ou d'une convention
Règles pour l’Etat : cf articles L 2122-6
Règles pour les collectivités territoriales : Article L. 2122-20
Les collectivités territoriales, leurs groupements et leurs établissements publics peuvent :
1\degre  Soit conclure sur leur domaine public un bail emphytéotique administratif dans les conditions déterminées
par les articles L. 1311-2 à L. 1311-4-1 du code général des collectivités territoriales ;
2\degre  Soit délivrer des autorisations d'occupation constitutives de droit réel dans les conditions déterminées par les
articles L. 1311-5 à L. 1311-8 du code général des collectivités territoriales.
Cf Article L 1311-13 : Les maires, les présidents des conseils généraux et les présidents des conseils régionaux,
les présidents des établissements publics rattachés à une collectivité territoriale ou regroupant ces collectivités
et les présidents des syndicats mixtes sont habilités à recevoir et à authentifier, en vue de leur publication au
bureau des hypothèques, les actes concernant les droits réels immobiliers ainsi que les baux, passés en la forme
administrative par ces collectivités et établissements publics.79
Lorsqu'il est fait application de la procédure de réception et d'authentification des actes mentionnée au premier
alinéa, la collectivité territoriale ou l'établissement public partie à l'acte est représenté, lors de la signature de
l'acte, par un adjoint ou un vice-président dans l'ordre de leur nomination.
Cf CE 18 nov 2015 \No  390461 SCI LES II C et autres
6. Considérant, en troisième lieu, qu’aux termes de l’article L. 2122-21 du code général des collectivités
territoriales : « Sous le contrôle du conseil municipal et sous le contrôle administratif du représentant de l’Etat
dans le département, le maire est chargé, d’une manière générale, d’exécuter les décisions du conseil municipal
et, en particulier : / 1\degre  De conserver et d’administrer les propriétés de la commune et de faire, en conséquence,
tous actes conservatoires de ses droits (...) » ; qu’il résulte de ces dispositions que, s’il appartient au conseil
municipal de délibérer sur les conditions générales d’administration et de gestion du domaine public communal,
le maire est seul compétent pour délivrer les autorisations d’occupation du domaine public ; qu’il est également
compétent, sur le fondement de ces mêmes dispositions, pour les retirer ou les abroger ; que, par suite, le juge
des référés n’a pas commis d’erreur de droit en jugeant que le moyen tiré de ce que le maire n’aurait pas été
compétent pour abroger l’autorisation d’occupation du domaine public n’était pas de nature à faire naître un
doute sérieux sur la légalité de l’arrêté du 17 février 2015 ;
A – Les conditions de fond de la constitution du droit réel.
1\degre ) Conditions des baux emphytéotiques des collectivités (applicables tant aux communes qu’aux
établissements publics des collectivités territoriales et aux groupements de ces collectivités)
Un bien immobilier appartenant à une collectivité territoriale peut faire l'objet d'un bail emphytéotique prévu à
l'article L. 451-1 du code rural et de la pêche maritime en vue de la réalisation d'une opération d'intérêt général
relevant de sa compétence ou en vue de l'affectation à une association cultuelle d'un édifice du culte ouvert au
public. Ce bail emphytéotique est dénommé bail emphytéotique administratif.
Un tel bail peut être conclu même si le bien sur lequel il porte, en raison notamment de l'affectation du bien
résultant soit du bail ou d'une convention non détachable de ce bail, soit des conditions de la gestion du bien ou
du contrôle par la personne publique de cette gestion, constitue une dépendance du domaine public, sous
réserve que cette dépendance demeure hors du champ d'application de la contravention de voirie.
Un tel bail ne peut avoir pour objet l'exécution de travaux, la livraison de fournitures, la prestation de services,
ou la gestion d'une mission de service public, avec une contrepartie économique constituée par un prix ou un
droit d'exploitation, pour le compte ou pour les besoins d'un acheteur soumis à l'ordonnance \no  2015-899 du 23
juillet 2015 relative aux marchés publics ou d'une autorité concédante soumise à l'ordonnance \no  2016-65 du 29
janvier 2016 relative aux contrats de concession.
Dans le cas où un tel bail serait nécessaire à l'exécution d'un contrat de la commande publique, ce contrat
prévoit, dans le respect des dispositions du présent code, les conditions de l'occupation du domaine.
Application de ce texte par TA Grenoble 1 er oct 2010 Préfet de la Drôme AJDA 2011 510 Une opération
financière sur les locaux d’une gendarmerie ne présente pas d’intérêt général
CAA Marseille 29 octobre 2012 \No  10MA02128 sur l’exception de voirie routière
4. Considérant qu'il résulte des motifs de la délibération attaquée que les contrats qu'elle autorise le maire à
signer s'inscrivent dans la droite ligne de l'adoption, par délibération en date du 16 février 2005, d'un nouveau
plan de circulation de la ville " pour assurer un meilleur fonctionnement urbain, améliorer la circulation
automobile et piétonne, créer des liaisons inter-quartiers, obtenir une plus grande hiérarchisation du réseau et,
enfin améliorer les conditions de stationnement en centre-ville " et qu'ils " participent de la mise en oeuvre du
plan de circulation par la création d'un grand parc public de stationnement de 381 places visant à satisfaire les
besoins actuels et à venir de la commune " ; que la délibération attaquée autorise en particulier le maire à
donner à bail emphytéotique administratif un lot de volume \no  1000 afin de réaliser un parc de stationnement
public souterrain comprenant 340 places au niveau N-1 et 41 places au niveau N-2 destiné à être mis à
disposition de la commune dans les conditions fixées par la convention de mise à disposition à la commune qui
l'accompagne, laquelle précise que le bien en cause ne peut avoir d'autre affectation que celle de parc public de
stationnement ; que toutefois, la seule circonstance que ce parc est partie intégrante d'un nouveau plan de
circulation de la ville, dont les objectifs sont d'améliorer la fluidité du trafic et la sécurité des usagers, ne permet
pas de considérer que ledit parc est intégré dans le domaine public routier et que le terrain, objet des contrats,
sur lequel est projetée la réalisation d'un programme mixte de logements, revêt le caractère d'une dépendance
de la voirie routière ; que, par suite, la société Cinergie est fondée à soutenir que c'est à tort que, par le
jugement attaqué, le tribunal administratif de Toulon a annulé la délibération précitée du 15 mai 2007 au motif80
qu'elle aurait autorisé la passation d'un bail emphytéotique sur un bien constituant une dépendance du domaine
public routier
CAA de BORDEAUX 2 février 2017 \No  14BX02682,14BX02684
10. Il résulte de l'instruction que l'objet du contrat que la commune de Rieumes se proposait de conclure au
moyen d'un bail emphytéotique était de confier à l'OPH de Haute-Garonne la construction d'une maison de
retraite médicalisée sur un terrain d'une superficie de 9 053 mètres carrés relevant de son domaine privé. Eu
égard à l'intérêt général communal de l'opération, celle-ci présentait ainsi le caractère d'un bail emphytéotique
administratif au sens des dispositions précitées de l'article L. 1311-2 du code général des collectivités
territoriales.
11. Le projet de la commune, tel qu'il a été instruit par les autorités compétentes et examiné par le comité
régional d'organisation sanitaire et sociale dans sa séance du 17 septembre 2002, portait précisément sur la
création d'un établissement d'hébergement pour personnes âgées dépendantes d'une capacité de 65 lits dont 15
pour personnes désorientées répartis dans 61 chambres individuelles et deux chambres doubles. Après sa
construction, l'ouvrage devait être mis à disposition d'un établissement public local créé spécialement par la
commune afin de l'exploiter puis devenir la propriété de la commune au terme du bail. La rémunération de
l'office public d'habitat consistait dans la perception d'un loyer pendant la même durée que celle du bail
emphytéotique. Ainsi, le contrat de bail emphytéotique conclu entre la commune de Rieumes et l'OPH de la
Haute-Garonne et la convention de mise à disposition conclue entre l'office et l'EHPAD La Prade de Rieumes
doivent être regardés come ayant formé un ensemble contractuel indivisible.
12. La cour, pour l'application du droit national mettant en oeuvre la directive susmentionnée, ne saurait être
tenue par la qualification donnée par les parties à l'opération en litige. Ainsi, le fait que, en raison du montage
de l'opération, l'ouvrage a été réalisé par l'OPH en son nom propre et doit être exploité par ce dernier jusqu'à
sa rétrocession à la commune en pleine propriété au terme du bail emphytéotique ou bien à la suite de l'exercice
par la commune de l'option de rachat anticipé de l'EHPAD stipulée par le protocole additionnel, n'est pas de
nature à enlever à cette dernière la qualité de pouvoir adjudicateur par rapport à la réalisation d'un tel ouvrage
alors même que la commune n'a joué ni pendant la réalisation de l'ouvrage ni avant le terme du bail le rôle de
maître d'ouvrage.
13. La nature de cette opération et ses modalités ont été définies de manière précise par la commune, en fonction
du besoin de cette dernière de se doter d'un établissement d'accueil pour personnes âgées dépendantes dont elle
était dépourvue. Dans ces conditions, ledit contrat et ladite convention, formant une seule et même opération,
présentent en réalité le caractère d'un marché public de travaux ayant pour objet, dans le cadre d'un contrat à
titre onéreux conclu entre l'OPH et la commune, la réalisation d'un ouvrage répondant aux besoins de la
commune, au sens et pour l'application de la directive 2004/18/CE susvisée. En vertu de cet accord, l'OPH de la
Haute-Garonne a fourni à la collectivité un EHPAD d'abord exploité par un établissement public émanant
directement de cette dernière, contre une rémunération prenant la forme d'un loyer payé par le gestionnaire de
l'établissement et calculé sur la base du coût réel de l'opération incluant les emprunts souscrits par l'OPH, ainsi
qu'il est stipulé à l'article 6 de la convention de mise à disposition susmentionnée. L'accord prévoyait en outre
une régularisation du paiement des indemnités d'occupation dues pour la période de 2008 à 2013.
14. En vertu de l'article 40 du code des marchés publics applicable aux faits du litige, la passation de ce marché
public, d'un montant de plus de 7,5 millions d'euros dépassant le seuil fixé par la législation communautaire
mise en oeuvre par le droit français, était soumise aux obligations de publicité instituées par la directive.
15. Dès lors, il appartenait à la commune de Rieumes d'assurer, préalablement à la passation du marché, des
conditions de mise en concurrence conformes aux objectifs de la directive \no  2004/18/CE et du code des
marchés publics. Or, il est constant que tel n'a pas été le cas, la collectivité ayant décidé dès l'origine de faire
réaliser son projet de création d'une maison de retraite médicalisée par l'OPH de la Haute-Garonne. Par suite,
les requérants sont fondés à soutenir que les délibérations en litige sont entachées d'illégalité.
Conseil d'État 10 février 2017 \No  395433
1. Considérant qu'il ressort des énonciations de l'arrêt attaqué que, par une délibération des 22 et 23 avril 2013,
le Conseil de Paris a approuvé la division en volumes du site de l'Institut des cultures d'Islam (ICI) situé 56, rue
Stephenson et 23, rue Doudeauville, dans le 18ème arrondissement, ainsi que la conclusion avec la société des
Habous et des Lieux Saints de l'Islam d'un bail emphytéotique administratif sur les volumes destinés à servir
d'assiette à des locaux cultuels pour une durée de quatre-vingt-dix-neuf ans moyennant un loyer capitalisé d'un
euro, et la cession à cette association, dans le cadre d'une vente d'immeuble à construire, des constructions à
vocation cultuelle devant être réalisées par la ville sur le site ; que, par cette même délibération, le Conseil de
Paris a approuvé les caractéristiques juridiques, techniques et financières, essentielles et déterminantes,
nécessaires à la mise en oeuvre de ces opérations, et a autorisé le maire à signer tous les actes nécessaires à81
cette mise en oeuvre, notamment à constituer toutes les servitudes nécessaires et à participer à toute association
syndicale libre dont la ville de Paris sera membre ; que M. B...a saisi le tribunal administratif de Paris d'une
demande tendant à l'annulation pour excès de pouvoir de cette délibération ; qu'il a demandé également
l'annulation de la décision du maire de conclure le bail emphytéotique administratif consenti à la société des
Habous et des Lieux Saints de l'Islam ; que par un jugement du 20 mai 2014, le tribunal administratif de Paris a
rejeté cette demande ; que par un arrêt du 26 octobre 2015 contre lequel la ville de Paris se pourvoit en
cassation, la cour administrative d'appel de Paris a, sur appel de M.B..., annulé le jugement du 20 mai 2014
ainsi que la délibération des 22 et 23 avril 2013 et la décision du maire de conclure le bail emphytéotique
administratif ;
4. Considérant, par ailleurs, que l'article L. 451-1 du code rural dispose : " Le bail emphytéotique de biens
immeubles confère au preneur un droit réel susceptible d'hypothèque ; ce droit peut être cédé et saisi dans les
formes prescrites pour la saisie immobilière. / Ce bail doit être consenti pour plus de dix-huit années et ne peut
dépasser quatre-vingt-dix-neuf ans ; il ne peut se prolonger par tacite reconduction. " ; qu'aux termes de
l'article L. 1311-2 du code général des collectivités territoriales, dans sa rédaction en vigueur à la date des
actes attaqués : " Un bien immobilier appartenant à une collectivité territoriale peut faire l'objet d'un bail
emphytéotique prévu à l'article L. 451-1 du code rural, (...) en vue de l'affectation à une association cultuelle
d'un édifice du culte ouvert au public (...) " ;
5. Considérant qu'il résulte des dispositions précitées de la loi du 9 décembre 1905 que les collectivités
publiques peuvent seulement financer les dépenses d'entretien et de conservation des édifices servant à l'exercice
public d'un culte dont elles sont demeurées ou devenues propriétaires lors de la séparation des Eglises et de
l'Etat ou accorder des concours aux associations cultuelles pour des travaux de réparation d'édifices cultuels et
qu'il leur est interdit d'apporter une aide à l'exercice d'un culte ; que les collectivités publiques ne peuvent donc,
aux termes de ces dispositions, apporter aucune contribution directe ou indirecte à la construction de nouveaux
édifices cultuels ;
6. Considérant, toutefois, que l'article L. 1311-2 du code général des collectivités territoriales a ouvert aux
collectivités territoriales la faculté, dans le respect du principe de neutralité à l'égard des cultes et du principe
d'égalité, d'autoriser un organisme qui entend construire un édifice du culte ouvert au public à occuper pour une
longue durée une dépendance de leur domaine privé ou de leur domaine public, dans le cadre d'un bail
emphytéotique, dénommé bail emphytéotique administratif et soumis aux conditions particulières posées par
l'article L. 1311-3 du code général des collectivités territoriales ; que le législateur a ainsi permis aux
collectivités territoriales de conclure un tel contrat en vue de la construction d'un nouvel édifice cultuel, avec
pour contreparties, d'une part, le versement, par l'emphytéote, d'une redevance qui, eu égard à la nature du
contrat et au fait que son titulaire n'exerce aucune activité à but lucratif, ne dépasse pas, en principe, un
montant modique, d'autre part, l'incorporation dans leur patrimoine, à l'expiration du bail, de l'édifice construit,
dont elles n'auront pas supporté les charges de conception, de construction, d'entretien ou de conservation ;
qu'il a, ce faisant, dérogé aux dispositions précitées de la loi du 9 décembre 1905 ; que cependant, cette faculté
n'est ouverte qu'à la condition que l'affectataire du lieu de culte édifié dans le cadre de ce bail soit, ainsi que
l'impliquent les termes mêmes de l'article L. 1311-2 du code général des collectivités territoriales, une
association cultuelle, c'est-à-dire une association satisfaisant aux prescriptions du titre IV de la loi du 9
décembre 1905 ; que, dans l'hypothèse où l'affectataire ne serait pas l'emphytéote, un tel bail n'est légal que s'il
comporte une clause résolutoire garantissant l'affectation du lieu à une association cultuelle satisfaisant aux
prescriptions du titre IV de la loi du 9 décembre 1905 ;
Article L1311-3 : Les baux passés en application de l'article L. 1311-2 satisfont aux conditions particulières
suivantes :
1\degre  Les droits résultant du bail ne peuvent être cédés, avec l'agrément de la collectivité territoriale, qu'à une
personne subrogée au preneur dans les droits et obligations découlant de ce bail et, le cas échéant, des
conventions non détachables conclues pour l'exécution du service public ou la réalisation de l'opération d'intérêt
général ;
2\degre  Le droit réel conféré au titulaire du bail de même que les ouvrages dont il est propriétaire sont susceptibles
d'hypothèque uniquement pour la garantie des emprunts contractés par le preneur en vue de financer la
réalisation ou l'amélioration des ouvrages situés sur le bien loué.
Ces emprunts sont pris en compte pour la détermination du montant maximum des garanties et cautionnements
qu'une collectivité territoriale est autorisée à accorder à une personne privée.
Le contrat constituant l'hypothèque doit, à peine de nullité, être approuvé par la collectivité territoriale ;
3\degre  Seuls les créanciers hypothécaires peuvent exercer des mesures conservatoires ou des mesures d'exécution
sur les droits immobiliers résultant du bail.82
La collectivité territoriale a la faculté de se substituer au preneur dans la charge des emprunts en résiliant ou en
modifiant le bail et, le cas échéant, les conventions non détachables. Elle peut également autoriser la cession
conformément aux dispositions du 1\degre  ci-dessus ;
4\degre  Les litiges relatifs à ces baux sont de la compétence des tribunaux administratifs ;
5\degre  Les constructions réalisées dans le cadre de ces baux peuvent donner lieu à la conclusion de contrats de
crédit-bail. Dans ce cas, le contrat comporte des clauses permettant de préserver les exigences du service public
6\degre  Lorsqu'une rémunération est versée par la personne publique au preneur, cette rémunération distingue, pour
son calcul, les coûts d'investissement, de fonctionnement et de financement.
2) Conditions des BEA passés par l’Etat
Article L 2341-1 Créé par LOI \no 2010-853 du 23 juillet 2010 - art. 11 (cf article de E Fatôme et M Raunet
AJDA 2010 2475), modifié en 2016
– Un bien immobilier appartenant à l'Etat ou à un établissement public mentionné au onzième alinéa
de l'article L. 710-1 du code de commerce, au premier alinéa de l'article 5-1 du code de l'artisanat ou
à l'article L. 510-1 du code rural et de la pêche maritime peut faire l'objet d'un bail emphytéotique
prévu à l'article L. 451-1 du même code, en vue de sa restauration, de sa réparation ou de sa mise en
valeur. Ce bail est dénommé bail emphytéotique administratif.
Un tel bail peut être conclu même s'il porte sur une dépendance du domaine public. Il ne peut avoir
pour objet l'exécution de travaux, la livraison de fournitures, la prestation de services, ou la gestion
d'une mission de service public, avec une contrepartie économique constituée par un prix ou un droit
d'exploitation, pour le compte ou pour les besoins d'un acheteur soumis à l'ordonnance \no  2015-899
du 23 juillet 2015 relative aux marchés publics ou d'une autorité concédante soumise à l'ordonnance
\no  2016-65 du 29 janvier 2016 relative aux contrats de concession.
Dans le cas où un tel bail serait nécessaire à l'exécution d'un contrat de la commande publique, ce
contrat prévoit, dans le respect des dispositions du présent code, les conditions de l'occupation du
domaine.
Il peut prévoir l'obligation pour le preneur de se libérer du paiement de la redevance d'avance, pour
tout ou partie de la durée du bail.
II. – Lorsque le bien objet du bail emphytéotique fait partie du domaine public de la personne
publique, le bail conclu en application du I satisfait aux conditions particulières suivantes :
1\degre  Les droits résultant du bail ne peuvent être cédés, avec l'agrément de la personne publique
propriétaire, qu'à une personne subrogée au preneur dans les droits et obligations découlant de ce
bail et, le cas échéant, des conventions non détachables conclues pour la réalisation de l'opération.
Par dérogation à l'alinéa précédent, les droits résultant du bail ne peuvent faire l'objet d'une cession
lorsque le respect des obligations de publicité et de sélection préalables à la délivrance d'un titre,
prévues à l'article L. 2122-1-1, s'y oppose ;
2\degre  Le droit réel conféré au preneur et les ouvrages dont il est propriétaire ne peuvent être
hypothéqués qu'en vue de garantir des emprunts contractés par le preneur pour financer la réalisation
des obligations qu'il tient du bail ; le contrat constituant l'hypothèque doit, à peine de nullité, être
approuvé par la personne publique propriétaire ;
3\degre  Seuls les créanciers hypothécaires peuvent exercer des mesures conservatoires ou des mesures
d'exécution sur les droits immobiliers résultant du bail. La personne publique propriétaire peut se
substituer au preneur dans la charge des emprunts en résiliant ou en modifiant le bail et, le cas
échéant, les conventions non détachables ;
4\degre  Les modalités de contrôle de l'activité du preneur par la personne publique propriétaire sont
prévues dans le bail ;
5\degre  Les constructions réalisées dans le cadre de ce bail peuvent donner lieu à la conclusion de contrats
de crédit-bail. Dans ce cas, le contrat comporte des clauses permettant de préserver les exigences du
service public.
III. – L'une ou plusieurs de ces conditions peuvent également être imposées au preneur lorsque le bien
fait partie du domaine privé de la personne publique .
3\degre ) Conditions des Autorisations d’Occupation Temporaires (AOT)83
- conditions de fond larges pour l’Etat
*Toute constitution de droit réel est interdite sur le domaine public naturel de l’Etat (art L 2122-5 )
* Le droit réel doit être compatible avec l’utilisation du domaine public.
* l’AOT peut être anticipée : art L 2122-1 al 2 et 3 : Le titre mentionné à l'alinéa précédent peut être
accordé pour occuper ou utiliser une dépendance du domaine privé d'une personne publique par
anticipation à l'incorporation de cette dépendance dans le domaine public, lorsque l'occupation ou
l'utilisation projetée le justifie.
Dans ce cas, le titre fixe le délai dans lequel l'incorporation doit se produire, lequel ne peut être
supérieur à six mois, et précise le sort de l'autorisation ainsi accordée si l'incorporation ne s'est pas
produite au terme de ce délai.
- Conditions de fond plus strictes pour les AOT des collectivités territoriales : Article L1311-5 CGCL
Les collectivités territoriales peuvent délivrer sur leur domaine public des autorisations d'occupation temporaire
constitutives de droits réels ou en vue de la réalisation d'une opération d'intérêt général relevant de leur
compétence. Le titulaire de ce titre possède un droit réel sur les ouvrages, constructions et installations de
caractère immobilier qu'il réalise pour l'exercice de cette activité.
Ce droit réel confère à son titulaire, pour la durée de l'autorisation et dans les conditions et les limites précisées
dans la présente section, les prérogatives et obligations du propriétaire.
Le titre fixe la durée de l'autorisation, en fonction de la nature de l'activité et de celle des ouvrages autorisés, et
compte tenu de l'importance de ces derniers, sans pouvoir excéder soixante-dix ans.
Ces dispositions sont applicables aux groupements et aux établissements publics des collectivités territoriales,
tant pour leur propre domaine public que pour celui mis à leur disposition.
II. - Dans les ports et les aéroports, sont considérées comme satisfaisant à la condition d'intérêt public local
mentionnée au premier alinéa du I les activités ayant trait à l'exploitation du port ou de l'aéroport ou qui sont de
nature à contribuer à leur animation ou à leur développement.
III.-Les collectivités territoriales ne peuvent utiliser ces autorisations d'occupation temporaire constitutives de
droits réels pour l'exécution de travaux, la livraison de fournitures, la prestation de services, ou la gestion d'une
mission de service public, avec une contrepartie économique constituée par un prix ou un droit d'exploitation,
pour leur compte ou pour leurs besoins.
Dans le cas où une autorisation d'occupation temporaire constitutive de droits réels serait nécessaire à
l'exécution d'un contrat de la commande publique, ce contrat prévoit, dans le respect des dispositions du I et du
code général de la propriété des personnes publiques, les conditions de l'occupation du domaine.
IV. - Les constructions mentionnées au présent article peuvent donner lieu à la conclusion de contrats de crédit-
bail. Dans ce cas, le contrat comporte des clauses permettant de préserver les exigences du service public.
Possibilité de passer des AOT même pour des parties du domaine public ou s’appliquent des contraventions de
voirie
La durée n’est pas une condition de validité ; CE 5 fev 2009 \no  305021 AJDA 2009 704 : Considérant que, si les
autorisations d'occupation du domaine public doivent en principe être délivrées pour une durée déterminée,
ainsi que le rappelle l'article L. 2122-2 du code général de la propriété des personnes publiques, la seule
circonstance qu'une convention ne conférant pas de droits réels à l'occupant du domaine public ne contenait
aucune précision relative à sa durée n'est pas de nature à entacher celle-ci de nullité ; qu'en effet, dans le
silence sur ce point de la convention, le principe d'inaliénabilité du domaine public, qui s'applique sauf texte
législatif contraire, implique que l'autorité gestionnaire du domaine peut mettre fin à tout moment, sous réserve
de justifier cette décision par un motif d'intérêt général, à l'autorisation d'occupation qu'elle a consentie ; que,
par suite, en jugeant que la clause qui, dans les conventions autorisant l'occupation du domaine public, en fixe
la durée revêt un caractère substantiel dont l'absence est de nature à entacher une telle convention de nullité, la
cour administrative d'appel de Marseille a commis une erreur de droit ;
B Conditions de forme et de mise en concurrence
1) Nécessité d’une mise en concurrence
Depuis l’ordonnance du 19 avril 2017
Article L2122-1-1
Sauf dispositions législatives contraires, lorsque le titre mentionné à l'article L. 2122-1 permet à son titulaire
d'occuper ou d'utiliser le domaine public en vue d'une exploitation économique, l'autorité compétente organise
librement une procédure de sélection préalable présentant toutes les garanties d'impartialité et de transparence,
et comportant des mesures de publicité permettant aux candidats potentiels de se manifester.84
Lorsque l'occupation ou l'utilisation autorisée est de courte durée ou que le nombre d'autorisations disponibles
pour l'exercice de l'activité économique projetée n'est pas limité, l'autorité compétente n'est tenue que de
procéder à une publicité préalable à la délivrance du titre, de nature à permettre la manifestation d'un intérêt
pertinent et à informer les candidats potentiels sur les conditions générales d'attribution.
Article L2122-1-2
L'article L. 2122-1-1 n'est pas applicable :
1\degre  Lorsque la délivrance du titre mentionné à l'article L. 2122-1 s'insère dans une opération donnant lieu à une
procédure présentant les mêmes caractéristiques que la procédure déterminée par le premier alinéa de l'article
L. 2122-1-1 ;
2\degre  Lorsque le titre d'occupation est conféré par un contrat de la commande publique ou que sa délivrance
s'inscrit dans le cadre d'un montage contractuel ayant, au préalable, donné lieu à une procédure de sélection ;
3\degre  Lorsque l'urgence le justifie. La durée du titre ne peut alors excéder un an ;
4\degre  Sans préjudice des dispositions figurant aux 1\degre  à 5\degre  de l'article L. 2122-1-3, lorsque le titre a pour seul objet
de prolonger une autorisation existante, sans que sa durée totale ne puisse excéder celle prévue à l'article L.
2122-2 ou que cette prolongation excède la durée nécessaire au dénouement, dans des conditions acceptables
notamment d'un point de vue économique, des relations entre l'occupant et l'autorité compétente.
Article L2122-1-3
L'article L. 2122-1-1 n'est pas non plus applicable lorsque l'organisation de la procédure qu'il prévoit s'avère
impossible ou non justifiée. L'autorité compétente peut ainsi délivrer le titre à l'amiable, notamment dans les cas
suivants :
1\degre  Lorsqu'une seule personne est en droit d'occuper la dépendance du domaine public en cause ;
2\degre  Lorsque le titre est délivré à une personne publique dont la gestion est soumise à la surveillance directe de
l'autorité compétente ou à une personne privée sur les activités de laquelle l'autorité compétente est en mesure
d'exercer un contrôle étroit ;
3\degre  Lorsqu'une première procédure de sélection s'est révélée infructueuse ou qu'une publicité suffisante pour
permettre la manifestation d'un intérêt pertinent est demeurée sans réponse ;
4\degre  Lorsque les caractéristiques particulières de la dépendance, notamment géographiques, physiques,
techniques ou fonctionnelles, ses conditions particulières d'occupation ou d'utilisation, ou les spécificités de son
affectation le justifient au regard de l'exercice de l'activité économique projetée ;
5\degre  Lorsque des impératifs tenant à l'exercice de l'autorité publique ou à des considérations de sécurité publique
le justifient.
Lorsqu'elle fait usage de la dérogation prévue au présent article, l'autorité compétente rend publiques les
considérations de droit et de fait l'ayant conduite à ne pas mettre en œuvre la procédure prévue à l'article L.
2122-1-1.
Article L2122-1-4
Lorsque la délivrance du titre mentionné à l'article L. 2122-1 intervient à la suite d'une manifestation d'intérêt
spontanée, l'autorité compétente doit s'assurer au préalable par une publicité suffisante, de l'absence de toute
autre manifestation d'intérêt concurrente.
2)Nécessité d’une autorisation
*Les modalités de la demande et de l’autorisation sont précisées aux articles R 2122-1 et suivants du
CG3P
* L’autorité compétente pour délivrer l’autorisation est mentionnée aux articles R 2122-14 et R 2122-
15.
Cf D 2012- 1093 du 27 sept 2012 « Art. R. 2122-30-1. - Tout projet de bail soumis à la réalisation
d'une évaluation préalable en application de l'article R. 2122-30 donne lieu à une étude réalisée par
l'autorité administrative visant à évaluer l'ensemble des conséquences de l'opération sur les finances
publiques et la disponibilité des crédits ainsi que sa compatibilité avec les orientations de la politique
immobilière de l'Etat.
« L'étude est réalisée concomitamment à l'évaluation préalable.
« Elle est transmise à l'autorité chargée du contrôle financier et aux ministres chargés de l'économie,
du budget et du domaine ainsi qu'à l'organisme expert mentionné à l'article 2 de l'ordonnance \no  2004-
559 du 17 juin 2004 sur les contrats de partenariat.
« Elle est actualisée sur demande du ministre chargé du budget ou si le projet connaît des évolutions
significatives. »85
C – Les caractères du droit réel conféré
Article R2122-6
Le titre fixe la durée de l'autorisation et les conditions juridiques et financières de l'occupation ou de l'utilisation
du domaine public
1\degre ) La nature
- Droit réel du preneur du bail emphytéotique art L1311-3 CGCL
2\degre  Le droit réel conféré au titulaire du bail de même que les ouvrages dont il est propriétaire sont susceptibles
d'hypothèque uniquement pour la garantie des emprunts contractés par le preneur en vue de financer la
réalisation ou l'amélioration des ouvrages situés sur le bien loué.
Ces emprunts sont pris en compte pour la détermination du montant maximum des garanties et cautionnements
qu'une collectivité territoriale est autorisée à accorder à une personne privée.
Le contrat constituant l'hypothèque doit, à peine de nullité, être approuvé par la collectivité territoriale ;
- Droit réel du titulaire d’une AOT
* de l’Etat : Article L. 2122-6
Le titulaire d'une autorisation d'occupation temporaire du domaine public de l'Etat a, sauf prescription
contraire de son titre, un droit réel sur les ouvrages, constructions et installations de caractère immobilier qu'il
réalise pour l'exercice d'une activité autorisée par ce titre.
Ce droit réel confère à son titulaire, pour la durée de l'autorisation et dans les conditions et les limites précisées
dans le présent paragraphe, les prérogatives et obligations du propriétaire.
Le titre fixe la durée de l'autorisation, en fonction de la nature de l'activité et de celle des ouvrages autorisés, et
compte tenu de l'importance de ces derniers, sans pouvoir excéder soixante-dix ans.
Une autorisation d'occupation temporaire ne peut avoir pour objet l'exécution de travaux, la livraison de
fournitures, la prestation de services, ou la gestion d'une mission de service public, avec une contrepartie
économique constituée par un prix ou un droit d'exploitation, pour le compte ou pour les besoins d'un acheteur
soumis à l'ordonnance \no  2015-899 du 23 juillet 2015 relative aux marchés publics ou d'une autorité concédante
soumise à l'ordonnance \no  2016-65 du 29 janvier 2016 relative aux contrats de concession.
Dans le cas où un titre d'occupation serait nécessaire à l'exécution d'un contrat de la commande publique, ce
contrat prévoit, dans le respect des dispositions du présent code, les conditions de l'occupation du domaine.
Quid du droit réel sur le sol ; cf infra en matière de cession
cf CE 11 mai 2016 commune de Fos sur mer \No  390118
5. Considérant que, contrairement à ce que relève l'arrêt attaqué, le droit réel dont bénéficie, en vertu de
l'article L. 34-1 du code du domaine de l'Etat, repris à l'article L. 2122-6 du code général de la propriété des
personnes publiques, le titulaire d'une autorisation d'occupation temporaire du domaine de l'Etat, ne porte pas
uniquement sur les ouvrages, constructions et installations que réalise le preneur mais inclut le terrain d'assiette
de ces constructions ; que, par suite, en jugeant que la convention, qualifiée par les parties de bail à
construction, conclue le 21 mars 2005 sur le domaine public du port autonome de Marseille était incompatible
avec les règles de gestion du domaine public au seul motif qu'un bail à construction confère au preneur, en vertu
des dispositions citées ci-dessus, des droits réels sur le sol au-delà des seuls " constructions, ouvrages et
installations " mentionnés à l'article L. 34-1 du code du domaine de l'Etat, la cour administrative d'appel de
Marseille a commis une erreur de droit ;
Article L. 2122-11 Les dispositions du présent paragraphe sont également applicables aux conventions de toute
nature ayant pour effet d'autoriser l'occupation du domaine public.
Lorsque ce droit d'occupation du domaine public résulte d'une concession de service public ou d'outillage
public, le cahier des charges précise les conditions particulières auxquelles il doit être satisfait pour tenir
compte des nécessités du service public.
Article R 2122-17 Créé par Décret \no 2011-1612 du 22 novembre 2011 Le titre d'occupation constitutif de droit
réel comporte la détermination précise de la consistance de ce droit et de la durée pour laquelle il est conféré
ainsi que toutes autres mentions nécessaires à la publicité foncière.
Il comporte aussi l'énoncé des conditions auxquelles ce droit est conféré, à savoir :
1\degre  Les ouvrages, constructions et installations de caractère immobilier à édifier et, le cas échéant, la liste de
ceux qui doivent être maintenus en état jusqu'à l'expiration de la durée de validité du titre ;
2\degre  Le montant et les modalités de paiement de la redevance domaniale ;86
3\degre  Le cas échéant, les obligations de service public imposées au titulaire de l'autorisation.
Il peut également préciser les règles de détermination de l'indemnité couvrant le préjudice direct, matériel et
certain causé par son retrait avant le terme prévu pour un motif autre que l'inexécution de ses conditions.
* d’une collectivité territoriale : art 1311-5 CGCL ; Le titulaire de ce titre possède un droit réel sur les
ouvrages, constructions et installations de caractère immobilier qu'il réalise pour l'exercice de cette activité.
Ce droit réel confère à son titulaire, pour la durée de l'autorisation et dans les conditions et les limites
précisées dans la présente section, les prérogatives et obligations du propriétaire.
Article L1100-1 du code de la commande publique
Ne sont pas soumis au présent code, outre les contrats de travail, les contrats ou conventions ayant pour objet
:...;
3\degre  L'occupation domaniale
2\degre ) Contenu
- droit et obligations du preneur bail emphytéotique (renvoi au Code rural
- droit variable pour AOT (cf R 2122-17 CGPPP)
3\degre ) Durée
- celle du bail emphytéotique (mais précaire et révocable :cf art L 2122-3)
- AOT des collectivités et de l’Etat : Art 1311-5 CGCL ; 2122-6 CGPPP Le titre fixe la durée de l'autorisation,
en fonction de la nature de l'activité et de celle des ouvrages autorisés, et compte tenu de l'importance de ces
derniers, sans pouvoir excéder soixante-dix ans.
- possibilité de retrait anticipé de l’autorisation L 2122-9 CGPPP ; 1311-7 CGCL
Toutefois, en cas de retrait de l'autorisation avant le terme prévu, pour un motif autre que l'inexécution de ses
clauses et conditions, le titulaire est indemnisé du préjudice direct, matériel et certain né de l'éviction anticipée.
Les règles de détermination de l'indemnité peuvent être précisées dans le titre d'occupation. Les droits des
créanciers régulièrement inscrits à la date du retrait anticipé sont reportés sur cette indemnité.
Mais indemnisation du préjudice direct, matériel et certain.
Article R2122-18 Créé par Décret \no 2011-1612 du 22 novembre 2011 Dans le cas où l'autorité qui a délivré le
titre constitutif de droit réel envisage, pour quelque motif que ce soit, de le retirer en totalité ou en partie avant
le terme fixé, le titulaire du titre à cette date en est informé par pli recommandé avec demande d'avis de
réception, deux mois au moins avant le retrait, sauf respect, en cas de concession, du délai particulier prévu par
le contrat.
Dans le cas où le retrait envisagé a pour motif l'inexécution des clauses et conditions de l'autorisation, l'autorité
qui l'a délivrée en informe les créanciers régulièrement inscrits, selon les mêmes modalités, deux mois au moins
avant le retrait.
D – La cession du droit réel conféré
-Principe de cessibilité du droit :
AOT Article L. 2122-7 CGPPP ; 1311-6 CGCT
Le droit réel conféré par le titre, les ouvrages, constructions et installations de caractère immobilier ne peuvent
être cédés, ou transmis dans le cadre de mutations entre vifs ou de fusion, absorption ou scission de sociétés,
pour la durée de validité du titre restant à courir, y compris dans le cas de réalisation de la sûreté portant sur
lesdits droits et biens et dans les cas mentionnés aux premier et deuxième alinéas de l'article L. 2122-8, qu'à une
personne agréée par l'autorité compétente, en vue d'une utilisation compatible avec l'affectation du domaine
public occupé.87
Le titulaire de l'autorisation d'occupation temporaire peut demander à l'autorité qui a délivré le titre de lui
indiquer si, au vu des éléments qui lui sont soumis à ce stade et sous réserve d'un changement ultérieur dans les
circonstances de fait ou de droit qui l'obligerait à revenir sur sa décision, elle délivrera l'agrément à une
personne déterminée qui lui sera substituée, pour la durée de validité du titre restant à courir, dans les droits et
obligations résultant de ce titre. Toutefois, ces dispositions ne sont pas applicables aux autorisations
d'occupation du domaine public qui ont été délivrées après une procédure de publicité et de mise en
concurrence.
Lors du décès d'une personne physique titulaire d'un titre d'occupation constitutif de droit réel, celui-ci peut être
transmis, dans les conditions mentionnées au premier alinéa, au conjoint survivant ou aux héritiers sous réserve
que le bénéficiaire, désigné par accord entre eux, soit présenté à l'agrément de l'autorité compétente dans un
délai de six mois à compter du décès.
Textes d’application : art R 2122-19 et s duCG3P
Article R2122-19
Préalablement à la signature de tout contrat ayant pour objet ou pour effet la transmission entre vifs, totale ou
partielle, du droit réel conféré par le titre d'occupation et des immeubles mentionnés à l'article L. 2122-7, la
personne qui, par l'effet de ce contrat, se trouve totalement ou partiellement substituée au titulaire de ce titre est
agréée par l'autorité qui l'a délivré. Il en va de même pour tout contrat produisant le même effet à la suite d'une
fusion, absorption ou scission de sociétés
BEA Article L1311- 3 1\degre  Les droits résultant du bail ne peuvent être cédés, avec l'agrément de la collectivité
territoriale, qu'à une personne subrogée au preneur dans les droits et obligations découlant de ce bail et, le cas
échéant, des conventions non détachables conclues pour l'exécution du service public ou la réalisation de
l'opération d'intérêt général ;
Par dérogation à l'alinéa précédent, les droits résultant du bail ne peuvent faire l'objet d'une cession lorsque le
respect des obligations de publicité et de sélection préalables à la délivrance d'un titre, prévues à l'article L.
2122-1-1 du code général de la propriété des personnes publiques, s'y oppose ;
BEA Etat art 2341-1 : 1\degre  Les droits résultant du bail ne peuvent être cédés, avec l'agrément de la
personne publique propriétaire, qu'à une personne subrogée au preneur dans les droits et obligations
découlant de ce bail et, le cas échéant, des conventions non détachables conclues pour la réalisation
de l'opération.
Par dérogation à l'alinéa précédent, les droits résultant du bail ne peuvent faire l'objet d'une cession
lorsque le respect des obligations de publicité et de sélection préalables à la délivrance d'un titre,
prévues à l'article L. 2122-1-1, s'y oppose ;
E – La situation du créancier sur le droit réel
1\degre ) Créanciers chirographaires
- interdiction pour eux d’exercer des mesures conservatoires ou d’exécution sur les droits immobiliers résultant
* du bail L 1311-3 CGCT : 3\degre  Seuls les créanciers hypothécaires peuvent exercer des mesures conservatoires
ou des mesures d'exécution sur les droits immobiliers résultant du bail.
La collectivité territoriale a la faculté de se substituer au preneur dans la charge des emprunts en résiliant ou en
modifiant le bail et, le cas échéant, les conventions non détachables. Elle peut également autoriser la cession
conformément aux dispositions du 1\degre  ci-dessus ;
* de l’AOT : Article L. 2122-8 CGPPP : Les créanciers chirographaires autres que ceux dont la créance est née
de l'exécution des travaux mentionnés à l'alinéa précédent ne peuvent pratiquer des mesures conservatoires ou
des mesures d'exécution forcée sur les droits et biens mentionnés au présent article.
(même chose dans l’article 1311-6-1 CGCT)
2\degre ) Créanciers hypothécaires88
Art 2122-8 CGPPP ; 1311-6-1 CGCT : Le droit réel conféré par le titre, les ouvrages, constructions et
installations ne peuvent être hypothéqués que pour garantir les emprunts contractés par le titulaire de
l'autorisation en vue de financer la réalisation, la modification ou l'extension des ouvrages, constructions et
installations de caractère immobilier situés sur la dépendance domaniale occupée.
Les hypothèques sur lesdits droits et biens s'éteignent au plus tard à l'expiration des titres d'occupation délivrés
en application des articles L. 2122-6 et L. 2122-10, quels qu'en soient les circonstances et le motif.
Art 1311-3 pour BEA collectivité : 2o Le droit réel conféré au titulaire du bail de même que les ouvrages dont il
est propriétaire sont susceptibles d'hypothèque uniquement pour la garantie des emprunts contractés par le
preneur en vue de financer la réalisation ou l'amélioration des ouvrages situés sur le bien loué.
Ces emprunts sont pris en compte pour la détermination du montant maximum des garanties et cautionnements
qu'une collectivité territoriale est autorisée à accorder à une personne privée.
Le contrat constituant l'hypothèque doit, à peine de nullité, être approuvé par la collectivité territoriale ;
règles communes
Art 2341-1 pour le BEA Etat : « 3\degre  Seuls les créanciers hypothécaires peuvent exercer des mesures
conservatoires ou des mesures d'exécution sur les droits immobiliers résultant du bail. La personne publique
propriétaire peut se substituer au preneur dans la charge des emprunts en résiliant ou en modifiant le bail et, le
cas échéant, les conventions non détachables ; »
-En cas de retrait pour inexécution : obligation d’avertir les créanciers 2 mois avant pour qu’il puisse proposer
quelqu’un susceptible de reprendre les engagements art L 2122-9 ; art 1311-7 : Deux mois au moins avant la
notification d'un retrait pour inexécution des clauses et conditions de l'autorisation, les créanciers régulièrement
inscrits sont informés des intentions de l'autorité compétente à toutes fins utiles, et notamment pour être mis en
mesure de proposer la substitution d'un tiers au permissionnaire défaillant ou de s'y substituer eux-mêmes.
Texte d’application art R 2122-25
-En cas de retrait sans inexécution : droit des créanciers reporté sur l’indemnité L 2122-9 al 3.
F La situation des locataires du preneur
Cour de cassation chambre civile 3 19 décembre 2012 \No  de pourvoi: 11-10372
Vu l'article L. 145-2 I 3\degre  du code de commerce, ensemble les articles L. 145-1 du code de commerce, L. 1311-2
et L. 1311-3 du code général des collectivités territoriales ;
Attendu que les dispositions du chapitre V du livre premier du code de commerce s'appliquent aux baux
d'immeubles ou de locaux principaux ou accessoires, nécessaires à la poursuite de l'activité des entreprises
publiques et établissements publics à caractère industriel ou commercial, dans les limites définies par les lois et
règlements qui les régissent et à condition que ces baux ne comportent aucune emprise sur le domaine public ;
Attendu, selon l'arrêt attaqué (Aix-en-Provence, 21 octobre 2010) que la Ville de Saint-Jean-Cap-Ferrat a
consenti par acte du 21 novembre 1994 à la société Andremax un bail emphytéotique pour une durée de 30 ans
expirant le 23 novembre 2024, portant sur deux cellules commerciales \no  8 et 9 situées sur le domaine public de
la commune ; que par acte du 5 janvier 2007, la société Andremax a sous-loué ces deux lots à M. X... pour une
durée de 18 mois à compter du 1er janvier 2008 ; que la société Andremax a délivré congé le 26 septembre 2008
en faisant injonction au preneur de quitter les lieux le 31 décembre 2008 ; que ce dernier n'ayant pas déféré, la
société Andremax l'a assigné en validité du congé et expulsion ; que M. X... a demandé reconventionnellement
que lui soit reconnu un nouveau bail soumis au statut des baux commerciaux par application de l'article L. 145-
5 du code de commerce ;
Attendu que pour accueillir la demande reconventionnelle l'arrêt retient que l'article L. 1311-2 du code général
des collectivités territoriales permet aux collectivités publiques de consentir sur leur domaine public des baux
emphytéotiques, qu' un tel bail investit le preneur d'un droit réel sur l'immeuble objet du bail et lui donne le droit
de le sous-louer et que l'occupation du domaine public par l'emphytéote échappe à la précarité et permet de
conclure un bail commercial ;
Qu'en statuant ainsi, alors que, nonobstant la qualité d'emphytéote du bailleur, le statut des baux commerciaux
ne s'applique pas aux conventions ayant pour objet des biens dépendant du domaine public, la cour d'appel a
violé les textes susvisés ;
G Dénouement de l’opération89
Article L. 2122-9 CGPPP, art 1311-7 CGCT
A l'issue du titre d'occupation, les ouvrages, constructions et installations de caractère immobilier existant sur la
dépendance domaniale occupée doivent être démolis soit par le titulaire de l'autorisation, soit à ses frais, à
moins que leur maintien en l'état n'ait été prévu expressément par le titre d'occupation ou que l'autorité
compétente ne renonce en tout ou partie à leur démolition.
Les ouvrages, constructions et installations de caractère immobilier dont le maintien à l'issue du titre
d'occupation a été accepté deviennent de plein droit et gratuitement la propriété de l'Etat, francs et quittes de
tous privilèges et hypothèques.
F La combinaison d’une AOT et d’un crédit bail
Article L. 2122-13 Dans le cadre des titres d'occupation prévus par les articles L. 2122-6 et L. 2122-11, la
réalisation des ouvrages, constructions et installations peut donner lieu à la conclusion de contrats de crédit-
bail. Lorsque ces contrats concernent le financement d'ouvrages, de constructions et d'installations qui sont
nécessaires à la continuité d'un service public, ils comportent des clauses permettant de préserver les exigences
de ce service public.
La conclusion de tels contrats de crédit-bail au bénéfice d'organismes dans lesquels l'Etat ou l'établissement
public gestionnaire du domaine apporte un concours financier ou détient, directement ou indirectement, une
participation financière permettant d'exercer un pouvoir prépondérant de décision ou de gestion est soumise à
un agrément de l'Etat. Cet agrément peut être refusé si l'opération se traduit par un accroissement des charges
ou une diminution des ressources de l'Etat. Un décret en Conseil d'Etat fixe les modalités de cet agrément.
Texte d’application : art R 2122-27 et s du CG3P
Art 1311-5 CGCT IV. - Les constructions mentionnées au présent article peuvent donner lieu à la conclusion
de contrats de crédit-bail. Dans ce cas, le contrat comporte des clauses permettant de préserver les exigences du
service public.
	\tableofcontents
\end{document}